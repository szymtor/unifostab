\section{Introduction and contributions}\label{sec:intro}

Nowhere dense classes of graphs were introduced 
by Ne\v set\v ril and Ossona de 
Mendez~\cite{nevsetvril2010first,nevsetvril2011nowhere} as a very 
general model
for uniform sparseness of graphs. These classes generalize many 
familiar classes of sparse graphs, such as planar graphs, graphs 
of bounded treewidth,  graphs of bounded degree, and, in fact, 
all classes that exclude a fixed 
topological minor.
Formally, a class $\CCC$ of graphs is {\em{nowhere dense}} if there is a function $t\colon \N\to \N$ such that for every $r\in \N$, no graph $G$ in~$\CCC$ contains the clique $K_{t(r)}$ on $t(r)$ vertices  as an {\em{$r$-shallow minor}},
i.e., as a subgraph of a graph obtained from $G$ by contracting mutually disjoint  subgraphs of radius at most $r$ to single vertices.
Classes of bounded expansion~\cite{nevsetvril2008grad} 
are important subclasses 
of nowhere dense graph classes which are defined by bounds on 
the edge density of $r$-shallow minors.


The concept of nowhere denseness
turns out to be very robust, as witnessed by the fact that it is equivalent 
to multiple other concepts studied in different areas of mathematics. 
One can equivalently characterize nowhere dense graph classes 
by bounds on the density of (topological) shallow 
minors~\cite{nevsetvril2010first,nevsetvril2011nowhere},
quasi-wideness~\cite{nevsetvril2011nowhere} (a notion introduced by
Dawar~\cite{dawar2010homomorphism} in his study of homomorphism
preservation properties), low tree-depth
colorings~\cite{nevsetvril2008grad}, generalized coloring
numbers~\cite{zhu2009coloring}, sparse neighborhood
covers~\cite{GroheKRSS15,grohe2014deciding}, by a game called the
splitter game~\cite{grohe2014deciding}, and by the model-theoretic
concepts of stability and independence~\cite{adler2014interpreting}.
For a broader discussion on the graph theoretic sparsity we refer to the book
of Ne\v{s}et\v{r}il and Ossona de Mendez~\cite{sparsity}.

\begin{comment}
These alternative characterizations have been very useful in 
the design of efficient algorithms. For instance, 
the {\sc{Subgraph Isomorphism}} and {\sc{Homomorphism}} problems 
are fixed-parameter tractable on any nowhere dense
class, parameterized by the size of the pattern graph~\cite{nevsetvril2010first}
and so is the {\sc Distance-$r$ Dominating Set} problem, parameterized
by the size of the solution~\cite{DawarK09}. In fact, 
the {\sc Distance-$r$ Dominating Set} problem admits
polynomial kernels~\cite{siebertz2016polynomial} and even 
almost linear kernels on nowhere dense classes of 
graphs~\cite{eickmeyer2016neighborhood}
(see also~\cite{drange2016kernelization} for the case $r=1$). 
It was shown in~\cite{grohe2014deciding}
that every first-order definable problem can be decided in
almost linear time on any nowhere dense graph class.

It is a natural question to ask for the most general classes of graphs
which admit efficient solutions for certain problems, or to 
classify them into tractable and intractable classes. It was shown 
that for the first-order model-checking problem~\cite{dvovrak2013testing} and for
the {\sc Distance-$r$ Dominating Set} problem~\cite{drange2016kernelization} 
the dividing line for algorithmic tractability 
on subgraph closed classes of graphs is exactly between the
nowhere dense and somewhere dense graph classes. 
\end{comment}

\paragraph{Uniform quasi-wideness.}

In this work we revisit the connections between the notion of nowhere 
denseness and notions from  model theory and finite model theory.
We first consider the connection to uniform quasi-wideness.

%, a
%notion  introduced by Dawar~\cite{dawar2010homomorphism}, who 
%proved that every quasi-wide class that is closed under taking substructures
%and disjoint unions has the homomorphism preservation property. 


Formally, a class $\CCC$ of graphs is \emph{uniformly quasi-wide} if there are
functions $N\colon \N\times\N\rightarrow \N$ and $s\colon \N\rightarrow \N$ such
that for all $r,m\in \N$ and all subsets $A\subseteq V(G)$ for
$G\in \CCC$ of size $\abs{A}\geq N(r,m)$ there is a set
$S\subseteq V(G)$ of size $\abs{S}\leq s(r)$ and a set
$B\subseteq A\setminus S$ of size $\abs{B}\geq m$ which is $r$-independent in
$G-S$. Recall that a set $B\subseteq V(G)$ is called {\em{$r$-independent}} in $G$ if all
distinct $u,v\in B$ are at distance 
larger than $r$ in $G$.

Ne\v{s}et\v{r}il and Ossona de Mendez proved that
the notions of uniform quasi-wideness and nowhere denseness coincide for 
classes of finite graphs~\cite{nevsetvril2011nowhere}. 
The proof of Ne\v{s}et\v{r}il 
and Ossona de Mendez goes back to a construction
of Kreidler and Seese~\cite{kreidler1998monadic} (see also Atserias et al.~\cite{atserias2006preservation}), 
and uses iterated Ramsey arguments. Hence the original bounds on 
the function $N$ are huge. Recently, Kreutzer, Rabinovich and the second author
 proved that we may always choose~$N$ to be a polynomial 
function of $m$, with the degree of the polynomial depending on $r$~\cite{siebertz2016polynomial}. However, the exact dependence of degree of the polynomial on $r$ and on the class $\CCC$ itself
 was  not specified in~\cite{siebertz2016polynomial}, as the existence of a polynomial bound is derived
from non-constructive arguments used by Adler and Adler in~\cite{adler2014interpreting} when showing that every nowhere dense class of graphs
is stable. We give a new construction 
which is considerably simpler than that of~\cite{siebertz2016polynomial}
and which gives explicit bounds on the degree of the polynomial. 
More precisely, we prove the following theorem; here, the notation $\Oof_{r,t}(\cdot)$ hides computable factors depending on $r$ and $t$.

\begin{theorem}\label{thm:new-uqw}
For all $r,t\in \N$ there is a polynomial  $N\colon \N\to \N$ with $N(m)=
\Oof_{r,t}{(m^{{(8t+1)}^{2rt}})}$, such that the following holds.
Let $G$ be a graph such that $K_t\not\minor_{\lfloor 9r/2\rfloor} G$, and
let $A\subseteq V(G)$ be a vertex subset of size at least $N(m)$, for a given $m$.
Then there exists a set $S\subseteq V(G)$ of size $|S|\leq t$ and a set $B\subseteq A\setminus S$ 
of size $|B|\geq m$ which is $r$-independent in $G-S$.
Moreover, given~$G$ and $A$, such sets $S$ and $B$ can be computed in time $\Oof_{r,t}(|A|\cdot |E(G)|)$. 
\end{theorem}

Let us remark
that even though the techniques employed to prove \cref{thm:new-uqw} are inspired by methods from stability theory, 
at the end we use only very simple graph theoretic notions. In particular, as asserted in the last sentence of the theorem, the
proof can easily be turned into an efficient algorithm.

\paragraph{Stability.}


The second topic of study is the
connection between nowhere denseness and stability theory. 
Fix
% Let $\cal C$ be a class of structures over
 a relational vocabulary $\Sigma$. Let 
$\phi(\tup{x},\tup{y})$ be a $\Sigma$-formula with the free variables
partitioned into two groups $\tup{x}, \tup{y}$. A \emph{$\phi$-ladder}
of length $n$ in a $\Sigma$-structure $\str A$ is a sequence $\tup{a}_1,\ldots, \tup{a}_{n},
\tup{b}_1,\ldots, \tup{b}_{n}$ of tuples of elements of $\str A$ 
such that for all $1\leq i,j\le n$,
\[\strA\models\phi(\tup{a}_i,\tup{b}_j)\Longleftrightarrow i\leq j. \]
The least  $n$ for which 
there is no $\phi$-ladder of length $n$ is 
the \emph{ladder index} 
of $\phi(\tup{x},\tup{y})$ in $\str A$ (which may depend on the way we split the
variables). A class of graphs $\CCC$ is \emph{stable} if
the ladder index of every first-order formula $\phi(\tup{x},\tup{y})$ over
graphs from $\CCC$ is bounded by a constant depending only on $\phi$ 
and~$\CCC$. 

Podewski and Ziegler~\cite{podewski1978stable} 
consider \emph{flat graphs}, a notion corresponding to uniform quasi-wideness in the 
infinite. They show that flat graphs are stable using an 
infinite Ramsey argument and compactness. Based on the 
results of Podewski and Ziegler, 
Adler and Adler~\cite{adler2014interpreting}
proved that every nowhere dense class is stable, in fact, they 
proved that for a subgraph-closed class $\CCC$, the notions of 
nowhere denseness and stability coincide.
However, their inherently non-constructive approach cannot give explicit upper bounds on parameters governing the stability of $\CCC$, for instance, the ladder indices.

Based on the approach of Podewski and Ziegler~\cite{podewski1978stable}, we give a combinatorial 
proof that every first-order formula has finite ladder index on every
nowhere dense class, which does not involve infinite combinatorics and model theory.
In particular, instead of compactness we use Gaifman's Locality Theorem for
first-order logic~\cite{gaifman1982local}. The following theorem summarizes our~result.

\begin{theorem}\label{thm:new-stable}
  There are computable functions $f\colon \N^3\to\N$ and $g\colon\N\to\N$ with the following property.
Suppose $\phi(\bar x,\bar y)$ is a formula of quantifier rank $q$ and with $d$ free variables,
and $G$ is a graph such that $K_t\not\minor_{g(q)} G$. Then the ladder index of $\phi(\bar x,\bar y)$ in $G$ is at most $f(q,d,t)$.
\end{theorem}

Note that in particular, \cref{thm:new-stable} implies that every nowhere dense graph is stable, which was the main conclusion of Adler and Adler~\cite{adler2014interpreting}.

We remark that the above connections between nowhere denseness and notions from model theory have recently found algorithmic applications.
Both uniform quasi-wideness and stability techniques are key tools used in the study of the complexity of the {\sc Distance-$r$ Dominating Set} problem on nowhere dense graph classes,
and in particular in the design of polynomial kernelization procedures for this problem~\cite{DawarK09,drange2016kernelization,eickmeyer2016neighborhood,siebertz2016polynomial}.

\paragraph{VC-dimension and VC-density.} The notion of VC-dimension was introduced by 
Vapnik and Chervonenkis~\cite{chervonenkis1971theory} as a measure of complexity of set systems, or equivalently of hypergraphs.
Over the years it
has found important applications in many areas of
statistics, discrete and computational geometry, 
and learning theory. The concept found its way to
model theory when Laskowski~\cite{laskowski1992vapnik} 
observed that a complete first-order theory does not have
the independence property (as introduced by 
Shelah~\cite{shelah1971stability}) if and only if, 
in each model, each definable family of sets has finite 
VC-dimension.

Formally, VC-dimension is defined as follows. 
Let $A$ be a set and let  $\FFF\subseteq 2^A$ 
be a family of subsets of $A$.
A subset $X\subseteq A$ is \emph{shattered by $\FFF$} if
$\{X\cap F\colon F\in \FFF\}=2^X$; that is, every subset of $X$ can be obtained as the intersection of some set from $\FFF$ with $X$. 
The \emph{Vapnik-Chervonenkis-dimension}, short 
\emph{VC-dimension},
of $\FFF$ is the maximum size of a set $X$ that is shattered by
$\FFF$. One of the main uses of VC-dimension is the
Sauer-Shelah-Lemma~\cite{chervonenkis1971theory,sauer1972density, shelah1972combinatorial}, which states that the
cardinality of a set family $\FFF$ on a ground set $A$ 
with VC-dimension $d$ satisfies 
$\abs{\FFF}\leq \sum_{i=0}^{d}\binom{|A|}{i}\in \Oof(|A|^d)$.
This motivates the definition of \emph{VC-density}, which
describes the asymptotic growth of finite set families
(as the size of the ground set $A$ goes to infinity). 
%
%Also the related notion of VC-density 
%turned out to be a very important measure for 
%the combinatorial complexity of a family of sets. 
%In the following years several authors 
%aimed to obtain explicit bounds on the VC-dimension 
%and VC-density of 
%definable families of sets in restricted 
%structures, see e.g.~\cite{AschenbrennerDHMS13, aschenbrenner2016vapnik, bobkov2017computations, johnson2010compression, karpinski1997approximating, 
%karpinski1997polynomial}. 

We refer to~\cite{aschenbrenner2016vapnik} for an overview of 
applications of VC-dimension and VC-density in model
theory and to the surveys~\cite{furedi1991traces,matouvsek1998geometric} 
on uses of VC-density in
combinatorics. 

Our attention was drawn to VC-dimension and 
VC-density when we studied the parameterized 
complexity of the {\sc Distance-$r$ Dominating Set} 
problem on nowhere dense graph classes~\cite{eickmeyer2016neighborhood}. 
One of the main contributions of \cite{eickmeyer2016neighborhood}
was to prove that nowhere dense classes of graphs have almost linear
\emph{distance neighborhood complexity}. More precisely, it
was proved that for every nowhere dense class~$\CCC$ of 
graphs, every positive integer $r$ and every $\epsilon>0$ there
exists a constant $c$ such that for every graph $G\in\CCC$ and
every vertex subset $A\subseteq V(G)$, the set of distance-$r$ neighborhood intersections with $A$,
$\{N_r[v]\cap A \colon v\in V(G)\}$,
has size bounded by $c\cdot |A|^{1+\epsilon}$. This result
generalizes a result of Reidl et al.~\cite{reidl2016characterising}, who proved that classes of bounded expansion have linear distance neighborhood complexity. 

Our goal is to generalizes the above result to the larger context of VC-density of first-order definable set families.
Note that the property of being at distance at most $r$ in a graph is 
a first-order definable property, for every constant $r$.
VC-density in model theory studies the asymptotic growth 
of arbitrary definable families. More precisely, 
let $\psi(\tup x,\tup y)$ be a first-order formula, where 
$\tup x$ is an $m$-tuple and $\tup y$ is an $n$-tuple of variables. 
Let $\strA$ be a structure and let $A$ be a set of elements of
$\strA$. Then the set of \emph{$\psi$-types} or 
\emph{$\psi$-traces} over $A$ in 
$\strA$ is the set
\[S_\psi(\strA,A)=\{\{\tup a\ \in A^m : \strA\models\psi(\tup a,\tup b)\} : \tup b\in V(\strA)^n\}.\]
The set $S_\psi(\strA,A)$ is also known as the {\em{Stone space}} induced by $\psi$ over $A\subseteq \strA$.

It is known that in (infinite) model theory,
measuring cardinality of Stone spaces can be used to characterize when a formula $\psi$ is stable on a class of models (i.e., does not define arbitrarily long ladders).
In a nutshell, this is the case if and only if for every model $\strA$ of interest and every infinite set of its elements $A$, say of cardinality at most $\kappa$ for a cardinal $\kappa$,
the cardinality of $S_\psi(\strA,A)$ is at most $\kappa$. We refer to~\cite{}\todo{Citation needed} for more information.

We prove the following theorem concerning the number of
types in sparse graph classes.

\begin{theorem}\label{thm:vc-density}
Let $\CCC$ be a class of graphs and let $\psi(\tup x,\tup y)$ be a first-order formula, where 
$\tup x$ is an $m$-tuple and $\tup y$ is an $n$-tuple of variables. 
\begin{enumerate}
\item If $\CCC$ is nowhere dense, then for every $\epsilon>0$ 
there exists a constant~$c$ such that for every $G\in \CCC$ and every nonempty
$A\subseteq V(G)$, we have $|S_\psi(A,G)|\leq c\cdot |A|^{n+\epsilon}.$

\item If $\CCC$ has bounded expansion, then there exists a constant~$c$ such that for every $G\in \CCC$ and every nonempty $A\subseteq V(G)$, we have $|S_\psi(A,G)|\leq c\cdot |A|^n$.
\end{enumerate}
\end{theorem}

Intuitively, \cref{thm:vc-density} states that the number
of $\psi$-types over $A$ in a nowhere dense class 
is bounded almost linearly in $|A|^n$, the total number number of $n$-tuples of vertices from $A$. 
\cref{thm:vc-density} also generalizes the recent result of Bobkov~\cite{bobkov2017computations} that nowhere dense classes
are {\em{dp-minimal}}, a notion introduced by Shelah~\cite{shelah2014strongly} that we will not use here.

%Formulated in terms of VC-density, for every nowhere 
%dense class of graphs and every first-order formula $\psi(\tup x,\tup y)$
%we have 
%\[\limsup_{a\rightarrow \infty}\max_{\substack{G\in\CCC\\A\subseteq
%    V(G), |A|=a}}\frac{\log |S_\psi(A,G)|}{\log a}\leq n.\]


%
%
%\sebi{The proof of \Cref{thm:vc-density} is (again) based on Gaifman's locality
%theorem for first-order logic, which allows us to focus at local 
%neighborhoods around the vertices $v\in V(G)$. 
%Using the closure lemma of~\cite{DrangeDFKLPPRVS16} for
%classes of bounded expansion and its generalization to
%nowhere dense classes~\cite{eickmeyer2016neighborhood}
%together with uniform quasi-wideness,
%\begin{change}{sz}we are able to bound 
%the number of interactions of vertices $v$ with the set $A$. \end{change}}
%
%
%
%
%\begin{change}{sz}
%\Cref{thm:vc-density} shows a strong new connection 
%	between first-order logic and the notions of bounded expansion
%	and nowhere denseness.
It is not difficult to see that \cref{thm:vc-density} also yields new
characterizations of nowhere dense classes and classes of bounded expansion in terms of VC-density, as explained in the following complelemtary result.

\begin{theorem}\label{thm:vc-density-lower-bound}
Let $\CCC$ be a class of graphs which 
is closed under taking subgraphs. 
\begin{enumerate}[(1)]
\item If $\CCC$ is somewhere dense, then there is a formula 
$\psi(x,y)$ such that for every $n\in \N$ there are $G\in\CCC$ and $A\subseteq V(G)$ 
with $|A|=n$ and $S_\psi(A,G)=2^{A}$. 
\item If $\CCC$ has unbounded expansion, then there is a formula 
$\psi(x,y)$ such that for every $c\in \mathbb{R}$ there exist $G\in\CCC$ and $A\subseteq V(G)$ with $|S_\psi(A,G)|>c|A|$. 
\end{enumerate}
\end{theorem}

\paragraph{Organization.} In \cref{sec:uqw} we recall standard concepts from the theory of sparse graphs and prove \cref{thm:new-uqw}.
\cref{sec:gaifman} we discuss Gaifman locality for first-order logic and derive some tools that will be used later.
Then, in \cref{sec:stable} we prove \cref{thm:new-stable}, whereas in \cref{sec:types} we prove \cref{thm:vc-density} and \cref{thm:vc-density-lower-bound}.

%\end{change}

%\pagebreak


% Finally, we observe that an argument of Bousquet and
% Thomasse\'e~\cite{BousquetT15} can be slightly modified to prove that
% the VC-dimension of the $r$th power $G^r$ of a graph $G$
% with $K_t\not\minor_r G$ is bounded by $t-1$.
% Boundedness of VC-dimension is a weaker condition than the stability of a class, however it is sufficient in many contexts.
% Therefore, we consider investigating explicit bounds on the VC-dimension potentially useful for future research.
%
% \begin{theorem}\label{thm:new-vc}
% Let $G$ be a graph such that $K_t\not\minor_r G$. Then the
% VC-dimension of $G^r$, the $r$th power of the graph~$G$, is bounded by $t-1$.
% \end{theorem}

%\paragraph{Organization.} We give background from graph theory in \cref{sec:uqw}, where we also
%prove \cref{thm:new-uqw}. 
%We provide background 
%on logic and prove \cref{thm:new-stable} in \cref{sec:stable}. 



