\section{VC dimension}\label{sec:vc}

We now come to the proof of \cref{thm:new-vc}. A set $X$ of vertices 
in a graph is \emph{shattered} if for every
subset $Y\subseteq X$ there exists 
a vertex $v$ such that $N[v]\cap X=Y$. The \emph{Vapnik-Chervonenkis dimension}, short \emph{VC-dimension}~\cite{chervonenkis1971theory} of a graph is the maximum size of a shattered set. 
The VC-dimension as a measure of complexity of set systems found has many applications, e.g.\ in learnability theory~\cite{haussler1987}, computational geometry~\cite{chazelle1989quasi},
and graph theory~\cite{alon2006dominating,BousquetT15,chepoi2007covering,eickmeyer2016neighborhood}.
We define notions of a {\em{$2$-shattered}} set and the {\em{2VC-dimension}} of a graph by restricting subsets $Y\subseteq X$ considered in the definition only to subsets of size exactly $2$.

The \emph{$r$th power of a graph $G$} is the graph $G^r$
with vertex set $V(G)$, where there is an edge between two 
vertices $u$ and $v$ if and only if their distance in $G$ is at most $r$. 

We observe that an argument of Bousquet and 
Thomass\'e~\cite{BousquetT15} can be slightly modified to prove that 
the $2$VC-dimension of the $r$-power graph $G^r$ of a graph $G$
with $K_t\not\minor_r G$ is small. Obviously, the $2$VC-dimension of $G$
bounds its VC-dimension. Hence, we in fact prove the following strengthening of \cref{thm:new-vc} stated in the introduction. 

\begin{theorem}
Let $r\in \N$ and let $G$ be a graph. 
If $K_t\not\minor_r G$, then the $2$VC-dimension of $G^r$
is at most $t-1$. 
\end{theorem}
\begin{proof}
Assume there is a set $A=\{a_1,\ldots, a_t\}$ of size $t$ such that
for all subsets $\{i,j\}\subseteq \{1,\ldots,t\}$ of size $2$ 
there is an vertex $v_{ij}$ with 
$N_r[v_{ij}]\cap A=\{a_i,a_j\}$.
For each subset $\{i,j\}\subseteq \{1,\ldots,t\}$ of size $2$, choose a vertex $u_{ij}$ so that:
\begin{enumerate}[(1)]
\item\label{p:i} $\dist(v_{ij},u_{ij})+\dist(u_{ij},a_i)\leq r$;
\item\label{p:j} $\dist(v_{ij},u_{ij})+\dist(u_{ij},a_j)\leq r$; and
\item\label{p:min} subject to conditions \eqref{p:i} and \eqref{p:j}, $\max(\dist(u_{ij},a_i),\dist(u_{ij},a_j))$ is minimized.
\end{enumerate}
Observe that $u_{ij}$ is well-defined since setting $u_{ij}=v_{ij}$ satisfies the first two conditions.

Let $P^i_{ij}$ and $P^j_{ij}$ be arbitrarily chosen shortest paths between $u_{ij}$ and~$a_i$, and between $u_{ij}$ and~$a_j$, respectively.
We now establish some basic properties of paths $P^i_{ij}$ and $P^j_{ij}$ following from the choice of $u_{ij}$.

\begin{claim}\label{cl:ineq}
For each vertex $x$ on $P^i_{ij}$ we have $\dist(v_{ij},x)+\dist(x,a_i)\leq r$, and
for each vertex $y$ on $P^j_{ij}$ we have $\dist(v_{ij},y)+\dist(y,a_j)\leq r$.
\end{claim}
\begin{clproof}
We prove only the first statement for the second is symmetric.
We have
$$\dist(v_{ij},x)+\dist(x,a_{i})\leq \dist(v_{ij},u_{ij})+\dist(u_{ij},x)+\dist(x,a_{i})=\dist(v_{ij},u_{ij})+\dist(u_{ij},a_{i})\leq r,$$
where the last equality is due to $x$ lying on a shortest path between $u_{ij}$ and $a_i$, and the last inequality is by condition~\eqref{p:i}.
\end{clproof}

\begin{claim}\label{cl:closer}
Suppose $x$ is a vertex on $P^i_{ij}$ that is different from $u_{ij}$. Then $\dist(x,a_i)<\dist(x,a_j)$.
Symmetrically, if $y$ lies on $P^j_{ij}$ and is different from $u_{ij}$, then $\dist(y,a_i)>\dist(y,a_j)$.
Consequently, paths $P^i_{ij}$ and $P^j_{ij}$ share only one vertex, being the endpoint $u_{ij}$.
\end{claim}
\begin{clproof}
We prove only the first claim, for the second is symmetric and the third directly follows from the first two.
Suppose for contradiction that $\dist(x,a_i)\geq \dist(x,a_j)$.
By \cref{cl:ineq} we have 
$$\dist(v_{ij},x)+\dist(x,a_i)\leq r.$$
On the other hand, since $\dist(x,a_i)\geq \dist(x,a_j)$, we have
$$\dist(v_{ij},x)+\dist(x,a_j)\leq\dist(v_{ij},x)+\dist(x,a_i)\leq r.$$
We conclude that $x$ satisfies conditions \eqref{p:i} and \eqref{p:j} from the definition of $u_{ij}$.
However, since $x\neq u_{ij}$ and $x$ lies on a shortest path between $u_{ij}$ and $a_i$, we have $\dist(x,a_i)<\dist(u_{ij},a_i)$.
Therefore,
$$\dist(x,a_j)\leq \dist(x,a_i)<\dist(u_{ij},a_i)\leq \max(\dist(u_{ij},a_i),\dist(u_{ij},a_j)).$$
Thus, the existence of $x$ contradicts condition \eqref{p:min} from the definition of $u_{ij}$.
\end{clproof}

Now, define paths $Q^i_{ij}$ and $Q^j_{ij}$ as follows:
\begin{itemize}
\item if $\dist(u_{ij},a_i)<\dist(u_{ij},a_j)$, then $Q^{i}_{ij}=P^{i}_{ij}$ and $Q^{j}_{ij}=P^{j}_{ij} - \{u_{ij}\}$;
\item if $\dist(u_{ij},a_i)>\dist(u_{ij},a_j)$, then $Q^{i}_{ij}=P^{i}_{ij} - \{u_{ij}\}$ and $Q^{j}_{ij}=P^{j}_{ij}$;
\item if $\dist(u_{ij},a_i)=\dist(u_{ij},a_j)$, then define $Q^i_{ij}$ and $Q^j_{ij}$ using any of the above.
\end{itemize}
Thus, by \cref{cl:closer} we have that paths $Q^{i}_{ij}$ and $Q^{j}_{ij}$ are disjoint. Moreover, for each vertex $x$ on $Q^{i}_{ij}$ we have $\dist(x,a_i)\leq \dist(x,a_j)$, and for each
vertex $y$ on $Q^{j}_{ij}$ we have $\dist(y,a_i)\geq \dist(y,a_j)$.

\begin{claim}\label{cl:intersect}
Let $\{i,j\}$ and $\{i',j'\}$ be two different subsets of size $2$ of $\{1,\ldots,t\}$.
Suppose that paths $Q^i_{ij}$ and $Q^{i'}_{i'j'}$ intersect.
Then $i=i'$.
\end{claim}
\begin{clproof}
Let $x$ be a vertex lying both on $Q^i_{ij}$ and $Q^{i'}_{i'j'}$. We first consider the corner case when $x=u_{ij}$.
Suppose first that $\dist(v_{ij},x)\geq \dist(v_{i'j'},x)$. Then by \cref{cl:ineq} we have
$$\dist(v_{i'j'},a_i)\leq \dist(v_{i'j'},x)+\dist(x,a_i)\leq \dist(v_{ij},x)+\dist(x,a_i)\leq r,$$
and analogously $\dist(v_{i'j'},a_{j})\leq r$. However, we assumed that $a_{i'}$ and $a_{j'}$ are the only vertices of~$A$ that are at distance at most $r$ from $v_{i'j'}$, hence $\{i,j\}=\{i',j'\}$,
a contradiction. Suppose then that $\dist(v_{ij},x)<\dist(v_{i'j'},x)$. 
Then we have
$$\dist(v_{ij},a_{i'})\leq \dist(v_{ij},x)+\dist(x,a_{i'})<\dist(v_{i'j'},x)+\dist(x,a_{i'})\leq r,$$
where the last equality follows from \cref{cl:ineq}.
Since $a_i$ and $a_j$ are the only vertices of $A$ that are at distance at most $r$ from $v_{ij}$, we infer that $i'\in \{i,j\}$. 
If $i'=i$ then we would be done, so suppose $i'=j$.
Since $x=u_{ij}$ and $x$ lies on $Q^i_{ij}$, by the definition of $Q^i_{ij}$ we have that $\dist(x,a_i)\leq \dist(x,a_j)=\dist(x,a_{i'})$. Therefore,
$$\dist(v_{i'j'},a_i)\leq \dist(v_{i'j'},x)+\dist(x,a_{i})\leq \dist(v_{i'j'},x)+\dist(x,a_{i'})\leq r.$$
where the last inequality follows from \cref{cl:ineq}.
Again, we assumed that $a_{i'}$ and $a_{j'}$ are the only vertices of $A$ that are at distance at most $r$ from $v_{i'j'}$, so $i\in \{i',j'\}$. If $i=i'$ then we are done, and otherwise we have $i=j'$.
Together with $i'=j$ this implies $\{i,j\}=\{i',j'\}$, a contradiction.

The second corner case when $x=u_{i'j'}$ leads to a contradiction in a symmetric manner.

We now move to the main case when $x\neq u_{ij}$ and $x\neq u_{i'j'}$.
Then by \cref{cl:closer} we have $\dist(x,a_i)<\dist(x,a_j)$ and $\dist(x,a_{i'})<\dist(x,a_{j'})$.
By symmetry, without loss of generality assume that $\dist(x,a_i)\leq \dist(x,a_{i'})$.
Observe now that
$$\dist(v_{i'j'},a_i)\leq \dist(v_{i'j'},x)+\dist(x,a_{i})\leq \dist(v_{i'j'},x)+\dist(x,a_{i'})\leq r,$$
where the last inequality follows from \cref{cl:ineq}.
Since we assumed that $a_{i'}$ and $a_{j'}$ are the only vertices of $A$ that are at distance at most $r$ from $v_{i'j'}$, we have $i\in \{i',j'\}$.
However, it cannot happen that $i=j'$, because $\dist(x,a_{i'})<\dist(x,a_{j'})$ and $\dist(x,a_{i'})\geq \dist(x,a_{i})$. We conclude that $i=i'$.
\end{clproof}

For each $i\in \{1,2,\ldots,t\}$ we define $X_i$ to be the union of vertex sets of paths $Q^i_{ij}$ for $j\neq i$.
Each of these paths has length at most $r$ and has $a_i$ as an endpoint, hence the subgraph induced by $X_i$ is connected and has radius at most $r$.
By \cref{cl:intersect}, sets $X_i$ are pairwise disjoint. Finally, observe that for each $\{i,j\}\subseteq \{1,\ldots,t\}$ with $i\neq j$, there is an edge between a vertex of $Q^{i}_{ij}$ and a vertex of $Q^{j}_{ij}$.
We conclude that $(X_i)_{i=1,\ldots,t}$ is a depth-$r$ minor model of $K_t$ in $G$, a contradiction.
\end{proof}

\begin{comment}
\begin{proof}
Assume there is a set $A=\{a_1,\ldots, a_t\}$ of size $t$ such that
for all subsets $\{a_i,a_j\}\subseteq A$ of size $2$ 
there is an element $v_{ij}\in V(G)\setminus A$ with 
$N_r[v_{ij}]\cap A=\{a_i,a_j\}$. Fix such $v_{ij}$ with the property
that $\max\left(\dist_G(v_{ij},a_i), \dist_G(v_{ij},a_j)\right)$ is 
minimized. 

A \emph{central walk} $W_{ij}$ is the concatenation of a minimum length
path $P_{ij}^i$ from $a_i$ to $v_{ij}$ and a minimum length path $P_{ij}^j$ from $v_{ij}$ to $a_j$. 
Note that a central walk is possibly not a path. For each pair $a_i,a_j$ fix
a central walk $W_{ij}$ and the corresponding paths $P_{ij}^i$ and $P_{ij}^j$. 

Now assume that a vertex $x$ is traversed by two different central 
walks $W_{ij}$, $W_{i'j'}$. By swapping indices if necessary, assume that $x$ lies on $P_{ij}^i$ and $P_{i'j'}^{i'}$. 

First, observe that if $\dist(x,a_i)=\dist(x,a_{i'})$, 
then $a_i=a_{i'}$. Indeed, if $\dist(x,a_i)=\dist(x,a_{i'})$ then $\dist(v_{ij},a_{i})=\dist(v_{ij},a_{i'})$, so $i'\in \{i,j\}$ because $a_i,a_j$ are the only vertices of $A$ at distance at most $r$ from $v_{ij}$.
Analogously $i\in \{i',j'\}$, so either $i=i'$, or $i'=j$ and $i=j'$. However, the latter case would imply $\{i,j\}=\{i',j'\}$, which contradicts the assumption that $W_{ij}$ and $W_{i'j'}$ are distinct. 

A similar argument yields that
$\dist(x,a_i)<\dist(x,a_j)$ 
and $\dist(x,a_{i'})<\dist(x,a_{j'})$. 
Now assume that $\dist(x,a_i)<\dist(x,a_{i'})$. By the same argument as 
above we have $a_{j'}=a_i$, hence $W_{i'j'}=W_{ij'}$. Here, we have
$\dist(x,a_i)<\dist(x,a_j)$ and $\dist(x,a_{i})<\dist(x,a_{i'})$, 
otherwise the walks are not distinct. 

Let us now construct connected subsets $X_i$ for all $1\leq i\leq t$. 
For every walk $W_{ij}$ the vertices of $W_{ij}$ closer to $a_i$ than to $a_j$ 
are added to $X_i$, the vertices of $W_{ij}$ closer to $a_j$ than to $a_i$ 
are added to $X_j$, ties are broken arbitrary.
Then the sets $X_i$ are pairwise disjoint by what we proved above. If a vertex $x$
appears in two distinct central walks, these are $W_{ij}$ and $W_{i\ell}$ for some
$i,j,\ell$ with $\dist(x,a_i)<\dist(x,a_j)$ and $\dist(x,a_i)<\dist(x,a_\ell)$. 
In both cases $x$ belongs to $X_i$. By construction, the sets $X_i$ are connected, 
have radius at most~$r$, and 
there is always an edge between a vertex of $X_i$ and a vertex of $X_j$ since $X_i\cup X_j$ 
contains the walk $W_{ij}$. Therefore, if the $2$VC-dimension is at least $t$, the 
graph contains $K_t$ as a depth-$r$ minor. 
\end{proof}
\end{comment}