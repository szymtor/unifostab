%% For double-blind review submission, w/o CCS and ACM Reference (max submission space)
\documentclass[acmsmall]{acmart}
\settopmatter{printfolios=true,printccs=false,printacmref=false}
%% For double-blind review submission, w/ CCS and ACM Reference
%\documentclass[sigplan,review,anonymous]{acmart}\settopmatter{printfolios=true}
%% For single-blind review submission, w/o CCS and ACM Reference (max submission space)
%\documentclass[sigplan,review]{acmart}\settopmatter{printfolios=true,printccs=false,printacmref=false}
%% For single-blind review submission, w/ CCS and ACM Reference
%\documentclass[sigplan,review]{acmart}\settopmatter{printfolios=true}
%% For final camera-ready submission, w/ required CCS and ACM Reference
%\documentclass[sigplan]{acmart}\settopmatter{}


%% Conference information
%% Supplied to authors by publisher for camera-ready submission;
%% use defaults for review submission.
\acmConference[LICS 2018]{Thirty-Third Annual ACM/IEEE Symposium on
Logic in Computer Science (LICS)}{July 9--12, 2018}{Oxford, Great Britain}
\acmYear{2018}
\acmISBN{} % \acmISBN{978-x-xxxx-xxxx-x/YY/MM}
\acmDOI{} % \acmDOI{10.1145/nnnnnnn.nnnnnnn}
\startPage{1}

%% Copyright information
%% Supplied to authors (based on authors' rights management selection;
%% see authors.acm.org) by publisher for camera-ready submission;
%% use 'none' for review submission.
\setcopyright{none}
%\setcopyright{acmcopyright}
%\setcopyright{acmlicensed}
%\setcopyright{rightsretained}
%\copyrightyear{2017}           %% If different from \acmYear

%% Bibliography style
\bibliographystyle{ACM-Reference-Format}
%% Citation style
%\citestyle{acmauthoryear}  %% For author/year citations
%\citestyle{acmnumeric}     %% For numeric citations
%\setcitestyle{nosort}      %% With 'acmnumeric', to disable automatic
                            %% sorting of references within a single citation;
                            %% e.g., \cite{Smith99,Carpenter05,Baker12}
                            %% rendered as [14,5,2] rather than [2,5,14].
%\setcitesyle{nocompress}   %% With 'acmnumeric', to disable automatic
                            %% compression of sequential references within a
                            %% single citation;
                            %% e.g., \cite{Baker12,Baker14,Baker16}
                            %% rendered as [2,3,4] rather than [2-4].


%%%%%%%%%%%%%%%%%%%%%%%%%%%%%%%%%%%%%%%%%%%%%%%%%%%%%%%%%%%%%%%%%%%%%%
%% Note: Authors migrating a paper from traditional SIGPLAN
%% proceedings format to PACMPL format must update the
%% '\documentclass' and topmatter commands above; see
%% 'acmart-pacmpl-template.tex'.
%%%%%%%%%%%%%%%%%%%%%%%%%%%%%%%%%%%%%%%%%%%%%%%%%%%%%%%%%%%%%%%%%%%%%%


%% Some recommended packages.
\usepackage{booktabs}   %% For formal tables:
                        %% http://ctan.org/pkg/booktabs
\usepackage{subcaption} %% For complex figures with subfigures/subcaptions
                        %% http://ctan.org/pkg/subcaption
                        
\usepackage[absolute]{textpos}
\newcounter{aux}
\newcounter{auxsec}
                        
\newcommand{\set}[1]{\{#1\}}
\newcommand{\setof}[2]{\left\{#1 \,\mid\, #2 \right\}}
\renewcommand{\subset}{\subseteq}

\newcommand{\wcol}{\mathrm{wcol}}
\newcommand{\col}{\mathrm{col}}
\newcommand{\adm}{\mathrm{adm}}
\newcommand{\tw}{\mathrm{tw}}
\newcommand{\WReach}{\mathrm{WReach}}
\newcommand{\SReach}{\mathrm{SReach}}
\newcommand{\wcolorder}{\sqsubseteq}
\newcommand{\Oof}{\mathcal{O}}
\newcommand{\CCC}{\mathscr{C}}
\newcommand{\cal}[1]{\mathcal{#1}}
\newcommand{\NNN}{\mathcal{N}}
\newcommand{\WWW}{\mathcal{W}}
\newcommand{\DDD}{\mathcal{D}}
\newcommand{\PPP}{\mathcal{P}}
\newcommand{\FFF}{\mathcal{F}}
\newcommand{\GGG}{\mathcal{G}}
\newcommand{\YYY}{\mathcal{Y}}
\newcommand{\nei}{\mathrm{nei}}
\renewcommand{\ker}{\mathrm{ker}}
\newcommand{\core}{\mathrm{core}}

\newcommand{\cutrk}{\mathrm{cutrk}}
\newcommand{\rank}{\mathrm{rank}}
\newcommand{\rw}{\mathrm{rw}}


\newcommand{\grad}{\nabla}
\newcommand{\ds}{\mathbf{ds}}
\newcommand{\cl}{\mathrm{cl}}
\newcommand{\cst}{\alpha}

\newcommand{\fnei}{f_{\nei}}
\newcommand{\fwcol}{f_{\wcol}}
\newcommand{\fker}{f_{\ker}}
\newcommand{\fproj}{f_{\mathrm{proj}}}
\newcommand{\fcl}{f_{\cl}}
\newcommand{\fgrad}{f_{\grad}}
\newcommand{\fpaths}{f_{\mathrm{pth}}}
\newcommand{\fapx}{f_{\mathrm{apx}}}
\newcommand{\fcore}{f_{\mathrm{core}}}
\newcommand{\ffin}{f_{\mathrm{fin}}}

\newcommand\blfootnote[1]{%
  \begingroup
  \renewcommand\thefootnote{}\footnote{#1}%
  \addtocounter{footnote}{-1}%
  \endgroup
}

\newcommand{\suchthat}{ \colon }
\newcommand{\sth}{ \colon }
\newcommand{\ie}{i.e.\@ }
\newcommand{\Pow}{\mathcal{P}}
\newcommand{\N}{\mathbb{N}}
\newcommand{\R}{\mathbb{R}}
\newcommand{\tup}[1]{\bar{#1}}
\renewcommand{\phi}{\varphi}
\renewcommand{\epsilon}{\varepsilon}
\newcommand{\str}{\mathbb}
\newcommand{\strA}{\str{A}}
\newcommand{\strB}{\str{B}}
\newcommand{\FO}{\mathrm{FO}}
\newcommand{\minor}{\preccurlyeq}
\newcommand{\dist}{\mathrm{dist}}
\newcommand{\indx}{\mathrm{index}}
\renewcommand{\mid}{~:~}

\newcommand{\profnum}{\widehat{\nu}}
\newcommand{\projnum}{\mu}
\newcommand{\projprof}{\widehat{\mu}}

\newcommand{\abs}[1]{\ensuremath{\left\lvert#1\right\rvert}}

\newcommand{\im}{\mathrm{im}}
\newcommand{\rg}{\mathrm{rg}}
\newcommand{\from}{\colon}

\renewcommand{\leq}{\leqslant}
\renewcommand{\geq}{\geqslant}
\renewcommand{\le}{\leqslant}
\renewcommand{\ge}{\geqslant}

\theoremstyle{remark}
\newtheorem{claim}{Claim}
\newtheorem*{clproof}{Proof}

\usepackage{mathrsfs}

\begin{document}

%% Title information
\title{On the number of types in sparse graphs}         %% [Short Title] is optional;
                                        %% when present, will be used in
                                        %% header instead of Full Title.
\titlenote{A preliminary version of this work appeared in the proceeding of LICS 2018~\cite{PilipczukST18a}. This full version contains complete proofs of all the claimed results. This work was done while S. Siebertz was affiliated with the University of Warsaw. The work of M.\ Pilipczuk and S.\ Siebertz was supported by the National Science Centre of 
Poland via POLONEZ grant agreement UMO-2015/19/P/ST6/03998, 
which has received funding from the European Union's Horizon 2020 research and 
innovation programme (Marie Sk\l odowska-Curie grant agreement No.\ 665778). The work of Sz.~Toru{\'n}czyk was supported by the National Science Centre of Poland grant 2016/21/D/ST6/01485.
M. Pilipczuk was supported by the Foundation for Polish Science (FNP) via the START stipend programme. 
\\
\mbox{}
\begin{textblock}{5}(12.8, 12.65)
\includegraphics[width=45px]{flag_bw}%
\end{textblock}
%
}
%% \titlenote is optional;
                                        %% can be repeated if necessary;
                                        %% contents suppressed with 'anonymous'
%\subtitle{Subtitle}                     %% \subtitle is optional
%\subtitlenote{with subtitle note}       %% \subtitlenote is optional;
                                        %% can be repeated if necessary;
                                        %% contents suppressed with 'anonymous'


%% Author information
%% Contents and number of authors suppressed with 'anonymous'.
%% Each author should be introduced by \author, followed by
%% \authornote (optional), \orcid (optional), \affiliation, and
%% \email.
%% An author may have multiple affiliations and/or emails; repeat the
%% appropriate command.
%% Many elements are not rendered, but should be provided for metadata
%% extraction tools.


%% Author with single affiliation.
\author{Micha\l~Pilipczuk}
\affiliation{
%  \position{Position1}
%  \department{Department1}              %% \department is recommended
  \institution{Institute of Informatics, University of Warsaw, Poland}            %% \institution is required
%  \streetaddress{Street1 Address1}
%  \city{City1}
%  \state{State1}
%  \postcode{Post-Code1}
%  \country{Country1}                    %% \country is recommended
}
\author{Sebastian Siebertz}
\affiliation{
%  \position{Position1}
%  \department{Department1}              %% \department is recommended
  \institution{Institut f\"ur Informatik, Humboldt-Universit\"at zu Berlin, Germany}            %% \institution is required
%  \streetaddress{Street1 Address1}
%  \city{City1}
%  \state{State1}
%  \postcode{Post-Code1}
%  \country{Country1}                    %% \country is recommended
}
\email{siebertz@informatik.hu-berlin.de}
\author{Szymon Toru{\'n}czyk}
%\authornote{with author1 note}          %% \authornote is optional;
                                        %% can be repeated if necessary
%\orcid{nnnn-nnnn-nnnn-nnnn}             %% \orcid is optional
\affiliation{
%  \position{Position1}
%  \department{Department1}              %% \department is recommended
  \institution{Institute of Informatics, University of Warsaw, Poland}            %% \institution is required
%  \streetaddress{Street1 Address1}
%  \city{City1}
%  \state{State1}
%  \postcode{Post-Code1}
%  \country{Country1}                    %% \country is recommended
}
\email{{michal.pilipczuk,szymtor}@mimuw.edu.pl}  %% \email is recommended

%% Author with two affiliations and emails.
%\author{First2 Last2}
%\authornote{with author2 note}          %% \authornote is optional;
%                                        %% can be repeated if necessary
%\orcid{nnnn-nnnn-nnnn-nnnn}             %% \orcid is optional
%\affiliation{
%  \position{Position2a}
%  \department{Department2a}             %% \department is recommended
%  \institution{Institution2a}           %% \institution is required
%  \streetaddress{Street2a Address2a}
%  \city{City2a}
%  \state{State2a}
%  \postcode{Post-Code2a}
%  \country{Country2a}                   %% \country is recommended
%}
%\email{first2.last2@inst2a.com}         %% \email is recommended
%\affiliation{
%  \position{Position2b}
%  \department{Department2b}             %% \department is recommended
%  \institution{Institution2b}           %% \institution is required
%  \streetaddress{Street3b Address2b}
%  \city{City2b}
%  \state{State2b}
%  \postcode{Post-Code2b}
%  \country{Country2b}                   %% \country is recommended
%}
%\email{first2.last2@inst2b.org}         %% \email is recommended


%% Abstract
%% Note: \begin{abstract}...\end{abstract} environment must come
%% before \maketitle command
\begin{abstract}
We prove that for every nowhere dense class of graphs $\CCC$, as
defined by Ne\v set\v ril and Ossona de
Mendez~\cite{nevsetvril2010first,nevsetvril2011nowhere}, and for every
first order formula $\phi(\tup x,\tup y)$, whenever one draws a graph
$G\in \CCC$ and a subset of its vertices $A$, the number of subsets of
$A^{|\tup y|}$ that are of the form $\set{\tup v\in A^{|\tup y|}\,
\colon\, G\models\phi(\bar u,\tup v)}$ for some valuation~$\tup u$ of
$\tup x$ in $G$ is bounded by $\Oof(|A|^{|\tup x|+\epsilon})$, for
every $\epsilon>0$. This provides optimal bounds on the VC-density of
first-order definable set systems in nowhere dense graph classes.
%
We also give two new proofs of upper bounds on quantities in nowhere
dense classes that are relevant for their logical treatment. First, we
provide a new proof of the fact that nowhere dense classes are
uniformly quasi-wide, implying explicit, polynomial upper bounds on
the functions relating the two notions. Second, we give a new
combinatorial proof of the result of Adler and
Adler~\cite{adler2014interpreting} stating that every nowhere dense
class of graphs is stable. In contrast to the previous proofs of the
above results, our proofs are completely finitistic and constructive,
and yield explicit and computable upper bounds on quantities related
to uniform quasi-wideness (margins) and stability (ladder indices).
\end{abstract}



%% 2012 ACM Computing Classification System (CSS) concepts
%% Generate at 'http://dl.acm.org/ccs/ccs.cfm'.
\begin{CCSXML}
<ccs2012>
<concept>
<concept_id>10011007.10011006.10011008</concept_id>
<concept_desc>Software and its engineering~General programming languages</concept_desc>
<concept_significance>500</concept_significance>
</concept>
<concept>
<concept_id>10003456.10003457.10003521.10003525</concept_id>
<concept_desc>Social and professional topics~History of programming languages</concept_desc>
<concept_significance>300</concept_significance>
</concept>
</ccs2012>
\end{CCSXML}

\ccsdesc[500]{Software and its engineering~General programming languages}
\ccsdesc[300]{Social and professional topics~History of programming languages}
%% End of generated code


%% Keywords
%% comma separated list
\keywords{Nowhere dense graphs, Stone space, first-order types, VC-density, stability, uniform quasi-wideness}  %% \keywords are mandatory in final camera-ready submission


%% \maketitle
%% Note: \maketitle command must come after title commands, author
%% commands, abstract environment, Computing Classification System
%% environment and commands, and keywords command.
\maketitle



\input{full-intro-short-new}
\input{full-prelims}
\input{full-nd-uqw}
\input{full-gaifman}
\input{full-vc-density}
\section{Packing and traversal numbers for nowhere dense graphs}\label{sec:ep}
In this section, we give an application 
ofTheorem~\ref{thm:vc-density}, proving a 
duality result for nowhere dense graph classes.

A \emph{set system} is a family  $\cal F$ of subsets of a set $X$.
Its  \emph{packing} is a subfamily of $\cal F$ of pairwise disjoint subsets, and its \emph{traversal} (or \emph{hitting set}) is a subset of $X$ which intersects every member of $\cal F$.
The \emph{packing number} of~$\cal F$, denoted $\nu(\cal F)$, is the largest cardinality of a packing in $\cal F$,
and the \emph{transversality} of $\cal F$, denoted
$\tau(\cal F)$, is the smallest cardinality of a traversal of $\cal F$.
Note that if $\cal G$ is a finite set system, then
$\nu({\cal G})\le \tau(\cal G)$. 
The set system $\cal F$ has the \emph{Erd{\H o}s-P\'{o}sa property} if there is a function $f\from\N\to\N$ such that every finite subfamily $\cal G$ of $\cal F$
satisfies $\tau({\cal G})\le f(\nu(\cal G))$. 

We prove that set systems defined by first order formulas in nowhere dense graph classes have the Erd{\H o}s-P\'{o}sa property, in the following sense.

 \setcounter{aux}{\thetheorem}
 \setcounter{theorem}{\theep}
\begin{theorem}
	Fix a nowhere dense class of graphs $\CCC$ and a 
	formula $\phi(x,y)$ with two free variables $x,y$.
	Then there is a function $f\from \N\to\N$ with the following property.
	Let $G\in \CCC$ be a graph and let $\cal G$
	be a family of subsets of $V(G)$ consisting of sets of the form $\setof{v\in V(G)}{\phi(u, v)}$, where~$u$ is some vertex of $V(G)$.
Then~$\tau({\cal G})\le f(\nu(\cal G))$.
\end{theorem}
 \setcounter{theorem}{\theaux}

%\erdosposa*
% \begin{theorem}\label{thm:erdos-posa}
% 	Fix a nowhere dense class of graphs $\CCC$ and a
% 	formula $\phi(x,y)$ with free variables $x,y$.
% 	 % where $x$ is a single variable and $\bar y$ is a tuple of variables.
% 	There is a function $f\from \N\to\N$ with the following property.
% 	Let $G\in \CCC$ be a graph and let $\cal G$
% 	be a family of subsets of $V(G)$ consisting of sets of the form $\setof{b\in V(G)}{\phi(a, b)}$, where~$a\in V(G)$.
% Then~$\tau({\cal G})\le f(\nu(\cal G))$.
% \end{theorem}

We will apply the following result of Matou{\v s}ek~\cite{Matousek:2004:BVI:1005787.1005789},
which relies on the proof of Alon and Kleitman~\cite{ALON1992103} of the conjecture of Hardwiger and Debrunner. 
In the result of Matou{\v s}ek, the set system $\cal F$ is infinite. For $m\in \N$, by $\pi_{\cal F}^*(m)$ we denote the \emph{dual shatter function} of $\cal F$, which is defined as the maximal number 
of occupied cells in the Venn diagram of $m$ sets in $\cal F$.


\begin{theorem}[Matou{\v s}ek, \cite{Matousek:2004:BVI:1005787.1005789}]\label{thm:pq}
	Let $\cal F$ be a set system with $\pi^*_{\cal F}(m)=o(m^k)$,
	for some integer $k$, and let $p\ge k$.
	Then there is a constant $T$ such that the following holds for every finite family $\cal G\subset \cal F$: 
	if $\cal G$ has the $(p,k)$-property, meaning that 
	among every $p$ sets in $\cal G$ some $k$ have a non-empty intersection, then $\tau ({\cal G})\le T$.
\end{theorem}
\begin{proof}[of Lemma~\ref{thm:erdos-posa}]
For a graph $G$, define the set system ${\cal F}_G$ on the ground set $V(G)$ as
$${\cal F}_G = \setof{\setof{v\in V(G)}{\phi(u, v)}}{u\in V(G)}.$$
Let then $\cal F$ be the disjoint union of set systems ${\cal F}_G$ for $G\in \CCC$. That is, 
the ground set of $\cal F$ is the disjoint union of the vertex sets $V(G)$ for $G\in \CCC$, and for each $G\in \CCC$ we add to ${\cal F}$
a copy of ${\cal F}_G$ over the copy of relevant $V(G)$.
Then the following claim follows directly fromTheorem~\ref{thm:vc-density}.

\begin{claim}
The dual shatter function of $\cal F$ satisfies $\pi^*_{\cal F}(m)=\Oof(m^{1+\epsilon})$,
for every fixed $\epsilon>0$. In particular, $\pi^*_{\cal F}(m)=o(m^{2})$.
\end{claim}

Consider the function $f\from \N \to \N$ defined so that $f(\nu)$ is the value $T$ obtained fromTheorem~\ref{thm:pq} applied to $\cal F$, $k=2$, and $p=\nu+1$.
Suppose now that $G\in \CCC$ is a graph and $\GGG\subseteq \FFF_G$
is a family of subsets of $V(G)$ consisting of sets of the form $\{v\in V(G)\,\colon\,\phi(u,v)\}$, where $u$ is some vertex of $G$.
We identify $\GGG$ with a subfamily of $\FFF$ in the natural way, following the embedding of $\FFF_G$ into $\FFF$ used in the construction of the latter.
Let $\nu$ be the packing number of $\GGG$.
In particular, for every $\nu+1$ subsets of $\GGG$
there is a vertex $v\in V(G)$
which is contained in two elements of~$\GGG$.
Hence, $\GGG$ is a $(p,2)$-family for $p=\nu+1$.
ByTheorem~\ref{thm:pq}, $\tau(\GGG)\le T=f(\nu)=f(\nu(\GGG))$, as required.
\end{proof}

\input{full-uqw-stable}


%% Acknowledgments
%\begin{acks}                            %% acks environment is optional
%                                        %% contents suppressed with 'anonymous'
%  %% Commands \grantsponsor{<sponsorID>}{<name>}{<url>} and
%  %% \grantnum[<url>]{<sponsorID>}{<number>} should be used to
%  %% acknowledge financial support and will be used by metadata
%  %% extraction tools.
%  This material is based upon work supported by the
%  \grantsponsor{GS100000001}{National Science
%    Foundation}{http://dx.doi.org/10.13039/100000001} under Grant
%  No.~\grantnum{GS100000001}{nnnnnnn} and Grant
%  No.~\grantnum{GS100000001}{mmmmmmm}.  Any opinions, findings, and
%  conclusions or recommendations expressed in this material are those
%  of the author and do not necessarily reflect the views of the
%  National Science Foundation.
%\end{acks}


%% Bibliography
\bibliography{ref}


%% Appendix
%\appendix
%\section{Appendix}
%
%Text of appendix \ldots

\end{document}
