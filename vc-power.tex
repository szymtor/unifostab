
 

\section{VC-dimension of power graphs}\label{sec:vc}

We now prove \cref{thm:new-vc}. 
In fact, we prove a slightly stronger result. The $2$VC-dimension
of a graph is the largest set which has a neighbour for each 
subset of size $2$. Obviously, the $2$VC-dimension of $G$
bounds its VC-dimension. 

\begin{theorem}
If $K_t\not\minor_rG$, then 
$G^r$ has $2$VC-dimension at most $t-1$. 
\end{theorem}
\begin{proof}
Assume there is a set $A=\{a_1,\ldots, a_t\}$ of size $t$ such that
for all subsets $\{a_i,a_j\}\subseteq A$ of size $2$ 
there is an element $v_{ij}\in V(G)\setminus A$ with 
$N_r[v_{ij}]\cap A=\{a_i,a_j\}$. Fix $v_{ij}$ with the property
that $\max\{\dist_G(v_{ij},a_i), \dist_G(v_{ij},a_j)\}$ is 
minimised. 

A \emph{central walk} $W_{ij}$ is the concatenation of a minimum length
path $P_{ij}^i$ from $a_i$ to $v_{ij}$ and a minimum length path $P_{ij}^j$ from $v_{ij}$ to $a_j$. 
Note that a central walk is possibly not a path. For each pair $a_i,a_j$ fix
a central walk $W_{ij}$ and the corresponding paths $P_{ij}^i$ and $P_{ij}^j$. 

Now assume that a vertex $x$ belongs to two distinct central 
walks $W_{ij}$, $W_{i'j'}$. Assume that $x$ lies on $P_{ij}^i$ and $P_{ij}^{i'}$,
otherwise, rename the elements. First, observe that if $\dist(x,a_i)=\dist(x,a_{i'})$, 
then $a_i=a_{i'}$. Otherwise, $\dist(v_{ij},a_{i})=\dist(v_{ij},a_{i'})$ and hence 
$a_j=a_i$, and analogously, $a_{j'}=a_j$, contradicting the assumption 
that $W_{ij}$ and $W_{i'j'}$ are distinct. By the same argument we have 
$\dist(x,a_i)<\dist(x,a_j)$ and $\dist(x,a_{i'})<\dist(x,a_{j'})$. 
Now assume that $\dist(x,a_i)<\dist(x,a_{i'})$. By the same argument as 
above we have $a_{j'}=a_i$, hence $W_{i'j'}=W_{ij'}$. Here, we have
$\dist(x,a_i)<\dist(x,a_j)$ and $\dist(x,a_{i})<\dist(x,a_{i'})$, 
otherwise the walks are not distinct. 

Let us now construct connected subsets $X_i$ for all $1\leq i\leq t$. 
For every walk $W_{ij}$ the vertices of $W_{ij}$ closer to $a_i$ than to $a_j$ 
are added to $X_i$, the vertices of $W_{ij}$ closer to $a_j$ than to $a_i$ 
are added to $X_j$, ties are broken arbitrary.
Then the sets $X_i$ are pairwise disjoint by what we proved above. If a vertex $x$
appears in two distinct central walks, these are $W_{ij}$ and $W_{i\ell}$ for some
$i,j,\ell$ with $\dist(x,a_i)<\dist(x,a_j)$ and $\dist(x,a_i)<\dist(x,a_\ell)$. 
In both cases $x$ belongs to $X_i$. By construction, the sets $X_i$ are connected, 
have radius at most~$r$, and 
there is always an edge between a vertex of $X_i$ and a vertex of $X_j$ since $X_i\cup X_j$ 
contains the walk $W_{ij}$. Therefore, if the $2$VC-dimension is at least $t$, the 
graph contains $K_t$ as a depth-$r$ minor. 
\end{proof}