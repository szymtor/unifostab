\section{Introduction}

Nowhere dense classes of graphs were introduced 
by Ne\v set\v ril and Ossona de 
Mendez~\cite{nevsetvril2010first,nevsetvril2011nowhere} as a very 
general model
for uniform sparseness of graphs. These classes generalize many 
familiar classes of sparse graphs, such as planar graphs, graphs 
of bounded treewidth,  graphs of bounded degree, and, in fact, 
all classes that exclude a fixed 
topological minor.
The concept of nowhere denseness
turns out to be very robust as witnessed by the fact that it is equivalent 
to multiple other concepts studied in different areas of mathematics. 
One can equivalently characterize nowhere dense graph classes 
by bounds on the density of shallow (topological)
minors~\cite{nevsetvril2010first,nevsetvril2011nowhere},
quasi-wideness~\cite{nevsetvril2011nowhere} (a notion introduced by
Dawar~\cite{dawar2010homomorphism} in his study of homomorphism
preservation properties), low tree-depth
colorings~\cite{nevsetvril2008grad}, generalized coloring
numbers~\cite{zhu2009coloring}, sparse neighborhood
covers~\cite{GroheKRSS15,grohe2014deciding}, by a game called the
splitter game~\cite{grohe2014deciding} and by the model-theoretic
concepts of stability and independence~\cite{adler2014interpreting}.
For a broader discussion on the graph theoretic sparsity we refer to the book
of Ne\v{s}et\v{r}il and Ossona de Mendez~\cite{sparsity}.

These alternative characterizations have been very useful in 
the design of efficient algorithms. For instance, 
the {\sc{Subgraph Isomorphism}} and {\sc{Homomorphism}} problems 
are fixed-parameter tractable on any nowhere dense
class, parameterized by the size of the pattern graph~\cite{nevsetvril2010first}
and so is the {\sc Distance-$r$ Dominating Set} problem, parameterized
by the size of the solution~\cite{DawarK09}. In fact, 
the {\sc Distance-$r$ Dominating Set} problem admits
polynomial kernels~\cite{siebertz2016polynomial} and even 
almost linear kernels on nowhere dense classes of 
graphs~\cite{eickmeyer2016neighborhood}
(see also~\cite{drange2016kernelization} for the case $r=1$). 
It was shown in~\cite{grohe2014deciding}
that every first-order definable problem can be decided in
almost linear time on any nowhere dense graph class.

It is a natural question to ask for the most general classes of graphs
which admit efficient solutions for certain problems, or to 
classify them into tractable and intractable classes. It was shown 
that for the first-order model-checking problem~\cite{dvovrak2013testing} and for
the {\sc Distance-$r$ Dominating Set} problem~\cite{drange2016kernelization} 
the dividing line for algorithmic tractability 
on subgraph closed classes of graphs is exactly between the
nowhere dense and somewhere dense graph classes. 

\smallskip

In (infinite) model theory, we are a different classification program is followed. An important 
problem is to count the number and classify the models of a complete theory. According
to the upward L\"ownheim-Skolem theorem, every theory $T$ with an infinite model has
models of arbitrary infinite cardinalities (larger then the size of the language). One of the central problems
 is to determine, for any fixed infinite cardinal~$\kappa$, how many models of cardinality 
$\kappa$ the theory $T$ can have (up to isomorphism). Note that the number of
models of cardinality $\kappa$ must lie between $1$ and $2^\kappa$ for all infinite cardinals $\kappa$ 
bigger than the cardinality of the language. By a result of 
Morley~\cite{morley1965categoricity}, if $T$ is a countable
theory which has only one model of cardinality $\kappa$ for some uncountable $\kappa$, 
then $T$ has only one model for all uncountable cardinals; a result which 
is often considered the starting point of modern model theory. In his famous classification
project~\cite{shelah1990classification}, Shelah essentially provided a complete
classification for all countable theories (the classification was completed 
in~\cite{hart2000uncountable}). Shelah identified several dividing lines and showed that
all theories on the non-structure side of the dividing line have $2^\kappa$ models
of cardinality $\kappa$. On
the structure side he showed that there are only few models whose isomorphism 
types can be described by small invariants. One of the most important dividing 
lines is \emph{stability}. Very roughly, if a theory is not stable then its models 
are too many and too complicated to classify, while this may be possible if the theory is stable, 
especially if the theory is \emph{superstable} or \emph{totally transcendental}.

Finite model theory is the study of the expressive power 
of logics on the class of finite structures. Many theorems and 
methods of classical model theory fail when only finite structures 
are considered. These include the compactness theorem, the completeness 
theorem and various interpolation and preservation theorems.
On the other hand, quite different questions, which often arise in
computer science, in particular questions from complexity theory and database
theory, are interesting in finite model theory. Hence, instead of focusing 
on negative results, much work has been invested in finding 
subclasses of the class of all finite structures that may be better behaved.
Surprisingly, it turns out that particularly classes which are 
algorithmically well-behaved turn out to have good model-theoretic 
properties. The two prime examples in our context are the result of 
Dawar~\cite{dawar2010homomorphism}, who introduced the notion of
quasi-wideness and proved any quasi-wide class that is closed under taking substructures
and disjoint unions has the homomorphism preservation property. It was 
later proved by Ne\v{s}et\v{r}il and Ossona de Mendez that 
the notions of uniform quasi-wideness and nowhere denseness coincide for 
graphs~\cite{nevsetvril2011nowhere}. The second result is by 
Adler and Adler~\cite{adler2014interpreting}, who observed that 
nowhere denseness essentially corresponds to the stability theoretic notion 
of \emph{superflatness} introduced by Podewski and 
Ziegler in~\cite{podewski1978stable}. In fact, Adler and Adler observed 
that on subgraph closed classes of graphs, the notions of nowhere denseness, 
stability and the non-independence property (NIP) are equivalent. 
Before we can give an 
overview of some of the techniques that can be brought down from 
the infinite to the finite and state our contributions, we need some
definitions. 

\paragraph{Stability.}
We start from recalling some notions from stability theory. 



\begin{definition}
Let $\cal C$ be a class of structures in a first-order language $L$. Let 
$\phi(\tup{x},\tup{y})$ be an $L$-formula with the free variables
divided into two groups $\tup{x}, \tup{y}$. A \emph{$\phi$-ladder}
of length $n$ is a sequence $(\tup{a}_1,\ldots, \tup{a}_{n},
\tup{b}_1,\ldots, \tup{b}_{n})$ of tuples in some model $\strA\in \cal C$, such that for all $1\leq i,j\le n$,
\[\strA\models\phi(\tup{a}_i,\tup{b}_j)\Longleftrightarrow i\leq j. \]
The least  $n$ for which 
there is no $\phi$-ladder of length $n$ is 
the \emph{ladder index} 
of $\phi(\tup{x},\tup{y})$ in $\cal C$ (which may depend on the way we split the
variables).
A formula $\phi$ is \emph{stable} (for $\cal C$) if its ladder index is finite, and \emph{unstable} otherwise. The class of structures $\cal C$ is \emph{stable} if  every formula is stable.
\end{definition}

By abuse of notation, we will apply the above definition to a single structure $\strA$ in place of a class $\cal C$; formally, this amounts to considering the class $\cal C=\set\strA$.


Usually, in model theory one fixes a complete theory $T$
and takes as the class $\cal C$ above the class of all models of the theory $T$. The theory $T$ is then called stable if the class $\cal C$ is stable.
We remark that a formula with ladder index $n$ is said to have the
$n$-order property in~\cite{adler2014interpreting} and~\cite{ensley1996finite}.
We follow the notation of the textbook~\cite{hodges1993model} and remark
that it is easy to see that a class $\CCC$ is stable if and only if there is no 
formula $\psi(\tup{x},\tup{y})$ such that for every $n\in \N$
there exist a structure $\strA\in \CCC$ and tuples $\tup{a}_1,\ldots, \tup{a}_{n}$
of elements of $\strA$ such that $\strA\models\psi(\tup{a}_i,\tup{a}_j)\Leftrightarrow i<j$,
that is, $\psi$ orders the tuples linearly. 

The first application of stability theory concerns the existence of long
indiscernible sequences. If $\tup{a}=(a_0,\ldots)$ is a non-repeating sequence of elements of $\strA$, we write \[[\tup{a}]^k\coloneqq \{\tup{b}=(a_{i_1},\ldots, a_{i_k}) \sth 0\leq i_1<\ldots <i_k\}\] for the set of of all subsequences
of $\tup{a}$ of length $k$. When we speak of a subsequence $\tup{a}'$
of a sequence $\tup{a}$, we always mean an increasing subsequence. 

\begin{definition}
Let $\tup{a}=(a_0,\ldots)$ be a non-repeating sequence of elements of $\strA$.
Let $f$ be a map with domain $[\tup{a}]^k$. A subsequence $\tup{a}'$ 
of $\tup{a}$ is \emph{$f$-indiscernible} if for any two 
subsequences $\tup{v},\tup{w}\in [\tup{a}']^k$ 
it holds that $f(\tup{w})=f(\tup{w})$.

If $\Phi$ is a set of formulas of the form $\phi(x_0,\ldots, x_{k-1})$, 
we say that $\tup{a}'$ 
is \emph{$\Phi$-indiscernible} (in a structure $\strA$)
if for every $\phi\in \Phi$ and any two 
subsequences $\tup{v},\tup{w}\in [\tup{a}]^k$ it
holds that $\strA\models\phi(\tup{a})\Leftrightarrow\strA\models\phi(\tup{b})$. 
\end{definition}

%We often identify ordered sets with sequences of elements. 
%A sequence $(a_1,\ldots, a_\ell)$ of elements of~$\strA$ is
%\emph{$\Phi$-indiscernible} if for every formula
%$\phi(x_1,\ldots, x_k)\in \Phi$ with $k$ free variables and any two
%increasing sequences
%$1\leq i_1<\ldots <i_k\leq \ell, 1\leq j_1< \ldots< j_k\leq \ell$ of
%integers, it holds that
%\[\strA\models\phi(a_{i_1},\ldots, a_{i_k})\Leftrightarrow \strA\models\phi(a_{j_1},
%\ldots, a_{j_k}).\]

Results about the existence of indiscernible sequences are known as 
\emph{partition theorems}, because the map $f$ induces a partition
on the set $[\tup{a}]^k$ (which can be identified with the ordered 
set $A=\{a_0,\ldots\}$). Partitioning theorems play an important role in 
combinatorics and graph theory, and also in stability theory. The 
crucial distinction in stability theory is between \emph{Ramsey type 
theorems}, which hold
for reason of cardinality alone, and \emph{stability type theorems},
which take into account the underlying structure (theory). The following
theorem shows that we need the sequence $A$ to be only polynomially larger
than the indiscernible sequence $B$ we are aiming at if we are in a stable theory. 
It is easy to make the theorem algorithmic, see 
e.g.~\cite{siebertz2016polynomial}. 

\begin{theorem}[Malliaris and Shelah~\cite{malliaris2014regularity}, Theorem 3.5, Item (2)]\label{thm:malshelah}
  Let  $\Delta$ be a finite
  set of stable first-order formulas.  There is a polynomial $p(x)$ such that
  for all $\strA\in \CCC$, every positive integer $m$ and every non-repeating sequence
  $\tup{a}=(a_0,\ldots, a_{\ell-1})$ of elements of $\strA$ of length $\ell=p(m)$, there
  exists a sub-sequence $\tup{a}'$ of
  $\tup{a}$ of length $m$ which is
  $\Delta$-indiscernible.
\end{theorem}

Another perspective on stability theory is to see it as a way of 
classifying definable sets in a structure and describing the interaction 
between definable sets. For this, we need the notion of types. 

\begin{definition}
Let $\Delta$ be a set of $L$-formulas, let $\strA$ be an $L$-structure and let
$A$ be a set of elements of $\strA$. Let $a$ be an element of $\strA$. The 
\emph{$\Delta$-type of~$a$ in $\strA$ over the parameters $A$} is the set
\begin{align*}
  \mathrm{tp}_\Delta(\strA, A, a) & \coloneqq  \{ \phi(x_1,a_1,\ldots, a_k)  : &\\
  &  \hspace{3.1cm}
                                \phi(x_1,y_1,\ldots, y_k)\in \Delta,
                                a_1,\ldots, a_k\in A,
                                \strA\models\phi(a,a_1,\ldots, a_k)\}.
\end{align*}
The set of \emph{$\Delta$-types realised} in $\strA$ over $A$ is the set
$S_\Delta(\strA,A) \coloneqq \{ \mathrm{tp}_\Delta(\strA, A, a) \sth a$ element of $\strA\}$.
\end{definition}

We usually assume that $\Delta$ is closed under negations. 
The number of realised types over a parameter set of cardinality
$\kappa$ can be anything between $0$ and $\abs{\Delta}\cdot 2^\kappa$. 
Shelah proved that
in a \emph{dependent} theory the number of realised types will always be
at most polynomial, and also this theorem carries over to the finite. 

\begin{definition}
Let $\cal C$ be a class of structures in a first-order language $L$. Let 
$\phi(\tup{x},\tup{y})$ be an $L$-formula with the free variables
divided into two groups $\tup{x}, \tup{y}$. The formula $\phi$ has
the \emph{$n$-independence property} if for some structure $\strA$ in $\cal C$, there are tuples $\tup{a}_1,\ldots, \tup{a}_{n}$ and
$\tup{b}_J$ for $J\subseteq \{1,\ldots, n\}$ such that
\[\strA\models\phi(\tup{a}_i,\tup{b}_J)\Longleftrightarrow i\in J. \]
A formula $\phi$ is \emph{dependent} (for $\cal C$) if there is 
some $n\in \N$ such that $\phi$ does not have the $n$-independence property.
A class $\CCC$ of structures is \emph{dependent} or has the \emph{non-independence property}, 
short \emph{NIP}, if every formula~$\phi$ is dependent for $\cal C$. 
\end{definition}

\begin{theorem}[\cite{shelah1990classification}, Theorem II.4.10(4) and II.4.11(4)]
Let $\Delta$ be a finite set of first-order formulas which are 
dependent on $\strA$. Then there exists a positive integer $k$ such that 
for any set~$A$ of elements of $\strA$ with $\abs{A}\geq 2$ it holds that
$\abs{S_\Delta(\strA,A)}\leq \abs{A}^k$. 
\end{theorem}

The independence property of a formula can be understood as its 
\emph{VC-dimension}~\cite{vapnik2015uniform,laskowski1992vapnik}, 
a concept with many applications in computational
learning theory, see e.g.~\cite{}. 

\paragraph{Nowhere denseness.}
We now come to the definition of nowhere dense graph classes, which is based
on excluded bounded-depth minors. 

\begin{definition}
A {\em{minor model}} of a graph $H$ in $G$ is a family $(I_u)_{u\in V(H)}$ of pairwise vertex-disjoint connected subgraphs of $G$
such that whenever $uv$ is an edge in~$H$, there are $u'\in I_u$ and $v'\in I_v$ for which $u'v'$ 
is an edge in $G$.
The graph $H$ is a {\em{depth-$r$ minor}} of $G$, denoted $H\minor_rG$, if there is a minor model
$(I_u)_{u\in V(H)}$ of~$H$ in $G$ such that each subgraph $I_u$ has radius at most $r$.

A graph $H$ is a \emph{topological minor} of a graph $G$ if there is a
function~$\delta$ mapping vertices $v\in V(H)$ to vertices of $V(G)$ and 
edges $e\in E(H)$ to directed paths in $G$ such that 
$\delta(v)\neq \delta(u)$ for all distinct $u,v\in V(H)$, and 
if $e=(u,v)\in E(H)$, then $\delta(e)$ is a path from 
$\delta(u)$ to $\delta(v)$ in~$G$ which is internally vertex disjoint from all 
$\delta(e')$ with $e'\in E(H)$, $e'\neq e$. 
For $r\geq 0$, $H$ is a \emph{topological depth-$r$ minor} of $G$, 
written $H\minor_r^tG$, if it is a topological minor and all paths~$\delta(e)$
have length at most $2r$. If all paths~$\delta(e)$ have length exactly
$r$, we say that $G$ contains an \emph{$r$-subdivision} of $H$ (as a 
subgraph). 
\end{definition}

\begin{definition}
A class $\CCC$ of graphs is \emph{nowhere dense} if there is a function 
$f:\N\rightarrow \N$ such that for all $r\in \N$ it holds that $K_{f(r)}\not\minor_r G$
for all $G\in \CCC$. 
\end{definition}

Equivalently, a 
class $\CCC$ of graphs is nowhere dense if there is a function 
$g:\N\rightarrow \N$ such that for all $r\in \N$ it holds that 
$K_{g(r)}\not\minor_r^t G$ for all $G\in \CCC$, or that the
$r$-subdivision of $K_{g(r)}$ is not a subgraph of all $G\in \CCC$. 
The latter is in fact the definition of \emph{superflat classes of graphs}
(for classes of finite and infinite graphs), 
given by Podewski and Zieger~\cite{podewski1978stable}, who also
proved that superflat classes of graphs are stable. 
The notion of uniform
quasi-wideness introduced by Dawar~\cite{dawar2010homomorphism}
can be seen as a finite analogue of the property 
$(\ast)$ introduced by Podewski and Ziegler. 

\begin{definition}
A set $B\subseteq V(G)$ is called {\em{$r$-independent}} in $G$ if for all
distinct $u,v\in B$ we have $\dist_G(u,v)>r$.
A class $\CCC$ of graphs is \emph{uniformly quasi-wide} if there are
functions $N\colon \N\times\N\rightarrow \N$ and $s:\N\rightarrow \N$ such
that for all $r,m\in \N$ and all subsets $A\subseteq V(G)$ for
$G\in \CCC$ of size $\abs{A}\geq N(r,m)$ there is a set
$S\subseteq V(G)$ of size $\abs{S}\leq s(r)$ and a set
$B\subseteq A\setminus S$ of size $\abs{B}\geq m$ which is $r$-independent in
$G-S$. 
\end{definition}

As shown by Ne\v{s}et\v{r}il and Ossona de Mendez~\cite{nevsetvril2010first},
a class $\CCC$ of graphs is nowhere dense if and only if it
is uniformly quasi-wide.

\paragraph{Our contributions.} In this work we revisit the 
connections between nowhere denseness, quasi-wideness and
stability. 

Our first result is  a new proof of a result of
Ne\v{s}et\v{r}il and Ossona de Mendez~\cite{nevsetvril2010first},
which states that a class $\CCC$ of graphs is nowhere dense if and only if it
is uniformly quasi-wide. The proof of Ne\v{s}et\v{r}il 
and Ossona de Mendez goes back to a construction
of Kreidler and Seese~\cite{kreidler1998monadic} (see also Atserias et al.~\cite{atserias2006preservation}), 
and uses iterated Ramsey arguments. Hence the original bounds on 
the function $N$ are huge. Recently, 
it was proved that we may always choose $N$ to be a polynomial 
function~\cite{siebertz2016polynomial}. The degree of the polynomial 
in~\cite{siebertz2016polynomial} was  not specified, its existence 
depends on Adler and Adler's result that nowhere dense classes of graphs
are stable and hence every fixed formula has bounded ladder index 
on every nowhere dense class of graphs. We give a new construction 
which is considerably simpler than that of~\cite{siebertz2016polynomial}
and which gives explicit bounds on the degree of the polynomial. 
We prove the following theorem. 

\begin{theorem}\label{thm:new-uqw}
Let $G$ be a graph such that $K_t\not\minor_{3r+1} G$. 
If $A\subseteq V(G)$ of size $\Omega(m^{(6t+3)^{t+r}})$, then we can find a set
$S\subseteq V(G)$ of size $|S|\leq t$ and a set $B\subseteq A\setminus S$ 
of size $|B|\geq m$ which is $r$-independent in $G-S$.  
\end{theorem}

We prove \cref{thm:new-uqw} in \cref{sec:uqw}. We want to highlight
that even though our methods are the methods from stability theory, 
we use only very simple graph theoretic notions. In particular, the
proof can easily be turned into an efficient algorithm which does not
call a model-checking algorithm as a subroutine. 

\bigskip
Podewski and Ziegler's proof that flat graphs are stable uses an 
infinite Ramsey argument. Based on Gaifman's Locality Theorem for
first-order logic~\cite{gaifman1982local}, we give a combinatorial 
proof that every first-order formula has finite ladder index on every
nowhere dense class of graphs. Our proof gives explicit bounds for the
ladder indices of formulas. 

\begin{theorem}\label{thm:new-stable}
  There are computable functions $f:\N^3\to\N$ and $g:\N\to\N$ with the following property.
If $\phi(\bar x)$ is a formula of quantifier rank $q$ and with $d$ free variables
and  $G$ is a graph such that $K_t\not\minor_{g(q)} G$, then the ladder index of $\phi$ on $G$ is at most $f(q,d,t)$. 
\end{theorem}

We prove \cref{thm:new-stable} in \cref{sec:stable}. 

\bigskip

Finally, we again consider the formula stating that the distance
between two elements is at most $r$. 
We observe that an argument of Bousquet and 
Thomasse\'e~\cite{BousquetT15} can be slightly modified to prove that 
the VC-dimension of the $r$-power graph $G^r$ of a graph $G$
with $K_t\not\minor_r G$ is bounded by $t-1$.

\begin{theorem}\label{thm:new-vc}
Let $G$ be a graph such that $K_t\not\minor_r G$. Then the
VC-dimension of the $r$-power graph~$G^r$ is bounded by $t-1$. 
\end{theorem}

We prove \cref{thm:new-vc} in \cref{sec:vc}. 
