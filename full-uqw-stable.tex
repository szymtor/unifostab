
\section{Bounds for stability}\label{sec:stable}
As mentioned, Adler and Adler~\cite{adler2014interpreting}, 
proved that every nowhere dense class of graphs is stable. In this section,
we prove its effective variant,~Theorem~\ref{thm:new-stable}, which 
we repeat for convenience.
%\newstable*
 \setcounter{aux}{\value{theorem}}
 \setcounter{theorem}{\value{stable}}
  \setcounter{auxsec}{\value{section}}
 \setcounter{section}{1}
 \begin{theorem}
 There are computable functions $f\colon \N^3\to\N$ and $g\colon\N\to\N$ with the following property.
 Suppose $\phi(\bar x,\bar y)$ is a formula of quantifier rank at most $q$ and with $d$ free variables.
 Suppose further that $G$ is a graph excluding $K_t$ as a depth-$g(q)$ minor. Then the ladder index of $\phi(\bar x,\bar y)$ in $G$ is at most $f(q,d,t)$.
 \end{theorem}
 \setcounter{theorem}{\value{aux}}
 \setcounter{section}{\value{auxsec}}

Recall that a class $\CCC$ is stable if and only if for every first order formula $\varphi(\bar x,\bar y)$, 
its ladder index over graphs from $\CCC$ is bounded by a constant depending only on $\CCC$ and $\varphi$;
see Section~\ref{sec:intro} to recall the background on stability.
Thus the result of Adler and Adler is implied by Theorem~\ref{thm:new-stable},
and is weaker in the following sense: Theorem~\ref{thm:new-stable} asserts in addition that there is a computable bound on the ladder index
of any formula that depends only on the size of an excluded clique minor at depth bounded in terms of formula's quantifier rank and number of free variables. 
We now prove~Theorem~\ref{thm:new-stable}.

\begin{proof}[Proof of~Theorem~\ref{thm:new-stable}]
Fix a formula $\phi(\bar x,\bar y)$ of quantifier rank $q$ and
a partitioning of its 
free variables into  $\bar y$ and $\bar z$.
Let $d=|\bar x|+|\bar y|$ be the total number of free variables of $\phi$.
Let $r\in \N$ be the number given by~Corollary~\ref{cor:bound},
which depends on $\phi$ only.
Let $\CCC$ be the class of all graphs 
such that  $K_t\not\minor_{18r} G$.
By~Theorem~\ref{thm:uqw-tuples}, 
$\CCC$ satisfies $\uqw^d_{r}(N^d_{r},s^d_r)$,
for some  polynomial  $N^d_r\from\N\to\N$ and number $s=s^d_r\in \N$ computable from $d,t,r$.
Let $T$ be the number given by~Corollary~\ref{cor:bound} for $\phi$ and~$s$.
 Finally, let 
$\ell=N^d_r(2T+1)$.
We show that 
every $\phi$-ladder in any $G\in\CCC$ has length smaller than~$\ell$.


For the sake of contradiction, assume that there is a graph $G\in\CCC$
and tuples $\bar u_1,\ldots,\bar u_\ell\in V(G)^{|\bar x|}$ and $ \bar v_1,\ldots, \bar v_\ell\in V(G)^{|\bar y|}$
which form a $\phi$-ladder in $G$, i.e., 
$\phi(\bar u_i,\bar v_j)$ holds in~$G$ if and only if $i\le j$.
	Let $A=\setof{ \bar u_i \bar v_i}{i=1,\ldots,\ell}\subset V(G)^d$. Note that $|A|=\ell\ge N^d_r(2T+1)$, since tuples $\bar u_i$ have to be pairwise different.
  
Applying property  $\uqw^d_r(N^d_r,s^d_r)$ to the set $A$, radius $r$, and target size $m=2T+1$
		 yields a set $S\subset V(G)$ with $|S|\le s$
	and a set $B\subset A$ with $|B|\geq 2T+1$ 
  of tuples which are  mutually $r$-separated by $S$  in $G$.
  Let $J\subseteq \set{1,\ldots,\ell}$
  be the set of indices corresponding to $B$,
  i.e., $J=\set{j\colon\bar u_j\bar v_j\in B}$.
  
  Since $|J|=2T+1$, we may partition $J$ into $J_1\uplus J_2$ with $|J_1|=T+1$ so that the following condition holds:
  for each $i,k\in J_1$ satisfying $i<k$, there exists $j\in J_2$ with $i<j<k$. Indeed, it suffices to order the indices of $J$ and put every second index to $J_1$, and every other to $J_2$.
  Let~$X$ be the set of vertices appearing in the tuples $\bar u_i$ with $i\in J_1$, and let $Y$ be the set of vertices appearing in the tuples $\bar v_j$ with $j\in J_2$.
  Since the tuples of $B$ are mutually $r$-separated by $S$ in~$G$, it follows that~$X$ and~$Y$ are $r$-separated by $S$.
  As $|J_1|=T+1$, by Corollary~\ref{cor:bound} we infer that there are distinct indices $i,k\in J_1$, say $i<k$, such that $\tp^\phi(\bar u_i/Y)=
    \tp^\phi(\bar u_{k}/Y)$. This implies that for each $j\in J_2$, we have $G,\bar u_i,\bar v_j\models \phi(\bar x,\bar y)$ if and only if $G,\bar u_{k},\bar v_j\models \phi(\bar x,\bar y)$.
    However, there is an index $j\in J_2$ such that $i<j<k$, and for this index we should have $G,\bar u_i,\bar v_j\models \phi(\bar x,\bar y)$ and $G,\bar u_{k},\bar v_j\not\models \phi(\bar x,\bar y)$
    by the definition of a ladder. This contradiction concludes the proof.
\end{proof}


