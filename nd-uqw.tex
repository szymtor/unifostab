\section{From nowhere denseness to uniform quasi-wideness}\label{sec:uqw}

This section is devoted to the proof of \cref{thm:new-uqw}. 
In the presentation we focus on proving the existential statement, and at the end we briefly argue how the proof can be turned into an algorithm with the promised running time guarantee.
We first recall necessary preliminaries from graph theory. 

\paragraph*{Preliminaries.}
All graphs in this paper are finite, undirected and simple, that is, 
they do not have loops or parallel edges. Our notation is standard,
we refer to~\cite{diestel2012graph} for more background on 
graph theory. 
We write $V(G)$ for the vertex set of a graph $G$ and
$E(G)$ for its edge set. 
The {\em{distance}} between vertices $u$ and $v$ in $G$, denoted $\dist_G(u,v)$, is the length of a shortest path between $u$ and $v$ in~$G$.
If there is no path between $u$ and $v$ in $G$, we put $\dist_G(u,v)=\infty$.
For a vertex $u$ and nonnegative integer $s$, by $N_s[u]$ we denote the {\em{$s$-neighborhood of $u$}} which comprises vertices at distance at most $s$ from $u$.

A {\em{minor model}} of a graph $H$ in $G$ is a family $(I_u)_{u\in V(H)}$ of pairwise vertex-disjoint connected subgraphs of $G$, called {\em{branch sets}},
such that whenever $uv$ is an edge in~$H$, there are $u'\in I_u$ and $v'\in I_v$ for which $u'v'$ 
is an edge in $G$.
The graph $H$ is a {\em{depth-$r$ minor}} of $G$, denoted $H\minor_rG$, if there is a minor model
$(I_u)_{u\in V(H)}$ of~$H$ in $G$ such that each $I_u$ has radius at most $r$.

A class $\CCC$ of graphs is \emph{nowhere dense} if there is a function 
$t\colon \N\rightarrow \N$ such that for all $r\in \N$ it holds that $K_{t(r)}\not\minor_r G$
for all $G\in \CCC$. 

A set $B\subseteq V(G)$ is called {\em{$r$-independent}} in a graph $G$ if for all
distinct $u,v\in B$ we have $\dist_G(u,v)>r$.
A class $\CCC$ of graphs is \emph{uniformly quasi-wide} if there are
functions $N\colon \N\times\N\rightarrow \N$ and $s:\N\rightarrow \N$ such
that for all $r,m\in \N$, all graphs $G\in \CCC$, and all subsets $A\subseteq V(G)$ of size $\abs{A}\geq N(r,m)$, there is a set
$S\subseteq V(G)$ of size $\abs{S}\leq s(r)$ and a set
$B\subseteq A\setminus S$ of size $\abs{B}\geq m$ which is $r$-independent in
$G-S$. 

\paragraph{General strategy.}
Our proof follows exactly the same lines as the original proof of Ne\v set\v ril and Ossona de Mendez, with the difference that in one technical lemma (\cref{lem:apex} below), we improve the bounds significantly by replacing a Ramsey argument by bounds on neighborhood
complexity in nowhere dense graph classes, due to~\cite{gajarsky2017kernelization}.
For sake of completeness, we present the entire proof of~\cref{thm:new-uqw}.


To prove
\cref{thm:new-uqw}, we first prove its variant~\cref{lem:engine} below, in which we assume that $A$ is already $(r-1)$-independent, and then chain the lemma for $r$ ranging from $1$ to the target value of $r$ to obtain the general statement.

\begin{lemma}\label{lem:engine}
For all $t\in \N$ there exists a function $L\colon \N\to \N$
bounded by a polynomial
and a number $d$ such that the following holds.
Given numbers $r,m\in \N$ and graph $G$ such that $K_t\not\minor_d G$ and
$(r-1)$-independent set of vertices $A\subset V(G)$ of size at least $L(m)$ there is a set $S\subseteq V(G)-A$ of size at most $t$ such that $A$ contains a subset $A'$ of size $m$ which is $r$-independent in $G-S$.

Moreover, if $r$ is odd then $S$ is empty, and if $r$ is even,
then every vertex of $S$ is at distance exactly $ r/2$ from every vertex of $A'$.
Finally, given $G$ and $A$, the sets $A'$ and $S$ can be computed in time $\Oof_{r,t}(|A|^c\cdot |E(G)|)$, where $c$ is some universal constant.
\end{lemma}

\subsection{Proof of~\cref{lem:engine}}

To prove~\cref{lem:engine}, we distinguish two special cases: the case of $r=1$ and the case $r=2$. The case of general $r$ then reduces to one of these two cases, depending on the parity of $r$, by observing that a $(2s+1)$-independent set $A$ in $G$
induces a $1$-independent set in $G$ with the balls of radius $s$ around the vertices of $A$ contracted, and,
similarly, a $(2s+2)$-independent set $A$ in $G$
induces a $2$-independent set in $G$ with the balls of radius $s$ around the vertices of $A$ contracted.

We start by proving~\cref{lem:engine} in the case $r=1$,
then we prove it in the case $r=2$. Next we show how the general case reduces to one of those two, and, finally, we deduce~\cref{thm:new-uqw} from~\cref{lem:engine}.

\paragraph{Case $r=1$.}
We prove~\cref{lem:engine} in the case $r=1$.
Let $d=0$, so $K_t\not\minor_d G$ amounts to saying that $G$
does not contain a clique of size $t$. By Ramsey's Theorem, in any graph every set of size $\binom{m+t-2}{t-1}$ contains an
independent set $A'$ of size $m$ or a clique of size $t$. Therefore, 
taking $L(m)$ as the above binomial coefficient yields~\cref{lem:engine} in case $r=0$, for $S=\emptyset$. Moreover, the set $A'$ can be computed from $G$ and $A$
in time~$\Oof(|A|\cdot |E(G)|)$, by simulating the proof of Ramsey's theorem.

\paragraph{Case $r=2$.}
We prove~\cref{lem:engine} in the case $r=2$. The main tool is the following Ramsey-like result.

\begin{lemma}\label{lem:apex}
For every integer $t\in \N$ and positive real $\alpha<\frac 1 2$ 
there is an integer $\ell_0\in\N$ with the following property.
Let $m,\ell$ be integers with $\ell\ge \ell_0$. 
If~$G$ is a graph and $A$ is its $1$-independent subset
with at least $(m+\ell)^{t}$ elements,
then at least one of the following conditions hold:
\begin{enumerate}
  % \item $A$ contains the principal vertices of a $1$-subdivision of a clique $K_t$ contained in $G$,
  \item $K_t\minor_{2} G$,
\item  $A$ contains a $2$-independent set of size $m$, 
\item  some vertex $v$ of $G$
has at least $\ell^{\alpha}$ neighbors in $A$.
\end{enumerate}
Moreover, in each case, the corresponding structure (a depth-$2$ minor model, an $2$-independent set of size $m$ or the vertex $v$) can be computed in $\Oof(|A|^c\cdot |E(G)|)$, where $c$  is some universal constant.
\end{lemma}

Note that a similar statement as that of~\cref{lem:apex}
can be obtained without much effort by employing Ramsey's theorem, as has been done in~\cite{}; however, the bound $(m+\ell)^t$ above is then replaced by a bound which is not polynomial in $m$, and is therefore insufficient for our needs.

\medskip
To prove~\cref{lem:apex}, we employ the following result: the bound on the number of distinct neighborhoods in a graph from a nowhere dense class.

\begin{lemma}[adaptation of Lemma 4.11 in \cite{gajarsky2017kernelization}]\label{lem:diversity}
Let $G$ be a graph such that $K_t\not\minor_{1} G$ for some constant $t\in \N$. 
Then for every $\epsilon>0$ there exists $n_0$, depending only on $t$ and $\epsilon$, such that for all $A\subseteq V(G)$ with $|A|\geq n_0$ it holds that
\[\abs{\{N(v)\cap A \colon v\in V(G)\}}\leq\abs{A}^{1+\epsilon}.\]
\end{lemma}



We remark that the proof of \cref{lem:diversity} uses only the fact that
nowhere dense classes of graphs do not have dense 
shallow minors~\cite{dvorak2007asymptotical,jiang2011compact}, and does not rely on any non-constructive arguments from the stability theory.
From \cref{lem:diversity} we derive the following.

\begin{corollary}\label{cor:diversity}
  Let $G,t,\epsilon,n_0$ be as above.
  If $|A|=n\ge n_0$ and every pair of elements of $A$ has a common neighbor in $G$,
  then there is a vertex $v$ in $G$ which has at least $n^{(1-\epsilon)/2}$ neighbors in $A$.
\end{corollary}
\begin{proof}Let ${\cal F}=\set{N(v)\cap A\colon v\in V(G)}$. 
  Let $k$ be the maximal cardinality of a set $F$ in ${\cal F}$.
  Say that a pair $(a,b)$ of elements of $A$ is \emph{covered} by $F\in {\cal F}$
  if  $a,b\in F$.
  By assumption, each of the $n^2$ pairs in $A^2$ is covered by some element $F\in {\cal F}$, and,
  clearly, every $F\in {\cal F}$ can cover at most $k^2$ pairs. 
  Hence, $n^2/k^2\le |{\cal F}|\le  n^{1+\epsilon}$, proving $k\ge n^{(1-\epsilon)/2}$.
\end{proof}

\begin{corollary}\label{cor:biversity}
    For every positive real $\alpha<\frac 1 2$ and integer $t\in\N$ there exists $\ell_0$ with the following property.
  Let $G$ be a graph such that $K_t\not\minor_2 G$
  % which does not contain
  % the complete bipartite graph $K_{t,t}$ 
  and let
  $A$ be a $1$-independent subset of $G$ with $|A|\ge \ell_0$ such that every pair of vertices in $A$ has a common neighbor in $G$.
   Then there is a vertex $v\in G$
  with at least $|A|^{\alpha}$ neighbors in $A$.
\end{corollary}
\begin{proof}\label{pf:}
    % Note that if $K_{2t}$ has a depth $1$ minor model in
    % a bipartite graph $H$, then there is a depth $1$ minor model of $K_{t}$ in $H$ with all centers of branch sets contained in one part of $H$. This, in turns, implies that $K_{t,t}$ is a subgraph of $H$.
  Define $H$ to be the bipartite graph induced 
  by $G$, with parts $A$ and $V(G)-A$, i.e.,  $v\in A$ and $w\in V(G)-A$ are adjacent in $H$ iff they are in $G$. 
  We claim that $K_{2t^2}\not\minor_1 H$. Indeed, it is easy to see that 
  since $H$ is bipartite, $K_{2t^2}\minor_1 H$ would imply that $H$ contains the 
  complete bipartite graph $K_{t^2,t^2}$,  which, in turn, implying that $K_{t}\minor_2 G$, contrary to our assumption.  
  % Since $G$ does not contain the complete bipartite graph $K_{t,t}$, neither does $H$.  
  % By the preceding remark, this implies  that $K_{2t}\not\minor_1 H$.

  Let $\ell_0$
    be the value $n_0$ from~\cref{lem:diversity} applied to $2t^2$ in place of $t$ and $\epsilon$ such that $(1-\epsilon)/2=\alpha$. Applying~\cref{cor:diversity} to $H,2t,\epsilon,\ell_0$, we conclude that if every pair of elements of $A$ has a common neighbor in $G$
    (hence also in $H$), then there is a vertex $v$
    of $H$ with at least $|A|^{\alpha}$ neighbors in~$A$.
\end{proof}

% \begin{corollary}\label{lem:gajarsky}
% Let $G$ be a graph, let $s\in \N$, and suppose there is a constant $t\in \N$ such that $K_t\not\minor_{3s+1} G$.
% Then for every $\epsilon>0$ there exists $n_0$, depending only on $\epsilon,t,s$, such that for each vertex subset $A\subseteq V(G)$ that is $(2s+1)$-independent in $G$ and satisfies $|A|\geq n_0$, it holds that
% \[\abs{\{N_{s+1}[v]\cap A \colon v\in V(G)\}}\leq\abs{A}^{1+\epsilon}.\]
% \end{corollary}
% \begin{proof}
% As $A$ is $(2s+1)$-independent, we have that the $s$-neighborhoods of vertices from $A$ are pairwise disjoint.
% Obtain an $s$-shallow minor $H$ of $G$ by contracting $N_s[u]$ for each $u\in A$.
% We implicitly identify each vertex $u\in A$ with the vertex of $H$ obtained from contracting $N_s[u]$, thus $A\subseteq V(H)$.
% It is known that every $1$-shallow minor of $H$ is also a $(3s+1)$-shallow minor of $G$ (cf.~\cite[Proposition~4.1]{sparsity}), hence $K_t\not\minor_{1} H$.
%
% Let us fix $\epsilon>0$.
% Take any $v\in V(G)$. If $v\in N_s[u]$ for some $u\in A$, then since $A$ is $(2s+1)$-independent, we have that $v$ is at distance larger than $2s+1$ from any other vertex of $A$.
% Hence in this case we have $N_{s+1}[v]\cap A=\{u\}$ and there can be at most $|A|$ neighborhoods of this type.
% Next, suppose $v\notin N_s[u]$ for any $u\in A$. Then by the construction of $H$ we have that $N_{s+1}[v]\cap A=N^H[v]\cap A$, where $N^H[v]$ is the neighborhood of $v$ in $H$.
% Since $K_t\not\minor_{1} H$, by \cref{lem:diversity} we have that the number of such neighborhoods is at most $|A|^{1+\epsilon/2}$,
% provided $|A|\geq n_0$ for some $n_0$ depending only on $t$ and $\varepsilon$.
% Thus, we conclude that $\abs{\{N_{s+1}[v]\cap A \colon v\in V(G)\}}\leq |A|+|A|^{1+\epsilon/2}$, which is bounded by $|A|^{1+\epsilon}$ if we choose $n_0$ large enough.
% \end{proof}
%
% %Observe that nowhere dense classes are closed under
% %taking bounded depth minors, as stated in the following lemma.
% %It is an immediate consequence of Proposition~4.1
% %of~\cite{sparsity}.
%
% %\begin{lemma}
% %Let $\CCC$ be a nowhere dense class of graphs and let $s\in \N$.
% %Then also the class $\{H \minor_s G \colon G\in \CCC\}$ is nowhere dense.
% %\end{lemma}





To prove~\cref{lem:apex}, we will arrange the elements of $A$ in a binary tree
and prove that the tree contains a long path. From this path, we will 
extract the set $A'$. In stability theory, similar trees are called \emph{type trees} and they are used to extract long indiscernible sequences, see e.g.~\cite{malliaris2014regularity}. 

\begin{proof}
	\newcommand{\dau}{\mathrm{D}}
	\newcommand{\son}{\mathrm{S}}
	
	We identify words in $\set{\dau,\son}^*$ with \emph{nodes}
	of the infinite rooted binary tree. For $w\in \set{\dau,\son}^*$,
	 the nodes $w\dau$ and $w\son$ are called, respectively, the \emph{daughter} and the \emph{son} of $w$,
	and $w$ is the \emph{father} of both $w\son$ and $w\dau$.
	We consider
	 finite, labeled, rooted, binary trees, which are called simply trees below, and are defined as follows.
	 A \emph{tree} is a partial labeling $\tau\from \set{\dau,\son}^*\to U$ whose domain is a finite set of nodes, called the \emph{nodes of $\tau$}, which is closed under taking fathers.
  
  Let $G$ be a graph, $A\subset V(G)$ a $1$-independent set of its vertices
  and $\bar a$ an enumeration of $A$.
We define  a binary tree $\tau$ which is 
  labeled by vertices of $G$. The tree is defined by processing all elements of the sequence $\bar a$ sequentially. We start with $\tau$ being the  tree with empty domain, and for each element $a$ of the sequence $\bar a$, execute the following procedure, which extends the domain of $\tau$ by one element.
  
When processing the vertex $a$, do the following. Start with $w$ being the empty word. While $w$ is in the domain of $\tau$, repeat the following step: 
  if the distance from $a$ to $\tau(w)$ in the graph 
  $G$ is two, replace $w$ by its son, otherwise, replace $w$ by its daughter.
  % Repeat the step, unless $\tau(w)$ is undefined.
  Once $w$ is not in the domain of $\tau$, extend $\tau$ to $w$    so that  $\tau(w)=a$. In this way, we have processed the element $a$, and now
    proceed to the next element $a$ of $\bar a$, until all elements are processed. This ends the construction of $\tau$.
	
  
  
  
  \medskip
For a tree $\tau$, define its
\emph{height} as 
the length of the longest word in the domain of $\tau$.
For a word $w$, define an \emph{alternation} to be 
a position $i$ such that $w_i\neq w_{i-1}$, where $w_0$ is assumed to be $\dau$.
 The \emph{alternation rank} of the tree $\tau$ is the maximum of the number of alternations of $w$, over all nodes $w$ of $\tau$.


\begin{lemma}\label{lem:number-of-nodes}
Let $h,t\ge 2$.	If $\tau$ has alternation rank at most $t-1$ and height at most $h-1$, then $\tau$ has fewer than $h^{t}$
	nodes.
\end{lemma}
\begin{proof}		
	To each node $w$ of $\tau$ assign 
	the function $f_w:\set{1,\ldots,t}\to\set{1,\ldots,h}$ which
	maps a number $i$ to the $i$th smallest index $j\ge 1$
	such that $w_j\neq w_{j-1}$, if it exists, 
	where $w_0$ is assumed to be equal $\dau$,
	and if such an index $j$ does not exist, then $f_w(i)=|w|+1$.
		The mapping $w\mapsto f_w$ is injective
and its image is contained in monotone functions, hence the domain of $\tau$ 
		has fewer than $h^{t}$ elements.
\end{proof}

\begin{lemma}\label{thm:alternation-rank-type-tree}
Suppose that  $K_t\not\minor_{2} G$.
Then $\tau$ has alternation rank at most $2t-1$.
\end{lemma}
\begin{proof}
	Let $w$ be a node of $\tau$ with alternation rank at least $2k$.  
	In particular, there are vertices $a_1,b_1,\ldots,a_k,b_k$ in $A$
	such that for each $i=1,\ldots,k$, 
	the nodes in $\tau$ corresponding to $b_i,a_{i+1},b_{i+1},\ldots,a_k,b_k$ are  descendants of the son of the node which corresponds to $a_i$,
	and the nodes corresponding to $a_{i+1},b_{i+1},\ldots,a_k,b_k$
	are descendants of the daughter of the node which corresponds to $b_i$.
	
	\begin{claim}\label{claim:minor}
		For every pair $a_i,b_j$ with $1\le i\le j\le k$, there is a vertex $z_{ij}\not\in A$		which is a common neighbor of $a_i$ and $b_j$,
		and is not a neighbor of $b_s$, for $s\neq j$.
	\end{claim}
	\begin{proof}
		Note that if $i\le j$, then $d(a_i,b_j)\le 2$, however, $d(a_i,b_j)>1$ since $A$
		is independent. In particular, there is a vertex $z_{ij}$ which is a common neighbor of $a_i$ and $b_j$. 
		Suppose that $z_{ij}$ is a neighbor of $b_s$, for some $s\neq j$. This implies that $d(b_j,b_s)\le 2$, which is impossible, 
since
		 the nodes corresponding to $b_s$ and $b_j$ in $\tau$ are such that one is a descendant of the daughter of the other, implying that $d(b_s,b_j)>2$.
	\end{proof}
  


For each $j=1,\ldots,k$, define the graph $B_j$
as the subgraph of $G$ induced by the set
$\set{a_j,b_j}\cup\set{z_{ij}\mid 1\le i\le  j}$.
By the claim above we have that the graphs $B_j$
are pairwise disjoint, for $j=1,\ldots,k$.
Moreover, for $1\le i\le j\le k$, there is an edge between $B_i$
and $B_j$, namely, the edge between $z_{ij}\in B_j$
and $a_i\in B_i$.
Hence, the graphs $B_j$, for $1\le j\le l$, define a depth-$2$ minor model of $K_k$ in $G$. Since $K_t\not\minor_{2}G$, this implies that $k<t$, proving~\cref{thm:alternation-rank-type-tree}.
\end{proof}

To prove~\cref{lem:apex}, let $\ell_0$
be the number obtained from~\cref{cor:biversity}.
Fix integers $\ell\ge \ell_0$ and $m$, and define $h=m+\ell$.
Let $A$ be an independent subset of $G$
of size at least $h^{t}$.

Suppose that the first case of~\cref{lem:apex} does not hold. In particular $K_t\not\minor_2 G$, so by~\cref{thm:alternation-rank-type-tree}, $\tau$ has alternation rank at most $2t-1$. From~\cref{lem:number-of-nodes} 
we conclude that $\tau$  has height at least~$h$.
As $h=m+\ell$, it follows that either $\tau$  has a node $w$ which contains at least $m$ letters $\dau$, or $\tau$ has a node which contains  at least $\ell$ letters $\son$.

Consider the first case, i.e., there is a node $w$ of $\tau$
which contains at least $m$ letters $\dau$, and let $X$
be the set of all nodes $\tau(v)$ such that $v\dau$ is a prefix of $w$. Then, by construction, $X$ is a $2$-independent subset of $G$ of size at least $m$, so the second case of the lemma holds.

Finally, consider the second case, i.e., there is anode $w$ in $\tau$ which contains at least $\ell$ letters $\son$, and let 
$Y$ be the set of all nodes $\tau(v)$ such that $v\son$ is a prefix of $w$. Then, by construction, $Y\subset A$ is a set of vertices which are mutually at distance exactly $2$ in $G$. By~\cref{cor:biversity}, there is a vertex $v\in G$
with at least $\ell^{\alpha}$ neighbors in $Y$.
This finishes the proof of the first part of~\cref{lem:apex}.

\medskip
The proof above yields an algorithm which first constructs the tree $\tau$, by 
processing each vertex $w$ of $A$, and testing whether the distance between $w$ and each vertex processed already is equal to $2$. Whenever a node with $2t$ alternations 
is inserted, we can exhibit in $G$ a depth-$2$ minor model of $K_t$.
Whenever a node with least $m$ letters $\dau$ is added to $\tau$,
we have constructed an $m$-independent set. Whenever a node with at least $\ell$ letters $\son$ is added to $\tau$, as argued, there must be some vertex $v\in V(G)-A$ with at least $\ell^\alpha$ neighbors in $A$. To find such a vertex, scan through all neighborhoods of vertices $v\in A$ in the graph $G$, and then select a vertex $w\in V(G)$
which belongs to the largest number of those neighborhoods. The overall running time is 
$\Oof_{r,t}(|A|^c\cdot |E(G)|)$, for some constant $c$, as required.
\end{proof}

We will apply~\cref{lem:apex} in the following form.
\begin{corollary}\label{cor:apex}
	Let $t$ be an integer ant $\beta>2\cdot t$ a real. 
	There is an integer $m_0$ such that for all $m\ge m_0,k\ge 1$, the following holds.
	If $G$ is a graph such that $K_t\not\minor_2 G$,
	$A\subset V(G)$ is an independent set which does not contain a $2$-independent set of size $m$, and $|A|\ge ((k+1)\cdot m)^{\beta ^k}$
	then there is a vertex $v$ of $G$ such that $|N_G(v)\cap A| \ge (k\cdot m)^{\beta^{k-1}}$.
\end{corollary}
\begin{proof}
Let $\alpha=t/\beta$, so that $\alpha<1/2$, 
 and let 
$\ell=(k\cdot m)^{\beta^{k-1}/\alpha}$.
Take $m_0= \ell_0^{t/\beta}$; then $m\ge m_0$ implies that $\ell\ge\ell_0$.
Since
$$|A|\ge \left((k+1)\cdot m\right)^{\beta^k}=\left ((((m+ k\cdot m)^{\beta^{k-1}/\alpha)}) \right)^t
\ge (m+(k\cdot m)^{\beta^{k-1}/\alpha})^t=(m+\ell)^t,$$
we may  apply Lemma~\ref{lem:apex},
yielding a vertex $v$ with $\ell^\alpha=(k\cdot m)^{\beta^{k-1}}$ neighbors in~$A$.
\end{proof}

% \begin{proof}  [of~\cref{lem:engine}]
We now prove~\cref{lem:engine} in the case $r=2$, and for $d=2$, i.e., we show  that if $A$ is a sufficiently large $1$-independent subset of a graph $G$ such that $K_t\not\minor_2 G$, then $A$ contains a subset of size $m$
which is  $2$-independent in  $G-S$, where $S$ has at most $t$ elements. 
To this end, we iteratively apply \cref{lem:apex} as long as  it results in the third case, yielding a vertex $v$ with many neighbhors in $A$. In this case, we add $v$ vertex to the set $S$, and apply the lemma again,
restricting $A$ to $A\cap N(v)$. 
The precise calculations follow.

Let $G,t$ be such that $K_t\not\minor_2 G$, and fix some number $\beta>2\cdot t$. Let $m_0$ be the number granted by~\cref{cor:apex}. If $m<m_0$, replace $m$ by $m_0$. We will find 
a subset of $A$ of size $m$ which is $2$-independent in $G-S$, where $|S|\le t$.


Assume that $|A|\ge ((t+1)\cdot m)^{\beta^{t}}$. By induction, we
 construct a sequence  $A=A_0\supseteq A_1\supseteq\ldots$ 
of $1$-independent subsets of $G$
of length at most $t$,
such that $|A_i|\ge ((t+1-i)\cdot m)^{\beta^{t-i}}$,
 as follows. Start with $A_0=A$. We maintain a set $S$ of vertices of $G$ which is initially empty.
For $i=1,2,\ldots$,
 apply~\cref{cor:apex} to the graph $G-S$ playing the role of $G$, 
$t-i$ playing the role of $k$ and $A_{i-1}$
 playing the role of $A$.
If $A_{i-1}$ contains a $2$-independent set of size $m$ in $G-S$, terminate.
 Otherwise, there is a vertex $v_i$ of $G-S$
 whose neighborhood in $G-S$ contains at least
 $((t+1-i)\cdot m)^{\beta^{t-i}}$ vertices of $A_{i-1}$.
 Let $A_{i}$ consist of those vertices, and add $v_i$
 to the set~$S$.  
  Proceed by replacing $i$ by $i+1$.

\begin{claim}\label{claim:at-most-t}
	The construction halts after fewer than $t$ steps. 
\end{claim}
\begin{proof}
Suppose that the construction proceeds for $k\le t$ steps.
By construction, each vertex $v_i$, for $i\le k$, is connected in $G$
 to all vertices of $A_{j}$, for $i\le j\le k$. In particular, all the vertices $v_1,\ldots,v_k$ are neighbors of all the vertices of $A_{k}$
 and $|A_k|\ge t+1-k$.
We may choose pairwise distinct vertices $w_1,\ldots,w_k$,
 so that $w_i\in A_i$ and $w_i$ is a neighbor of $v_i$.
 Hence, the connected subgraphs $\set{w_i,v_i}$ of $G$ yield a depth-$1$ minor model of $K_k$ in $G$.
 Since $K_t\not\minor_2 G$, we must have $k<t$, so the process terminates before $t$ steps.
 \end{proof}
 
 Therefore, at some step $k<t$ of the construction we must have obtained a $2$-independent subset $A'$ of $G-S$ of size $m$. Moreover, $|S|\le k<t$.
 
 
 
 This proves~\cref{lem:engine} in the case $r=2$, for $d=2$, and for the function $L=L_t$, defined by $L_t(m)=((t+1)\cdot m)^{\beta^t}$
 for $m\ge m_0$ and $L(m)=L(m_0)$ for $m<m_0$, where $\beta>2\cdot t$ is any fixed constant and $m_0$ is obtained from~\cref{cor:apex}.
 The proof yields an algorithm constructing the sets $A'$ and $S$,
 which amounts to applying at most $t$ times the algorithm obtained in~\cref{lem:apex}.
 Hence, its running time  is $\Oof_{r,t}(|A|^c\cdot |E(G)|)$ as required.
% \end{proof}


\paragraph{Odd case.}
We prove~\cref{lem:engine} in the case $r=2s+1$, for some $s\in \N$. Take $d=s$.
Let $G$ be a graph such that $K_t\not\minor_s G$, and 
 let $A$ be an $2s$-independent subset of $G$. Consider the graph $G'$ obtained from $G$
by contracting the (pairwise disjoint) balls of radius $s$ around each vertex $v\in A$.
 Let $A'$ denote the set of vertices of $G'$ corresponding to the contracted balls. There is an obvious bijection between $A$ and $A'$.
From $K_t\not\minor_s G$ it follows that $G'$ does not contain $K_t$. Applying the already proved case $r=1$ to $G'$ and $A'$, we conclude that 
if $|A|=|A'|\ge {m+t-2\choose t-1}$ then
 $A'$ contains an independent subset $B$ of size $m$,
 which corresponds to a $2s+1$-independent subset of $G$.
 
 
 \paragraph{Even case.}
 Finally,
 we prove~\cref{lem:engine} in the case $r=2s+2$, for some $s\in \N$.  Take  $d=5s+2$.
Let $G$  be such that 
 $K_t\not\minor_{d} G$, and
let $A$ be an $(2s+1)$-independent subset of $G$. Consider the graph $G'$ obtained from $G$
by contracting the (pairwise disjoint) balls of radius $s$ around each vertex $v\in A$.
 Let $A'$ denote the set of vertices of $G'$ corresponding to the contracted balls. Again, there is a bijection between $A$ and $A'$. Note that
this time, $A'$ is a $1$-independent subset of $G'$.
From $K_t\not\minor_{5s+2} G$ it follows that $K_t\not\minor_2 G'$. Apply the already proved case $r=2$ to $G'$ and $A'$. Then, if $|A|=|A'|\ge L_t(m)$, where $L_t$ is the function as defined in the case $r=2$, then
 $A'$ contains a subset $B'$ of size $m$
which is  $2$-independent in $G'-S'$, for some $S'\subset V(G')-A'$.
The set $S'$ corresponds to some set of vertices $S\subset V(G)$
which are at distance at least $s+1$ from each vertex in $A$,
and the set  $B'$ corresponds to some subset $B$ of $A$
which is $(2s+2)$-independent in $G-S$. Moreover, as each vertex of $S'$
is a neighbor of each vertex of $B'$, we have that each vertex of $S$
has distance exactly $s+1=r/2$ from each vertex of $B$.

\medskip
An algorithm computing the sets $B$ and $S$ (in either the odd or even case)
simply runs a breadth-first search from each vertex of $A$ to compute the graph $G'$ with the balls of radius  $\lfloor \frac{r-1}2 \rfloor$  around the vertices in $A$ contracted to single vertices, and then runs the algorithm for the case $r=1$ or $r=2$.
This yields a running time of  $\Oof_{r,t}(|A|^c\cdot |E(G)|)$, where $c$ is some constant.
 \medskip
  
This finishes  the proof of~\cref{lem:engine}.

\begin{remark}
	It is not difficult to decrease the bound from $d=5s$ to $d= 3s+2$
	in the  case when $r=2s+2$. To do so, one would need to formulate~\cref{lem:apex}
	a bit more carefully, and replace the first possibility $K_t\minor_2 G$
	by the following condition, corresponding to the statement of~\cref{claim:minor}:
\begin{quote}
	There are vertices $a_1,b_1,\ldots,a_t,b_t\in A$ 
	such that for each $1\le i\le j\le t$, there is a vertex $z_{ij}\in V(G)-A$
	which is a neighbor of $a_i$ and $b_j$, but not of $b_s$, for $s\neq j$.	
\end{quote}
Similarly, the assumption of~\cref{thm:alternation-rank-type-tree}
should be then replaced by the negation of the above condition.
\end{remark}



\subsection*{Proof of \cref{thm:new-uqw}.}
We now wrap up the proof of \cref{thm:new-uqw} by iteratively applying~\cref{lem:engine}.

%\begin{lemma}\label{lem:ramsey2}
%Let $G$ be a graph such that $K_t\not\minor_r G$. 
%If $A$ is $2r$-independent and
%has size at least $\binom{m+t-2}{t-1}$, then there exists
%a subset $B\subseteq A$ of size at least $m$ which is a
%$(2r+1)$-independent set. 
%\end{lemma}
%\begin{proof}
%As $A$ is $2r$-independent, we can contract the $r$-neighborhood
%of each $v\in A$. The corresponding elements $N_r[v]$ in the resulting 
%depth-$r$ minor $H$ form a set $Z$ that is in \mbox{$1$-to-$1$} correspondence 
%with $A$. By assumption, 
%$H[Z]$ excludes $K_t$ as a subgraph, and hence by \cref{lem:ramsey1},
%it contains an independent set $B'\subseteq Z$ of size $m$ in $H$. 
%This set $B'$ corresponds to a $(2r+1)$-independent set $B\subseteq A$ of $G$. 
%\end{proof}
%
%\begin{lemma}\label{lem:distance-apex}
%Let $\CCC$ be a nowhere dense class of graphs. 
%Let $n_0$ be the constant of for $\epsilon=1/3$. 
%Assume $K_t\not\minor_{r+2} G$. 
%Let $m\geq n_0$ be an integer. 
%Let $A$ be a $(2r+1)$-independent set in $G$ of size at least $R(m,t)$. 
%Then there is a subset $A'\subseteq A$ of size at least~$m$ and a 
%a set $S\subseteq V(G)\setminus A$ of size at most $t-1$ such that
%\begin{enumerate}
%\item every vertex of $S$ is connected to a vertex at distance $r$ of $w$ for 
%every $w\in A'$, and
%\item $A'$ is $(2r+2)$-independent in $G-S$. 
%\end{enumerate} 
%\end{lemma}
%\begin{proof}
%As $A$ is $(2r+1)$-independent, we can contract the $r$-neighborhood
%of each $v\in A$. The corresponding elements $N_r[v]$ in the resulting 
%depth-$r$ minor $H$ form a set $Z$ that is in $1$-to-$1$ correspondence 
%with $A$. As $A$
%is $(2r+1)$-independent, $Z$ is independent in $H$. We now apply 
%\cref{lem:iterate-apex} to $Z$ in $H$. Note that the depth-$2$ minor
%we construct in \cref{thm:alternation-rank-type-tree} (now applied to $H$) 
%uses as connecting vertices $z_{ij}$ original vertices of the graph and
%not contracted vertices. Hence, when we apply the lemma, we may 
%use the assumption that $K_t\not\minor_{r+2} G$ (in general, 
%a depth-$2$ minor of a depth-$r$ minor may be a depth-$5r$ minor
%of the original graph~see Proposition~4.1 of~\cite{sparsity}). 
%Also, the vertices $v$ returned by 
%\cref{lem:iterate-apex} correspond to vertices of the graph $G$ and not
%to contracted neighborhoods. In particular, as $A$ is $(2r+1)$-independent, 
%the vertices $v$ returned by the lemma have distance exactly $r$ to 
%the vertices $w\in A$. The set $Z'$ returned by \cref{lem:iterate-apex}
%for $H$ is $2$-independent in $H$, and hence the corresponding 
%subset $B\subseteq A$ of $G$ is $(2r+2)$-independent. 
%in $G$. 
%\end{proof}

\begin{proof}[of \cref{thm:new-uqw}]
Fix integers $r,t$,  and a graph $G$ such that $K_t\not\minor_{d} G$,
for $d=\lceil 5r/2 \rceil$. Let $\beta>2\cdot t$ be a fixed real, and let $m_0$ be the number from~\cref{cor:apex}. Without loss of generality suppose $m\geq t$ and $m\ge m_0$.
 Denote $\gamma=\beta^t$, and
define a function $L_t$ by $L_t(m)=((t+1)\cdot m)^\gamma$.

Define the sequence $m_0,m_1,\ldots,m_r$ recursively so that $m_i=
(t+1)^{\gamma^{r+1-i}}\cdot m^{\gamma^{r-i}}$. Note that $m_r\ge m$
and $m_i\ge L_t(m_{i+1})$, for $i=0,\ldots,r-1$.

Suppose that $A$ is a set of vertices of $G$ such that $|A|\ge m_0=(t+1)^{\gamma^{r+1}}\cdot m^{\gamma^{r}}$. We inductively construct sequences of sets $A= A_0\supseteq A_1\supseteq \ldots \supseteq A_r$ and $\emptyset=S_0\subseteq S_1\subseteq S_2\ldots$
satisfying the following conditions:
\begin{itemize}
	\item $|A_i|\ge m_i \ge L_t(m_{i+1})$,
	\item $A_i$ is $i$-independent in $G-S_i$.
\end{itemize}
To construct $A_{i+1}$ out of $A_i$, apply~\cref{lem:engine} to the graph $G-S_i$ and 
$i$-independent subset $A_i$ of size at least $L_t(m_{i+1})$. This yields a set $S\subset V(G)$ which is disjoint from $S_i\cup A_i$, and a subset $A_{i+1}$ of $A_i$ of size 
at least $m_{i+1}$
which is $(i+1)$-independent in $G-S_{i+1}$, where $S_{i+1}=S\cup S_i$. This completes the inductive construction.

In particular, we have that $|A_r|\ge m$ and that $A_r$ is a subset of $A$ which is $r$-independent in $G-S_r$.
Observe that by construction, $|S_r|\le r\cdot t/2$, as in the odd steps, the constructed set $S$ is empty, and in the even steps, it has at most $t$ elements.  We claim that in fact we have $|S_r|<t$. The argument below is similar as in the proof of \cref{claim:at-most-t}.


By the last part of the statement of~\cref{lem:engine},  at the $i$th step of the construction, each vertex of the set $S$
has distance exactly $i/2$ from all vertices in $A_{i+1}$ in the graph 
$G-S_i$. In particular, 
each vertex of the final set $S_r$ has distance at most $r/2$ from all vertices in $A_r$
in $G-S_r$. 

For $a\in A_r$, let $N(a)$ denotes the $r$-neighborhood of $a$ in $G-S_r$.
The above remark amounts to saying that for $s\in S_r$ and $a\in A_r$, $s$ has a neighbor in $N(a)$ in $G-S_r$.

By assumption that $m\ge t$, we may choose vertices $a_1,\ldots,a_t\in A_r$.
To reach a contradiction, suppose that $S_r$ contains $t$ distinct vertices $s_1,\ldots,s_t$. 
By the above, the sets $N(a_i)\cup\set{s_i}$ 
form a minor model of $K_t$ in $G$ at depth-$(\lfloor r/2\rfloor+1)$.
This contradicts the assumption that $K_t\not\minor_d G$ for $d=\lceil 5r/2 \rceil$.
Hence, $|S|<t$.

This proves the first part of~\cref{thm:new-uqw}, where as the function $N$ we may take the following function
$$N(m)=(m(t+1))^{{(2t+1)}^{r(t+1)}},$$
and the sets $B$ and $S$ are, respectively, the sets $A_r$ and $S_r$ constructed above.

\medskip
The proof presented above can be turned into an algorithm that, 
given $G$ and $A$, outputs the sets $S$ and $B$, and amounts to
applying the algorithm given by~\cref{lem:engine} at most $r$ times.
Hence, the total running time is $\Oof_{r,t}(|A|^c\cdot |E(G)|)$ for some universal constant~$c$.
This finishes the proof~\cref{thm:new-uqw}.
\end{proof}


\section{VC dimension}\label{sec:vc}

We now come to the proof of \cref{thm:new-vc}. A set $X$ of vertices 
in a graph is \emph{shattered} if for every
subset $Y\subseteq X$ there exists 
a vertex $v$ such that $N[v]\cap X=Y$. The \emph{Vapnik-Chervonenkis dimension}, short \emph{VC-dimension}~\cite{chervonenkis1971theory} of a graph is the maximum size of a shattered set. 
The VC-dimension as a measure of complexity of set systems found has many applications, e.g.\ in learnability theory~\cite{haussler1987}, computational geometry~\cite{chazelle1989quasi},
and graph theory~\cite{alon2006dominating,BousquetT15,chepoi2007covering,eickmeyer2016neighborhood}.
We define notions of a {\em{$2$-shattered}} set and the {\em{2VC-dimension}} of a graph by restricting subsets $Y\subseteq X$ considered in the definition only to subsets of size exactly $2$.

The \emph{$r$th power of a graph $G$} is the graph $G^r$
with vertex set $V(G)$, where there is an edge between two 
vertices $u$ and $v$ if and only if their distance in $G$ is at most $r$. 

We observe that an argument of Bousquet and 
Thomass\'e~\cite{BousquetT15} can be slightly modified to prove that 
the $2$VC-dimension of the $r$-power graph $G^r$ of a graph $G$
with $K_t\not\minor_r G$ is small. Obviously, the $2$VC-dimension of $G$
bounds its VC-dimension. Hence, we in fact prove the following strengthening of \cref{thm:new-vc} stated in the introduction. 

\begin{theorem}
Let $r\in \N$ and let $G$ be a graph. 
If $K_t\not\minor_r G$, then the $2$VC-dimension of $G^r$
is at most $t-1$. 
\end{theorem}
\begin{proof}
Assume there is a set $A=\{a_1,\ldots, a_t\}$ of size $t$ such that
for all subsets $\{i,j\}\subseteq \{1,\ldots,t\}$ of size $2$ 
there is an vertex $v_{ij}$ with 
$N_r[v_{ij}]\cap A=\{a_i,a_j\}$.
For each subset $\{i,j\}\subseteq \{1,\ldots,t\}$ of size $2$, choose a vertex $u_{ij}$ so that:
\begin{enumerate}[(1)]
\item\label{p:i} $\dist(v_{ij},u_{ij})+\dist(u_{ij},a_i)\leq r$;
\item\label{p:j} $\dist(v_{ij},u_{ij})+\dist(u_{ij},a_j)\leq r$; and
\item\label{p:min} subject to conditions \eqref{p:i} and \eqref{p:j}, $\max(\dist(u_{ij},a_i),\dist(u_{ij},a_j))$ is minimized.
\end{enumerate}
Observe that $u_{ij}$ is well-defined since setting $u_{ij}=v_{ij}$ satisfies the first two conditions.

Let $P^i_{ij}$ and $P^j_{ij}$ be arbitrarily chosen shortest paths between $u_{ij}$ and~$a_i$, and between $u_{ij}$ and~$a_j$, respectively.
We now establish some basic properties of paths $P^i_{ij}$ and $P^j_{ij}$ following from the choice of $u_{ij}$.

\begin{claim}\label{cl:ineq}
For each vertex $x$ on $P^i_{ij}$ we have $\dist(v_{ij},x)+\dist(x,a_i)\leq r$, and
for each vertex $y$ on $P^j_{ij}$ we have $\dist(v_{ij},y)+\dist(y,a_j)\leq r$.
\end{claim}
\begin{clproof}
We prove only the first statement for the second is symmetric.
We have
$$\dist(v_{ij},x)+\dist(x,a_{i})\leq \dist(v_{ij},u_{ij})+\dist(u_{ij},x)+\dist(x,a_{i})=\dist(v_{ij},u_{ij})+\dist(u_{ij},a_{i})\leq r,$$
where the last equality is due to $x$ lying on a shortest path between $u_{ij}$ and $a_i$, and the last inequality is by condition~\eqref{p:i}.
\end{clproof}

\begin{claim}\label{cl:closer}
Suppose $x$ is a vertex on $P^i_{ij}$ that is different from $u_{ij}$. Then $\dist(x,a_i)<\dist(x,a_j)$.
Symmetrically, if $y$ lies on $P^j_{ij}$ and is different from $u_{ij}$, then $\dist(y,a_i)>\dist(y,a_j)$.
Consequently, paths $P^i_{ij}$ and $P^j_{ij}$ share only one vertex, being the endpoint $u_{ij}$.
\end{claim}
\begin{clproof}
We prove only the first claim, for the second is symmetric and the third directly follows from the first two.
Suppose for contradiction that $\dist(x,a_i)\geq \dist(x,a_j)$.
By \cref{cl:ineq} we have 
$$\dist(v_{ij},x)+\dist(x,a_i)\leq r.$$
On the other hand, since $\dist(x,a_i)\geq \dist(x,a_j)$, we have
$$\dist(v_{ij},x)+\dist(x,a_j)\leq\dist(v_{ij},x)+\dist(x,a_i)\leq r.$$
We conclude that $x$ satisfies conditions \eqref{p:i} and \eqref{p:j} from the definition of $u_{ij}$.
However, since $x\neq u_{ij}$ and $x$ lies on a shortest path between $u_{ij}$ and $a_i$, we have $\dist(x,a_i)<\dist(u_{ij},a_i)$.
Therefore,
$$\dist(x,a_j)\leq \dist(x,a_i)<\dist(u_{ij},a_i)\leq \max(\dist(u_{ij},a_i),\dist(u_{ij},a_j)).$$
Thus, the existence of $x$ contradicts condition \eqref{p:min} from the definition of $u_{ij}$.
\end{clproof}

Now, define paths $Q^i_{ij}$ and $Q^j_{ij}$ as follows:
\begin{itemize}
\item if $\dist(u_{ij},a_i)<\dist(u_{ij},a_j)$, then $Q^{i}_{ij}=P^{i}_{ij}$ and $Q^{j}_{ij}=P^{j}_{ij} - \{u_{ij}\}$;
\item if $\dist(u_{ij},a_i)>\dist(u_{ij},a_j)$, then $Q^{i}_{ij}=P^{i}_{ij} - \{u_{ij}\}$ and $Q^{j}_{ij}=P^{j}_{ij}$;
\item if $\dist(u_{ij},a_i)=\dist(u_{ij},a_j)$, then define $Q^i_{ij}$ and $Q^j_{ij}$ using any of the above.
\end{itemize}
Thus, by \cref{cl:closer} we have that paths $Q^{i}_{ij}$ and $Q^{j}_{ij}$ are disjoint. Moreover, for each vertex $x$ on $Q^{i}_{ij}$ we have $\dist(x,a_i)\leq \dist(x,a_j)$, and for each
vertex $y$ on $Q^{j}_{ij}$ we have $\dist(y,a_i)\geq \dist(y,a_j)$.

\begin{claim}\label{cl:intersect}
Let $\{i,j\}$ and $\{i',j'\}$ be two different subsets of size $2$ of $\{1,\ldots,t\}$.
Suppose that paths $Q^i_{ij}$ and $Q^{i'}_{i'j'}$ intersect.
Then $i=i'$.
\end{claim}
\begin{clproof}
Let $x$ be a vertex lying both on $Q^i_{ij}$ and $Q^{i'}_{i'j'}$. We first consider the corner case when $x=u_{ij}$.
Suppose first that $\dist(v_{ij},x)\geq \dist(v_{i'j'},x)$. Then by \cref{cl:ineq} we have
$$\dist(v_{i'j'},a_i)\leq \dist(v_{i'j'},x)+\dist(x,a_i)\leq \dist(v_{ij},x)+\dist(x,a_i)\leq r,$$
and analogously $\dist(v_{i'j'},a_{j})\leq r$. However, we assumed that $a_{i'}$ and $a_{j'}$ are the only vertices of~$A$ that are at distance at most $r$ from $v_{i'j'}$, hence $\{i,j\}=\{i',j'\}$,
a contradiction. Suppose then that $\dist(v_{ij},x)<\dist(v_{i'j'},x)$. 
Then we have
$$\dist(v_{ij},a_{i'})\leq \dist(v_{ij},x)+\dist(x,a_{i'})<\dist(v_{i'j'},x)+\dist(x,a_{i'})\leq r,$$
where the last equality follows from \cref{cl:ineq}.
Since $a_i$ and $a_j$ are the only vertices of $A$ that are at distance at most $r$ from $v_{ij}$, we infer that $i'\in \{i,j\}$. 
If $i'=i$ then we would be done, so suppose $i'=j$.
Since $x=u_{ij}$ and $x$ lies on $Q^i_{ij}$, by the definition of $Q^i_{ij}$ we have that $\dist(x,a_i)\leq \dist(x,a_j)=\dist(x,a_{i'})$. Therefore,
$$\dist(v_{i'j'},a_i)\leq \dist(v_{i'j'},x)+\dist(x,a_{i})\leq \dist(v_{i'j'},x)+\dist(x,a_{i'})\leq r.$$
where the last inequality follows from \cref{cl:ineq}.
Again, we assumed that $a_{i'}$ and $a_{j'}$ are the only vertices of $A$ that are at distance at most $r$ from $v_{i'j'}$, so $i\in \{i',j'\}$. If $i=i'$ then we are done, and otherwise we have $i=j'$.
Together with $i'=j$ this implies $\{i,j\}=\{i',j'\}$, a contradiction.

The second corner case when $x=u_{i'j'}$ leads to a contradiction in a symmetric manner.

We now move to the main case when $x\neq u_{ij}$ and $x\neq u_{i'j'}$.
Then by \cref{cl:closer} we have $\dist(x,a_i)<\dist(x,a_j)$ and $\dist(x,a_{i'})<\dist(x,a_{j'})$.
By symmetry, without loss of generality assume that $\dist(x,a_i)\leq \dist(x,a_{i'})$.
Observe now that
$$\dist(v_{i'j'},a_i)\leq \dist(v_{i'j'},x)+\dist(x,a_{i})\leq \dist(v_{i'j'},x)+\dist(x,a_{i'})\leq r,$$
where the last inequality follows from \cref{cl:ineq}.
Since we assumed that $a_{i'}$ and $a_{j'}$ are the only vertices of $A$ that are at distance at most $r$ from $v_{i'j'}$, we have $i\in \{i',j'\}$.
However, it cannot happen that $i=j'$, because $\dist(x,a_{i'})<\dist(x,a_{j'})$ and $\dist(x,a_{i'})\geq \dist(x,a_{i})$. We conclude that $i=i'$.
\end{clproof}

For each $i\in \{1,2,\ldots,t\}$ we define $X_i$ to be the union of vertex sets of paths $Q^i_{ij}$ for $j\neq i$.
Each of these paths has length at most $r$ and has $a_i$ as an endpoint, hence the subgraph induced by $X_i$ is connected and has radius at most $r$.
By \cref{cl:intersect}, sets $X_i$ are pairwise disjoint. Finally, observe that for each $\{i,j\}\subseteq \{1,\ldots,t\}$ with $i\neq j$, there is an edge between a vertex of $Q^{i}_{ij}$ and a vertex of $Q^{j}_{ij}$.
We conclude that $(X_i)_{i=1,\ldots,t}$ is a depth-$r$ minor model of $K_t$ in $G$, a contradiction.
\end{proof}

\begin{comment}
\begin{proof}
Assume there is a set $A=\{a_1,\ldots, a_t\}$ of size $t$ such that
for all subsets $\{a_i,a_j\}\subseteq A$ of size $2$ 
there is an element $v_{ij}\in V(G)\setminus A$ with 
$N_r[v_{ij}]\cap A=\{a_i,a_j\}$. Fix such $v_{ij}$ with the property
that $\max\left(\dist_G(v_{ij},a_i), \dist_G(v_{ij},a_j)\right)$ is 
minimized. 

A \emph{central walk} $W_{ij}$ is the concatenation of a minimum length
path $P_{ij}^i$ from $a_i$ to $v_{ij}$ and a minimum length path $P_{ij}^j$ from $v_{ij}$ to $a_j$. 
Note that a central walk is possibly not a path. For each pair $a_i,a_j$ fix
a central walk $W_{ij}$ and the corresponding paths $P_{ij}^i$ and $P_{ij}^j$. 

Now assume that a vertex $x$ is traversed by two different central 
walks $W_{ij}$, $W_{i'j'}$. By swapping indices if necessary, assume that $x$ lies on $P_{ij}^i$ and $P_{i'j'}^{i'}$. 

First, observe that if $\dist(x,a_i)=\dist(x,a_{i'})$, 
then $a_i=a_{i'}$. Indeed, if $\dist(x,a_i)=\dist(x,a_{i'})$ then $\dist(v_{ij},a_{i})=\dist(v_{ij},a_{i'})$, so $i'\in \{i,j\}$ because $a_i,a_j$ are the only vertices of $A$ at distance at most $r$ from $v_{ij}$.
Analogously $i\in \{i',j'\}$, so either $i=i'$, or $i'=j$ and $i=j'$. However, the latter case would imply $\{i,j\}=\{i',j'\}$, which contradicts the assumption that $W_{ij}$ and $W_{i'j'}$ are distinct. 

A similar argument yields that
$\dist(x,a_i)<\dist(x,a_j)$ 
and $\dist(x,a_{i'})<\dist(x,a_{j'})$. 
Now assume that $\dist(x,a_i)<\dist(x,a_{i'})$. By the same argument as 
above we have $a_{j'}=a_i$, hence $W_{i'j'}=W_{ij'}$. Here, we have
$\dist(x,a_i)<\dist(x,a_j)$ and $\dist(x,a_{i})<\dist(x,a_{i'})$, 
otherwise the walks are not distinct. 

Let us now construct connected subsets $X_i$ for all $1\leq i\leq t$. 
For every walk $W_{ij}$ the vertices of $W_{ij}$ closer to $a_i$ than to $a_j$ 
are added to $X_i$, the vertices of $W_{ij}$ closer to $a_j$ than to $a_i$ 
are added to $X_j$, ties are broken arbitrary.
Then the sets $X_i$ are pairwise disjoint by what we proved above. If a vertex $x$
appears in two distinct central walks, these are $W_{ij}$ and $W_{i\ell}$ for some
$i,j,\ell$ with $\dist(x,a_i)<\dist(x,a_j)$ and $\dist(x,a_i)<\dist(x,a_\ell)$. 
In both cases $x$ belongs to $X_i$. By construction, the sets $X_i$ are connected, 
have radius at most~$r$, and 
there is always an edge between a vertex of $X_i$ and a vertex of $X_j$ since $X_i\cup X_j$ 
contains the walk $W_{ij}$. Therefore, if the $2$VC-dimension is at least $t$, the 
graph contains $K_t$ as a depth-$r$ minor. 
\end{proof}
\end{comment}