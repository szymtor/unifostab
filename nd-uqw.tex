\section{Bounds for uniform quasi-wideness}\label{sec:uqw}

In this section we prove \cref{thm:new-uqw}. 
% In the presentation we focus on proving the existential statement, and at the end we briefly argue how the proof can be turned into an algorithm with the promised running time.
We first recall some basic notions from graph theory. 

\paragraph*{Preliminaries.}
All graphs in this paper are finite, undirected and simple, that is, 
they do not have loops or parallel edges. Our notation is standard,
we refer to~\cite{diestel2012graph} for more background on 
graph theory. 
We write $V(G)$ for the vertex set of a graph $G$ and
$E(G)$ for its edge set. 
The {\em{distance}} between vertices $u$ and $v$ in $G$, denoted $\dist_G(u,v)$, is the length of a shortest path between $u$ and $v$ in~$G$.
If there is no path between $u$ and $v$ in $G$, we put $\dist_G(u,v)=\infty$.
The {\em{(open) neighborhood}} of a vertex $u$, denoted $N(u)$, is the set of neighbors of $u$, excluding $u$ itself.
For a non-negative integer $r$, by $N_r[u]$ we denote the {\em{(closed) $r$-neighborhood}} of $u$ which comprises vertices at distance at most $r$ from $u$; note that $u$ is always contained in its closed $r$-neighborhood. The \emph{radius} of a graph $G$ is the least integer $r$ such that there is some vertex $v$ of $G$ with $N_r[v]=V(G)$.


A {\em{minor model}} of a graph $H$ in $G$ is a family $(I_u)_{u\in V(H)}$ of pairwise vertex-disjoint connected subgraphs of $G$, called {\em{branch sets}},
such that whenever $uv$ is an edge in~$H$, there are $u'\in I_u$ and $v'\in I_v$ for which $u'v'$ 
is an edge in $G$.
The graph $H$ is a {\em{depth-$r$ minor}} of $G$, denoted $H\minor_rG$, if there is a minor model
$(I_u)_{u\in V(H)}$ of~$H$ in $G$ such that each $I_u$ has radius at most $r$.

A class $\CCC$ of graphs is \emph{nowhere dense} if there is a function 
$t\colon \N\rightarrow \N$ such that for all $r\in \N$ it holds that $K_{t(r)}\not\minor_r G$
for all $G\in \CCC$, where $K_{t(r)}$ denotes the clique on $t(r)$ vertices.
The class $\CCC$ moreover has \emph{bounded expansion}
if there is a function $d\colon\N\rightarrow\N$ such that for all 
$r\in \N$ and all $H\minor_rG$ with $G\in\CCC$, the edge density
of $H$, i.e. $|E(H)|/|V(H)|$, is bounded by $d(r)$. Note that every 
class of bounded expansion is nowhere dense. The converse is not necessarily true in general~\cite{sparsity}.

% A set $B\subseteq V(G)$ is called {\em{$r$-independent}} in a graph $G$ if  $\dist_G(u,v)>r$ for all
% distinct $u,v\in B$.

 

A set $B\subseteq V(G)$ is called {\em{$r$-independent}} in a graph $G$ if  $\dist_G(u,v)>r$ for all
distinct $u,v\in B$. 
A class $\CCC$ of graphs is \emph{uniformly quasi-wide} if for every $r\in \N$ there is a number $s_r\in \N$
and a function $N_r\from\N\to\N$ such that
for all $m\in \N$, all graphs $G\in \CCC$, and all subsets $A\subseteq V(G)$ of size $\abs{A}\geq N_r(m)$, there is a set
$S\subseteq V(G)$ of size $\abs{S}\leq s_r$ and a set
$B\subseteq A-S$ of size $\abs{B}\geq m$ 
which is $r$-independent in $G-S$.



\paragraph{General strategy.}
Our proof follows the same lines as the original proof of Ne\v set\v ril and Ossona de Mendez, with the difference that in the key technical lemma (\cref{lem:apex} below), 
we improve the bounds significantly by replacing a Ramsey argument with a new combinatorial reasoning.
The new argument essentially originates in the concept of {\em{branching index}} from stability theory. 
%, 
%and also uses the almost linear bound on neighborhood complexity in nowhere dense graph classes, due to Gajarsk\'y et al.~\cite{gajarsky2017kernelization}. \marginpar{no longer}
%For sake of completeness, we present the entire proof of~\cref{thm:new-uqw}.

We first prove a restricted variant,~\cref{lem:engine} below, in which we assume that $A$ is already $(r-1)$-independent. Then, in order to derive
\cref{thm:new-uqw}, we apply the lemma iteratively for $r$ ranging from $1$ to the target value.

\begin{lemma}\label{lem:engine}
For every pair of integers $t,r\in \N$ there exists an integer $d<9r/2$ and a function $L\colon \N\to \N$ with $L(m)=\Oof_{r,t}{(m^{{(8t+1)}^{2rt}})}$ such that the following holds.
For each $m\in \N$, graph~$G$ with $K_t\not\minor_{d} G$, and
$(r-1)$-independent set $A\subseteq V(G)$ of size at least $L(m)$, there is a set $S\subseteq V(G)-A$ of size at most $t$ such that $A$ contains a subset $A'$ of size $m$ which is $r$-independent in $G-S$.
Moreover, if $r$ is odd then $S$ is empty, and if $r$ is even,
then every vertex of $S$ is at distance exactly $r/2$ from every vertex of $A'$.
Finally, given $G$ and $A$, the sets $A'$ and $S$ can be computed in time $\Oof_{r,t}(|A|\cdot |E(G)|)$.
\end{lemma}

We prove~\cref{lem:engine} in \Cref{sec:engine}, but  a very rough sketch is as follows.
The  case of general~$r$ reduces to the case $r=1$ or $r=2$, depending on the parity of $r$,
by contracting the balls of radius $\lfloor \frac {r-1} 2\rfloor $ around the vertices in $A$ to single vertices.
The case of $r=1$ follows immediately from Ramsey's theorem. 
It turns out that the case $r=2$ is substantially more difficult.
We start by formulating and proving the main technical result needed for proving the case $r=2$.







\subsection{The main technical lemma}
\label{sec:main-tech}

The following, Ramsey-like result is the main technical lemma used in the proof of~\cref{thm:new-uqw}. 

\begin{lemma}\label{lem:apex}
Let $\ell,m,t\in \N$ and assume $\ell\geq t^{8}$. 
If~$G$ is a graph and $A$ is a $1$-independent set in~$G$
with at least $(m+\ell)^{2t}$ elements,
then at least one of the following conditions hold:
\begin{itemize}
  \item $K_t\minor_{4} G$,
\item  $A$ contains a $2$-independent set of size $m$, 
\item  some vertex $v$ of $G$
has at least $\ell^{1/8}$ neighbors in $A$.
\end{itemize}
Moreover, if $K_t\not\minor_4G$, the
structures described in the other two cases (a $2$-independent set 
of size~$m$, or a vertex $v$ as above) can be 
computed in time $\Oof_t(|A|\cdot |E(G)|)$. 
\end{lemma}
We remark that a statement similar to that of \cref{lem:apex}
can be obtained by employing Ramsey's theorem, as has been done in~\cite{nevsetvril2011nowhere}. This, however, 
%yields
%in place of the bound $(m+\ell)^{2t}$ 
%a bound of the form $R(m,\underbrace{q,q,\ldots,q}_{k\text{ times}})$,
%where $k\sim\ell^{1/8}$ and $R(m_1,\ldots,m_c)$
%is the Ramsey number for $c$ colors.
%In particular, this 
does not give a bound which is polynomial in $m+\ell$, and thus cannot be used to prove~\cref{thm:new-uqw}.

\medskip
\cref{sec:main-tech} is devoted to the proof of~\cref{lem:apex}.
We will use the following bounds on the edge density
of graphs with excluded shallow minors obtained
by Alon et al.~\cite{alon2003turan}. 

\begin{lemma}[Theorem 2.2 in~\cite{alon2003turan}]\label{lem:densitynd}
Let $H$ be a bipartite graph with maximum degree
$d$ on one side. Then exists a constant $c_H$, depending 
only on $H$, such that every $n$-vertex graph $G$
excluding~$H$ as a subgraph has at most $c_H\cdot n^{2-1/d}$
edges. 
\end{lemma} 

Observe that if $K_t\not\minor_1G$, then in particular
the $1$-subdivision of $K_t$ is excluded as a subgraph
of $G$ (the $1$-subdivision of a graph $H$ is obtained by 
replacing every edge of $H$ by a path of length $2$). 
Moreover, the $1$-subdivision of 
$K_t$ is a bipartite graph with maximum degree $2$ on one
side. Furthermore, it is easy to check in the 
proof of Theorem 2.2 in~\cite{alon2003turan} 
that $c_H\leq |V(H)|$
in case $d=2$. Since the $1$-subdivision of $K_t$ has 
less than $t^2$ vertices we can choose $c_t=t^2$ and
conclude the following.   

\begin{corollary}\label{crl:densitynd}
Let $G$ be an $n$-vertex graph such that $K_t\not\minor_1 G$ for
some constant $t\in \N$. Then $G$ has at most 
$t^2\cdot n^{3/2}$ edges.
\end{corollary}

The following lemma will allow us to get a grip of 
shallow minors at depth larger than $1$.

\begin{lemma}[adaptation of Proposition 4.11 in~\cite{sparsity}]\label{lem:combineminors}
Let $a,b\in \N$ and let $c=\frac{(2a+1)(2b+1)-1}{2}$. 
Let $G$ be a graph, let $H\minor_aG$ and let 
$J\minor_bH$. Then $J\minor_cG$. 
\end{lemma}

We will need one more technical lemma.

\begin{lemma}\label{lem:diversity}
  Let $G$ be a graph such that $K_t\not\minor_4G$ for some
  $t\in\N$ and let $A\subseteq V(G)$ with $|A|\geq t^{8}$. 
  Assume furthermore that every pair of elements of $A$ has a common neighbor in $V(G)\setminus A$.
  Then there exists a vertex $v$ in $V(G)\setminus A$ which has at least $|A|^{1/8}$ neighbors in $A$.
\end{lemma}
\begin{proof}
%Our approach to prove the lemma is inspired by 
%the proof of Lemma 4.3 in~\cite{gajarsky2017kernelization}. 
We construct a depth-$1$ 
minor~$H$ of $G$ with $|A|$ vertices as follows. Recall that a 
minor model of a graph~$H$ in $G$ is a family $(I_u)_{u\in V(H)}$ of pairwise vertex-disjoint connected subgraphs of $G$
such that whenever~$uv$ is an edge in~$H$, there 
are $u'\in I_u$ and $v'\in I_v$ for which $u'v'$ 
is an edge in $G$. 

Let $G^0=G$, $V(H^0)=A$ and $I_u^0=G[\{u\}]$ for 
all $u\in V(H)$. We iteratively define a sequence 
$G^1,G^2,\ldots,G^\ell$ and $(I_u)^1_{u\in V(H)},
\ldots, (I_u)^\ell_{u\in V(H)}$ of subgraphs of $G$ 
of radius at most $1$ 
as follows. Observe
that for $0\leq i\leq \ell$, the sequence $(I_u)^i_{u\in V(H)}$
uniquely defines a depth-$1$ minor $H^i\minor_1G$. 

If $G^i$ and $(I_u)^i_{u\in V(H)}$ (with corresponding
depth-$1$ minor $H^i$) have been 
constructed, $0\leq i<\ell$, we proceed as follows. 
If there is a vertex $v\in V(G^i)\setminus A$ with 
$a,b\in A \cap N(v)$ and $ab\not\in E(H^i)$, then 
remove $v$ from $G^i$ to obtain $G^{i+1}$ 
and add $v$ to $I_a^i$ to obtain $I_a^{i+1}$ (this 
adds the edge $ab$ to $E(H^{i+1})$). 
For $u\neq a$ set $I_u^{i+1}=I_u$. We terminate
in iteration $\ell$ if no such pair $a,b$ can be found 
and let $H=H^\ell$. 

By assumption, $K_t\not\minor_4 G$, which by 
\Cref{lem:combineminors} implies $K_t\not\minor_1 H$. 
This by \Cref{crl:densitynd} implies that $H$ has at most
$t^2\cdot|A|^{3/2}$ edges. Our assumption $|A|\geq t^8$
implies $t^2\cdot |A|^{3/2}\leq |A|^{7/4}$. 

Now, let ${\cal F}=\set{N(v)\cap A\colon v\in V(G)}$. 
  Let $k$ be the maximal cardinality of a set $F$ in ${\cal F}$.
  Say that an unordered pair $\{a,b\}$ of distinct elements of $A$ is \emph{covered} by $F\in {\cal F}$
  if  $a,b\in F$. Clearly, every $F\in {\cal F}$ can cover at most $\binom{k}{2}$ pairs. When we removed some $v\in V(G^i)$ to
  introduce a new edge to $H^{i+1}$, we therefore potentially
  created $\binom{k}{2}$ uncovered pairs. On the other hand, 
  by assumption, each of the $\binom{|A|}{2}$ unordered pairs of
   distinct elements of $A$ is covered in $G^0$ by some element 
   $F\in {\cal F}$. We conclude that we have at least 
   $\binom{|A|}{2}/\binom{k}{2}$ edges in $H$. 
   This implies $k\geq |A|^{1/8}$, as otherwise
   $\binom{|A|}{2}/\binom{k}{2}> |A|^2/k^2\geq |A|^{7/4}$,
   a contradiction.
\end{proof}




%\paragraph*{Neighborhood complexity.} Let us first recall the following result of Gajarsk\'y et al.~\cite{gajarsky2017kernelization}
%which provides an upper bound on the number of distinct neighborhoods in terms of density of shallow minors. 
%
%\begin{lemma}[adaptation of Lemma 4.11 in \cite{gajarsky2017kernelization}]\label{lem:pre-diversity}
%Let $G$ be a graph such that $K_t\not\minor_{1} G$ for some constant $t\in \N$ and let $a\in \N$. 
%Let \[p\coloneqq \max
%\left\{\frac{|E(H)|}{|V(H)|} : H\minor_1 G, |V(H)|= a\right\}.\] 
%Then for all $A\subseteq V(G)$ with $|A|= a$ we have 
%\[\abs{\{N(v)\cap A \colon v\in V(G)\}}\leq(2p)^t\cdot \abs{A}.\]
%\end{lemma}
%
%Nowhere dense classes have small edge density as
%first shown by Dvo\v{r}\'ak~\cite{dvorak2007asymptotical}. 
%We apply the best currently known bounds due to 
%Jiang~\cite{jiang2011compact}. 
%
%\begin{lemma}[adaptation of Theorem 1.6 in~\cite{jiang2011compact}]\label{lem:densityjiang}
%Let $s\in\N$, $\epsilon<1/2$ and $r=\left\lfloor 10/\epsilon\right\rfloor$. There exists 
%$n_0$ such that for all $n\geq n_0$, if $G$ is an 
%$n$-vertex graph with $K_s\not\minor_rG$, 
%then $G$ has less than $n^{1+\epsilon}$ many edges. 
%\end{lemma}
%
%
%\begin{corollary}\label{lem:diversity}
%Let $G\in \CCC$ be a graph such that $K_t\not\minor_1G$ 
%and $K_s\not\minor_{90t}G$ for
%some constants $s,t\in\N$. Then 
%there exists $n_0$ such that for all $A\subseteq V(G)$ 
%with $|A|\geq n_0$ we have 
%\[\abs{\{N(v)\cap A \colon v\in V(G)\}}\leq\abs{A}^{1+1/3}.\]
%\end{corollary}
%\begin{proof}
%Let $H$ be the densest depth-$1$ minor of $G$ with 
%$|A|$ vertices. Now apply \Cref{lem:densityjiang}
%to the graph $H$ with parameters $s$, $1/(6t)$ and $r=30t$.
%Observe that according to \Cref{lem:combineminors}, 
%a depth-$30t$ minor of $H$ is a depth-$90t$ minor 
%of $G$, hence the lemma can be applied to $H$ with 
%parameter $s$. According to the lemma, there is 
%$n_0$ depending only on $s$ and $t$ such that 
%if $|A|\geq n_0$, then $H$ has at most $|A|^{1+1/(6t)}$
%many edges, hence $p\coloneqq |E(H)|/|V(H)|\leq a^{1/(6t)}$. Without loss of generality assume 
%that $n_0\geq 2^{6t}$. 
%We now apply \Cref{lem:pre-diversity} to conclude that
%\[\abs{\{N(v)\cap A \colon v\in V(G)\}}\leq(2p)^t\cdot \abs{A}\leq (2|A|^{1/(6t)})^t\cdot \abs{A}\leq n_0^{1/6}\cdot \abs{A}^{1+1/6}\leq \abs{A}^{1+1/3}.\]
%\end{proof}
%
%We remark that the proof of \cref{lem:diversity} uses only the fact that
%nowhere dense classes of graphs do not have dense 
%shallow minors~\cite{dvorak2007asymptotical,jiang2011compact}, and does not rely on any non-constructive arguments from stability theory. In particular, $n_0$ depends in a computable manner on $s$ and $t$.
%From \cref{lem:diversity} we derive the following.



\begin{comment}
\begin{corollary}\label{cor:biversity}
    For every positive real $\alpha<\frac 1 2$ and integer $t\in\N$ there exists an integer $\ell_0$ with the following property.
  Let $G$ be a graph such that $K_t\not\minor_2 G$
  and let
  $A$ be a $1$-independent subset of $G$ with $|A|\ge \ell_0$ such that every pair of vertices in $A$ has a common neighbor in $G$.
   Then there is a vertex $v\in G$
  with at least $|A|^{\alpha}$ neighbors in $A$.
\end{corollary}
\begin{proof}\label{pf:}
  Define $H$ to be the bipartite graph induced 
  by $G$, with partite sets $A$ and $V(G)-A$; that is,  $v\in A$ and $w\in V(G)-A$ are adjacent in $H$ iff they are in $G$. 
  We claim that $K_{2t^2}\not\minor_1 H$. Indeed, it is easy to see that 
  since $H$ is bipartite, $K_{2t^2}\minor_1 H$ would imply that $H$ contains the 
  complete bipartite graph $K_{t^2,t^2}$,  which, in turn, implying that $K_{t}\minor_2 G$, contrary to our assumption.  
  
  Let $\ell_0$
    be the value $n_0$ from~\cref{lem:diversity} applied to $2t^2$ in place of $t$ and $\epsilon$ such that $(1-\epsilon)/2=\alpha$. Applying~\cref{cor:diversity} to $H,2t,\epsilon,\ell_0$, we conclude that if every pair of elements of $A$ has a common neighbor in $G$
    (hence also in $H$), then there is a vertex $v$
    of $H$ with at least $|A|^{\alpha}$ neighbors in~$A$.
\end{proof}
\end{comment}

% \begin{corollary}\label{lem:gajarsky}
% Let $G$ be a graph, let $s\in \N$, and suppose there is a constant $t\in \N$ such that $K_t\not\minor_{3s+1} G$.
% Then for every $\epsilon>0$ there exists $n_0$, depending only on $\epsilon,t,s$, such that for each vertex subset $A\subseteq V(G)$ that is $(2s+1)$-independent in $G$ and satisfies $|A|\geq n_0$, it holds that
% \[\abs{\{N_{s+1}[v]\cap A \colon v\in V(G)\}}\leq\abs{A}^{1+\epsilon}.\]
% \end{corollary}
% \begin{proof}
% As $A$ is $(2s+1)$-independent, we have that the $s$-neighborhoods of vertices from $A$ are pairwise disjoint.
% Obtain an $s$-shallow minor $H$ of $G$ by contracting $N_s[u]$ for each $u\in A$.
% We implicitly identify each vertex $u\in A$ with the vertex of $H$ obtained from contracting $N_s[u]$, thus $A\subseteq V(H)$.
% It is known that every $1$-shallow minor of $H$ is also a $(3s+1)$-shallow minor of $G$ (cf.~\cite[Proposition~4.1]{sparsity}), hence $K_t\not\minor_{1} H$.
%
% Let us fix $\epsilon>0$.
% Take any $v\in V(G)$. If $v\in N_s[u]$ for some $u\in A$, then since $A$ is $(2s+1)$-independent, we have that $v$ is at distance larger than $2s+1$ from any other vertex of $A$.
% Hence in this case we have $N_{s+1}[v]\cap A=\{u\}$ and there can be at most $|A|$ neighborhoods of this type.
% Next, suppose $v\notin N_s[u]$ for any $u\in A$. Then by the construction of $H$ we have that $N_{s+1}[v]\cap A=N^H[v]\cap A$, where $N^H[v]$ is the neighborhood of $v$ in $H$.
% Since $K_t\not\minor_{1} H$, by \cref{lem:diversity} we have that the number of such neighborhoods is at most $|A|^{1+\epsilon/2}$,
% provided $|A|\geq n_0$ for some $n_0$ depending only on $t$ and $\varepsilon$.
% Thus, we conclude that $\abs{\{N_{s+1}[v]\cap A \colon v\in V(G)\}}\leq |A|+|A|^{1+\epsilon/2}$, which is bounded by $|A|^{1+\epsilon}$ if we choose $n_0$ large enough.
% \end{proof}
%
% %Observe that nowhere dense classes are closed under
% %taking bounded depth minors, as stated in the following lemma.
% %It is an immediate consequence of Proposition~4.1
% %of~\cite{sparsity}.
%
% %\begin{lemma}
% %Let $\CCC$ be a nowhere dense class of graphs and let $s\in \N$.
% %Then also the class $\{H \minor_s G \colon G\in \CCC\}$ is nowhere dense.
% %\end{lemma}





\paragraph*{Proof of \cref{lem:apex}.} We proceed with the proof of \cref{lem:apex}, which spans the  remainder of~\cref{sec:main-tech}.
We will arrange the elements of $A$ in a binary tree
and prove that the tree contains a long path. From this path, we will 
extract the set $A'$. In stability theory, similar trees are called \emph{type trees} and they are used to extract long indiscernible sequences, see e.g.~\cite{malliaris2014regularity}. 


\newcommand{\dau}{\mathrm{D}}
\newcommand{\son}{\mathrm{S}}
	
	We identify words in $\set{\dau,\son}^*$ with \emph{nodes}
	of the infinite rooted binary tree. 
  The \emph{depth} of a node $w$ is the length of $w$.
  For $w\in \set{\dau,\son}^*$,
	 the nodes $w\dau$ and $w\son$ are called, respectively, the \emph{daughter} and the \emph{son} of $w$,
	and $w$ is the \emph{parent} of both $w\son$ and $w\dau$. A node $w'$ is a {\em{descendant}} of a node $w$ if $w'$ is a prefix of $w$ (possibly $w'=w$).
	We consider
	 finite, labeled, rooted, binary trees, which are called simply trees below, and are defined as follows.
	 A \emph{tree} is a partial function $\tau\from \set{\dau,\son}^*\to U$ whose domain is a finite set of nodes, called the \emph{nodes of $\tau$}, which is closed under taking parents. If $v$ is a node of $\tau$, then $\tau(v)$ is called its \emph{label}.
  
  Let $G$ be a graph, $A\subset V(G)$ be a $1$-independent set in $G$,
  and $\bar a$ be an enumeration of $A$.
We define  a binary tree $\tau$ which is 
  labeled by vertices of $G$. The tree is defined by processing all elements of $\bar a$ sequentially. We start with $\tau$ being the  tree with empty domain, and for each element $a$ of the sequence $\bar a$, processed in the order given by $\bar a$, execute the following procedure which results in adding a node with label $a$ to $\tau$.
  
When processing the vertex $a$, do the following. Start with $w$ being the empty word. While~$w$ is a node of $\tau$, repeat the following step: 
  if the distance from $a$ to $\tau(w)$ in the graph 
  $G$ is at most~$2$, replace $w$ by its son, otherwise, replace $w$ by its daughter.
  % Repeat the step, unless $\tau(w)$ is undefined.
  Once $w$ is not a node of $\tau$, extend $\tau$ by setting  $\tau(w)=a$. In this way, we have processed the element $a$, and now
    proceed to the next element $a$ of $\bar a$, until all elements are processed. This ends the construction of $\tau$.
	

Define the
\emph{depth} of $\tau$ as 
the maximal depth of a node of $\tau$.
For a word $w$, define an \emph{alternation} to be 
a position $i$ such that $w_i\neq w_{i-1}$, where $w_0$ is assumed to be $\dau$.
 The \emph{alternation rank} of the tree $\tau$ is the maximum of the number of alternations of $w$, over all nodes $w$ of $\tau$.


\begin{lemma}\label{lem:number-of-nodes}
Let $h,t\ge 2$.	If $\tau$ has alternation rank at most $2t-1$ and depth at most $h-1$, then~$\tau$ has fewer than $h^{2t}$ nodes.
\end{lemma}
\begin{proof}		
	To each node $w$ of $\tau$ assign 
	the function $f_w\colon \set{1,\ldots,2t}\to\set{1,\ldots,h}$ defined as follows:
	$f_w$ maps each $i\in \set{1,\ldots,2t}$ to the $i$th alternation of $w$, provided $i$ is at most the number of alternations of $w$, and otherwise we put $f_w(i)=|w|+1$.
	It is clear that the mapping $w\mapsto f_w$ for nodes $w$ of $\tau$ is injective
and its image is contained in monotone functions, hence the domain of~$\tau$ 
		has fewer than $h^{2t}$ elements.
\end{proof}

\begin{lemma}\label{thm:alternation-rank-type-tree}
Suppose that  $K_t\not\minor_{2} G$.
Then $\tau$ has alternation rank at most $2t-1$.
\end{lemma}
\begin{proof}
	Let $w$ be a node of $\tau$ with at least $2k$ alternations, for some $k\in \N$.
	In particular, there are vertices $a_1,b_1,\ldots,a_k,b_k$ in $A$
	such that for each $i\in \set{1,\ldots,k}$, 
	the nodes in $\tau$ corresponding to $b_i,a_{i+1},b_{i+1},\ldots,a_k,b_k$ are  descendants of the son of the node which corresponds to $a_i$,
	and the nodes corresponding to $a_{i+1},b_{i+1},\ldots,a_k,b_k$
	are descendants of the daughter of the node which corresponds to $b_i$.
	
	\begin{claim}\label{claim:minor}
		For every pair $a_i,b_j$ with $1\le i\le j\le k$, there is a vertex $z_{ij}\not\in A$		which is a common neighbor of $a_i$ and $b_j$,
		and is not a neighbor of any $b_s$ with $s\neq j$.
	\end{claim}
	\begin{clproof}
		Note that since $i\le j$, the node of corresponding to $b_j$ is a descendant of the son of the node corresponding to $a_i$, hence we have $\dist_G(a_i,b_j)\le 2$ by the construction of $\tau$.
		However, we also have $\dist_G(a_i,b_j)>1$ since $A$
		is $1$-independent. Therefore $\dist_G(a_i,b_j)=2$, so there is a vertex $z_{ij}$ which is a common neighbor of $a_i$ and $b_j$. 
		Suppose that $z_{ij}$ was a neighbor of $b_s$, for some $s\neq j$. This would imply that $\dist_G(b_j,b_s)\le 2$, which is impossible, 
since
		 the nodes corresponding to $b_s$ and $b_j$ in $\tau$ are such that one is a descendant of the daughter of the other, implying that $\dist_G(b_s,b_j)>2$.
	\end{clproof}
  
Note that whenever $i\leq j$ and $i'\leq j'$ are such that $j\neq j'$, the vertices $z_{ij}$ and $z_{i'j'}$ are different, because $z_{ij}$ is adjacent to $b_{j}$ but not to $b_{j'}$, and the converse holds for $z_{i'j'}$.
However, it may happen that $z_{ij}=z_{i'j}$ even if $i\neq i'$. This will not affect our further reasoning.

For each $j\in\set{1,\ldots,k}$, define the graph $B_j$
as the subgraph of $G$ induced by the set
$\set{a_j,b_j}\cup\set{z_{ij}\colon 1\le i\le  j}$.
By \cref{thm:alternation-rank-type-tree} and the discussion from the previous paragraph, the graphs $B_j$ for $j\in \set{1,\ldots,k}$
are pairwise disjoint.
Moreover, for all $1\le i\le j\le k$, there is an edge between $B_i$
and $B_j$, namely, the edge between $z_{ij}\in B_j$
and $a_i\in B_i$.
Hence, the graphs $B_j$, for $j\in \set{1,\ldots,k}$, define a depth-$2$ minor model of $K_k$ in $G$. Since $K_t\not\minor_{2}G$, this implies that $k<t$, proving~\cref{thm:alternation-rank-type-tree}.
\end{proof}

We continue with the proof of~\cref{lem:apex}. 
Fix integers $\ell\ge t^8$ and~$m$, and define $h=m+\ell$.
Let $A$ be a $1$-independent set in $G$
of size at least $h^{2t}$.

Suppose that the first case of \cref{lem:apex} does not hold. In particular $K_t\not\minor_2 G$, so by \cref{thm:alternation-rank-type-tree},~$\tau$ has alternation rank at most $2t-1$. From \cref{lem:number-of-nodes} 
we conclude that $\tau$  has depth at least~$h$.
As $h=m+\ell$, it follows that either $\tau$  has a node $w$ which contains at least $m$ letters~$\dau$, or $\tau$ has a node $w$ which contains  at least $\ell$ letters $\son$.

Consider the first case, i.e., there is a node $w$ of $\tau$
which contains at least $m$ letters $\dau$, and let $X$
be the set of all vertices $\tau(u)$ such that $u\dau$ is a prefix of $w$. Then, by construction, $X$ is a $2$-independent set in $G$ of size at least $m$, so the second case of the lemma holds.

Finally, consider the second case, i.e., there is a node $w$ in $\tau$ which contains at least $\ell$ letters~$\son$. Let 
$Y$ be the set of all vertices $\tau(u)$ such that $u\son$ is a prefix of $w$. Then, by construction, $Y\subset A$ is a set of vertices which are mutually at distance exactly $2$ in $G$. 
Since $K_t\not\minor_4 G$, by~\cref{lem:diversity} we infer that there is a vertex $v\in G$
with at least $\ell^{1/8}$ neighbors in $Y$.
This finishes the proof of the first part of~\cref{lem:apex}.

The proof above yields an algorithm which first constructs the tree $\tau$, by 
iteratively processing each vertex $w$ of $A$ and testing whether the distance between $w$ and each vertex processed already is equal to $2$.
This amounts to running a breadth-first search from every vertex of $A$, which can be done in time $\Oof(|A|\cdot |E(G)|)$.
Whenever a node with $2t$ alternations 
is inserted to $\tau$, we can exhibit in $G$ a depth-$2$ minor model of $K_t$.
Whenever a node with least $m$ letters $\dau$ is added to~$\tau$,
we have constructed an $m$-independent set. Whenever a node with at least $\ell$ letters $\son$ is added to $\tau$, as argued, there must be some vertex $v\in V(G)-A$ with at least $\ell^{1/8}$ neighbors in~$A$. 
To find such a vertex, scan through all neighborhoods of vertices $v\in A$ in the graph $G$, and then select a vertex $w\in V(G)$
which belongs to the largest number of those neighborhoods; this can be done in time $\Oof(|E(G)|)$.
The overall running time is $\Oof(|A|\cdot |E(G)|)$, as required.\hfill$\square$

%This finishes the proof of~\cref{lem:apex}.

\subsection{Proof of~\cref{lem:engine}}
\label{sec:engine}
% To prove~\cref{lem:engine}, we distinguish two special cases: the case of $r=1$ and the case $r=2$. The case of general $r$ then reduces to one of these two cases, depending on the parity of $r$, by observing that a $(2s+1)$-independent set $A$ in $G$
% induces a $1$-independent set in $G$ with the balls of radius $s$ around the vertices of $A$ contracted, and,
% similarly, a $(2s+2)$-independent set $A$ in $G$
% induces a $2$-independent set in $G$ with the balls of radius $s$ around the vertices of $A$ contracted.

With \cref{lem:apex} proved, we can proceed with~\cref{lem:engine}. 
We start with the case $r=1$, then we move to the case $r=2$. 
Next we show how the general case reduces to one of those two cases.
% , and, finally, we deduce~\cref{thm:new-uqw} from~\cref{lem:engine}.

\paragraph{Case $r=1$.}
We put $d=0$, thus we assume that $K_t\not\minor_0 G$; that is, $G$ does not contain a clique of size $t$ as a subgraph. By Ramsey's Theorem, in every graph every set of size $\binom{m+t-2}{t-1}$ contains an
independent set of size $m$ or a clique of size $t$. Therefore, 
taking $L(m)$ as the above binomial coefficient yields~\cref{lem:engine} in case $r=0$, for $S=\emptyset$. Note here that $\binom{m+t-2}{t-1}\in\Oof_{t}{(m^{{(8t+1)}^{2t}})}$.
Moreover, such independent set or clique can be computed from $G$ and $A$ in time~$\Oof(|A|\cdot |E(G)|)$ by simulating the proof of Ramsey's theorem.

\paragraph{Case $r=2$.}
We put $d=2$, thus we assume that $K_t\not\minor_4 G$.
We show that if $A$ is a sufficiently large $1$-independent set in a graph $G$ such that $K_t\not\minor_4 G$, 
then there is a set of vertices~$S$ of size at most $t$ such that $A\setminus S$ contains a subset of size $m$ which is $2$-independent in $G-S$. 
Here, by ``sufficiently large'' we mean of size of size at least $L(m)$, for $L(m)$ emerging from the proof.
To this end, we shall iteratively apply \cref{lem:apex} as long as  it results in the third case, 
yielding a vertex $v$ with many neighbors in $A$. In this case, we add $v$ vertex to the set $S$, and apply the lemma again,
restricting $A$ to $A\cap N(v)$. 
The precise calculations follow.

\newcommand{\mbull}{\widehat{m}}

Fix a number $\beta>8t$. For $k\ge 0$,
define $m_k=((k+1)\cdot m)^{(2\beta)^k}$.
In the following we will always assume that $m\geq t^8$. 
We will apply~\cref{lem:apex} in the following form.
\begin{claim}\label{cor:apex}
	If $G$ is a graph such that $K_t\not\minor_4 G$, and
	$A\subset V(G)$ is an $1$-independent set in $G$ which does not contain a $2$-independent set of size $m$ and satisfies $|A|\ge m_k$,
	then there exists a vertex $v$ of $G$ such that $|N_G(v)\cap A| \ge m_{k-1}$.
\end{claim}
\begin{clproof}
Let $\ell=(k\cdot m)^{8\cdot(2\beta)^{k-1}}$.
Then $m\ge t^8$ implies that $\ell\ge t^8$.
Observe that
\[|A|\ge \left((k+1)\cdot m\right)^{(2\beta)^k}\ge\left ((m+ k\cdot m)^{8\cdot(2\beta)^{k-1}} \right)^{2t}
\ge \left(m+(k\cdot m)^{8\cdot (2\beta)^{k-1}}\right)^{2t}=(m+\ell)^{2t}.\]
Therefore, we may  apply \cref{lem:apex}, yielding a vertex $v$ with at least $\ell^{1/8}=(k\cdot m)^{(2\beta)^{k-1}}=m_{k-1}$ neighbors in~$A$.
\end{clproof}

%In the following we assume that $m\geq t^8$, since we may always ask for finding a $2$-independent set of size $t^8$ instead of $m$.
We will now find 
a subset of $A$ of size $m$ which is $2$-independent in $G-S$, where $|S|\le t$.
Assume that $|A|\ge m_t$. By induction, we
 construct a sequence  $A=A_0\supseteq A_1\supseteq\ldots$ 
of \mbox{$1$-independent} subsets of $G$
of length at most $t$
such that $|A_i|\ge m_{t-i}$,
 as follows. Start with $A_0=A$. We maintain a set $S$ of vertices of $G$ which is initially empty, and we maintain the invariant that $A_i$ is disjoint with $S$ at each step of the construction.

For $i=0,1,2,\ldots$ do as follows.
If $A_{i}$ contains a $2$-independent set of size $m$ in $G-S$, terminate.
 Otherwise, 
 apply~\cref{cor:apex} to the graph $G-S$ with $1$-independent set
 $A_{i}$ of size $|A_i|\ge m_{t-i}$. This yields a vertex $v_{i+1}$ of $G-S$
 whose neighborhood in $G-S$ contains at least
 $m_{t-i-1}$ vertices of $A_{i}$.
 Let $A_{i+1}$ consist of those vertices, and add $v_{i+1}$
 to the set~$S$.  
  Increment $i$ and repeat.

\begin{claim}\label{claim:at-most-t}
	The construction halts after less than $t$ steps.
\end{claim}
\begin{clproof}
Suppose that the construction proceeds for $k\le t$ steps.
By construction, each vertex~$v_i$, for $i\le k$, is adjacent in $G$
 to all the vertices of $A_{j}$, for each $i\le j\le k$. In particular, all the vertices $v_1,\ldots,v_k$ are adjacent to all the vertices of $A_{k}$
 and $|A_k|\ge m_{t-k}\ge m\ge t$.
Choose any pairwise distinct vertices $w_1,\ldots,w_k\in A_k$ and observe that the connected subgraphs $\set{w_i,v_i}$ of~$G$ yield a depth-$1$ minor model of $K_k$ in $G$.
 Since $K_t\not\minor_2 G$, we must have $k<t$.
 \end{clproof}
 
 Therefore, at some step $k<t$ of the construction we must have obtained a $2$-independent subset $A'$ of $G-S$ of size $m$. Moreover, $|S|\le k<t$.
 
 
 
 This proves~\cref{lem:engine} in the case $r=2$, for the function $L(m)=m_t=((t+1)\cdot m)^{\beta^{2t}}$
 for $m\ge t^8$ and $L(m)=L(t^8)$ for $m<t^8$, where $\beta>8t$ is any fixed constant.
 It is easy to see that then $L_t(m)=L(m)\in \Oof_{t}{(m^{{(8t+1)}^{2t}})}$, provided we put $\beta=8t+1$.
 Also, the proof easily yields an algorithm constructing the sets~$A'$ and $S$,
 which amounts to applying at most $t$ times the algorithm of~\cref{lem:apex}.
 Hence, its running time  is $\Oof_{r,t}(|A|\cdot |E(G)|)$, as required.
% \end{proof}


\paragraph{Odd case.}
We prove~\cref{lem:engine} in the case when $r=2s+1$, for some integer $s\geq 1$. We put $d=s=\frac{r-1}{2}$.
Let $G$ be a graph such that $K_t\not\minor_s G$, and 
 let $A$ be a $2s$-independent set in $G$. Consider the graph $G'$ obtained from $G$
by contracting the (pairwise disjoint) balls of radius $s$ around each vertex $v\in A$.
 Let $A'$ denote the set of vertices of $G'$ corresponding to the contracted balls. There is a natural bijection between $A$ and $A'$.
From $K_t\not\minor_s G$ it follows that~$G'$ does not contain $K_t$ as a subgraph. Applying the already proved case $r=1$ to $G'$ and $A'$, we conclude that 
if $|A|=|A'|\ge {m+t-2\choose t-1}$, then
 $A'$ contains a $1$-independent subset $B$ of size $m$,
 which corresponds to a $(2s+1)$-independent set in $G$ that is contained in $A$.
 Hence, the obtained bound is $L(m)={m+t-2\choose t-1}$, and we have already argued that then $L(m)\in \Oof_{r,t}{(m^{{(8t+1)}^{2t}})}$.
 
 
 \paragraph{Even case.}
 Finally,
 we prove~\cref{lem:engine} in the case $r=2s+2$, for some integer $s\geq 1$. We put $d=9s+4=9r/2-5$.
Let $G$  be such that 
 $K_t\not\minor_{d} G$, and
let $A$ be a $(2s+1)$-independent set in~$G$. Consider the graph $G'$ obtained from $G$
by contracting the (pairwise disjoint) balls of radius $s$ around each vertex $v\in A$.
 Let $A'$ denote the set of vertices of $G'$ corresponding to the contracted balls. Again, there is a bijection between $A$ and $A'$. Note that
this time, $A'$ is a $1$-independent set in $G'$.
From $K_t\not\minor_{9s+4} G$ it follows by \Cref{lem:combineminors} that $K_t\not\minor_4 G'$. Apply the already proved case $r=2$ to $G'$ and $A'$. 
Then, if $|A|=|A'|\ge L_t(m)$, where $L_t(m)$ is the function as defined in the case $r=2$, then
 $A'$ contains a subset $B'$ of size $m$
which is  $2$-independent in $G'-S'$, for some $S'\subset V(G')-A'$.
The set $S'$ corresponds to some set of vertices $S\subset V(G)$
which are at distance at least $s+1$ from each vertex in $A$,
and the set  $B'$ corresponds to some subset $B$ of $A$
which is $(2s+2)$-independent in $G-S$. Moreover, as each vertex of $S'$
is a neighbor of each vertex of $B'$,  each vertex of $S$
has distance exactly $s+1=r/2$ from each vertex of $B$.

\medskip
An algorithm computing the sets $B$ and $S$ (in either the odd or even case) can be given as follows:
simply run a breadth-first search from each vertex of $A$ to compute the graph $G'$ with the balls of radius  $\lfloor \frac{r-1}2 \rfloor$  around the vertices in $A$ contracted to single vertices, 
and then run the algorithm for the case $r=1$ or $r=2$.
This yields a running time of  $\Oof_{r,t}(|A|\cdot |E(G)|)$.
 \medskip
  
This finishes  the proof of~\cref{lem:engine}.

%\begin{remark}
%A more careful analysis of the case $r=2s+2$, in particular of the construction of $G'$ and the properties of $G'$ actually used in the proof of \cref{claim:minor},
%shows that the bound of $d=5s+2$ may be decreased to $d=3s+2$. We refrain from giving the details for the sake of clarity of the argument.
%\end{remark}

%\begin{comment}
%\begin{remark}
%	It is not difficult to decrease the bound from $d=5s+2$ to $d= 3s+2$
%	in the  case when $r=2s+2$. To do so, one would need to formulate~\cref{lem:apex}
%	a bit more carefully, and replace the first possibility $K_t\minor_2 G$
%	by the following condition, corresponding to the statement of~\cref{claim:minor}:
%\begin{quote}
%	There are vertices $a_1,b_1,\ldots,a_t,b_t\in A$ 
%	such that for each $1\le i\le j\le t$, there is a vertex $z_{ij}\in V(G)-A$
%	which is a neighbor of $a_i$ and $b_j$, but not of $b_s$, for $s\neq j$.	
%\end{quote}
%\end{remark}
%\end{comment}


\subsection{Proof of \cref{thm:new-uqw}}
We now wrap up the proof of \cref{thm:new-uqw} by iteratively applying~\cref{lem:engine}.

%\begin{lemma}\label{lem:ramsey2}
%Let $G$ be a graph such that $K_t\not\minor_r G$. 
%If $A$ is $2r$-independent and
%has size at least $\binom{m+t-2}{t-1}$, then there exists
%a subset $B\subseteq A$ of size at least $m$ which is a
%$(2r+1)$-independent set. 
%\end{lemma}
%\begin{proof}
%As $A$ is $2r$-independent, we can contract the $r$-neighborhood
%of each $v\in A$. The corresponding elements $N_r[v]$ in the resulting 
%depth-$r$ minor $H$ form a set $Z$ that is in \mbox{$1$-to-$1$} correspondence 
%with $A$. By assumption, 
%$H[Z]$ excludes $K_t$ as a subgraph, and hence by \cref{lem:ramsey1},
%it contains an independent set $B'\subseteq Z$ of size $m$ in $H$. 
%This set $B'$ corresponds to a $(2r+1)$-independent set $B\subseteq A$ of $G$. 
%\end{proof}
%
%\begin{lemma}\label{lem:distance-apex}
%Let $\CCC$ be a nowhere dense class of graphs. 
%Let $n_0$ be the constant of for $\epsilon=1/3$. 
%Assume $K_t\not\minor_{r+2} G$. 
%Let $m\geq n_0$ be an integer. 
%Let $A$ be a $(2r+1)$-independent set in $G$ of size at least $R(m,t)$. 
%Then there is a subset $A'\subseteq A$ of size at least~$m$ and a 
%a set $S\subseteq V(G)\setminus A$ of size at most $t-1$ such that
%\begin{enumerate}
%\item every vertex of $S$ is connected to a vertex at distance $r$ of $w$ for 
%every $w\in A'$, and
%\item $A'$ is $(2r+2)$-independent in $G-S$. 
%\end{enumerate} 
%\end{lemma}
%\begin{proof}
%As $A$ is $(2r+1)$-independent, we can contract the $r$-neighborhood
%of each $v\in A$. The corresponding elements $N_r[v]$ in the resulting 
%depth-$r$ minor $H$ form a set $Z$ that is in $1$-to-$1$ correspondence 
%with $A$. As $A$
%is $(2r+1)$-independent, $Z$ is independent in $H$. We now apply 
%\cref{lem:iterate-apex} to $Z$ in $H$. Note that the depth-$2$ minor
%we construct in \cref{thm:alternation-rank-type-tree} (now applied to $H$) 
%uses as connecting vertices $z_{ij}$ original vertices of the graph and
%not contracted vertices. Hence, when we apply the lemma, we may 
%use the assumption that $K_t\not\minor_{r+2} G$ (in general, 
%a depth-$2$ minor of a depth-$r$ minor may be a depth-$5r$ minor
%of the original graph~see Proposition~4.1 of~\cite{sparsity}). 
%Also, the vertices $v$ returned by 
%\cref{lem:iterate-apex} correspond to vertices of the graph $G$ and not
%to contracted neighborhoods. In particular, as $A$ is $(2r+1)$-independent, 
%the vertices $v$ returned by the lemma have distance exactly $r$ to 
%the vertices $w\in A$. The set $Z'$ returned by \cref{lem:iterate-apex}
%for $H$ is $2$-independent in $H$, and hence the corresponding 
%subset $B\subseteq A$ of $G$ is $(2r+2)$-independent. 
%in $G$. 
%\end{proof}

\begin{proof}[of \cref{thm:new-uqw}]
Fix integers $r,t$,  and a graph $G$ such that $K_t\not\minor_{d} G$,
for $d=\lfloor 9r/2 \rfloor$. Let $\beta>8t$ be a fixed real. As in the proof of \cref{lem:engine}, without loss of generality suppose $m\geq t^8$.
 Denote $\gamma=\beta^{2t}$, and
define the function $L(m)$ as $L(m)=((t+1)\cdot m)^\gamma$.

Define the sequence $m_0,m_1,\ldots,m_r$ as follows:
$$m_i=(t+1)^{\gamma^{r-i}}\cdot m^{\gamma^{r-i}}.$$ 
Note that $m_r\geq m$ and $m_i=L(m_{i+1})$, for all $i\in \set{0,\ldots,r-1}$.

Suppose that $A$ is a set of vertices of $G$ such that $|A|\ge m_0=(t+1)^{\gamma^{r}}\cdot m^{\gamma^{r}}$. 
We inductively construct sequences of sets $A= A_0\supseteq A_1\supseteq \ldots \supseteq A_r$ and $\emptyset=S_0\subseteq S_1\subseteq S_2\ldots$
satisfying the following conditions:
\begin{itemize}
	\item $|A_i|\ge m_i \ge L(m_{i+1})$,
	\item $A_i\cap S_i=\emptyset$ and $A_i$ is $i$-independent in $G-S_i$.
\end{itemize}
To construct $A_{i+1}$ out of $A_i$, apply~\cref{lem:engine} to the graph $G-S_i$ and 
the $i$-independent set $A_i$ of size at least $L(m_{i+1})$. This yields a set $S\subseteq V(G)$ which is disjoint from $S_i\cup A_i$, and a subset $A_{i+1}$ of $A_i$ of size 
at least $m_{i+1}$
which is $(i+1)$-independent in $G-S_{i+1}$, where $S_{i+1}=S\cup S_i$. This completes the inductive construction.

In particular,  $|A_r|\ge m$ and $A_r$ is a subset of $A$ which is $r$-independent in $G-S_r$.
Observe that by construction, $|S_r|\le r t/2$, as in the odd steps, the constructed set $S$ is empty, and in the even steps, it has at most $t$ elements. 
We show that in fact we have $|S_r|<t$ using the following argument, similar to the one used in~\cref{claim:at-most-t}.


By the last part of the statement of~\cref{lem:engine},  at the $i$th step of the construction, each vertex of the set $S$
has distance exactly $i/2$ from all vertices in $A_{i+1}$ in the graph 
$G-S_i$. 
For $a\in A_r$, let $\overline{N}(a)$ denote the $\lfloor r/2\rfloor$-neighborhood of $a$ in $G-S_r$; note that sets $\overline{N}(a)$ are pairwise disjoint.
The above remark implies that each vertex $v$ of the final set $S_r$ has a neighbor in the set $\overline{N}(a)$ for each $a\in A_r$.
Indeed, suppose $v$ belonged to the set $S$ added to $S_r$ in the $i$th step of the construction; i.e. $v\in S_{i+1}\setminus S_i$.
Then there exists a path in $G-S_i$ from $v$ to $a$ of length exactly~$i/2$, which traverses only vertices at distance less than $i/2$ from $a$.
Since in this and further steps of the construction we were removing only vertices at distance at least $i/2$ from $a$, this path stays intact in $G-S_r$ and hence is completely contained in $\overline{N}(a)$.

By assumption that $m\ge t$, we may choose pairwise different vertices $a_1,\ldots,a_t\in A_r$.
To reach a contradiction, suppose that $S_r$ contains $t$ distinct vertices $s_1,\ldots,s_t$. 
By the above, the sets $\overline{N}(a_i)\cup\set{s_i}$ 
form a minor model of $K_t$ in $G$ at depth-$(\lfloor r/2\rfloor+1)$.
This contradicts the assumption that $K_t\not\minor_d G$ for $d=\lfloor 9r/2 \rfloor$.
Hence, $|S|<t$.


Define the function  $N:\N\to\N$
as $N(m)=((t+1)m)^{\gamma^{r}}=((t+1)m)^{\beta^{2rt}}$
for $m\ge t^8$  and $N(m)=N(t^8)$ for $m<t^8$.
Note that $N(m)\in \Oof_{r,t}{(m^{{(8t+1)}^{2rt}})}$, provided we put $\beta=8t+1$.
The argument above shows that if $|A|\ge N(m)$ then 
there is a set $S\subset V(G)$, equal to $S_r$ above,
and a set $B\subset A$, equal to $A_r$ above,
so that $B$ is $r$-independent in $G-S$.
Given~$G$ and $A$, the sets~$S$ and $B$ can be computed by applying the algorithm given by~\cref{lem:engine} at most~$r$ times, so in time $\Oof_{r,t}(|A|\cdot |E(G)|)$.
This finishes the proof of~\cref{thm:new-uqw}.
\end{proof}



\section{Uniform quasi-widness for tuples}\label{sec:uqw-tuples}
We now formulate and prove an extension of~\cref{thm:new-uqw}, namely \cref{thm:uqw-tuples},
which applies to sets of tuples of vertices, rather than sets of vertices. This more general result will be used later on in the paper. 
The result and its proof are essentially adaptations to the finite of their infinite analogues introduced by Podewski and Ziegler (cf.~\cite{podewski1978stable},  Corollary 3),
modulo the numerical bounds.








We generalize the notion of independence to sets of tuples of vertices.
Fix a graph $G$ and a number $r\in \N$.
Say that two tuples $\bar u,\bar v$ of vertices of $G$ are \emph{$r$-separated} by $S\subset V(G)$ if every path from a vertex in $\bar u$ to a vertex in $\bar v$ of length at most $r$ passes through $S$.

If $S\subset V(G)$ is a set of vertices and $A\subset V(G)^d$ is a set of $d$-tuples of vertices, for some $d\in\N$,
then we say that $A$ is \emph{mutually $r$-independent} in $G-S$ 
if any two distinct $\bar u,\bar v\in A$
are $r$-separated by $S$ in $G$.
%  Throughout this section we use the convention that if $x\in S$, then the distance in $G-S$ between $x$ and any vertex, including~$x$ itself, is infinity.
% For instance, in the definition above, the tuples from $A$ may contain vertices of~$S$, and such a vertex is considered infinitely far from every vertex.

\medskip
\newcommand{\uqw}{\mathrm{UQW}}
\newcommand{\puqw}{\mathrm{PUQW}}
Fix a class $\cal C$ and numbers $r,d\in\N$.
For a function $N\from\N\to\N$
and number $s\in\N$,
we say that $\cal C$ satisfies the property
$\uqw^d_r(N,s)$ if the following condition holds:
   \begin{quote}\itshape 
      for all $m\in \N$ and all subsets 
     $A\subset V(G)^d$ with $|A|\ge N(m)$ there is a set $S\subset V(G)$ of size $|S|\le s$ and a subset $B\subset A$ of size $|B|\ge m$ which is mutually $r$-independent in $G-S$.
   \end{quote}   
    We say that $\cal C$ satisfies the property $\uqw^d_r$ if  $\cal C$ satisfies $\uqw^d_r(N,s)$ for 
	some $N\from\N\to\N$ and $s\in\N$.
	If moreover $N$ can be taken polynomial,
	then we say that $\cal C$ satisfies the property $\puqw^d_r$.
	When $d=1$, we omit it from the superscripts.

  Note that there is a slight discrepancy 
  in the definition of uniform quasi-wideness 
  and the property of satisfying $\uqw_r$, for all $r\in \N$,
  which is due to the fact that in the original definition,
  the set $B$ must be disjoint from $S$,
  whereas in the property $\uqw_r$, 
  some vertices of $S$ may belong to $B$. This distinction is inessential when it comes to dimension $1$, since $|S|\le s_r$ for some constant $s_r$,
  so passing from one definition to the other requires 
  modifying the function $N_r$ by an additive constant $s_r$.
In particular, a class of graphs $\cal C$ is uniformly quasi-wide if and only if it 
	satisfies $\uqw_r$, for all $r\in \N$.  
  However, generalizing to tuples of dimension $d$ requires the use of the definition above, where the tuples in $B$ are allowed to contain  vertices which occur $S$. For example, if the graph $G$ is a star with many arms and $A$ consists of all pairs of adjacent vertices in $G$, then $S$
  needs to contain the central vertex of $G$,
  and therefore $S$ will contain a vertex from every tuple in $A$. We may take $B$ to be equal to $A$ in this case.
  
\medskip
	Using the above terminology, \cref{thm:new-uqw}
	states that for every fixed $r\in\N$, if there is a number $t\in\N$
	such that $K_t\not\minor_{\lfloor 9r/2\rfloor} G$ for all $G\in \cal C$,
	then $\cal C$ satisfies $\puqw_r$,
	and more precisely $\uqw_r(N_r,s_r)$
	for a polynomial $N_r\from\N\to\N$ and number $s_r\in \N$, where $N_r$ and $s_r$ can be computed from $r$ and $t$.
	The following result provides a generalization to higher dimensions $d>1$.
	
	% and 	\cref{thm:}
	
	
	
 %
%
%
%
% Let $\uqw^d_r$ denote the following statement concerning a class of graphs $\cal C$:
%    \begin{quote}\itshape There exists a constant $s^d_r\in \N$ and a polynomial $N^d_r\from\N\to\N$ such that
%       for all $m\in \N$ and all subsets
%      $A\subset V(G)^d$ with $|A|\ge N^d_r(m)$ there is a set $S\subset V(G)$ of size $|S|\le s^d_r$ and a subset $B\subset A$ of size $|B|\ge m$ which is mutually $r$-independent in $G-S$.
%    \end{quote}
%
% In the case $d=1$ we write simply $\uqw_r$
% instead of $\uqw^1_r$, and
% $s^1_r$ and $N^1_r$ are denoted $s_r$ and $N_r$, respectively.
%
% Therefore, \cref{thm:new-uqw}
% states that if there is a number $t\in\N$
% such that $K_t\not\minor_{\lfloor 9r/2\rfloor} G$ for all $G\in \cal C$,
% then $\cal C$ satisfies $\uqw_r$.
% Moreover, the proof is effective,
% meaning that  $s_r$ and $N_r$
% can be computed, given $t$.
% The following result provides a generalization to $\uqw^d_r$, for $d>1$.

\begin{theorem}\label{thm:uqw-tuples}If $\cal C$
	is a nowhere dense class of graphs,
	then for all $r,d\in\N$,
	the class $\cal C$ satisfies
	 $\puqw^d_r$.
	More precisely, for any class of graphs $\cal C$ and numbers $r,t\in\N$,
	if  	$K_t\not\minor_{18r} G$ for all $G\in \cal C$,
then for all $d\in \N$ the class $\cal C$ satisfies $\uqw^d_{r}(N^d_r,s^d_r,)$	for 
some number $s^d_r\in \N$ and polynomial $N^d_r\from \N\to\N$ that can be computed given $r,t$ and $d$.
\end{theorem}


\cref{thm:uqw-tuples} is an immediate consequence  of~\cref{thm:new-uqw} and of the following result.

\begin{proposition}\label{prop:uqw-tuples}
% 	Let $\cal C$ be a uniformly quasi-wide class of graphs and $d\in\N$ be an integer.
% 	Then, for every $d,r\in\N$,
% 	$\cal C$ satisfies
% 	$\puqw^d_r$.
% More precisely,
For all $r,d\in\N$,
if $\cal C$ satisfies $\uqw_{2r}(N_{2r},s_{2r})$ 
for some $s_{2r}\in\N$ and $N_{2r}\in \N$,
then $\cal C$
satisfies $\uqw^d_r(N^d_r,s^d_r)$
for  $s^d_r=d\cdot s_{2r}$ and 
a function $N^d_r$ defined by  $N^d_r=T_d(N_{2r},s_{2r})$, for some 
term operation $T_d$ using $+,\times, N_{2r}, s_{2r}$, which is computable from $d$.
	% Then there are functions $N^d\colon \N\times\N\to\N$ and $s^d\colon \N\to\N$
	% such that for all $r,m\in\N$ and all subsets $A\subset V(G)^d$
	% with $|A|\ge N^d(r,m)$ there  is a set $S\subset V(G)$
	% of size $|S|\le s^d(r)$ and a subset $B\subset A$ of size $|B|\ge m$ which is mutually $r$-independent in $G-S$. Moreover, $N^d$ is polynomial in the second argument and $s^d$ is polynomial.
\end{proposition}
% Combining with~\cref{thm:new-uqw} yields the following.


The rest of this section is devoted to the proof of \cref{prop:uqw-tuples}.
Fix a class $\cal C$
such that $\uqw_{2r}(N_{2r},s_{2r})$ holds for some number $s_{2r}\in \N$ and  function $N_{2r}\from \N\to \N$.
In the proof below, we will invoke the property $\uqw_{2r}$,
but also the property $\uqw_r$, which also holds for $\cal C$, as witnessed e.g. by the function $N_r\coloneqq N_{2r}$ and constant $s_r\coloneqq s_{2r}$. 

\medskip

	Let $d\in \N$ be a fixed dimension.
		For a fixed graph $G\in \cal C$  and
	  coordinate $i\in\set{1,\ldots,d}$, let $\pi_i$ denote the projection from $V(G)^d$ onto the $i$th coordinate.

	

\begin{lemma}\label{lem:step1} For any $r,m\in \N$ there is an integer $K_r(m)$ such that
	for any given $A\subset V(G)^d$ with $|A|\ge K_r(m)$,
	there is a set $B\subset A$ with $|B|\ge m$ and a set $S\subset V(G)$ with $|S|\le d\cdot s_r$, 
	such that for each coordinate $i\in\set{1,\ldots,d}$ and all distinct $\bar x,\bar y\in B$,
 $\pi_i(\bar x)$ and $\pi_i(\bar y)$ are $r$-separated by $S$. 
\end{lemma}
\begin{proof}
We will iteratively apply the following claim.

%Let $f\colon \N\to \N$ be defined as $f(m)=N(r,m)\cdot m$ for $m\in\N$.

\begin{claim}\label{claim:ith-coord}
Fix a coordinate $i\in\set{1,\ldots,d}$, an integer $m'\in\N$, and a  set $A'\subset V(G)^d$ with  $|A'|\ge N_r(m')\cdot m'$.
Then there is a set $B'\subset A'$ with $|B'|\ge m'$
and a set $S'\subset V(G)$ with $|S'|\le  s_r$, such that for all distinct $\bar x,\bar y\in B$,
 $\pi_i(\bar x)$ and $\pi_i(\bar y)$ are $r$-separated by $S$.
 %\marginpar{observe that we prove something stronger here: if we choose elements of $S$, then these are equal on each coordinate and this will be needed in the later lemma.}
\end{claim}
\begin{clproof}
We consider two cases, depending on whether $|\pi_i(A')|\ge N_r(m')$.

\smallskip
If $\pi_i(A')\subset V(G)$ has at least $N_r(m')$ elements, then we apply the property $\uqw_r$  to the set  $\pi_i(A')\subset V(G)$. This yields sets $S'\subset V(G)$ and $B''\subset \pi_i(A')$
such that $|B''|\ge m'$, $|S'|\le s_r$, and $B''$ is $r$-independent in $G-S'$. 
Let $B'\subseteq A'$ be a subset of tuples constructed as follows: for each $u\in B''$, include in $B'$ one arbitrarily chosen tuple $\bar x\in A'$ such that $\pi_i(\bar x)=u$.
Clearly $|B'|=|B''|\ge m'$ and for all distinct $\bar x,\bar y\in B'$, we have that $\pi_i(\bar x)$ and $\pi_i(\bar y)$ are distinct and $r$-separated by $S$; this is because $B''$ is $r$-independent
in $G-S$. Hence $B'$ and $S'$ satisfy all the required properties.

If $\pi_i(A')$ has less than $N_r(m')$ elements, then choose the element $a\in\pi_i(A')$ whose inverse image $\pi_i^{-1}(\set a)\cap A'$ has the largest cardinality. Let $S'=\set{a}$ 
and let $B'=\pi_i^{-1}(\set a)$. Then $$|B'|\ge \frac{|A'|}{|\pi_i(A')|}\ge \frac{|A'|}{N_r(m')}\ge \frac {N_r(m')\cdot m'}{N_r(m')}=m',$$
and $|S'|=1$. 
% Observe that $\pi_i(\bar x)=a$ for all $\bar x\in A'$. As $a\in S$, by the adopted convention, $\pi_i(\bar x)$ and $\pi_i(\bar y)$ are at infinite distance for all distinct $\bar x,\bar y\in B$.
\end{clproof}

We proceed with the proof of \cref{lem:step1}.
Let $f(m')=N_r(m')\cdot m'$ for $m'\in\N$; by $f^k$ we denote the $k$-fold composition of $f$ with itself.
Let $A\subset V(G)^d$ be such that $|A|\ge f^d(m)$. 
Define $B_0=A$, $S_0=\emptyset$, and for $i=1,\ldots,d$,
let $B_{i}$ and $S_i$ be the $B'$ and $S'$ obtained from \cref{claim:ith-coord} applied to the set of tuples $B_{i-1}\subset V(G)^d$, the coordinate $i$, and $m'=f^{d-i}(m)$. 
The invariant is that $|B_i|\ge f^{d-i}(m)$.
In particular, 
taking $B=B_d$ and $S=S_1\cup\ldots \cup S_d$, we obtain that $|B|\ge m$ and $|S|\le d\cdot s_r$, and, by construction, $\pi_i(B)$
is $r$-independent in $G-S$ for every coordinate $i\in\set{1,\ldots,d}$. Letting $K_r(m)=f^d(m)$ yields the lemma.
\end{proof}


\begin{lemma}\label{lem:step2}
	Let $B\subset V(G)^d$ and $S\subset V(G)$ be such that 
  %\begin{itemize}
	%\item
   for all $i\in \set{1,\ldots,d}$,
	$\pi_i(B)$ is $2r$-independent in $G-S$.
  % \item for each tuple $\bar a\in B$, the set of entries of $\bar a$ is $r$-independent in $G-S$.
%	\end{itemize}
	Then there is a set $C$ with $C\subset B$ 
	such that $C$ is mutually $r$-independent in $G-S$
	and $|C|\geq\frac{|B|}{d^2+1}$.
\end{lemma}
\begin{proof}
We construct a sequence $C_0\subset C_1\subset \ldots$ of subsets of $B$ which are mutually $r$-independent in $G-S$, as follows.

We start with $C_0=\emptyset$. Suppose that $C_s\subset B$ is 
 already constructed for some $s\ge 0$
 and is mutually $r$-independent in $G-S$; we construct $C_{s+1}$. With each element $a\in B-C_s$,
we associate an arbitrarily chosen function $f_a\colon \set{1,\ldots,d}^2\to C_s\cup \set{\bot}$
with the following properties:
\begin{itemize}
	\item If $f_a(i,j)=b$ then the $i$th coordinate of $a$
	and the $j$th coordinate of $b$ are not $r$-separated by $S$.
	\item If $f_a(i,j)=\bot$ then there is no element $b\in C_s$ 
	such that the $i$th coordinate of $a$ and the $j$th coordinate of $b$ are at not $r$-separated by $S$.
\end{itemize}
Observe that whenever $a_1, a_2$ are two distinct elements of $B-C_s$,
then for all $i,j\in \set{1,\ldots,d}^2$, the values $f_{a_1}(i,j)$ and $f_{a_2}(i,j)$
cannot be equal to the same element $b\in C_s$:
otherwise, we would have that the $i$th coordinate of $a_1$
and the $i$th coordinate of $a_2$ are not $2r$-separated by $S$, which is impossible by the assumption on $B$.
In particular, if $|B-C_s|> |C_s|\cdot d^2$
then there must be some element  $a\in B-C_s$  
such that $f_a(i,j)=\bot$  for all $i,j\in\set{1,\ldots,d}$.
Let $C_{s+1}=C_s\cup \set a$.
By construction, $C_{s+1}$ is mutually $r$-independent in $G-S$.

We may repeat the construction as long as $|B|>|C_s|\cdot (d^2+1)=s\cdot (d^2+1)$, and we stop when this inequality no longer holds. Define the set $C$ as the last constructed set $C_s$.
By construction, $|C_s|=s\ge 
\frac{|B|}{d^2+1}$.	
\end{proof}

To finish the proof of \cref{prop:uqw-tuples},
given a set $A\subset V(G)^d$ and integers $r,m\in\N$,
first apply 
\cref{lem:step1} 
  with $r'=2r$ and
 $m'= m\cdot (d^2+1)$.
 Assuming that $|A|\ge K_{r'}(m')$, 
we obtain a set $B\subset A$ with $|B|\ge m\cdot (d^2+1)$ and a set $S\subset V(G)$ with $|S|\le s_{2r}$.
Apply \cref{lem:step2} to $B$ and $S$, yielding a set $C\subset B$ which is mutually $r$-independent in $G-S$ and has size at least $m$. This concludes the proof of \cref{prop:uqw-tuples},
where the obtained bounds are $N^d_r(m)=K_{r'}(m')=K_{2r}(m\cdot (d^2+1))$ and $s^d_r=d\cdot s_{2r}$.

\begin{comment}
\begin{remark}\label{rem:local-tuples}
  A detailed analysis of the presented proof
  allows to obtain a stronger statement than in
   \cref{prop:uqw-tuples}, which we now describe.
   For $d,r,m\in\N$,
  let
    $\textrm{UQW}(d,r)$ denote the following statement:
   \begin{quote}\itshape There exists a constant $s^d_r\in \N$ and a polynomial $N^d_r\from\N\to\N$ such that 
      for all $m\in \N$ and all subsets 
     $A\subset V(G)^d$ with $|A|\ge N^d_r(m)$ there is a set $S\subset V(G)$ of size $|S|\le s^d_r$ and a subset $B\subset A$ of size $|B|\ge m$ which is mutually $r$-independent in $G-S$.
   \end{quote}   
   Our proof shows that  the statement $\textrm{UQW}(d,r)$
   can be concluded from the statement $\textrm{UQW}(1,4r)$.
     
This is because in our proof,  we have~obtained:
  $$
  N^d_{2r}(m)=(N_{4r})^d(m(d^2+1))\qquad\textrm{and}\qquad s^d_{2r}=d\cdot s_{4r}.
  $$
  Thus, when establishing the values of $N^d_{2r}(m)$ and $s^d_{2r}$, we refer to the quasi-wideness of $\CCC$ only by using numbers $s_{4r}$ and $N_{4r}(m')$ for $m'\in \N$.
  
  On the other hand, by \cref{thm:new-uqw}, the statement $\textrm{UQW}(1,4r)$
  follows from the existence of a number $t\in\N$ such that $K_t\not\minor_{10r} G$ for $G\in \CCC$.
  To summarize, for all $r\in\N$, the existence of a number $t\in\N$ such that $K_t\not\minor_{10r} G$ for $G\in \CCC$ implies the statement $\textrm{UQW}(d,r)$,
  for all $d\in\N$.
    %
  % By \cref{thm:new-uqw}, it suffices to assume that $K_t\not\minor_{10r} G$ to have $s(4r)\leq t$ and $N(4r,m')\leq c(r,t)\cdot (m')^{(8t+1)^{2rt}}$ for some computable function $c(r,t)$.
  % Hence, this supposition alone, instead of full quasi-wideness of $\CCC$, is sufficient to claim that the conclusion of \cref{thm:new-uqw} holds with
  % $s^d(2r)$ bounded by a computable function of $t$, $d$, and $q$, and $N^d(2r,m)$ bounded by a computable function of $m$, $t$, $d$, and $q$.
\end{remark}
\end{comment}