\section{From nowhere denseness to uniform quasi-wideness}\label{sec:uqw}

This section is devoted to the proofs of \cref{thm:new-uqw} and \cref{thm:new-vc}. 
We first recall necessary preliminaries from graph theory. 

All graphs in this paper are finite, undirected and simple, that is, 
they do not have loops or parallel edges. Our notation is standard,
we refer to~\cite{diestel2012graph} for more background on 
graph theory. 
We write $V(G)$ for the vertex set of a graph $G$ and
$E(G)$ for its edge set. 
The {\em{distance}} between vertices $u$ and $v$ in $G$, denoted $\dist_G(u,v)$, is the length of a shortest path between $u$ and $v$ in~$G$.
If there is no path between $u$ and $v$ in $G$, we put $\dist_G(u,v)=\infty$.
For a vertex $u$ and nonnegative integer $s$, by $N_s[u]$ we denote the {\em{$s$-neighborhood of $u$}} which comprises vertices at distance at most $s$ from $u$.

A {\em{minor model}} of a graph $H$ in $G$ is a family $(I_u)_{u\in V(H)}$ of pairwise vertex-disjoint connected subgraphs of $G$, called {\em{branch sets}},
such that whenever $uv$ is an edge in~$H$, there are $u'\in I_u$ and $v'\in I_v$ for which $u'v'$ 
is an edge in $G$.
The graph $H$ is a {\em{depth-$r$ minor}} of $G$, denoted $H\minor_rG$, if there is a minor model
$(I_u)_{u\in V(H)}$ of~$H$ in $G$ such that each $I_u$ has radius at most $r$.

A class $\CCC$ of graphs is \emph{nowhere dense} if there is a function 
$t\colon \N\rightarrow \N$ such that for all $r\in \N$ it holds that $K_{t(r)}\not\minor_r G$
for all $G\in \CCC$. 

A set $B\subseteq V(G)$ is called {\em{$r$-independent}} in a graph $G$ if for all
distinct $u,v\in B$ we have $\dist_G(u,v)>r$.
A class $\CCC$ of graphs is \emph{uniformly quasi-wide} if there are
functions $N\colon \N\times\N\rightarrow \N$ and $s:\N\rightarrow \N$ such
that for all $r,m\in \N$, all graphs $G\in \CCC$, and all subsets $A\subseteq V(G)$ of size $\abs{A}\geq N(r,m)$, there is a set
$S\subseteq V(G)$ of size $\abs{S}\leq s(r)$ and a set
$B\subseteq A\setminus S$ of size $\abs{B}\geq m$ which is $r$-independent in
$G-S$. 

On high level, to prove \cref{thm:new-uqw} we proceed by finding large independent sets within $A$ for larger and larger radii, just as in~\cite{nevsetvril2010first}.
If for some $s\leq \left\lceil r/2\right\rceil$
we have found a large $2s$-independent set, we want to find a large 
$(2s+1)$-independent subset. This case can easily be handled by the classical Ramsey 
Theorem on nowhere dense classes of graphs. The more
intricate case is to find a few elements to delete such that
we find a large $(2s+2)$-independent set in a large $(2s+1)$-independent set. 
We alternately carry out the constructions for $s=1,\ldots, \left\lceil r/2\right\rceil$ to 
obtain the statement of the theorem. 

The following lemma encapsulates the step from a large $2s$-independent set to a large $(2s+1)$-independent set.

\begin{lemma}\label{lem:ramsey1}
Let $G$ be a graph and let $s\in \N$. If $K_t\not\minor_{s} G$ and 
if $A\subseteq V(G)$ is a $2s$-independent set in $G$ of
size at least $\binom{m+t-2}{t-1}$, then there exists
a subset $B\subseteq A$ of size at least $m$ which $(2s+1)$-independent in $G$. 
\end{lemma}
\begin{proof}
By Ramsey's Theorem\todo{citation?}, in any graph every set of size $\binom{m+t-2}{t-1}$ contains an
independent set of size $m$ or a clique of size $t$. 

As $A$ is $2s$-independent, the $s$-neighborhoods of vertices from $A$ are pairwise disjoint.
Let $H$ be a graph with $A$ as the vertex set, where we put an edge between vertices $u$ and $v$ if and only if there is an edge in $G$ between $N_s[u]$ and $N_s[v]$.
Clearly, $H$ is a depth-$s$ minor of $G$, as we can map each $u\in A$ to $I_u=G[N_s[u]]$ to obtain an appropriate minor model.
By assumption, 
$H$ excludes $K_t$ as a subgraph, and hence by Ramsey's Theorem,
we can find an independent set $B$ of size $m$ in $H$. 
By the construction of $H$, we have that $B\subseteq A$ and $B$ is also a $(2s+1)$-independent set in $G$. 
\end{proof}

For the second, more complicated case, 
we will employ one more lemma, namely the bound on the number of distinct neighborhoods in a graph from a nowhere dense class.

\begin{lemma}[adaptation of Lemma 4.11 in \cite{gajarsky2017kernelization}]\label{lem:diversity}
Let $G$ be a graph such that $K_t\not\minor_{1} G$ for some constant $t\in \N$. 
Then for every $\epsilon>0$ there exists $n_0$, depending only on $t$ and $\epsilon$, such that for all $A\subseteq V(G)$ with $|A|\geq n_0$ it holds that
\[\abs{\{N(v)\cap A \colon v\in V(G)\}}\leq\abs{A}^{1+\epsilon}.\]
\end{lemma}

We remark that the proof of \cref{lem:diversity} uses only the fact that
nowhere dense classes of graphs do not have dense 
shallow minors~\cite{dvorak2007asymptotical,jiang2011compact}, and does not rely on any non-constructive arguments from the stability theory.
From \cref{lem:diversity} we can derive the following.

\begin{corollary}\label{lem:gajarsky}
Let $\CCC$ be a nowhere dense graph class and let $s\in \N$. 
Let $G\in \CCC$ and let $A\subseteq V(G)$ be a $(2s+1)$-independent
set.
Then for every $\epsilon>0$ there exists $n_0$ such that if $|A|\geq n_0$, then 
\[\abs{\{N_{s+1}[v]\cap A \colon v\in V(G)\}}\leq\abs{A}^{1+\epsilon}.\]
\end{corollary}
\begin{proof}
As $A$ is $(2s+1)$-independent, we have that the $s$-neighborhoods of vertices from $A$ are pairwise disjoint.
Obtain an $s$-shallow minor $H$ of $G$ by contracting $N_s[u]$ for each $u\in A$. 
We implicitly identify each vertex $u\in A$ with the vertex of $H$ obtained from contracting $N_s[u]$, thus $A\subseteq V(H)$.
Observe that $H$ belongs to the class $\CCC\triangledown s$ of all $s$-shallow minors of graphs from $\CCC$. This class is also nowhere dense;
see, e.g., Proposition~4.1 in~\cite{sparsity}.

Let us fix $\epsilon>0$
Take any $v\in V(G)$. If $v\in N_s[u]$ for some $u\in A$, then since $A$ is $(2s+1)$-independent, we have that $v$ is at distance larger than $2s+1$ from any other vertex of $A$.
Hence in this case we have $N_{s+1}[v]\cap A=\{u\}$ and there can be at most $|A|$ neighborhoods of this type.
Next, suppose $v\notin N_s[u]$ for any $u\in A$. Then by the construction of $H$ we have that $N_{s+1}[v]\cap A=N^H[v]\cap A$, where $N^H[v]$ is the neighborhood of $v$ in $H$.
Since $H\in \CCC\triangledown s$ and $\CCC\triangledown s$ is nowhere dense, by \cref{lem:diversity} we have that the number of such neighborhoods is at most $|A|^{1+\epsilon/2}$,
provided $|A|\geq n_0$ for some $n_0$.
Thus, we conclude that $\abs{\{N_{s+1}[v]\cap A \colon v\in V(G)\}}\leq |A|+|A|^{1+\epsilon/2}$, which is bounded by $|A|^{1+\epsilon}$ if we choose $n_0$ large enough.
\end{proof}

%Observe that nowhere dense classes are closed under
%taking bounded depth minors, as stated in the following lemma. 
%It is an immediate consequence of Proposition~4.1 
%of~\cite{sparsity}. 

%\begin{lemma}
%Let $\CCC$ be a nowhere dense class of graphs and let $s\in \N$. 
%Then also the class $\{H \minor_s G \colon G\in \CCC\}$ is nowhere dense. 
%\end{lemma}



We can now state the main building block of our proof as a lemma. 

\begin{lemma}\label{lem:apex}
Let $\CCC$ be a nowhere dense class of graphs and let $s\in \N$. 
Assume that $K_t\not\minor_{2s+2} G$. 
Let $n_0$ be the constant given by \cref{lem:gajarsky} for $\epsilon=1/3$ 
and let $m\geq n_0$ be an integer. 
If $A$ is an $(2s+1)$-independent set in $G$ of size at least $(4(m+1)^3+2)^{2t+1}$ 
then either 
\begin{itemize}
\item $A$ contains a $(2s+2)$-independent set $A'$ of size $m$, or
\item there is a subset $A'\subseteq A$ of size at least~$m$ and
a vertex $v\in V(G)$ which is at distance exactly $s+1$ to all vertices of $A'$. 
\end{itemize}
\end{lemma}


%Even though we prove a purely graph theoretic statement, 
%it is useful to see our methods in a more general model-theoretic context. 


%\begin{definition}
%If $\tau$ is a word over an alphabet $\Sigma$ and
%$a\in \Sigma$, then $\tau\cdot a$ denotes the concatenation of~$\tau$
%and $a$.  The \emph{branching index} of a formula $\psi(\tup{x},\tup{y}$
%over a graph $G$ is the largest number
%$\ell$ such that there are tuples of elements
%$\tup{u}_{\sigma_1},\ldots, \tup{u}_{\sigma_{2^\ell}}\in V(G)$, indexed by the
%words over the alphabet $\{0,1\}$ of length exactly $\ell$, and
%tuples of elements $\tup{v}_{\tau_1},\ldots, \tup{v}_{\tau_{2^\ell-1}}$, indexed by the
%words over $\{0,1\}$ of length strictly smaller than $\ell$, such that
%if $\tau_j\cdot a$ is a (not necessarily proper) prefix of~$\sigma_i$, then
%$G\models \psi(\tup{u}_{\sigma_i},\tup{v}_{\tau_j})$ if, and only if, $a=1$. The tuples
%$\tup{u}_{\sigma_1},\ldots, \tup{u}_{\sigma_{2^\ell}}\in V(G)$ are called the 
%\emph{leaves} of the tree, the tuples $\tup{v}_{\tau_1},\ldots, \tup{v}_{\tau_{2^\ell-1}}$
%are its \emph{inner nodes}. Intuitively, a leaf $\tup{u}$ is connected to its 
%predecessors~$\tup{v}$ such
%that $\tup{u}$ is a \emph{right successor} of $\tup{v}$ and not to its predecessors such that 
%it is a \emph{left
%successor}. 
%\end{definition}
     

%\begin{lemma}[\cite{hodges1993model}, Lemma 6.7.9, p.\
%  313]\label{lem:branching}
%  Let $\psi(\tup{x},\tup{y})$ be a formula and let $G$ be a graph. 
%  If $\psi$ has branching index~$k$ over $G$, 
%  then~$\psi$ has ladder index smaller than $2^{k+1}$ over $G$. 
%  If $\psi$ has  has
%  ladder index $k$ over $G$, then $\psi$ has branching index smaller than
%  $2^{k+2}-2$ over $G$.
% \end{lemma}

%In the proof of~\cref{thm:malshelah} one constructs a tree, in which
%elements are iteratively classified according to their types. The depth of 
%the type tree is directly related to the ladder index of the formula, 
%more precisely to the \emph{branching index} of the formula. We refrain
%from defining the branching index here and refer to~\cite{malliaris2014regularity}
%and to Lemma 6.7.9 of the textbook \cite{hodges1993model}, which relates
%ladder index and branching index. In this section we will be working with 
%formula $\phi(x,y)$ with exactly two free variables only. 

To prove the lemma, we will arrange the elements of $A$ in a binary tree
and prove that the tree contains a long path. From this path, we will 
extract the set $A'$. In stability theory, such trees are called \emph{type trees} and they are used to extract long indiscernible sequences, see e.g.~\cite{malliaris2014regularity}. 

Let $T$ be a (rooted) binary tree, where each vertex (except the root) is 
marked as a left or right successor of its predecessor. We call $w$ 
a \emph{left (right) descendant} of $v$ if the first successor on the unique
$v$-$w$ path in $T$ is a left (right) successor.

Fix an enumeration $a_1,\ldots, a_{\ell}$ of $A$ and let $s\in \N$. 
The \emph{distance-$s$ tree}
of $(a_1,\ldots,a_{\ell})$ is a binary tree which is constructed recursively as 
follows. We make $a_1$ the root of the tree. Assume that $a_1,\ldots, a_i$
have already been inserted into the tree. In order to insert the next element $a_{i+1}$, we follow a root-leaf path to find a position for it. 
Starting from the root $a_1$, at each point we are at some node $a_j$ and we are to decide whether we continue along the left or to the right branch at $a_j$.
If $\dist(a_j,a_{i+1})\leq s$, we continue along the right branch at $a_j$, otherwise we follow the left branch. If 
there is no right successor (or left successor, respectively), we insert $a_{i+1}$ 
as a right (or left child, respectively) of $a_j$. 

A root-leaf path in $T$ has \emph{$k$ alternations} 
if it contains $k+1$ nodes $a_{i_1},\ldots, a_{i_{k+1}}$ (which appear
in that order on the path, possibly not consecutively) such that $a_{i_{j+1}}$ is a left descendant of $a_{i_j}$
if and only if $a_{i_{j+2}}$ is a right descendant of $a_{i_{j+1}}$, for all $1\leq i\leq k-1$. 
The \emph{alternation rank} of $T$ is the largest number $t$ such that 
$T$ contains a path with $t$ alternations.

\begin{lemma}\label{lem:number-of-nodes}
Let $T$ be a rooted binary tree of alternation rank at most $t$. If
$T$ has height at most~$h$, then $T$ has at most $1+2h^{t+1}$
vertices. 
\end{lemma}
\begin{proof}
For each vertex $u$ of $T$, say at level $g\leq h$, let the {\em{characteristic vector}} of $u$ be the vector $(p_0,\ldots,p_{g-1})$ such that each $p_i\in \{\mathsf{L},\mathsf{R}\}$ denotes
whether $u$ is the left of right successor of its unique ancestor at level $i$. Clearly, every vertex of $T$ has a different characteristic vector.

For a characteristic vector $(p_0,\ldots,p_{g-1})$ of a vertex $u$, let $X\subseteq \{0,1,\ldots,g-2\}$ be the subset of those indices $i$ for which $p_i\neq p_{i+1}$. 
Note that the path from the root to $u$ has $|X|$ alternations, so since $T$ has alternation rank at most $t$, we have that $|X|\leq t$.
Observe also that the characteristic vector $(p_0,\ldots,p_{g-1})$ can be uniquely described by specifying $p_0\in \{\mathsf{L},\mathsf{R}\}$ and the set $X$, which is a subset of size at most $t$ of
a set of size $g-1$.
It follows that the number of different characteristic vectors of vertices at level $g\geq 1$ is at most $2\sum_{s=0}^{t}\binom{g-1}{s}$.

We conclude that the number of vertices in $T$ is upper bounded by the number of different characteristic vectors of length at most $h$, which is at most
\begin{align*}
1+2\sum_{g=1}^h\, \sum_{s=0}^t\, \binom{g-1}{s}=1+2\sum_{s=0}^t\, \sum_{g=1}^h\, \binom{g-1}{s}= 1+2\sum_{s=1}^{t+1} \binom{h}{s}\leq 1+2h^{t+1}.
\end{align*}
This concludes the proof.
\end{proof}


%Denote by $k_\ell^s$ the maximum possible number of vertices in a binary tree of height at most $\ell$ and alternation rank at most $s$. We have $k_0^s\leq 1$ for all $s\geq 1$
%and $k_\ell^1\leq 2\ell-1$ for all $\ell$. 
%Take now any binary tree $T$ of height at most $\ell$ and alternation rank at most $s$, and let $L$ and $R$ be the root-leaf paths in $T$ that take only left, respectively only right successors.
%As the height of $T$ is at most $\ell$, we have that the removal of $L$ and $R$ from $T$ results in a forest consisting of at most $2(\ell-1)$ subtrees of $T$. 
%Observe that each of these subtrees has alternation rank at most $s-1$
%and height at most $\ell-2$,
%hence we have
%\begin{align*}
%k_\ell^s\leq 2(\ell-1)\cdot k_{\ell-2}^{s-1}+2\ell+1
%\end{align*}
%for all $s\geq 2,\ell\geq 2$. In the last inequality, we count the vertices on 
%the two paths of length at most 
%$\ell$ staring in the root which have no alternations (at most $2\ell-1$). 
%All nodes branching from 
%these paths have one less alternation available, hence there are at most
%$ 2(k_{\ell-2}^{s-1}+k_{\ell-3}^{s-1}+\ldots + k_{0}^{s-1})$ of them. 
%Now it is easy to check that the function $(2\ell+2)^s$ satisfies these requirements (use $k_0^{s-1}\leq 1$ here). 


We now prove that in nowhere dense classes the alternation ranks of 
distance trees are small. 

\begin{lemma}\label{thm:alternation-rank-type-tree}
Let $G$ be a graph, let $s,t\in \N$ be such that $K_t\not\minor_{2s+2} G$, and let
$(v_1,\ldots, v_\ell)$ be an enumeration of a $(2s+1)$-independent set 
in $G$. Then the alternation rank of the distance-$(2s+2)$ tree of 
$(v_1,\ldots,v_\ell)$ is at most~$2t$. 
\end{lemma}
\begin{proof}
Let $T$ be the distance-$(2s+2)$ tree of $(v_1,\ldots,v_\ell)$.
Assume that we find a root-leaf path in $T$ which has alternation $2k$.
That is, this path contains nodes $v_{i_1},\ldots, v_{i_{2k+1}}$ such that $v_{i_{j+1}}$ is a left descendant of $v_{i_j}$
if and only if $v_{i_{j+2}}$ is a right descendant of $v_{i_{j+1}}$, for all $1\leq j\leq 2k-1$.
Let $a_1,b_1,\ldots, a_k,b_k$ be the subsequence of $v_{i_1},\ldots, v_{i_{2k+1}}$ from $v_{i_1}$ to $v_{i_{2k}}$ if $v_{i_2}$ is a right descendant of $v_{i_1}$,
and from $v_{i_2}$ to $v_{i_{2k+1}}$ otherwise. Thus, $b_1$ is a right descendant of $a_1$ in $T$.

\begin{claim}\label{cl:zij}
For every pair $a_i,b_j$ with $1\leq i\leq j\leq k$, there
is a vertex $z_{ij}$ which is at distance exactly $s+1$ from 
$a_i$ and from $b_j$, and which is at distance greater than $s+1$ 
from all $b_\ell$ with~$\ell\neq j$. 
\end{claim}
\begin{clproof}
Since for $j\geq i$, $b_j$ is a right descendant of $a_i$ in $T$, 
there is a path of length at most $2s+2$ between $a_i$ and $b_j$ in $G$. 
Since $\{v_1,\ldots, v_n\}$ is $(2s+1)$-independent in~$G$, this path
must have length exactly $2s+2$, and hence there is a vertex $z_{ij}$ 
at distance exactly $s+1$ both from $a_i$ and from $b_j$. 

Suppose now that $z_{ij}$ was at distance at most $s+1$ from some $b_\ell$ with $\ell\neq j$.
Then $b_\ell$ and $b_j$ would be at distance at most $2s+2$ in $G$.
However, if $\ell<j$ then $b_j$ is a left descendant of $b_\ell$ in $T$, and if $\ell>j$ then $b_\ell$ is a left descendant of $b_j$ in $T$.
In any case $b_j$ and $b_\ell$ have to be at distance more than $2s+2$ in $G$ by the construction of $T$, which is a contradiction.
\end{clproof}

Now for each $1\leq j\leq k$, let $H_j$ be the subgraph induced by $N_s[b_j]\cup N_s[a_j]\cup\{z_{ij} \colon 1\leq i\leq j\}$ in $G$.
By \cref{cl:zij}, $H_j$ is connected and has radius at most $2s+2$; the center is $z_{jj}$.
Since each vertex $z_{ij}$ is at distance $s+1$ from $a_i$ and $b_j$, it cannot be at distance at most $s$ from any other vertex $a_\ell$ for $\ell\neq i$, or $b_\ell$ for $\ell\neq j$;
this would contradict the assumption that $a_1,b_1,\ldots,a_k,b_k$ are $(2s+1)$-independent.
Moreover, we have that $z_{ij}\neq z_{i'j'}$ whenever $j\neq j'$, since $z_{ij}$ is at distance greater than $s+1$ from $b_{j'}$.
It follows that subgraphs $H_j$ for $1\leq j\leq k$ are pairwise disjoint.
However, for each $1\leq i<j\leq k$, the vertex $z_{ij}$ certifies that there is an edge in $G$ between $H_i$ and $H_j$.
Hence, subgraphs $H_j$ for $1\leq j\leq k$ yield a depth-$(2s+2)$ minor of $K_k$ in $G$.
Since $K_t\not\minor_{2s+2} G$, we infer that $k<t$, which proves the claim.
\end{proof}

%
%Note that the above proof does not give a proof that the ladder
%index of the distance-$2$ formula is at most $2(t-1)$. In the type
%tree we have the stronger statement that the vertices $b_i$
%are not connected by a path of length $2$. 
%We make no statement about the connections
%of these elements in the ladder. 


We can now prove \cref{lem:apex}. 

\begin{proof}[of \cref{lem:apex}]
Fix any enumeration $v_1,\ldots, v_\ell$ of $A$ and
build the distance-$(2s+2)$ tree~$T$ for
$v_1,\ldots,v_\ell$. By~\cref{thm:alternation-rank-type-tree}, 
the alternation rank of $T$ is at most~$2t$. 

By~\cref{lem:number-of-nodes}, $T$ has depth at least $2(m+1)^3$, hence, 
it contains a path of length at least $2(m+1)^3$. We choose the largest set $X$
of vertices on that path without alternations, which has hence length
at least $(m+1)^3$. 

Either, $X$ is a set with all its successors in $T$ on the left, 
then $X$ is a $(2s+2)$-independent set and we let $A'=X$ and finish the proof.

Otherwise, the vertices of $X$ are pairwise at distance $(2s+2)$. Let $\ell+1=|X|$. 
We claim that we find an element which is at distance exactly $s+1$ 
to at least $\ell^{1/3}$
of the vertices of~$X$. There are $\binom{\ell+1}{2}\geq \ell^2/2$ pairs in $X$ which 
must be connected via an intermediate vertex. A vertex which is at distance
exactly $s+1$ to $d$ vertices of $A$ can create $\binom{d}{2}\leq d^2/2$ of such connections. 
Hence, if there is no vertex with $|N_{s+1}(v)\cap A|\geq \ell^{1/3}$, 
we need $\ell^{4/3}$ vertices with distinct $(s+1)$-neighborhoods
to create all connections. As $\ell\geq n_0(1/3)$, 
this contradicts \cref{lem:gajarsky}. Hence there exists a vertex $v$ which is 
at distance exactly $s+1$ to at least $m$ vertices of $A$, 
we let $A'=N_r(v)\cap A$ and finish the proof. 
\end{proof}

We iteratively apply \cref{lem:apex}, always adding the apex vertex 
returned by the lemma to the set $S$, until we find a large 
$(2s+2)$-independent subset of $A$ in $G-S$. For the induction, we
define a function $R$ such that $R(m,1)=(4(m+1)^3+2)^{2t+1}$
and $R(m, i+1)=(4(R(m,i)+1)^3+2)^{2t+1}$.

\begin{lemma}\label{lem:iterate-apex}
Let $\CCC$ be a nowhere dense class of graphs. 
Let $n_0$ be the constant of \cref{lem:gajarsky} for $\epsilon=1/3$. 
Let $s\in \N$ and assume $K_t\not\minor_{3s+1} G$. 
Let $m\geq n_0$ be an integer. 
Let $A$ be a $(2s+1)$-independent set in $G$ of size at least $R(m,t)$. 
Then there is a subset $A'\subseteq A$ of size at least~$m$ and a 
a set $S\subseteq V(G)\setminus A$ of size at most $t-1$ such that
\begin{enumerate}
\item every vertex of $S$ is at distance exactly $s+1$ to 
every vertex of $A'$, and
\item $A'$ is $(2s+2)$-independent in $G-S$. 
\end{enumerate} 
\end{lemma}
\begin{proof}
Apply \cref{lem:apex} inductively. In step $i$ of the induction,
either the lemma returns a $(2s+2)$-independent $A'$ or a vertex $v$
with $N_{s+1}(v)\cap A=A'$ of size $R(m,t-i)$. In the second
case, add the vertex $v$ to $S$. We claim that after $t-1$
iterations we must be in the first case. Otherwise, enumerate the set $S$ 
as $(s_1,\ldots, s_t)$ and consider the sets
$X_i=N_s^{G-S}[a_i]\cup \{s_i\}$, $1\leq i\leq t$. Each~$X_i$ has radius
$s+1$, the $X_i$ are disjoint, and $X_i$ has a vertex connected to $s_j\in X_j$, $i\neq j$, 
hence we have found $K_t$ as a depth-$s+1$ minor, contradicting
our assumption. 
\end{proof}

We now proceed by induction on $r$. 

%\begin{lemma}\label{lem:ramsey2}
%Let $G$ be a graph such that $K_t\not\minor_r G$. 
%If $A$ is $2r$-independent and
%has size at least $\binom{m+t-2}{t-1}$, then there exists
%a subset $B\subseteq A$ of size at least $m$ which is a
%$(2r+1)$-independent set. 
%\end{lemma}
%\begin{proof}
%As $A$ is $2r$-independent, we can contract the $r$-neighborhood
%of each $v\in A$. The corresponding elements $N_r[v]$ in the resulting 
%depth-$r$ minor $H$ form a set $Z$ that is in \mbox{$1$-to-$1$} correspondence 
%with $A$. By assumption, 
%$H[Z]$ excludes $K_t$ as a subgraph, and hence by \cref{lem:ramsey1},
%it contains an independent set $B'\subseteq Z$ of size $m$ in $H$. 
%This set $B'$ corresponds to a $(2r+1)$-independent set $B\subseteq A$ of $G$. 
%\end{proof}
%
%\begin{lemma}\label{lem:distance-apex}
%Let $\CCC$ be a nowhere dense class of graphs. 
%Let $n_0$ be the constant of for $\epsilon=1/3$. 
%Assume $K_t\not\minor_{r+2} G$. 
%Let $m\geq n_0$ be an integer. 
%Let $A$ be a $(2r+1)$-independent set in $G$ of size at least $R(m,t)$. 
%Then there is a subset $A'\subseteq A$ of size at least~$m$ and a 
%a set $S\subseteq V(G)\setminus A$ of size at most $t-1$ such that
%\begin{enumerate}
%\item every vertex of $S$ is connected to a vertex at distance $r$ of $w$ for 
%every $w\in A'$, and
%\item $A'$ is $(2r+2)$-independent in $G-S$. 
%\end{enumerate} 
%\end{lemma}
%\begin{proof}
%As $A$ is $(2r+1)$-independent, we can contract the $r$-neighborhood
%of each $v\in A$. The corresponding elements $N_r[v]$ in the resulting 
%depth-$r$ minor $H$ form a set $Z$ that is in $1$-to-$1$ correspondence 
%with $A$. As $A$
%is $(2r+1)$-independent, $Z$ is independent in $H$. We now apply 
%\cref{lem:iterate-apex} to $Z$ in $H$. Note that the depth-$2$ minor
%we construct in \cref{thm:alternation-rank-type-tree} (now applied to $H$) 
%uses as connecting vertices $z_{ij}$ original vertices of the graph and
%not contracted vertices. Hence, when we apply the lemma, we may 
%use the assumption that $K_t\not\minor_{r+2} G$ (in general, 
%a depth-$2$ minor of a depth-$r$ minor may be a depth-$5r$ minor
%of the original graph~see Proposition~4.1 of~\cite{sparsity}). 
%Also, the vertices $v$ returned by 
%\cref{lem:iterate-apex} correspond to vertices of the graph $G$ and not
%to contracted neighborhoods. In particular, as $A$ is $(2r+1)$-independent, 
%the vertices $v$ returned by the lemma have distance exactly $r$ to 
%the vertices $w\in A$. The set $Z'$ returned by \cref{lem:iterate-apex}
%for $H$ is $2$-independent in $H$, and hence the corresponding 
%subset $B\subseteq A$ of $G$ is $(2r+2)$-independent. 
%in $G$. 
%\end{proof}

\begin{proof}[of \cref{thm:new-uqw}]
Alternatingly apply \cref{lem:ramsey1} and \cref{lem:iterate-apex} 
for $r$ times. Note that throughout the process we delete a set $S$ of 
size at most $t-1$, if $K_t\not\minor_{r+1} G$, as we can argue 
for the union of the sets $S$ found in each iteration, as in the proof
of \cref{lem:iterate-apex}. Furthermore, 
$\binom{m+t-2}{t-1}\leq (4(m+1)^3+2)^{2t+1}$. 
Then, if we start with a set $A$ of size at least 
$R(m,r+t)\in\Oof_{r,t}(m^{(6t+3)^{t+r}})$, the 
subset $B\subseteq A$ will have size $m$. This proves \cref{thm:new-uqw}. 
\end{proof}

TODO: Algorithmic version

We now come to the proof of \cref{thm:new-vc}. A set $X$ of vertices 
in a graph is \emph{shattered} (resp. \emph{$2$-shattered}) if for every
subset $Y\subseteq X$ (resp. every subset $Y\subseteq X$ of size $2$) there exists 
a vertex $v$ such that $N[v]\cap X=Y$. The \emph{Vapnik-Chervonenkis dimension}, short \emph{VC-dimension}~\cite{chervonenkis1971theory} (resp.
$2$VC-dimension) of a graph is the maximum size of a shattered set (resp. $2$-shattered set). The measure VC-dimension found has many applications, e.g.\ in learnability theory~\cite{haussler1987}, computational geometry~\cite{chazelle1989quasi} and graph theory~\cite{alon2006dominating,BousquetT15,chepoi2007covering,eickmeyer2016neighborhood}.


The \emph{$r$th power of a graph $G$} is the graph $G^r$
with vertex set $V(G)$, where there is an edge between two 
vertices $u$ and $v$ if and only if their distance in $G$ is at most $r$. 

We observe that an argument of Bousquet and 
Thomasse\'e~\cite{BousquetT15} can be slightly modified to prove that 
the $2$VC-dimension of the $r$-power graph $G^r$ of a graph $G$
with $K_t\not\minor_r G$ is small. Obviously, the $2$VC-dimension of $G$
bounds its VC-dimension. We hence prove a stronger result
than stated as \cref{thm:new-vc} in the introduction. 

\begin{theorem}
Let $r\in \N$ and let $G$ be a graph. 
If $K_t\not\minor_r G$, then the $2$VC-dimension of $G^r$
is at most $t-1$. 
\end{theorem}
\begin{proof}
Assume there is a set $A=\{a_1,\ldots, a_t\}$ of size $t$ such that
for all subsets $\{a_i,a_j\}\subseteq A$ of size $2$ 
there is an element $v_{ij}\in V(G)\setminus A$ with 
$N_r[v_{ij}]\cap A=\{a_i,a_j\}$. Fix $v_{ij}$ with the property
that $\max\{\dist_G(v_{ij},a_i), \dist_G(v_{ij},a_j)\}$ is 
minimised. 

A \emph{central walk} $W_{ij}$ is the concatenation of a minimum length
path $P_{ij}^i$ from $a_i$ to $v_{ij}$ and a minimum length path $P_{ij}^j$ from $v_{ij}$ to $a_j$. 
Note that a central walk is possibly not a path. For each pair $a_i,a_j$ fix
a central walk $W_{ij}$ and the corresponding paths $P_{ij}^i$ and $P_{ij}^j$. 

Now assume that a vertex $x$ belongs to two distinct central 
walks $W_{ij}$, $W_{i'j'}$. Assume that $x$ lies on $P_{ij}^i$ and $P_{ij}^{i'}$,
otherwise, rename the elements. First, observe that if $\dist(x,a_i)=\dist(x,a_{i'})$, 
then $a_i=a_{i'}$. Otherwise, $\dist(v_{ij},a_{i})=\dist(v_{ij},a_{i'})$ and hence 
$a_j=a_i$, and analogously, $a_{j'}=a_j$, contradicting the assumption 
that $W_{ij}$ and $W_{i'j'}$ are distinct. By the same argument we have 
$\dist(x,a_i)<\dist(x,a_j)$ and $\dist(x,a_{i'})<\dist(x,a_{j'})$. 
Now assume that $\dist(x,a_i)<\dist(x,a_{i'})$. By the same argument as 
above we have $a_{j'}=a_i$, hence $W_{i'j'}=W_{ij'}$. Here, we have
$\dist(x,a_i)<\dist(x,a_j)$ and $\dist(x,a_{i})<\dist(x,a_{i'})$, 
otherwise the walks are not distinct. 

Let us now construct connected subsets $X_i$ for all $1\leq i\leq t$. 
For every walk $W_{ij}$ the vertices of $W_{ij}$ closer to $a_i$ than to $a_j$ 
are added to $X_i$, the vertices of $W_{ij}$ closer to $a_j$ than to $a_i$ 
are added to $X_j$, ties are broken arbitrary.
Then the sets $X_i$ are pairwise disjoint by what we proved above. If a vertex $x$
appears in two distinct central walks, these are $W_{ij}$ and $W_{i\ell}$ for some
$i,j,\ell$ with $\dist(x,a_i)<\dist(x,a_j)$ and $\dist(x,a_i)<\dist(x,a_\ell)$. 
In both cases $x$ belongs to $X_i$. By construction, the sets $X_i$ are connected, 
have radius at most~$r$, and 
there is always an edge between a vertex of $X_i$ and a vertex of $X_j$ since $X_i\cup X_j$ 
contains the walk $W_{ij}$. Therefore, if the $2$VC-dimension is at least $t$, the 
graph contains $K_t$ as a depth-$r$ minor. 
\end{proof}