

\section{Uniform quasi-wideness}\label{sec:uqw}

The $k$-order property of a formula is strongly related
to its branching index. The largest $k$ such that 
$\psi$ has the $k$-order property over $G$ is
also called the \emph{ladder-index} of $\psi$ over $G$. 

If $\tau$ is a word over an alphabet $\Sigma$ and
$a\in \Sigma$, then $\tau\cdot a$ denotes the concatenation of~$\tau$
and $a$.  The \emph{branching index} of a formula $\psi(\tup{x},\tup{y}$
over a graph $G$ is the largest number
$\ell$ such that there are tuples of elements
$\tup{u}_{\sigma_1},\ldots, \tup{u}_{\sigma_{2^\ell}}\in V(G)$, indexed by the
words over the alphabet $\{0,1\}$ of length exactly $\ell$, and
tuples of elements $\tup{v}_{\tau_1},\ldots, \tup{v}_{\tau_{2^\ell-1}}$, indexed by the
words over $\{0,1\}$ of length strictly smaller than $\ell$, such that
if $\tau_j\cdot a$ is a (not necessarily proper) prefix of~$\sigma_i$, then
$G\models \psi(\tup{u}_{\sigma_i},\tup{v}_{\tau_j})$ if, and only if, $a=1$. The tuples
$\tup{u}_{\sigma_1},\ldots, \tup{u}_{\sigma_{2^\ell}}\in V(G)$ are called the 
\emph{leaves} of the tree, the tuples $\tup{v}_{\tau_1},\ldots, \tup{v}_{\tau_{2^\ell-1}}$
are its \emph{inner nodes}. Intuitively, a leaf $\tup{u}$ is connected to its 
predecessors~$\tup{v}$ such
that $\tup{u}$ is a \emph{right successor} of $\tup{v}$ and not to its predecessors such that 
it is a \emph{left
successor}. 
     

\begin{lemma}[\cite{hodges1993model}, Lemma 6.7.9, p.\
  313]\label{lem:branching}
  Let $\psi(\tup{x},\tup{y})$ be a formula and let $G$ be a graph. 
  If $\psi$ has branching index~$k$ over $G$, 
  then~$\psi$ has ladder index smaller than $2^{k+1}$ over $G$. 
  If $\psi$ has  has
  ladder index $k$ over $G$, then $\psi$ has branching index smaller than
  $2^{k+2}-2$ over $G$.
 \end{lemma}

In the proof of~\cref{thm:malshelah} we construct a type tree, in which
elements are iteratively classified according to their types. The depth of 
the type tree is directly related to the branching index of the formula. 
Our first theorem shows that we can avoid the exponential dependency 
between branching index and ladder index if we consider formulas $\psi(x,y)$
with exactly two free variables. 


Let $\psi(x,y)$ be a formula with $2$ free variables and let $(v_1,\ldots, v_n)$
be a sequence of vertices of $G$. The \emph{type tree}
of $\psi$ over $(v_1,\ldots,v_n)$ is constructed as 
follows. We make $v_1$ the root of the tree. Assume that $v_1,\ldots, v_i$
have been inserted to the tree. We follow a root-leaf path to find the
position for the next vertex $v_{i+1}$. If $G\models\psi(v_j,v_{i+1})$, we
go the right branch of the tree, otherwise we go to the left branch of 
the tree. 


\begin{theorem}
Let $\phi(x,y)$ be a formula with $2$ free variables and let
$(v_1,\ldots, v_n)$ be a sequence of elements of $G$. Then the 
largest complete binary subtree that is found as a topological minor 
of the type tree of $\psi$ over
$(v_1,\ldots, v_n)$ has depth at most twice the ladder
index of $\psi$. 
\end{theorem}
\begin{proof}
Consider the alternating path in the tree. 
\end{proof}

The following is implicit in~\cite{malliaris2014regularity}. 

\begin{lemma}[reference?]\label{lem:depth}
If a tree with $n$ vertices does not contain a complete binary 
tree of depth $k$ as a topological minor, then it has depth at most 
$x$. 
\end{lemma}

\begin{lemma}\label{lem:minor-to-tree}
Let $\phi(x,y)$ be a formula with $2$ free variables and let
$(v_1,\ldots, v_n)$ be a sequence of elements of $G$. Let $H$
be the topological minor model of a complete binary subtree in the
type tree of $\psi$ over $(v_1,\ldots, v_n)$. Denote the 
principal vertices of $H$ by $(w_1,\ldots, w_m)$. Then the type
tree of $\psi$ over $(w_1,\ldots, w_m)$ is a complete binary
tree. 
\end{lemma}

For the proof of our theorem we will use the formula 
$\dist_G(x,y)\leq 2$, for which we can give an explicit 
bound on the depth of the largest binary subtree.

\begin{theorem}
Let $\CCC$ be a nowhere dense class of graphs. Let 
$(v_1,\ldots, v_n)$ be the enumeration of an independent set 
in $G$. Assume that $K_t\not\minor_2G$. 
Then the largest complete binary subtree that is found as a topological minor 
of the type tree of $\psi$ over
$(v_1,\ldots, v_n)$ has depth at most $2(t-1)$. 
\end{theorem}
\begin{proof}
According to \cref{lem:minor-to-tree} we may assume that we find
the largest complete binary tree as a subgraph. We consider the
vertices $a_1,b_1,\ldots, a_k,b_k$ of the alternating path in the
type tree. Because $(v_1,\ldots, v_n)$ is independent in $G$, 
none of these vertices is adjacent and all of the vertices on the
paths of length $2$ which cause the creation of edges in the type
tree are distinct from $a_1,\ldots, b_k$. 

\begin{claim}
Every vertex $a_i$ is connected to every $b_j$, $j\geq i$,
via a vertex $z_{ij}$ which is not connected to any~$b_\ell$, $\ell\neq j$. 
\end{claim}

\noindent\textit{Proof.} Because $b_j$, $j\geq i$, is right of $a_i$, there is 
an element $z_{ij}$ connected to $a_i$ and $b_j$. If $z_{ij}$ was 
connected to $b_\ell$, $\ell\neq j$, then $b_\ell$ would be adjacent 
to $b_j$ in the type tree, which it is not. \hfill$\lrcorner$

\bigskip
We now contract $b_j$ and all $z_{ij}$ as well as $a_j$ to a single 
vertex for all $1\leq j\leq k$, $1\leq i\leq j$. We obtain a complete
graph $K_k$ as a depth-$2$ minor. 
\end{proof}

Note that the above proof does not give a proof that the ladder
index of the distance-$2$ formula is at most $2(t-1)$. In the type
tree we have the stronger statement that the vertices $b_i$
are not connected by a path of length $2$. 
We make no statement about the connections
of these elements in the ladder. 

\begin{theorem}
The function $N$ in the definition of uniform quasi-wideness
is small.
\end{theorem}
\begin{proof}
Assume $K_t\not\minor_2G$, in particular, $G$ excludes $K_t$
as a subgraph. We take a large independent subset $A'$ of $A$ and
enumerate it as $(a_1,\ldots, a_m)$. We build the type
tree, which has depth at least $x$ according to \cref{lem:depth}. 
We consider the longest branch of the type tree. On this branch, 
we take a set $X$ of maximum length such that every vertex has 
all its successors on the same side. This set has size at least
half the length of the branch. 

Either, $X$ is a set with all its successors on the left, then $X$ 
is a $2$-independent set and we are done with this step and
continue with $X$. Otherwise, all vertices of $X$ are at distance
$2$. Let $m=X$ and assume that $m$ is larger than $n_0$ to
be defined for neighbourhood complexity.
We claim that we find an element which is connected to at least $m^{1/3}$
of the vertices of $X$. Otherwise, every vertex
can create only $m^{2/3}$ connections, however, we need
to create $m^2$ connections. Hence, we need $m^{4/3}$ vertices
to create all connections. We have neighbourhood complexity
$m^{1+\epsilon}$ though, a contradiction. 

We delete the element of degree $m^{1/3}$ 
and continue with the subsequence
induced by its neighbours. This can happen at most 
$t(2)$ times, as we are constructing a complete minor at 
depth $2$ here. 
\end{proof}

