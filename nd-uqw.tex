\section{Bounds for uniform quasi-wideness}\label{sec:uqw}



%In this section we prove Theorem~\ref{thm:new-uqw}, which strengthens Theorem~\ref{thm:krs} by providing an explicit polynomial $N$ and bound $s$,
%whereas the bounds in~Theorem~\ref{thm:krs} rely on non-constructive arguments. 
%We note that~Theorem~\ref{thm:krs} is sufficient to prove our main result,~Theorem~\ref{thm:vc-density}, but is required in our proof of~Theorem~\ref{thm:new-stable}, which is the effective 
%variant of the result of Adler and Adler, Theorem~\ref{thm:new-stable}.

In this section we prove Theorem~\ref{thm:new-uqw}, which strengthens the 
result of Kreutzer et al.~\cite{siebertz2016polynomial} by providing an explicit polynomial $N$ and bound $s$. 
We remark that the result of Kreutzer et al.~is sufficient to prove our main result, Theorem~\ref{thm:vc-density}, but is required in our proof of~Theorem~\ref{thm:new-stable}, which is the effective 
variant of the result of Adler and Adler, Theorem~\ref{thm:new-stable}.
%
%
% \setcounter{theorem}{}
% \begin{theorem}\label{thm:new-uqw}
% For all $r,t\in \N$ there is a polynomial  $N\colon \N\to \N$ with $N(m)=
% \Oof_{r,t}{(m^{{(4t+1)}^{2rt}})}$, such that the following holds.
% Let $G$ be a graph such that $K_t\not\minor_{\lfloor 9r/2\rfloor} G$, and
% let $A\subseteq V(G)$ be a vertex subset of size at least $N(m)$, for a given $m$.
% Then there exists a set $S\subseteq V(G)$ of size $|S|<t$ and a set $B\subseteq A\setminus S$
% of size $|B|\geq m$ which is $r$-independent in $G-S$.
% Moreover, given~$G$ and $A$, such sets $S$ and $B$ can be computed in time $\Oof_{r,t}(|A|\cdot |E(G)|)$.
% \end{theorem}
%
%
%
%\paragraph{General strategy.}
Our proof follows the same lines as the original proof of Ne\v set\v ril and Ossona de Mendez~\cite{nevsetvril2010first}, with the difference that in the key technical lemma (Lemma 3.5 of~\cite{nevsetvril2010first}) %(Theorem~\ref{lem:apex} below), 
we improve the bounds significantly by replacing a Ramsey argument with a refined combinatorial analysis (Lemma~\ref{lem:apex} below).
The new argument essentially originates in the concept of {\em{branching index}} from stability theory. 
%, 
%and also uses the almost linear bound on neighborhood complexity in nowhere dense graph classes, due to Gajarsk\'y et al.~\cite{gajarsky2017kernelization}. \marginpar{no longer}
%For sake of completeness, we present the entire proof of~Theorem~\ref{thm:new-uqw}.
Due to space constraints we present only the proof of 
the key lemma and refer to the appended full version
for a complete proof of the main theorem. 

%
%We first prove a restricted variant,~Theorem~\ref{lem:engine} below, in which we assume that $A$ is already $(r-1)$-independent. Then, in order to derive
%Theorem~\ref{thm:new-uqw}, we apply the lemma iteratively for $r$ ranging from $1$ to the target value.
%
%\begin{lemma}\label{lem:engine}
%For every pair of integers $t,r\in \N$ there exists an integer $d<9r/2$ and a function $L\colon \N\to \N$ with $L(m)=\Oof_{r,t}{(m^{{(4t+1)}^{2rt}})}$ such that the following holds.
%For each $m\in \N$, graph~$G$ with $K_t\not\minor_{d} G$, and
%$(r-1)$-independent set $A\subseteq V(G)$ of size at least $L(m)$, there is a set $S\subseteq V(G)-A$ of size less than $t$ such that $A$ contains a subset $B$ of size $m$ which is $r$-independent in $G-S$.
%Moreover, if $r$ is odd then $S$ is empty, and if $r$ is even,
%then every vertex of $S$ is at distance exactly $r/2$ from every vertex of $B$.
%Finally, given $G$ and $A$, the sets $B$ and $S$ can be computed in time $\Oof_{r,t}(|A|\cdot |E(G)|)$.
%\end{lemma}
%
%We prove~Theorem~\ref{lem:engine} in \autoref{sec:engine}, but  a very rough sketch is as follows.
%The  case of general~$r$ reduces to the case $r=1$ or $r=2$, depending on the parity of $r$,
%by contracting the balls of radius $\lfloor \frac {r-1} 2\rfloor $ around the vertices in $A$ to single vertices.
%The case of $r=1$ follows immediately from Ramsey's theorem, as in~\cite{nevsetvril2011nowhere}.
%The case $r=2$ is substantially more difficult.
%We start by formulating and proving the main technical result needed for proving the case $r=2$.
%
%
%
%
%
%
%
%\subsection{The main technical lemma}
%\label{sec:main-tech}
%
%The following, Ramsey-like result is the main technical lemma used in the proof of~Theorem~\ref{thm:new-uqw}. 


\begin{lemma}\label{lem:apex}
Let $\ell,m,t\in \N$ and assume $\ell\geq t^{8}$. 
If~$G$ is a graph and $A$ is a $1$-independent set in~$G$
with at least $(m+\ell)^{2t}$ elements,
then at least one of the following conditions hold:
\begin{itemize}
  \item $K_t\minor_{4} G$,
\item  $A$ contains a $2$-independent set of size $m$, 
\item  some vertex $v$ of $G$
has at least $\ell^{1/4}$ neighbors in $A$.
\end{itemize}
Moreover, if $K_t\not\minor_4G$, the
structures described in the other two cases (a $2$-independent set 
of size~$m$, or a vertex $v$ as above) can be 
computed in time $\Oof_t(|A|\cdot |E(G)|)$. 
\end{lemma}
%We remark that a statement similar to that of Theorem~\ref{lem:apex}
%can be obtained by employing Ramsey's theorem, as has been done in~\cite{nevsetvril2010first}. This, however, 
%%yields
%%in place of the bound $(m+\ell)^{2t}$ 
%%a bound of the form $R(m,\underbrace{q,q,\ldots,q}_{k\text{ times}})$,
%%where $k\sim\ell^{1/8}$ and $R(m_1,\ldots,m_c)$
%%is the Ramsey number for $c$ colors.
%%In particular, this 
%does not give a bound which is polynomial in $m+\ell$, and thus cannot be used to prove~Theorem~\ref{thm:new-uqw}.
%%
%%\medskip
The remainder of this section is devoted to the proof of Lemma~\ref{lem:apex}.
%We will use the following bounds on the edge density
%of graphs with excluded shallow minors obtained
%by Alon et al.~\cite{alon2003turan}. 
%
%\begin{lemma}[Theorem 2.2 in~\cite{alon2003turan}]\label{lem:densitynd}
%Let $H$ be a bipartite graph with maximum degree
%$d$ on one side. Then there exists a constant $c_H$, depending 
%only on $H$, such that every $n$-vertex graph $G$
%excluding~$H$ as a subgraph has at most $c_H\cdot n^{2-1/d}$
%edges. 
%\end{lemma} 
%
%Observe that if $K_t\not\minor_1G$, then in particular
%the $1$-subdivision of $K_t$ is excluded as a subgraph
%of $G$ (the $1$-subdivision of a graph $H$ is obtained by 
%replacing every edge of $H$ by a path of length $2$). 
%Moreover, the $1$-subdivision of 
%$K_t$ is a bipartite graph with maximum degree $2$ on one
%side. Furthermore, it is easy to check in the 
%proof of Theorem 2.2 in~\cite{alon2003turan} 
%that $c_H\leq |V(H)|$
%in case $d=2$. Since the $1$-subdivision of $K_t$ has 
%$\binom{t+1}{2}$ vertices, we can choose $c_{K_t}=\binom{t+1}{2}$ and
%conclude the following.   
%
%\begin{corollary}\label{crl:densitynd}
%Let $G$ be an $n$-vertex graph such that $K_t\not\minor_1 G$ for
%some constant $t\in \N$. Then $G$ has at most
%$\binom{t+1}{2}\cdot n^{3/2}$ edges.
%\end{corollary}
%
%We will use the following standard lemma saying that a shallow minor of a shallow minor is a shallow minor, where the parameters of shallowness are appropriately chosen.
%
%\begin{lemma}[adaptation of Proposition 4.11 in~\cite{sparsity}]\label{lem:combineminors}
%Suppose $J,H,G$ are graphs such that $H\minor_a G$ and $J\minor_b H$, for some $a,b\in \N$.
%Then $J\minor_c G$, where $c=2ab+a+b$.
%\end{lemma}
%
We will need one more technical lemma.

\begin{lemma}[$\star$]\label{lem:diversity}
  Let $G$ be a graph such that $K_t\not\minor_4G$ for some
  $t\in\N$ and let $A\subseteq V(G)$ with $|A|\geq t^{8}$. 
  Assume furthermore that every pair of elements of $A$ has a common neighbor in $V(G)\setminus A$.
  Then there exists a vertex $v$ in $V(G)\setminus A$ which has at least $|A|^{1/4}$ neighbors in $A$.
\end{lemma}
%\begin{proof}
%Denote $k=\max\{\,|N(w)\cap A|\ \colon\ w\in V(G)-A\,\}$; our goal is to prove that $k\geq |A|^{1/4}$.
%
%Let $B\subseteq V(G)-A$ be the set of those vertices outside of $A$ that have a  neighbor in $A$. 
%Construct a function $f\colon B\to A$ by a random procedure as follows:
%for each vertex $v\in B$, choose $f(v)$ uniformly and independently at random from the set $N(v)\cap A$.
%Next, for each $u\in A$ define branch set $I_u=G[\{u\}\cup f^{-1}(u)]$. Observe that since, by construction, $v$ and $f(v)$ are adjacent for all $v\in B$, each branch set $I_u$ has radius at most $1$,
%with $u$ being the central vertex. Also, the branch sets $\{I_u\}_{u\in A}$ are pairwise disjoint.
%Finally, construct a graph $H$ on vertex set $A$ by making distinct $u,v\in A$ adjacent in $H$ whenever there is an edge in $G$ between the branch sets~$I_u$ and $I_v$.
%Then the branch sets $\{I_u\}_{u\in A}$ witness that $H$ is a $1$-shallow minor of $G$.
%
%For distinct $u,v\in A$, let us estimate the probability that the edge $uv$ appears in $H$.
%By assumption, there is a vertex $w\in B$ that is adjacent both to $u$ and to $v$. Observe that if it happens that $f(w)=u$ or $f(w)=v$, then $uv$ for sure becomes an edge in $H$. 
%Since $w$ has at most $k$ neighbors in $A$, the probability that $f(w)\in \{u,v\}$ is at least $\frac{2}{k}$.
%
%By the linearity of expectation, the expected number of edges in $H$ is at least $\binom{|A|}{2}\cdot \frac{2}{k}=\frac{|A|(|A|-1)}{k}$.
%Hence, for at least one run of the random experiment we have that $H$ indeed has at least this many edges. 
%On the other hand, observe that $K_t\not\minor_1 H$; indeed, since $H\minor_1 G$, by Lemma~\ref{lem:combineminors} we infer that $K_t\minor_1 H$ would imply $K_t\minor_4 G$, a contradiction with the assumptions on $G$.
%Then Corollary~\ref{crl:densitynd} implies $H$ has at most $\binom{t+1}{2}\cdot |A|^{3/2}$ edges.
%Observe that 
%$\binom{t+1}{2}\cdot |A|^{3/2}\leq 3t^2/4\cdot |A|^{3/2}\leq \frac{3}{4}|A|^{7/4}$,
%where the first inequality holds due to $t\geq 2$, while the second holds by the assumption that $|A|\geq t^8$.
%By combining the above bounds, we obtain
%$$\frac{|A|(|A|-1)}{k}\leq \frac{3}{4}|A|^{7/4},$$
%which implies $k\geq |A|^{1/4}$ due to $|A|\geq t^8\geq 64$.
%\end{proof}

We proceed with the proof of Lemma~\ref{lem:apex}.
The idea is to arrange the elements of $A$ in a binary tree
and prove that provided $A$ is large, this tree contains a long path. From this path, we will 
extract the set $B$. 
In stability theory, similar trees are called \emph{type trees} and they are used to extract long indiscernible sequences, see e.g.~\cite{malliaris2014regularity}. 


\newcommand{\dau}{\mathrm{D}}
\newcommand{\son}{\mathrm{S}}
	
	We will work with a two-symbol alphabet $\set{\dau,\son}$, for {\em{daughter}} and {\em{son}}.
	We identify words in $\set{\dau,\son}^*$ with \emph{nodes}
	of the infinite rooted binary tree. 
  The \emph{depth} of a node $w$ is the length of $w$.
  For $w\in \set{\dau,\son}^*$,
	 the nodes $w\dau$ and $w\son$ are called, respectively, the \emph{daughter} and the \emph{son} of $w$,
	and $w$ is the \emph{parent} of both $w\son$ and $w\dau$. A node $w'$ is a {\em{descendant}} of a node $w$ if $w'$ is a prefix of $w$ (possibly $w'=w$).
	We consider
	 finite, labeled, rooted, binary trees, which are called simply trees below, and are defined as follows.
	 For a set of labels $U$, a ($U$-labeled) \emph{tree} is a partial function $\tau\from \set{\dau,\son}^*\to U$ whose domain is a finite set of nodes, 
	 called the \emph{nodes of $\tau$}, which is closed under taking parents. 
	 If $v$ is a node of $\tau$, then $\tau(v)$ is called its \emph{label}.
  
  Let $G$ be a graph, $A\subset V(G)$ be a $1$-independent set in $G$,
  and $\bar a$ be any enumeration of $A$, that is, a sequence of length $|A|$ in which every element of $A$ appears exactly once.
  We define a binary tree $\tau$ which is 
  labeled by vertices of $G$. The tree is defined by processing all elements of~$\bar a$ sequentially. 
  We start with $\tau$ being the  tree with empty domain, and for each element $a$ of the sequence $\bar a$, processed in the order given by $\bar a$, 
  execute the following procedure which results in adding a node with label $a$ to $\tau$.
  
When processing the vertex $a$, do the following. Start with~$w$ being the empty word. While~$w$ is a node of $\tau$, repeat the following step: 
  if the distance from $a$ to $\tau(w)$ in the graph~$G$ is at most~$2$, replace $w$ by its son, otherwise, replace~$w$ by its daughter.
  % Repeat the step, unless $\tau(w)$ is undefined.
  Once $w$ is not a node of $\tau$, extend~$\tau$ by setting  $\tau(w)=a$. In this way, we have processed the element $a$, and now
    proceed to the next element of $\bar a$, until all elements are processed. This ends the construction of $\tau$.
    Thus, $\tau$ is a tree labeled with vertices of $A$, and every vertex of $A$ appears exactly once in $\tau$.
	

Define the
\emph{depth} of $\tau$ as 
the maximal depth of a node of $\tau$.
For a word $w$, an \emph{alternation} in~$w$ is any 
position $\alpha$, $1\leq \alpha\leq |w|$, such that $w_\alpha\neq w_{\alpha-1}$; here, $w_\alpha$ denotes the $\alpha$th symbol of~$w$, and~$w_0$ is assumed to be $\dau$.
The \emph{alternation rank} of the tree $\tau$ is the maximum of the number of alternations in $w$, over all nodes $w$ of $\tau$.


\begin{lemma}[$\star$]\label{lem:number-of-nodes}
Let $h,t\ge 2$.	If $\tau$ has alternation rank at most $2t-1$ and depth at most $h-1$, then~$\tau$ has fewer than $h^{2t}$ nodes.
\end{lemma}
%\begin{proof}		
%	With each node $w$ of $\tau$ associate
%	function $f_w\colon \set{1,\ldots,2t}\to\set{1,\ldots,h}$ defined as follows:
%	$f_w$ maps each $i\in \set{1,\ldots,2t}$ to the $i$th alternation of $w$, provided $i$ is at most the number of alternations of $w$, and otherwise we put $f_w(i)=|w|+1$.
%	It is clear that the mapping $w\mapsto f_w$ for nodes $w$ of $\tau$ is injective
%and its image is contained in monotone functions from $\set{1,\ldots,2t}$ to $\set{1,\ldots,h}$, whose number is less than $h^{2t}$.
%Hence, the domain of~$\tau$ 
%		has fewer than $h^{2t}$ elements.
%\end{proof}

\begin{lemma}\label{thm:alternation-rank-type-tree}
Suppose that  $K_t\not\minor_{2} G$.
Then $\tau$ has alternation rank at most $2t-1$.
\end{lemma}
\begin{proof}
	Let $w$ be a node of $\tau$ with at least $2k$ alternations, for some $k\in \N$.
	Suppose $\alpha_1,\beta_1,\ldots,\alpha_k,\beta_k$ be the first $2k$ alternations of $w$.
	By the assumption that $w_0=\dau$ we have that~$w$ contains symbol $\son$ at all positions $\alpha_i$ for $i=1,\ldots,k$, and symbol $\dau$ at all positions $\beta_i$ for $i=1,\ldots,k$.
	For each $i\in \set{1,\ldots,k}$, define $a_i\in V(G)$ to be the label in $\tau$ of the prefix of $w$ of length $\alpha_i-1$, and similarly define $b_i\in V(G)$ to be the label in $\tau$ of the prefix of $w$
	of length $\beta_i-1$. 
	It follows that for each $i\in \set{1,\ldots,k}$, the following assertions hold:
	the nodes in $\tau$ with labels $b_i,a_{i+1},b_{i+1},\ldots,a_k,b_k$ are  descendants of the son of the node with label $a_i$,
	and the nodes with labels $a_{i+1},b_{i+1},\ldots,a_k,b_k$
	are descendants of the daughter of the node with label $b_i$.
	
	\begin{claim}\label{claim:minor}
		For every pair $a_i,b_j$ with $1\le i\le j\le k$, there is a vertex $z_{ij}\not\in A$	 which is a common neighbor of $a_i$ and $b_j$,
		and is not a neighbor of any $b_s$ with $s\neq j$.
	\end{claim}
	\begin{clproof}
		Note that since $i\le j$, the node with label $b_j$ is a descendant of the son of the node with label $a_i$, hence we have $\dist_G(a_i,b_j)\le 2$ by the construction of $\tau$.
		However, we also have $\dist_G(a_i,b_j)>1$ since $A$
		is $1$-independent. Therefore $\dist_G(a_i,b_j)=2$, so there is a vertex $z_{ij}$ which is a common neighbor of $a_i$ and $b_j$. 
		Suppose that $z_{ij}$ was a neighbor of~$b_s$, for some $s\neq j$. This would imply that $\dist_G(b_j,b_s)\le 2$, which is impossible, 
because
		 the nodes with labels~$b_s$ and $b_j$ in $\tau$ are such that one is a descendant of the daughter of the other, implying that $\dist_G(b_s,b_j)>2$.\hfill$\lrcorner$
	\end{clproof}
  
Note that whenever $i\leq j$ and $i'\leq j'$ are such that $j\neq j'$, the vertices $z_{ij}$ and $z_{i'j'}$ are different, because $z_{ij}$ is adjacent to $b_{j}$ but not to $b_{j'}$, and the converse holds for $z_{i'j'}$.
However, it may happen that $z_{ij}=z_{i'j}$ even if $i\neq i'$. This will not affect our further reasoning.

For each $j\in\set{1,\ldots,k}$, let $B_j$
be the subgraph of $G$ induced by the set
$\set{a_j,b_j}\cup\set{z_{ij}\colon 1\le i\le  j}$.
Observe that~$B_j$ is connected and has radius at most $2$, with $b_j$ being the central vertex.
By Claim~\ref{claim:minor} and the discussion from the previous paragraph, the graphs $B_j$ for $j\in \set{1,\ldots,k}$
are pairwise disjoint.
Moreover, for all $1\le i\le j\le k$, there is an edge between~$B_i$
and $B_j$, namely, the edge between $z_{ij}\in B_j$
and $a_i\in B_i$.
Hence, the graphs $B_j$, for $j\in \set{1,\ldots,k}$, define a depth-$2$ minor model of $K_k$ in $G$. Since $K_t\not\minor_{2}G$, this implies that $k<t$, proving Lemma~\ref{thm:alternation-rank-type-tree}.
\end{proof}

We continue with the proof of Lemma~\ref{lem:apex}. 
Fix integers $\ell\ge t^8$ and~$m$, and define $h=m+\ell$.
Let $A$ be a $1$-independent set in $G$
of size at least $h^{2t}$.

Suppose that the first case of Lemma~\ref{lem:apex} does not hold. In particular $K_t\not\minor_2 G$, so by Lemma~\ref{thm:alternation-rank-type-tree},~$\tau$ has alternation rank at most $2t-1$. From Lemma~\ref{lem:number-of-nodes} 
we conclude that $\tau$  has depth at least~$h$.
As $h=m+\ell$, it follows that either $\tau$  has a node~$w$ which contains at least $m$ letters~$\dau$, or $\tau$ has a node~$w$ which contains  at least $\ell$ letters $\son$.

Consider the first case, i.e., there is a node $w$ of $\tau$
which contains at least $m$ letters $\dau$, and let $X$
be the set of all vertices $\tau(u)$ such that $u\dau$ is a prefix of $w$. Then, by construction, $X$ is a $2$-independent set in $G$ of size at least $m$, so the second case of the lemma holds.

Finally, consider the second case, i.e., there is a node $w$ in $\tau$ which contains at least $\ell$ letters~$\son$. Let 
$Y$ be the set of all vertices $\tau(u)$ such that $u\son$ is a prefix of $w$. Then, by construction, $Y\subset A$ is a set of at least $\ell$ vertices which are mutually at distance exactly $2$ in $G$. 
Since $K_t\not\minor_4 G$ and $\ell\geq t^8$, by Lemma~\ref{lem:diversity} we infer that there is a vertex $v\in G$
with at least $\ell^{1/4}$ neighbors in $Y$.
This finishes the proof of the existential part of Lemma~\ref{lem:apex}.
We defer the proof of the algorithmic statement of the lemma to the appendix.
%, the proof above yields an algorithm which first constructs the tree $\tau$, by 
%iteratively processing each vertex $w$ of $A$ and testing whether the distance between $w$ and each vertex processed already is equal to~$2$.
%This amounts to running a breadth-first search from every vertex of $A$, which can be done in time $\Oof(|A|\cdot |E(G)|)$.
%Whenever a node with $2t$ alternations 
%is inserted to $\tau$, we can exhibit in $G$ a depth-$2$ minor model of $K_t$.
%Whenever a node with least $m$ letters $\dau$ is added to~$\tau$,
%we have constructed an $m$-independent set. Whenever a node with at least $\ell$ letters $\son$ is added to $\tau$, as argued, there must be some vertex $v\in V(G)-A$ with at least $\ell^{1/8}$ neighbors in~$A$. 
%To find such a vertex, scan through all neighborhoods of vertices $v\in A$ in the graph $G$, and then select a vertex $w\in V(G)$
%which belongs to the largest number of those neighborhoods; this can be done in time $\Oof(|E(G)|)$.
%The overall running time is $\Oof(|A|\cdot |E(G)|)$, as required.

%This finishes the proof of~Theorem~\ref{lem:apex}.

%\subsection{Proof of~Theorem~\ref{lem:engine}}
%\label{sec:engine}
%% To prove~Theorem~\ref{lem:engine}, we distinguish two special cases: the case of $r=1$ and the case $r=2$. The case of general $r$ then reduces to one of these two cases, depending on the parity of $r$, by observing that a $(2s+1)$-independent set $A$ in $G$
%% induces a $1$-independent set in $G$ with the balls of radius $s$ around the vertices of $A$ contracted, and,
%% similarly, a $(2s+2)$-independent set $A$ in $G$
%% induces a $2$-independent set in $G$ with the balls of radius $s$ around the vertices of $A$ contracted.

%With Theorem~\ref{lem:apex} proved, we can proceed with~Theorem~\ref{lem:engine}. 
%We start with the case $r=1$, then we move to the case $r=2$. 
%Next, we show how the general case reduces to one of those two cases.
%% , and, finally, we deduce~Theorem~\ref{thm:new-uqw} from~Theorem~\ref{lem:engine}.
%
%\paragraph{Case $r=1$.}
%We put $d=0$, thus we assume that $K_t\not\minor_0 G$; that is, $G$ does not contain a clique of size $t$ as a subgraph. By Ramsey's Theorem, in every graph every vertex subset of size $\binom{m+t-2}{t-1}$ contains an
%independent set of size $m$ or a clique of size $t$. Therefore, 
%taking $L(m)$ to be the above binomial coefficient yields~Theorem~\ref{lem:engine} in case $r=0$, for $S=\emptyset$. Note here that $\binom{m+t-2}{t-1}\in\Oof_{t}{(m^{{(4t+1)}^{2t}})}$.
%Moreover, such independent set or clique can be computed from $G$ and $A$ in time~$\Oof(|A|\cdot |E(G)|)$ by simulating the proof of Ramsey's theorem.
%
%\paragraph{Case $r=2$.}
%We put $d=2$, thus we assume that $K_t\not\minor_4 G$.
%We show that if $A$ is a sufficiently large $1$-independent set in a graph $G$ such that $K_t\not\minor_4 G$, 
%then there is a set of vertices~$S$ of size less than $t$ such that $A\setminus S$ contains a subset of size $m$ which is $2$-independent in $G-S$. 
%Here, by ``sufficiently large'' we mean of size of size at least $L(m)$, for $L(m)$ emerging from the proof.
%To this end, we shall iteratively apply Theorem~\ref{lem:apex} as long as  it results in the third case, 
%yielding a vertex $v$ with many neighbors in $A$. In this case, we add $v$ vertex to the set $S$, and apply the lemma again,
%restricting $A$ to $A\cap N(v)$. 
%Precise calculations follow.
%
%\newcommand{\mbull}{\widehat{m}}
%
%Fix a number $\beta>4t$. For $k\ge 0$,
%define $m_k=((k+1)\cdot m)^{(2\beta)^k}$.
%In the following we will always assume that $m\geq t^8$. 
%We will apply~Theorem~\ref{lem:apex} in the following form.
%\begin{claim}\label{cor:apex}
%	If $G$ is a graph such that $K_t\not\minor_4 G$, and
%	$A\subset V(G)$ is an $1$-independent set in $G$ which does not contain a $2$-independent set of size $m$ and satisfies $|A|\ge m_k$, for some $k\geq 1$,
%	then there exists a vertex $v\in V(G)-A$ such that $|N_G(v)\cap A| \ge m_{k-1}$.
%\end{claim}
%\begin{clproof}
%Let $\ell=(k\cdot m)^{4\cdot(2\beta)^{k-1}}$.
%Then $m\ge t^8$ implies that $\ell\ge t^8$.
%Observe that
%\[|A|\ge \left((k+1)\cdot m\right)^{(2\beta)^k}\ge\left ((m+ k\cdot m)^{4\cdot(2\beta)^{k-1}} \right)^{2t}
%\ge \left(m+(k\cdot m)^{4\cdot (2\beta)^{k-1}}\right)^{2t}=(m+\ell)^{2t}.\]
%Therefore, we may  apply Theorem~\ref{lem:apex}, yielding a vertex $v$ with at least $\ell^{1/4}=(k\cdot m)^{(2\beta)^{k-1}}=m_{k-1}$ neighbors in~$A$.
%\end{clproof}
%
%%In the following we assume that $m\geq t^8$, since we may always ask for finding a $2$-independent set of size $t^8$ instead of $m$.
%We will now find 
%a subset of $A$ of size $m$ which is $2$-independent in $G-S$, for some $S$ with $|S|<t$.
%Assume that $|A|\ge m_t$. By induction, we
% construct a sequence  $A=A_0\supseteq A_1\supseteq\ldots$ 
%of \mbox{$1$-independent} vertex subsets of $G$
%of length at most $t$
%such that $|A_i|\ge m_{t-i}$,
% as follows. Start with $A_0=A$. We maintain a set $S$ of vertices of $G$ which is initially empty, and we maintain the invariant that $A_i$ is disjoint with $S$ at each step of the construction.
%
%For $i=0,1,2,\ldots$ do as follows.
%If $A_{i}$ contains a subset of size $m$ which is $2$-independent set in $G-S$, terminate.
% Otherwise, 
% apply~\autoref{cor:apex} to the graph $G-S$ with $1$-independent set
% $A_{i}$ of size $|A_i|\ge m_{t-i}$. This yields a vertex $v_{i+1}\in V(G)-(S\cup A_i)$
% whose neighborhood in $G-S$ contains at least
% $m_{t-i-1}$ vertices of $A_{i}$.
% Define $A_{i+1}$ as the set of neighbors of $v_{i+1}$ in $A_i$, and add $v_{i+1}$
% to the set~$S$.  
%  Increment $i$ and repeat.
%
%\begin{claim}\label{claim:at-most-t}
%	The construction halts after less than $t$ steps.
%\end{claim}
%\begin{clproof}
%Suppose that the construction proceeds for $k\le t$ steps.
%By construction, each vertex~$v_i$, for $i\le k$, is adjacent in $G$
% to all the vertices of $A_{j}$, for each $i\le j\le k$. In particular, all the vertices $v_1,\ldots,v_k$ are adjacent to all the vertices of $A_{k}$
% and $|A_k|\ge m_{t-k}\ge m\ge t$.
%Choose any pairwise distinct vertices $w_1,\ldots,w_k\in A_k$ and observe that the connected subgraphs $G[\set{w_i,v_i}]$ of~$G$ yield a depth-$1$ minor model of $K_k$ in $G$.
% Since $K_t\not\minor_2 G$, we must have $k<t$.
% \end{clproof}
% 
% Therefore, at some step $k<t$ of the construction we must have obtained a $2$-independent subset $B$ of $G-S$ of size $m$. Moreover, $|S|\le k<t$.
% 
% 
% 
% This proves~Theorem~\ref{lem:engine} in the case $r=2$, for the function $L(m)$ defined as $L(m)=m_t=((t+1)\cdot m)^{\beta^{2t}}$
% for $m\ge t^8$, and $L(m)=L(t^8)$ for $m<t^8$, where $\beta>4t$ is any fixed constant.
% It is easy to see that then $L(m)\in \Oof_{t}{(m^{{(4t+1)}^{2t}})}$, provided we put $\beta=4t+1$.
% Also, the proof easily yields an algorithm constructing the sets~$B$ and $S$,
% which amounts to applying at most $t$ times the algorithm of~Theorem~\ref{lem:apex}.
% Hence, its running time  is $\Oof_{r,t}(|A|\cdot |E(G)|)$, as required.
%% \end{proof}
%
%
%\paragraph{Odd case.}
%We now prove~Theorem~\ref{lem:engine} in the case when $r=2s+1$, for some integer $s\geq 1$. We put $d=s=\frac{r-1}{2}$.
%Let $G$ be a graph such that $K_t\not\minor_s G$, and 
% let $A$ be a $2s$-independent set in $G$. Consider the graph $G'$ obtained from $G$
%by contracting the (pairwise disjoint) balls of radius $s$ around each vertex $v\in A$.
% Let $A'$ denote the set of vertices of $G'$ corresponding to the contracted balls. There is a natural correspondence (bijection) between $A$ and $A'$, where each vertex $v\in A$ is associated with the
% vertex of $A'$ resulting from contracting the ball of radius $s$ around $v$.
%From $K_t\not\minor_s G$ it follows that~$G'$ does not contain $K_t$ as a subgraph. Applying the already proved case $r=1$ to $G'$ and $A'$, we conclude that 
%provided $|A|=|A'|\ge {m+t-2\choose t-1}$, the set
% $A'$ contains a $1$-independent subset $B'$ of size $m$,
% which corresponds to a $(2s+1)$-independent set $B$ in $G$ that is contained in $A$; thus, we may put $S=\emptyset$ again.
% Hence, the obtained bound is $L(m)={m+t-2\choose t-1}$, and we have already argued that then $L(m)\in \Oof_{r,t}{(m^{{(4t+1)}^{2t}})}$.
% 
% 
% \paragraph{Even case.}
% Finally,
% we prove~Theorem~\ref{lem:engine} in the case $r=2s+2$, for some integer $s\geq 1$. We put $d=9s+4=9r/2-5$.
%Let $G$  be such that 
% $K_t\not\minor_{d} G$, and
%let $A$ be a $(2s+1)$-independent set in~$G$. Consider the graph $G'$ obtained from $G$
%by contracting the (pairwise disjoint) balls of radius $s$ around each vertex $v\in A$.
% Let $A'$ denote the set of vertices of $G'$ corresponding to the contracted balls. Again, there is a natural correspondence (bijection) between $A$ and $A'$. Note that
%this time, $A'$ is a $1$-independent set in $G'$.
%Since $G'\minor_s G$, from $K_t\not\minor_{9s+4} G$ it follows by Theorem~\ref{lem:combineminors} that $K_t\not\minor_4 G'$. Apply the already proved case $r=2$ to $G'$ and $A'$. 
%Then, provided $|A|=|A'|\ge L_t(m)$, where $L_t(m)$ is the function as defined in the case $r=2$, we infer that
% $A'$ contains a subset $B'$ of size $m$
%which is  $2$-independent in $G'-S'$, for some $S'\subset V(G')-A'$ of size less than $t$.
%Since $S'\cap A'=\emptyset$, each vertex of $S'$ originates from a single vertex of $G$ before the contractions yielding $G'$; thus, $S'$ corresponds to a
%set $S$ consisting of less than $t$ vertices of~$G$ which are at distance at least $s+1$ from each vertex in $A$.
%In turn, the set $B'$ corresponds to some subset $B$ of $A$
%which is $(2s+2)$-independent in $G-S$. Moreover, as in $G'$ each vertex of~$S'$
%is a neighbor of each vertex of $B'$,  each vertex of $S$
%has distance exactly $s+1=r/2$ from each vertex of $B$.
%
%\medskip
%An algorithm computing the sets $B$ and $S$ (in either the odd or even case) can be given as follows:
%simply run a breadth-first search from each vertex of $A$ to compute the graph $G'$ with the balls of radius  $\lfloor \frac{r-1}2 \rfloor$  around the vertices in $A$ contracted to single vertices, 
%and then run the algorithm for the case $r=1$ or $r=2$.
%This yields a running time of  $\Oof_{r,t}(|A|\cdot |E(G)|)$.
% \medskip
%  
%This finishes  the proof of~Theorem~\ref{lem:engine}.
%
%\subsection{Proof of Theorem~\ref{thm:new-uqw}}
%We now wrap up the proof of Theorem~\ref{thm:new-uqw} by iteratively applying~Theorem~\ref{lem:engine}. 
%We repeat the statement for convenience.
%
%% \setcounter{aux}{\thetheorem}
%% \setcounter{theorem}{\theuqw}
% \begin{theorem}
% For all $r,t\in \N$ there is a polynomial  $N\colon \N\to \N$ with $N(m)=
% \Oof_{r,t}{(m^{{(4t+1)}^{2rt}})}$, such that the following holds.
% Let $G$ be a graph such that $K_t\not\minor_{\lfloor 9r/2\rfloor} G$, and
% let $A\subseteq V(G)$ be a vertex subset of size at least $N(m)$, for a given $m$.
% Then there exists a set $S\subseteq V(G)$ of size $|S|<t$ and a set $B\subseteq A\setminus S$
% of size $|B|\geq m$ which is $r$-independent in $G-S$.
% Moreover, given~$G$ and $A$, such sets $S$ and $B$ can be computed in time $\Oof_{r,t}(|A|\cdot |E(G)|)$.
% \end{theorem}
%% \setcounter{theorem}{\theaux}
%\begin{proof}
%Fix integers $r,t$,  and a graph $G$ such that $K_t\not\minor_{d} G$,
%for $d=\lfloor 9r/2 \rfloor$. Let $\beta>4t$ be a fixed real. As in the proof of Theorem~\ref{lem:engine}, we suppose $m\geq t^8$; this will be taken care by the final choice of the function $N(m)$.
%Denote $\gamma=\beta^{2t}$, and
%define the function $L(m)$ as $L(m)=((t+1)\cdot m)^\gamma$.
%
%Define sequence $m_0,m_1,\ldots,m_r$ as follows:
%\begin{eqnarray*}
%m_r & = & m\\
%m_i & = & L(m_{i+1}) \qquad \textrm{for }0\leq i<m.
%\end{eqnarray*}
%A straightforward induction yields that 
%\begin{equation*}
%m_i=(t+1)^{\frac{\gamma^{r-i}-1}{\gamma-1}}\cdot m^{\gamma^{r-i}}\qquad \textrm{for all }i\in \set{0,\ldots,r}.
%\end{equation*}
%
%Suppose that $A$ is a set of vertices of $G$ such that $|A|\ge m_0=(t+1)^{\frac{\gamma^{r}-1}{\gamma-1}}\cdot m^{\gamma^{r}}$. 
%We inductively construct sequences of sets $A= A_0\supseteq A_1\supseteq \ldots \supseteq A_r$ and $\emptyset=S_0\subseteq S_1\subseteq S_2\ldots$
%satisfying the following conditions:
%\begin{itemize}
%	\item $|A_i|\ge m_i=L(m_{i+1})$,
%	\item $A_i\cap S_i=\emptyset$ and $A_i$ is $i$-independent in $G-S_i$.
%\end{itemize}
%To construct $A_{i+1}$ out of $A_i$, apply~Theorem~\ref{lem:engine} to the graph $G-S_i$ and 
%the $i$-independent set $A_i$ of size at least $L(m_{i+1})$. This yields a set $S\subseteq V(G)$ which is disjoint from $S_i\cup A_i$, and a subset $A_{i+1}$ of $A_i-S$ of size 
%at least $m_{i+1}$
%which is $(i+1)$-independent in $G-S_{i+1}$, where $S_{i+1}=S\cup S_i$. This completes the inductive construction.
%
%In particular,  $|A_r|\ge m_r=m$ and $A_r$ is a subset of $A$ which is $r$-independent in $G-S_r$.
%Observe that by construction, $|S_r|<r t/2$, as in the odd steps, the constructed set $S$ is empty, and in the even steps, it has less than $t$ elements. 
%We show that in fact we have $|S_r|<t$ using the following argument, similar to the one used in~\autoref{claim:at-most-t}.
%
%By the last part of the statement of~Theorem~\ref{lem:engine},  at the $i$th step of the construction, each vertex of the set $S$ obtained from Theorem~\ref{lem:engine}
%is at distance exactly $i/2$ from all the vertices in $A_{i+1}$ in the graph 
%$G-S_i$. 
%For $a\in A_r$, let $\overline{N}(a)$ denote the $\lfloor r/2\rfloor$-neighborhood of $a$ in $G-S_r$; note that sets $\overline{N}(a)$ are pairwise disjoint.
%The above remark implies that each vertex $v$ of the final set $S_r$ has a neighbor in the set $\overline{N}(a)$ for each $a\in A_r$.
%Indeed, suppose $v$ belonged to the set $S$ added to $S_r$ in the $i$th step of the construction; i.e. $v\in S_{i+1}\setminus S_i$.
%Then there exists a path in $G-S_i$ from $v$ to $a$ of length exactly~$i/2$, which traverses only vertices at distance less than $i/2$ from $a$.
%Since in this and further steps of the construction we were removing only vertices at distance at least $i/2$ from $a$, this path stays intact in $G-S_r$ and hence is completely contained in $\overline{N}(a)$.
%
%By assumption that $m\ge t$, we may choose pairwise different vertices $a_1,\ldots,a_t\in A_r$.
%To reach a contradiction, suppose that $S_r$ contains $t$ distinct vertices $s_1,\ldots,s_t$. 
%By the above, the sets $\overline{N}(a_i)\cup\set{s_i}$ 
%form a minor model of $K_t$ in $G$ at depth-$(\lfloor r/2\rfloor+1)$.
%This contradicts the assumption that $K_t\not\minor_d G$ for $d=\lfloor 9r/2 \rfloor$.
%Hence, $|S|<t$.
%
%Define the function  $N:\N\to\N$
%as $N(m)=(t+1)^{\frac{\gamma^{r}-1}{\gamma-1}}\cdot m^{\gamma^{r}}$
%for $m\ge t^8$ and $N(m)=N(t^8)$ for $m<t^8$; this justifies the assumption $m\geq t^8$ made in the beginning.
%Recalling that $\gamma=\beta^{2t}$ and putting $\beta=4t+1$, we
%note that $N(m)\in \Oof_{r,t}{(m^{{(4t+1)}^{2rt}})}$.
%The argument above shows that if $|A|\ge N(m)$, then 
%there is a set $S\subset V(G)$, equal to $S_r$ above,
%and a set $B\subset A$, equal to $A_r$ above,
%so that $B$ is $r$-independent in $G-S$.
%Given~$G$ and $A$, the sets~$S$ and $B$ can be computed by applying the algorithm of Theorem~\ref{lem:engine} at most~$r$ times, so in time $\Oof_{r,t}(|A|\cdot |E(G)|)$.
%This finishes the proof of~Theorem~\ref{thm:new-uqw}.
%\end{proof}



\subsection{Uniform quasi-widness for tuples}\label{sec:uqw-tuples}
We now formulate and prove an extension of~Theorem~\ref{thm:new-uqw}
which applies to sets of tuples of vertices, rather than sets of vertices. 
This more general result will be used later on in the paper. 
The result and its proof are essentially adaptations to the finite of their infinite analogues introduced by Podewski and Ziegler (cf.~\cite{podewski1978stable},  Corollary 3),
modulo the numerical bounds.

Fix a graph $G$ and a number $r\in \N$, and let $S\subseteq V(G)$ be a subset of vertices of $G$.
We say that vertices $u$ and $v$ are {\em{$r$-separated}} by $S$ in $G$ if every path of length at most~$r$ connecting $u$ and $v$ in $G$ passes through a vertex of $S$.
We extend this notion to tuples:
two tuples $\bar u,\bar v$ of vertices of $G$ are \emph{$r$-separated} by $S$ every vertex appearing in $\bar u$ is $r$-separated by $S$ from every vertex appearing in $\bar{v}$.
Finally, if $A\subseteq V(G)^d$ is a set of $d$-tuples of vertices, for some $d\in\N$,
then we say that $A$ is \emph{mutually $r$-separated} by $S$ in $G$ 
if any two distinct $\bar u,\bar v\in A$ are $r$-separated by $S$ in $G$.

\newcommand{\uqw}{\mathrm{UQW}}
\newcommand{\puqw}{\mathrm{PUQW}}
With these definitions set, we may introduce the notion of uniform quasi-wideness for tuples.

\begin{definition}
Fix a class $\cal C$ and numbers $r,d\in\N$.
For a function $N\from\N\to\N$
and number $s\in\N$,
we say that $\cal C$ satisfies property
$\uqw^d_r(N,s)$ if the following condition holds:
   \begin{quote}\itshape 
      for every $m\in \N$ and every subset 
     $A\subseteq V(G)^d$ with $|A|\ge N(m)$, there is a set $S\subset V(G)$ with $|S|\le s$ and a subset $B\subset A$ with $|B|\ge m$ which is mutually $r$-separated by $S$ in $G$.
   \end{quote}   
    We say that $\cal C$ satisfies property $\uqw^d_r$ if  $\cal C$ satisfies property $\uqw^d_r(N,s)$ for 
	some $N\from\N\to\N$ and $s\in\N$.
	If moreover one can take $N$ to be a polynomial,
	then we say that $\cal C$ satisfies property $\puqw^d_r$.
\end{definition}

%When $d=1$, we omit it from the superscripts.
%  Note that there is a slight discrepancy 
%  in the definition of uniform quasi-wideness 
%  and the property of satisfying $\uqw_r$, for all $r\in \N$.
%  This is due to the fact that in the original definition,
%  the set $B$ must be disjoint from $S$,
%  whereas in the property $\uqw_r$, 
%  some vertices of $S$ may belong to $B$. This distinction is inessential when it comes to dimension $1$, since $|S|\le s_r$ for some constant $s_r$,
%  so passing from one definition to the other requires 
%  modifying the function $N_r$ by an additive constant~$s_r$.
%In particular, a class of graphs $\cal C$ is uniformly quasi-wide if and only if it 
%	satisfies $\uqw_r$, for all $r\in \N$.  
%  However, generalizing to tuples of dimension $d$ requires the use of the definition above, where the tuples in~$B$ are allowed to contain  vertices which occur in $S$. 
%  For example, if the graph $G$ is a star with many arms and $A$ consists of all pairs of adjacent vertices in $G$, then $S$
%  needs to contain the central vertex of $G$,
%  and therefore $S$ will contain a vertex from every tuple in $A$. We may take $B$ to be equal to $A$ in this case.
  
%\medskip
%	Using the above terminology, Theorem~\ref{thm:new-uqw}
%	states that for every fixed $r\in\N$, if there is a number $t\in\N$
%	such that $K_t\not\minor_{\lfloor 9r/2\rfloor} G$ for all $G\in \cal C$,
%	then $\cal C$ satisfies $\puqw_r$,
%	and more precisely $\uqw_r(N_r,s_r)$
%	for a polynomial $N_r\from\N\to\N$ and number $s_r\in \N$, where $N_r$ and $s_r$ can be computed from $r$ and $t$.
	The following result provides a generalization of Theorem~\ref{thm:new-uqw} to higher dimensions.

\begin{theorem}[$\star$]\label{thm:uqw-tuples}If $\cal C$
	is a nowhere dense class of graphs,
	then for all $r,d\in\N$,
	the class $\cal C$ satisfies
	 $\puqw^d_r$.
	More precisely, for any class of graphs $\cal C$ and numbers $r,t\in\N$,
	if  	${K_t\not\minor_{18r} G}$ for all $G\in \cal C$,
then for all $d\in \N$ the class $\cal C$ satisfies $\uqw^d_{r}(N^d_r,s^d_r,)$	for 
some number $s^d_r\in \N$ and polynomial $N^d_r\from \N\to\N$ that can be computed given $r$, $t$, and $d$.
\end{theorem}

%
%Theorem~\ref{thm:uqw-tuples} is an immediate consequence  of~Theorem~\ref{thm:new-uqw} (or~Theorem~\ref{thm:krs} if only the first part of the statement is concerned)
%and of the following result.
%
%\begin{proposition}\label{prop:uqw-tuples}
%For all $r,d\in\N$,
%if $\cal C$ satisfies $\uqw_{2r}(N_{2r},s_{2r})$ 
%for some $s_{2r}\in\N$ and \mbox{$N_{2r}\colon\N\to \N$},
%then $\cal C$
%satisfies $\uqw^d_r(N^d_r,s^d_r)$
%for  $s^d_r=d\cdot s_{2r}$ and 
%function $N^d_r\colon \N\to \N$ defined as $N^d_r(m)=f^d((d^2+1)\cdot m)$, where $f(m')=m'\cdot N_{2r}(m')$ and $f^d$ is the $d$-fold composition of $f$ with itself.
%\end{proposition}
%
%
%The rest of~\autoref{sec:uqw-tuples} is devoted to the proof of \autoref{prop:uqw-tuples}.
%Fix a class $\cal C$
%such that $\uqw_{2r}(N_{2r},s_{2r})$ holds for some number $s_{2r}\in \N$ and  function $N_{2r}\from \N\to \N$.
%We also fix the function $f$ defined in the statement of \autoref{prop:uqw-tuples}.
%
%%In the proof below, we will invoke property $\uqw_{2r}$ for $\cal C$,
%%but also property $\uqw_r$, which also holds for $\cal C$, as witnessed e.g. by the function $N_r(\cdot)\coloneqq N_{2r}(\cdot)$ and constant $s_r\coloneqq s_{2r}$.
%
%\medskip
%
%	Let us fix dimension $d\in \N$, radius $r\in \N$, and graph $G\in \cal C$.
%        For a coordinate $i\in\set{1,\ldots,d}$, by $\pi_i\colon V(G)^d\to V(G)$ we denote the {\em{projection}} onto the $i$th coordinate; that is,
%        for $\bar{x}\in V(G)^d$ by $\pi_i(\bar{x})$ we denote the $i$th coordinate of $\bar{x}$.
%        
%        %and set of tuples $A\subseteq V(G)^d$, by
%        %$\pi_i(A)$ we denote the {\em{multiset}} of vertices appearing on the $i$th coordinate of the tuples in $A$.
%        %That is, each tuple $\bar u\in A$ contributes with one element to $\pi_i(A)$, this element being the vertex on the $i$th coordinate of $\bar u$.
%        %The notion of mutual $r$-separation is naturally extended to multisets: a multiset $M$ of vertices of $G$ is {\em{mutually $r$-separated}} by a vertex subset $S\subseteq V(G)$ in $G$ if for
%        %any two distinct vertices $u,v$ drawn from $M$, it holds that $u$ and $v$ are $r$-separated by $S$ in $G$.
%        %Note that in case $M$ contains more than one copy of some vertex $u$, any set $S$ that mutually $r$-separates $S$ has to contain $u$.
%        %By a slight abuse of notation, for a single tuple $\bar{u}$ by $\pi_i(\bar{u})$ we denote the $i$th coordinate of $\bar{u}$.
%	
%	Our first goal is to find a large subset of tuples that are mutually $2r$-separated by some small~$S$ on each coordinate separately.
%	Note that in the following statement we ask for $2r$-separation, instead of $r$-separation.
%
%\begin{lemma}\label{lem:step1} For all $r,m\in \N$ and $A\subset V(G)^d$ with $|A|\ge f^d(m)$,
%	there is a set $B\subset A$ with $|B|\ge m$ and a set $S\subset V(G)$ with $|S|\le d\cdot s_{2r}$ 
%	such that for each coordinate $i\in\set{1,\ldots,d}$ and all distinct $\bar x,\bar y\in B$,
%        the vertices $\pi_i(\bar x)$ and $\pi_i(\bar y)$ are $2r$-separated by $S$. 
%\end{lemma}
%\begin{proof}
%We will iteratively apply the following claim.
%
%%Let $f\colon \N\to \N$ be defined as $f(m)=N(r,m)\cdot m$ for $m\in\N$.
%
%\begin{claim}\label{claim:ith-coord}
%Fix a coordinate $i\in\set{1,\ldots,d}$, an integer $m'\in\N$, and a  set $A'\subset V(G)^d$ with  $|A'|\ge f(m')$.
%Then there is a set $B'\subset A'$ with $|B'|\ge m'$
%and a set $S'\subset V(G)$ with $|S'|\le  s_{2r}$, such that for all distinct $\bar x,\bar y\in B$,
%the vertices $\pi_i(\bar x)$ and $\pi_i(\bar y)$ are $2r$-separated by $S$.
%\end{claim}
%\begin{clproof}
%We consider two cases, depending on whether $|\pi_i(A')|\geq N_{2r}(m')$.
%
%Suppose first that $\pi_i(A')$ contains at least $N_{2r}(m')$ distinct vertices.
%Then we may apply the property $\uqw_{2r}$ to $\pi_i(A')$, yielding sets $S'\subset V(G)$ and $X\subseteq \pi_i(A')$
%such that $|X|\ge m'$, $|S'|\le s_{2r}$, and $X$ is mutually $2r$-separated by $S'$ in $G$. 
%Let $B'\subseteq A'$ be a subset of tuples constructed as follows: for each $u\in X$, include in $B'$ one arbitrarily chosen tuple $\bar x\in A'$ such that the $i$th coordinate of $\bar x$ is $u$.
%Clearly $|B'|=|X|\ge m'$ and for all distinct $\bar x,\bar y\in B'$, we have that $\pi_i(\bar x)$ and $\pi_i(\bar y)$ are different and $2r$-separated by $S'$ in $G$; this is because $X$ is mutually $2r$-separated by $S'$
%in $G$. Hence $B'$ and $S'$ satisfy all the required properties.
%
%Suppose now that $|\pi_i(A')|<N_{2r}(m')$. 
%Then choose a vertex $a\in \pi_i(A')$ for which the pre-image $\pi_i^{-1}(a)$ has the largest cardinality.
%Since $|A'|\geq f(m')=m'\cdot N_{2r}(m')$, we have that 
%$$|\pi_i^{-1}(a)|\geq \frac{|A'|}{|\pi_i(A')|}\geq \frac{m'\cdot N_{2r}(m')}{N_{2r}(m')}=m'.$$
%Hence, provided we set $S'=\set{a}$ and $B'=\pi_i^{-1}(a)$, we have that $B'$ is mutually $2r$-separated by~$S'$, $|B'|\geq m$, and $|S'|=1$.
%\end{clproof}
%
%We proceed with the proof of Theorem~\ref{lem:step1}.
%Let $A\subset V(G)^d$ be such that $|A|\ge f^d(m)$.
%We inductively define subsets $B_0\supseteq B_1\supseteq \ldots \supseteq B_d$ of $A$ and sets $S_1,\ldots,S_d\subseteq V(G)$ as follows.
%First put $B_0=A$. Then, for each $i=1,\ldots,d$,
%let $B_{i}$ and $S_i$ be the $B'$ and $S'$ obtained from \autoref{claim:ith-coord} applied to the set of tuples $B_{i-1}\subset V(G)^d$, the coordinate $i$, and $m'=f^{d-i}(m)$. 
%It is straightforward to see that the following invariant holds for each $i\in \set{1,\ldots,d}$: $|B_i|\ge f^{d-i}(m)$ and for all $j\leq i$
%and distinct $\bar x,\bar y\in B_i$, the vertices $\pi_j(\bar x)$ and $\pi_j(\bar{y})$ are $2r$-separated by $S_1\cup\ldots\cup S_i$ in $G$.
%In particular, by taking $B=B_d$ and $S=S_1\cup\ldots \cup S_d$, we obtain that $|B|\ge m$, $|S|\le d\cdot s_{2r}$, and $B$ and $S$ satisfy the condition requested in the lemma statement.
%\end{proof}
%
%The next lemma will be used to turn mutual $2r$-separation on each coordinate to mutual $r$-separation of the whole tuple set.
%
%\begin{lemma}\label{lem:step2}
%	Let $B\subset V(G)^d$ and $S\subset V(G)$ be such that 
%   for each $i\in \set{1,\ldots,d}$ and all distinct $\bar{x},\bar{y}\in B$, the vertices $\pi_i(\bar{x})$ and $\pi_i(\bar{y})$ are $2r$-separated by $S$ in $G$.
%	Then there is a set $C$ with $C\subset B$ and $|C|\geq\frac{|B|}{d^2+1}$
%	such that $C$ is mutually $r$-separated by $S$ in $G$.
%\end{lemma}
%\begin{proof}
%Let $C$ be a maximal subset of $B$ that is mutually $r$-separated by $S$ in $G$.
%By the maximality of $C$, with each tuple $\bar a\in B-C$ we may associate a tuple $\bar b\in C$ and a pair of indices $(i,j)\in \set{1,\ldots,d}^2$ that witness that $a$ cannot be added to $C$, namely
%$\pi_i(\bar a)$ and $\pi_j(\bar b)$ are not $r$-separated by $S$ in $G$.
%Observe that two different tuples $\bar a,\bar a'\in B-C$ cannot be associated with exactly the same $\bar b\in C$ and same pair of indices $(i,j)$.
%Indeed, then both $\pi_i(\bar a)$ and $\pi_i(\bar a')$ would not be $r$-separated from $\pi_j(\bar b)$ by $S$ in $G$, 
%which would imply that $\pi_i(\bar a)$ and $\pi_i(\bar a')$ would not be $2r$-separated from each other by $S$,
%a contradiction with the assumption on $B$.
%Hence, $|B-C|$ is upper bounded by the number of tuples of the form $(\bar b,i,j)\in C\times \set{1,\ldots,d}^2$, which is $d^2|C|$.
%We conclude that $|B-C|\leq d^2|C|$, which implies $|C|\geq \frac{|B|}{d^2+1}$.
%\end{proof}
%
%\begin{comment}
%\begin{proof}
%We construct a sequence $C_0\subset C_1\subset \ldots$ of subsets of $B$ which are mutually $r$-independent in $G-S$, as follows.
%
%We start with $C_0=\emptyset$. Suppose that $C_s\subset B$ is 
% already constructed for some $s\ge 0$
% and is mutually $r$-independent in $G-S$; we construct $C_{s+1}$. With each element $a\in B-C_s$,
%we associate an arbitrarily chosen function $f_a\colon \set{1,\ldots,d}^2\to C_s\cup \set{\bot}$
%with the following properties:
%\begin{itemize}
%	\item If $f_a(i,j)=b$ then the $i$th coordinate of $a$
%	and the $j$th coordinate of $b$ are not $r$-separated by $S$.
%	\item If $f_a(i,j)=\bot$ then there is no element $b\in C_s$ 
%	such that the $i$th coordinate of $a$ and the $j$th coordinate of $b$ are at not $r$-separated by $S$.
%\end{itemize}
%Observe that whenever $a_1, a_2$ are two distinct elements of $B-C_s$,
%then for all $i,j\in \set{1,\ldots,d}^2$, the values $f_{a_1}(i,j)$ and $f_{a_2}(i,j)$
%cannot be equal to the same element $b\in C_s$:
%otherwise, we would have that the $i$th coordinate of $a_1$
%and the $i$th coordinate of $a_2$ are not $2r$-separated by $S$, which is impossible by the assumption on $B$.
%In particular, if $|B-C_s|> |C_s|\cdot d^2$
%then there must be some element  $a\in B-C_s$  
%such that $f_a(i,j)=\bot$  for all $i,j\in\set{1,\ldots,d}$.
%Let $C_{s+1}=C_s\cup \set a$.
%By construction, $C_{s+1}$ is mutually $r$-independent in $G-S$.
%
%We may repeat the construction as long as $|B|>|C_s|\cdot (d^2+1)=s\cdot (d^2+1)$, and we stop when this inequality no longer holds. Define the set $C$ as the last constructed set $C_s$.
%By construction, $|C_s|=s\ge 
%\frac{|B|}{d^2+1}$.	
%\end{proof}
%\end{comment}
%
%To finish the proof of \autoref{prop:uqw-tuples},
%given a set $A\subset V(G)^d$ and integer $m\in\N$,
%first apply 
%Theorem~\ref{lem:step1} 
%  with $m'= m\cdot (d^2+1)$.
% Assuming that $|A|\ge f^d(m')$, 
%we obtain a set $B\subseteq A$ with $|B|\ge m\cdot (d^2+1)$ and a set $S\subset V(G)$ with $|S|\le d\cdot s_{2r}$,
%such that for each $i\in \set{1,\ldots,d}$ and all distinct $\bar{x},\bar{y}\in B$, the vertices $\pi_i(\bar{x})$ and $\pi_i(\bar{y})$ are $2r$-separated by $S$ in $G$. 
%Then, apply Theorem~\ref{lem:step2} to $B$ and $S$, yielding a set $C\subset B$ which is mutually $r$-separated by $S$ and has size at least $m$. 
%This concludes the proof of \autoref{prop:uqw-tuples}.
%
%\begin{comment}
%\begin{remark}\label{rem:local-tuples}
%  A detailed analysis of the presented proof
%  allows to obtain a stronger statement than in
%   \autoref{prop:uqw-tuples}, which we now describe.
%   For $d,r,m\in\N$,
%  let
%    $\textrm{UQW}(d,r)$ denote the following statement:
%   \begin{quote}\itshape There exists a constant $s^d_r\in \N$ and a polynomial $N^d_r\from\N\to\N$ such that 
%      for all $m\in \N$ and all subsets 
%     $A\subset V(G)^d$ with $|A|\ge N^d_r(m)$ there is a set $S\subset V(G)$ of size $|S|\le s^d_r$ and a subset $B\subset A$ of size $|B|\ge m$ which is mutually $r$-independent in $G-S$.
%   \end{quote}   
%   Our proof shows that  the statement $\textrm{UQW}(d,r)$
%   can be concluded from the statement $\textrm{UQW}(1,4r)$.
%     
%This is because in our proof,  we have~obtained:
%  $$
%  N^d_{2r}(m)=(N_{4r})^d(m(d^2+1))\qquad\textrm{and}\qquad s^d_{2r}=d\cdot s_{4r}.
%  $$
%  Thus, when establishing the values of $N^d_{2r}(m)$ and $s^d_{2r}$, we refer to the quasi-wideness of $\CCC$ only by using numbers $s_{4r}$ and $N_{4r}(m')$ for $m'\in \N$.
%  
%  On the other hand, by Theorem~\ref{thm:new-uqw}, the statement $\textrm{UQW}(1,4r)$
%  follows from the existence of a number $t\in\N$ such that $K_t\not\minor_{10r} G$ for $G\in \CCC$.
%  To summarize, for all $r\in\N$, the existence of a number $t\in\N$ such that $K_t\not\minor_{10r} G$ for $G\in \CCC$ implies the statement $\textrm{UQW}(d,r)$,
%  for all $d\in\N$.
%    %
%  % By Theorem~\ref{thm:new-uqw}, it suffices to assume that $K_t\not\minor_{10r} G$ to have $s(4r)\leq t$ and $N(4r,m')\leq c(r,t)\cdot (m')^{(8t+1)^{2rt}}$ for some computable function $c(r,t)$.
%  % Hence, this supposition alone, instead of full quasi-wideness of $\CCC$, is sufficient to claim that the conclusion of Theorem~\ref{thm:new-uqw} holds with
%  % $s^d(2r)$ bounded by a computable function of $t$, $d$, and $q$, and $N^d(2r,m)$ bounded by a computable function of $m$, $t$, $d$, and $q$.
%\end{remark}
%\end{comment}