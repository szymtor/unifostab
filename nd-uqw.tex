\section{From nowhere denseness to uniform quasi-wideness}\label{sec:uqw}

This section is devoted to the proof of~\cref{thm:new-uqw}. 
Even though we prove a purely graph theoretic statement, 
it is useful to see our methods in a more general model-theoretic context. 


%\begin{definition}
%If $\tau$ is a word over an alphabet $\Sigma$ and
%$a\in \Sigma$, then $\tau\cdot a$ denotes the concatenation of~$\tau$
%and $a$.  The \emph{branching index} of a formula $\psi(\tup{x},\tup{y}$
%over a graph $G$ is the largest number
%$\ell$ such that there are tuples of elements
%$\tup{u}_{\sigma_1},\ldots, \tup{u}_{\sigma_{2^\ell}}\in V(G)$, indexed by the
%words over the alphabet $\{0,1\}$ of length exactly $\ell$, and
%tuples of elements $\tup{v}_{\tau_1},\ldots, \tup{v}_{\tau_{2^\ell-1}}$, indexed by the
%words over $\{0,1\}$ of length strictly smaller than $\ell$, such that
%if $\tau_j\cdot a$ is a (not necessarily proper) prefix of~$\sigma_i$, then
%$G\models \psi(\tup{u}_{\sigma_i},\tup{v}_{\tau_j})$ if, and only if, $a=1$. The tuples
%$\tup{u}_{\sigma_1},\ldots, \tup{u}_{\sigma_{2^\ell}}\in V(G)$ are called the 
%\emph{leaves} of the tree, the tuples $\tup{v}_{\tau_1},\ldots, \tup{v}_{\tau_{2^\ell-1}}$
%are its \emph{inner nodes}. Intuitively, a leaf $\tup{u}$ is connected to its 
%predecessors~$\tup{v}$ such
%that $\tup{u}$ is a \emph{right successor} of $\tup{v}$ and not to its predecessors such that 
%it is a \emph{left
%successor}. 
%\end{definition}
     

%\begin{lemma}[\cite{hodges1993model}, Lemma 6.7.9, p.\
%  313]\label{lem:branching}
%  Let $\psi(\tup{x},\tup{y})$ be a formula and let $G$ be a graph. 
%  If $\psi$ has branching index~$k$ over $G$, 
%  then~$\psi$ has ladder index smaller than $2^{k+1}$ over $G$. 
%  If $\psi$ has  has
%  ladder index $k$ over $G$, then $\psi$ has branching index smaller than
%  $2^{k+2}-2$ over $G$.
% \end{lemma}

In the proof of~\cref{thm:malshelah} one constructs a tree, in which
elements are iteratively classified according to their types. The depth of 
the type tree is directly related to the ladder index of the formula, 
more precisely to the \emph{branching index} of the formula. We refrain
from defining the branching index here and refer to~\cite{malliaris2014regularity}
and to Lemma 6.7.9 of the textbook \cite{hodges1993model}, which relates
ladder index and branching index. In this section we will be working with 
formula $\phi(x,y)$ with exactly two free variables only. 

Let $T$ be a binary tree, where each vertex (except the root) is 
marked as a left or right successor of its predecessor. We call $w$ 
a \emph{left (right) descendant} of $v$ if the first successor on the unique
$v$-$w$ path in $T$ is a left (right) successor. A root-leaf path in $T$ is called
$k$-alternating if it contains $k+1$ nodes $a_1,\ldots, a_{k+1}$ (which appear
in that order on the path, possibly not consecutively) such that $a_{i+1}$ is a left successor of $a_i$
if and only if $a_{i+2}$ is a right successor of $a_{i+1}$, $1\leq i\leq k-2$. 
The \emph{alternation rank} of $T$ is the largest number $t$ such that 
$T$ contains a $t$-alternating path of length $t$.

\begin{lemma}\label{lem:number-of-nodes}
Let $T$ be a tree of alternation rank $t$. If
$T$ has height at most~$h$, then $T$ has at most $(2h+2)^{t+1}$
vertices. 
\end{lemma}
\begin{proof}
Denote by $n_\ell^s$ the number of vertices a binary tree without a 
an $s$-alternating path can have. Then we have $n_0^s\leq 1$ for all $s\geq 1$
and $n_\ell^1\leq 2\ell-1$ for all $\ell$, and 
\begin{align*}
n_\ell^s\leq 2(n_{\ell-2}^{s-1}+n_{\ell-3}^{s-1}+\ldots + n_{0}^{s-1})+2\ell-1
\end{align*}
for all $s\geq 2,\ell\geq 1$. In the last inequality, we count the two paths of length
$\ell$ staring in the root which have no alternations. All nodes branching from 
these paths have one less alternation available. 
Now it is easy to check that the function $(2\ell+2)^s$ satisfies these requirements. 
\end{proof}

Let $\phi(x,y)$ be a formula with $2$ free variables and let $(v_1,\ldots, v_n)$
be a sequence of vertices of $G$. The \emph{type tree}
of $\phi$ over $(v_1,\ldots,v_n)$ is a binary tree which is constructed recursively as 
follows. We make $v_1$ the root of the tree. Assume that $v_1,\ldots, v_i$
have been inserted to the tree. We follow a root-leaf path to find the
position for the next vertex $v_{i+1}$. If $G\models\phi(v_j,v_{i+1})$, we
follow the right branch at $v_j$, otherwise we follow the left branch. If there is
no such successor, we insert $v_{i+1}$ as a right or left child of $v_j$, 
accordingly. 

\begin{lemma}
Let $\phi(x,y)$ be a formula with $2$ free variables and let
$(v_1,\ldots, v_n)$ be a sequence of elements of $G$. Then the 
alternation rank of the type tree of $\phi$ over $(v_1,\ldots, v_n)$
is at most twice the ladder index of $\psi$. 
\end{lemma}
\begin{proof}
An alternating path in $T$ is a $\phi$-ladder.
\end{proof}

We are now ready to prove~\cref{thm:new-uqw}. As in the 
original proof of Ne\v{s}et\v{r}il and Ossona de Mendez, the main 
idea is to find a few elements to delete such that a large
subset $A'\subseteq A$ is $2$-independent. We can then 
contract the disjoint neighborhoods of these elements and
continue with the depth-$1$ minor $G'$, where we 
identify $A'$ in $G$ with a set of vertices in $G'$. A $2$-independent
subset of $A'$ in $G'$ will be a $4$-independent set in $G$, 
and so on, until we finally arrive at an $r$-independent set. 

We can use Ramsey's Theorem to first find an independent subset
of $A$. 

\begin{lemma}
Let $G$ be a graph such that $K_t\not\subseteq G$. If $A$
has size at least $\binom{m+t-2}{t-1}$, then there exists
a subset $B\subseteq A$ of size at least $m$ which is an
independent set. 
\end{lemma}

We now consider the formula 
$\phi_2(x,y)\coloneqq \dist_G(x,y)\leq 2$, for which we can give an explicit 
bound on the alternation rank in its type tree. 

\begin{theorem}\label{thm:alternation-rank-type-tree}
Let $G$ be a graph such that $K_t\not\minor_2G$ and let
$(v_1,\ldots, v_n)$ be an enumeration of an independent set 
in $G$. Then the alternation rank of the type tree of $\phi$ 
over $(v_1,\ldots, v_n)$ is at most~$2t$. 
\end{theorem}
\begin{proof}
Assume that we find an alternating path $a_1,b_1,\ldots, a_k,b_k$ 
in the type tree, where $b_1$ is right of $a_1$. Otherwise, we consider
the path $b_1,a_2,b_2,\ldots,a_k, b_k$. Because $(v_1,\ldots, v_n)$ is independent in $G$, 
none of these vertices are adjacent and all of the vertices on the
paths of length $2$ which cause the creation of edges in the type
tree are distinct from $a_1,\ldots, b_k$. 

\begin{claim}
Every vertex $a_i$ is connected to every $b_j$, $j\geq i$,
via a vertex $z_{ij}$ which is not connected to any~$b_\ell$, $\ell\neq j$. 
\end{claim}

\noindent\textit{Proof.} Because $b_j$, $j\geq i$, is right of $a_i$, there is 
an element $z_{ij}$ connected to $a_i$ and $b_j$. If $z_{ij}$ was 
connected to $b_\ell$, $\ell\neq j$, then $b_\ell$ would be adjacent 
to $b_j$ in the type tree, which it is not. \hfill$\lrcorner$

\bigskip
Now let $X_j\coloneqq \{b_j\}\cup\{a_j\}\cup\{z_{ij} : 1\leq i\leq j\}$
for all $1\leq j\leq k$. $X_j$ is connected and has radius $2$. As $a_i$
is connected to $z_{ij}$ and $a_i\in X_i$ and $z_{ij}\in X_j$ for $i\neq j$, 
we have constructed a complete graph $K_k$ as depth-$2$ minor. Hence 
$k\leq t-1$, which proves the claim. 
\end{proof}

Note that the above proof does not give a proof that the ladder
index of the distance-$2$ formula is at most $2(t-1)$. In the type
tree we have the stronger statement that the vertices $b_i$
are not connected by a path of length $2$. 
We make no statement about the connections
of these elements in the ladder. 

We need one more lemma. 

\begin{lemma}[\cite{gajarsky2013kernelization}]\label{lem:diversity}
  Let $\CCC$ be a nowhere dense class of graphs. For every
  $\epsilon>0$ there is an integer~$n_0$ such that if
  $A\subseteq V(G)$ for $G\in \CCC$ with $\abs{A}\geq n_0$,
  then \[\abs{\{N(v)\cap A : v\in V(G)\}}\leq \abs{A}^{1+\epsilon}.\]
\end{lemma}

\begin{lemma}
Let $G$ be a graph such that $K_t\not\minor_2 G$ and let $m\in \N$. 
Let $\epsilon>0$ and let $n_0$ be the corresponding constant 
of~\cref{lem:diversity}.
Let $A$ be an independent set in $G$ of size at least $(m+n_0)^x$. 
Then either $A$ contains a $2$-independent set of size $m$ or
there is a subset $A'\subseteq A$ of size at least~$m$ and
a vertex $v\in V(G)$ which is adjacent to all vertices of $A'$. 
\end{lemma}
\begin{proof}
Fix any enumeration $(v_1,\ldots, v_{m^x})$ of $A$ and
build the type tree $T$ of the formula $\dist(x,y)\leq 2$ over
$(v_1,\ldots, v_{m^x})$. According to~\cref{thm:alternation-rank-type-tree}, 
the alternation rank of $T$ is at most $2t$. According 
to~\cref{lem:number-of-nodes}, $T$ has depth at least $h$, hence, 
a path of length at least $m^{3t}$. We choose the longest set $X$
of vertices on that path without alternations, which has hence length
at least $m^{3t}$. 

Either, $X$ is a set with all its successors on the left, then $X$ 
is a $2$-independent set and we are done.
Otherwise, all vertices of $X$ are at distance
$2$. Let $m=X$ and assume that $m$ is larger than $n_0$ to
be defined for neighbourhood complexity.
We claim that we find an element which is connected to at least $m^{1/3}$
of the vertices of $X$. Otherwise, every vertex
can create only $m^{2/3}$ connections, however, we need
to create $m^2$ connections. Hence, we need $m^{4/3}$ vertices
to create all connections. We have neighbourhood complexity
$m^{1+\epsilon}$ though, a contradiction. 

\end{proof}



