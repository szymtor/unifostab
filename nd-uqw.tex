\section{From nowhere denseness to uniform quasi-wideness}\label{sec:uqw}

This section is devoted to the proof of \cref{thm:new-uqw}. 
In the presentation we focus on proving the existential statement, and at the end we briefly argue how the proof can be turned into an algorithm with the promised running time guarantee.
We first recall necessary preliminaries from graph theory. 

\paragraph*{Preliminaries.}
All graphs in this paper are finite, undirected and simple, that is, 
they do not have loops or parallel edges. Our notation is standard,
we refer to~\cite{diestel2012graph} for more background on 
graph theory. 
We write $V(G)$ for the vertex set of a graph $G$ and
$E(G)$ for its edge set. 
The {\em{distance}} between vertices $u$ and $v$ in $G$, denoted $\dist_G(u,v)$, is the length of a shortest path between $u$ and $v$ in~$G$.
If there is no path between $u$ and $v$ in $G$, we put $\dist_G(u,v)=\infty$.
For a vertex $u$ and nonnegative integer $s$, by $N_s[u]$ we denote the {\em{$s$-neighborhood of $u$}} which comprises vertices at distance at most $s$ from $u$.

A {\em{minor model}} of a graph $H$ in $G$ is a family $(I_u)_{u\in V(H)}$ of pairwise vertex-disjoint connected subgraphs of $G$, called {\em{branch sets}},
such that whenever $uv$ is an edge in~$H$, there are $u'\in I_u$ and $v'\in I_v$ for which $u'v'$ 
is an edge in $G$.
The graph $H$ is a {\em{depth-$r$ minor}} of $G$, denoted $H\minor_rG$, if there is a minor model
$(I_u)_{u\in V(H)}$ of~$H$ in $G$ such that each $I_u$ has radius at most $r$.

A class $\CCC$ of graphs is \emph{nowhere dense} if there is a function 
$t\colon \N\rightarrow \N$ such that for all $r\in \N$ it holds that $K_{t(r)}\not\minor_r G$
for all $G\in \CCC$. 

A set $B\subseteq V(G)$ is called {\em{$r$-independent}} in a graph $G$ if for all
distinct $u,v\in B$ we have $\dist_G(u,v)>r$.
A class $\CCC$ of graphs is \emph{uniformly quasi-wide} if there are
functions $N\colon \N\times\N\rightarrow \N$ and $s:\N\rightarrow \N$ such
that for all $r,m\in \N$, all graphs $G\in \CCC$, and all subsets $A\subseteq V(G)$ of size $\abs{A}\geq N(r,m)$, there is a set
$S\subseteq V(G)$ of size $\abs{S}\leq s(r)$ and a set
$B\subseteq A\setminus S$ of size $\abs{B}\geq m$ which is $r$-independent in
$G-S$. 

\paragraph{General strategy.}
Our proof follows exactly the same lines as the original proof of Ne\v set\v ril and Ossona de Mendez, with the difference that in one technical lemma (\cref{lem:apex} below), we improve the bounds significantly by replacing a Ramsey argument by a more subtle argument about the neighborhood
complexity in nowhere dense graph classes, due to~\cite{gajarsky2017kernelization}.
For sake of completeness, we present the entire proof.


The general idea behind the proof of 
\cref{thm:new-uqw} is to prove the following 
weaker variant, for each radius $r$:

\begin{lemma}\label{lem:engine}
For all $r,t\in \N$ there exists a function $L\colon \N\to \N$
and a number $d$ such that the following holds.
Given any graph $G$ such that $K_t\not\minor_d G$ and
$(r-1)$-independent set of vertices $A\subset V(G)$ of size at least $L(m)$, there is a set $S\subseteq V(G)$ of size at most $t$ such that $A-S$ contains a subset of size $m$ which is $r$-independent in $G-S$.
\end{lemma}

Note that the main difference between the statement of this lemma and~\cref{thm:new-uqw} is that here we assume that 
$A$ is already $(r-1)$-independent, whereas in~\cref{thm:new-uqw} we make no assumptions on $A$. However, the theorem then follows by an easy induction: to find an $r$-independent set in a set $A$, we start with 
the very large ($0$-independent) set $A_0=A$, in which (after removing 
$L(1)$ vertices from $G$)
we find a very large $1$-independent subset $A_1$, in which (after further removing $L(2)$ vertices from $G$) we find a large $2$-independent subset $A_2$, etc., until arriving at an $r$-independent subset $A_r$ of size $m$.

To prove~\cref{lem:engine}, we distinguish two special cases: the case of $r=0$ and the case $r=1$. The case of general $r$ then reduces to one of these two cases, depending on the parity of $r$, since a $(2s+1)$-independent set $A$ in $G$
induces a $1$-independent set in $G$ with the balls of radius $s$ around the vertices of $A$ contracted, and,
similarly, a $(2s+2)$-independent set $A$ in $G$
induces a $2$-independent set in $G$ with the balls of radius $s$ around the vertices of $A$ contracted.

We start by proving~\cref{lem:engine} in the case $r=0$,
then we prove it in the case $r=1$. Next we show how the general case reduces to one of those two, and, finally, we deduce~\cref{thm:new-uqw} from~\cref{lem:engine}.

\paragraph{Case $r=0$.}
We prove~\cref{lem:engine} in the case $r=0$.
Let $d=0$, so $K_t\not\minor_d G$ amounts to saying that $G$
does not contain a clique of size $t$. By Ramsey's Theorem, in any graph every set of size $\binom{m+t-2}{t-1}$ contains an
independent set of size $m$ or a clique of size $t$. Therefore, 
taking $L(m)$ as the above binomial coefficient yields~\cref{lem:engine} in case $r=0$, by taking $S=\emptyset$.

\paragraph{Case $r=1$.}
We prove~\cref{lem:engine} in the case $r=1$.
For the second, more intricate case, 
we employ one more result: the bound on the number of distinct neighborhoods in a graph from a nowhere dense class.

\begin{lemma}[adaptation of Lemma 4.11 in \cite{gajarsky2017kernelization}]\label{lem:diversity}
Let $G$ be a graph such that $K_t\not\minor_{1} G$ for some constant $t\in \N$. 
Then for every $\epsilon>0$ there exists $n_0$, depending only on $t$ and $\epsilon$, such that for all $A\subseteq V(G)$ with $|A|\geq n_0$ it holds that
\[\abs{\{N(v)\cap A \colon v\in V(G)\}}\leq\abs{A}^{1+\epsilon}.\]
\end{lemma}

We remark that the proof of \cref{lem:diversity} uses only the fact that
nowhere dense classes of graphs do not have dense 
shallow minors~\cite{dvorak2007asymptotical,jiang2011compact}, and does not rely on any non-constructive arguments from the stability theory.
From \cref{lem:diversity} we derive the following.

% \begin{corollary}\label{lem:gajarsky}
% Let $G$ be a graph, let $s\in \N$, and suppose there is a constant $t\in \N$ such that $K_t\not\minor_{3s+1} G$.
% Then for every $\epsilon>0$ there exists $n_0$, depending only on $\epsilon,t,s$, such that for each vertex subset $A\subseteq V(G)$ that is $(2s+1)$-independent in $G$ and satisfies $|A|\geq n_0$, it holds that
% \[\abs{\{N_{s+1}[v]\cap A \colon v\in V(G)\}}\leq\abs{A}^{1+\epsilon}.\]
% \end{corollary}
% \begin{proof}
% As $A$ is $(2s+1)$-independent, we have that the $s$-neighborhoods of vertices from $A$ are pairwise disjoint.
% Obtain an $s$-shallow minor $H$ of $G$ by contracting $N_s[u]$ for each $u\in A$.
% We implicitly identify each vertex $u\in A$ with the vertex of $H$ obtained from contracting $N_s[u]$, thus $A\subseteq V(H)$.
% It is known that every $1$-shallow minor of $H$ is also a $(3s+1)$-shallow minor of $G$ (cf.~\cite[Proposition~4.1]{sparsity}), hence $K_t\not\minor_{1} H$.
%
% Let us fix $\epsilon>0$.
% Take any $v\in V(G)$. If $v\in N_s[u]$ for some $u\in A$, then since $A$ is $(2s+1)$-independent, we have that $v$ is at distance larger than $2s+1$ from any other vertex of $A$.
% Hence in this case we have $N_{s+1}[v]\cap A=\{u\}$ and there can be at most $|A|$ neighborhoods of this type.
% Next, suppose $v\notin N_s[u]$ for any $u\in A$. Then by the construction of $H$ we have that $N_{s+1}[v]\cap A=N^H[v]\cap A$, where $N^H[v]$ is the neighborhood of $v$ in $H$.
% Since $K_t\not\minor_{1} H$, by \cref{lem:diversity} we have that the number of such neighborhoods is at most $|A|^{1+\epsilon/2}$,
% provided $|A|\geq n_0$ for some $n_0$ depending only on $t$ and $\varepsilon$.
% Thus, we conclude that $\abs{\{N_{s+1}[v]\cap A \colon v\in V(G)\}}\leq |A|+|A|^{1+\epsilon/2}$, which is bounded by $|A|^{1+\epsilon}$ if we choose $n_0$ large enough.
% \end{proof}
%
% %Observe that nowhere dense classes are closed under
% %taking bounded depth minors, as stated in the following lemma.
% %It is an immediate consequence of Proposition~4.1
% %of~\cite{sparsity}.
%
% %\begin{lemma}
% %Let $\CCC$ be a nowhere dense class of graphs and let $s\in \N$.
% %Then also the class $\{H \minor_s G \colon G\in \CCC\}$ is nowhere dense.
% %\end{lemma}



We  now state the main building block of our proof as a lemma. 

\begin{lemma}\label{lem:apex}
Let $G$ be a graph and let $s\in \N$. 
Assume that $K_t\not\minor_{3s+2} G$ for some $t\in \N$. 
Let $n_0$ be the constant given by \cref{lem:diversity} for $\epsilon=1/3$ and $t,s$ as above, 
and let $m\geq n_0$ be an integer. 
If~$A$ is an $1$-independent set in $G$ of size at least $2+2(2m^3)^{2t+1}$,
then either 
\begin{itemize}
\item $A$ contains a $2$-independent set of size $m$, or
\item some vertex $v$ of $G$
has at least $m$ neighbors in $A$.
\end{itemize}
\end{lemma}


To prove the lemma, we will arrange the elements of $A$ in a binary tree
and prove that the tree contains a long path. From this path, we will 
extract the set $A'$. In stability theory, similar trees are called \emph{type trees} and they are used to extract long indiscernible sequences, see e.g.~\cite{malliaris2014regularity}. 

\begin{proof}
	We prove~\cref{lem:engine} in the case $r=1$.
	Let $d=1$, so we assume that $G$ is a graph which does not contain a clique $K_t$ as a minor at depth $1$.
	\newcommand{\dau}{D}
	\newcommand{\son}{S}
	
	We call elements of the set $\set{\dau,\son}^*$ \emph{nodes},
	and the nodes $w\dau$ and $w\son$ are, respectively, the \emph{daughter} and the \emph{son} of $w$,
	and $w$ is the \emph{ancestor} of every word of the form $wv$, where $v\in\set{\dau,\son}^*$. The empty word  is called the \emph{root}.
	 A (finite, rooted, binary) \emph{tree} is a partial labeling $\tau\from \set{\dau,\son}^*\to U$ whose domain is a finite set of nodes which is closed under taking ancestors. The \emph{size} of a tree is the cardinality of its domain. The \emph{empty} tree is the tree of size $0$. The \emph{height} of a nonempty tree is the length of the longest word in its domain. 
  
  Let $G$ be a graph, $A\subset V(G)$ a $1$-independent set of its vertices
  and $\bar a$ an enumeration of $A$.

We define  a binary tree $\tau$ which is 
  labeled vertices of $G$. The tree is defined by processing all elements of the sequence $\bar a$ sequentially. We start with $\tau$ being the empty tree, and for each element $a$ of the sequence $\bar a$, execute the following procedure, which extends the domain of $\tau$ by one element.
  
When processing the vertex $a$, do the following. Start with $w$ being the root. While $w$ is in the domain of $\tau$, repeat the following step: 
  if the distance from $a$ to $\tau(w)$ in the graph 
  $G$ is two, replace $w$ by its son, otherwise, replace $w$ by its daughter.
  % Repeat the step, unless $\tau(w)$ is undefined.
  Once $w$ is not in the domain of $\tau$, extend $\tau$ to $w$    so that  $\tau(w)=a$. In this way, we have processed the element $a$, and now
    proceed to the next element $a$ of $\bar a$, until all elements are processed. This ends the construction of $\tau$.
	
	Note that $\tau$ is a bijection from its domain to $A$.
  
  
  \medskip

We consider two cases:
\begin{itemize}
	\item  there is some $w\in\set{\son,\dau}^*$ in the domain of $\tau$ which contains at least $m^3$ $\son$'s,
	\item every $w\in \set{\son,\dau}^*$ in the domain of $\tau$ contains at most $m$ $\son$'s.
\end{itemize}

	Consider the first case.
	Let $B\subset A$ be the set of the labels of the 
	fathers of the sons on the path corresponding to $w$, i.e.,
	$$B=\set{\tau(u)\colon 
	u\son \text{ is a prefix of }w}.$$
	Note that $B$ has at least $m$ elements,
	and by construction, any two elements of $B$ are at mutual distance two. 
	
	Let 
	$$\cal F=\set {N[v]\cap B\colon v\in V(G)}.$$ 
	We say that a pair  $a,b \in B$
	is \emph{covered} if
	$\set{a,b}$ is contained in some $F\in \cal F$.


Let $k$ be the maximal size of a set in $\cal F$.
By~\cref{lem:diversity}, the family $\cal F$ contains 
 at most  $m$ elements,
 and each of them covers at most $k\choose 2$ 
pairs of elements of $B$. Hence, by the union bound,
at most $(k\choose 2)\cdot m$ pairs are covered, 
yet,
by the remarks above, every pair ${a,b}\subset B$.
It follows that $k\ge m^{1/4}$.



By the union bound, this means that the size of the family 
		$$\set{N[v]\cap A\colon v\in V(G)}$$
is at least 	
	
	
	


  \begin{itemize}
  	\item Every 
	
	\item There is a $w$ in the domain of $\tau$ which contains at 
	
  \end{itemize}
  
  

\begin{claim}
The domain of $\tau$ consists of words with at most $t$ occurrences of the letter $0$. 
\end{claim}
\begin{clproof}
  Observe that if $\tau(w)$ is defined and $w$ has $l$ occurrences of the letter $0$, then $G$ contains a clique on $l$ vertices as  a minor at depth $r$:
  the vertices of the clique are the words $u$ 
such that $u0$ is a prefix of $w$,
  and the branch set associated to $u$ 
  is $B_\tau(u)$.
\end{clproof}


\begin{claim}\label{cl:right-words}
  The  tree $\tau$  has size at most $h^{t+1}$, where $h$ is its height.
\end{claim}
\begin{clproof}
The number of words of length at most $h$ and with at most $t$ occurrences of the letter $0$ is at most $h^{t+1}$.
\end{clproof}

  
Suppose that
  $|A|>(t\cdot (m+1))^{t+1}$. By~\cref{cl:right-words},  
  the height of $\tau$ is larger than $t\cdot (m+1)$, i.e.,
  there is a word $w$ of length larger than $t\cdot (m+1)$ in the domain of $\tau$. Since $w$ has at most $t$ occurrences of the letter~$0$, it  must contain a block of $m$ consecutive $1$'s, i.e., there is a word $u$ such that $u1^k$ is a prefix of $w$, for $k=1,\ldots,m$.
Let $B$ be the set of corresponding vertices, i.e., $B=\set{v_\tau(u1^k):k=1,\ldots,m}$,
and let $S$ be the set of vertices occurring in some $B_\tau(v)$, for some prefix $v$ of $u$.
It follows by construction that $B$ is a subset of $A$ which is $r$-independent in $G-S$.  Moreover, $|B|=m$ and 
$S$ contains at most $r\cdot t^2$ vertices.
\end{proof}

To solve the odd case (finding a large $(2s+2)$-independent subset of a large $(2s+1)$-indepenent set),
we iteratively apply \cref{lem:apex}, always adding the apex vertex $v$
returned by the lemma to the set $S$, until we find a large 
$(2s+2)$-independent subset of $A$ in $G-S$. For the induction, we
define a function $R$ such that $R(m,0)=m$
and $R(m,i)=2+2(2(R(m,i-1))^3)^{2t+1}$ for $i\geq 1$.
Then it can be easily seen that $R(m,t)\in \Theta_{t}(m^{6t+3})$.

\begin{lemma}\label{lem:iterate-apex}
Let $G$ be a graph and let $s\in \N$. 
Assume that $K_t\not\minor_{3s+2} G$ for some $t\in \N$. 
Let $n_0$ be the constant given by \cref{lem:gajarsky} for $\epsilon=1/3$ and $t,s$ as above, 
and let $m\geq \max(t,n_0)$ be an integer. 
Suppose $A$ is a $(2s+1)$-independent set in $G$ of size at least $R(m,t)$.
Then there is a subset $A'\subseteq A$ of size at least~$m$ and a 
a set $S\subseteq V(G)\setminus A$ of size at most $t-1$ such that
\begin{enumerate}
\item every vertex of $S$ is at distance exactly $s+1$ from 
every vertex of $A'$, and
\item $A'$ is $(2s+2)$-independent in $G-S$. 
\end{enumerate} 
\end{lemma}
\begin{proof}
We apply \cref{lem:apex} iteratively, for less than $t$ steps.
At step $i$ we maintain a subset $S\subseteq V(G)$ with $|S|=i$ and a subset $A'\subseteq A$ of size at least $R(m,t-i)$ such that each vertex of $S$ is at distance exactly $s+1$ from each vertex of $A'$ in $G$.
In the beginning we have $i=0$, $S=\emptyset$, and $A'=A$, so the invariant is satisfied.
In step $i$ of the construction, where $i<t$, we apply \cref{lem:apex} to $A'$ in $G-S$.
This application either yields a subset $A''\subseteq A'$ that is $(2s+2)$-independent in $G-S$ and satisfies $|A''|\geq R(m,t-i-1)\geq m$,
or a subset $A''\subseteq A'$ of size at least $R(m,t-i-1)$ together with a vertex $v\in V(G)\setminus S$ that is at distance exactly $s+1$ from each vertex of $A''$ in $G-S$
In the former case we are done, as $A''$ and $S$ form a valid outcome of the lemma.
In the latter case, 
observe that $v$ is at distance exactly $s+1$ from each vertex of $A''$ also 
in~$G$, as each vertex of $S$ is at distance exactly $s+1$ from each vertex 
of $A''$ in~$G$,
so the removal of $S$ from the graph does change the balls around vertices $A''$ of radius $s$; that is, $N^{G-S}_s[a]=N^{G}_s[a]$ for each $a\in A''$.
Hence we may proceed with the construction having set $A':=A''$ and $S:=S\cup \{v\}$.

It remains to prove that if the construction was carried out for $t$ steps, 
yielding a set $S$ of size~$t$ and a subset $A'$ of size at least $R(m,0)=m$, then $K_t$ is a depth-$(s+1)$ minor of $G$, which is a contradiction.
Let $(v_1,\ldots,v_t)$ be the enumeration of $S$ in the order of selection: vertex~$v_i$ was selected in the $i$-th round.
By the construction, each vertex $v_i$ is at distance exactly $s+1$ from each vertex of $A'$ in $G$.
Select any distinct vertices $a_1,\ldots,a_t\in A'$; they exist since $|A'|\geq m\geq t$.
Next, for $i=1,\ldots,t$ define
$$X_i=N^{G}_s[a_i]\cup \{v_i\}.$$
Since $A'$ is $(2s+1)$-independent in $G$ and each vertex $v_i$ is at distance exactly $s+1$ from each vertex of $A'$ in $G$, we infer that sets $X_i$ are pairwise disjoint.
Moreover, since $v_i$ is at distance $s+1$ from $a_i$, we also have that each $X_i$ induces a connected graph of radius at most $s+1$.
Finally, observe that for each $i\neq j$, $v_i$ has a neighbor among $N^G_s[a_j]$, since $v_i$ is at distance $s+1$ from $a_j$ in $G$.
This implies that $(X_i)_{i=1,\ldots,t}$ is a depth-$(s+1)$ minor model of $K_t$ in $G$.
\end{proof}

\paragraph*{Finishing the proof of \cref{thm:new-uqw}.}
We now wrap up the proof of \cref{thm:new-uqw} by applying induction on the radius. 

%\begin{lemma}\label{lem:ramsey2}
%Let $G$ be a graph such that $K_t\not\minor_r G$. 
%If $A$ is $2r$-independent and
%has size at least $\binom{m+t-2}{t-1}$, then there exists
%a subset $B\subseteq A$ of size at least $m$ which is a
%$(2r+1)$-independent set. 
%\end{lemma}
%\begin{proof}
%As $A$ is $2r$-independent, we can contract the $r$-neighborhood
%of each $v\in A$. The corresponding elements $N_r[v]$ in the resulting 
%depth-$r$ minor $H$ form a set $Z$ that is in \mbox{$1$-to-$1$} correspondence 
%with $A$. By assumption, 
%$H[Z]$ excludes $K_t$ as a subgraph, and hence by \cref{lem:ramsey1},
%it contains an independent set $B'\subseteq Z$ of size $m$ in $H$. 
%This set $B'$ corresponds to a $(2r+1)$-independent set $B\subseteq A$ of $G$. 
%\end{proof}
%
%\begin{lemma}\label{lem:distance-apex}
%Let $\CCC$ be a nowhere dense class of graphs. 
%Let $n_0$ be the constant of for $\epsilon=1/3$. 
%Assume $K_t\not\minor_{r+2} G$. 
%Let $m\geq n_0$ be an integer. 
%Let $A$ be a $(2r+1)$-independent set in $G$ of size at least $R(m,t)$. 
%Then there is a subset $A'\subseteq A$ of size at least~$m$ and a 
%a set $S\subseteq V(G)\setminus A$ of size at most $t-1$ such that
%\begin{enumerate}
%\item every vertex of $S$ is connected to a vertex at distance $r$ of $w$ for 
%every $w\in A'$, and
%\item $A'$ is $(2r+2)$-independent in $G-S$. 
%\end{enumerate} 
%\end{lemma}
%\begin{proof}
%As $A$ is $(2r+1)$-independent, we can contract the $r$-neighborhood
%of each $v\in A$. The corresponding elements $N_r[v]$ in the resulting 
%depth-$r$ minor $H$ form a set $Z$ that is in $1$-to-$1$ correspondence 
%with $A$. As $A$
%is $(2r+1)$-independent, $Z$ is independent in $H$. We now apply 
%\cref{lem:iterate-apex} to $Z$ in $H$. Note that the depth-$2$ minor
%we construct in \cref{thm:alternation-rank-type-tree} (now applied to $H$) 
%uses as connecting vertices $z_{ij}$ original vertices of the graph and
%not contracted vertices. Hence, when we apply the lemma, we may 
%use the assumption that $K_t\not\minor_{r+2} G$ (in general, 
%a depth-$2$ minor of a depth-$r$ minor may be a depth-$5r$ minor
%of the original graph~see Proposition~4.1 of~\cite{sparsity}). 
%Also, the vertices $v$ returned by 
%\cref{lem:iterate-apex} correspond to vertices of the graph $G$ and not
%to contracted neighborhoods. In particular, as $A$ is $(2r+1)$-independent, 
%the vertices $v$ returned by the lemma have distance exactly $r$ to 
%the vertices $w\in A$. The set $Z'$ returned by \cref{lem:iterate-apex}
%for $H$ is $2$-independent in $H$, and hence the corresponding 
%subset $B\subseteq A$ of $G$ is $(2r+2)$-independent. 
%in $G$. 
%\end{proof}

\begin{proof}[of \cref{thm:new-uqw}]
Without loss of generality suppose $m\geq t$.
Define $Q(m,i)$ as follows: $$Q(m,0)=m\quad\textrm{and}\quad Q(m,i)=\max \left (R(Q(i-1),t),\binom{Q(i-1)+t-2}{t-1}\right)\textrm{ for }i>0.$$
Since $R(n,t)\in \Theta_{t}(n^{6t+3})$ and $\binom{n+t-2}{t-1}\in \Theta_t(n^{t-1})$, it can be easily seen that $Q(m,r)\in \Theta_{r,t}(m^{(6t+3)^r})$.
We set $N(m)=Q(m,r)$, so suppose we are given a vertex subset $A$ of size at least $Q(m,r)$.

We now proceed in $r$ rounds, applying \cref{lem:ramsey1} and \cref{lem:iterate-apex} alternately.
Precisely, we construct sets $S_0\subseteq S_1\subseteq \ldots\subseteq S_r$ and $A_0\supseteq A_1\supseteq \ldots\supseteq A_r$, where $S_0=\emptyset$ and $A_0=A$.
We maintain the invariant that $|S_i|\leq \lfloor i/2\rfloor$, $|A_i|\geq Q(m,r-i)$, and $A_i$ is $i$-independent in $G-S_i$. This is clearly satisfied for $i=0$, so we need to describe the construction
of $(A_i,S_i)$ based on $(A_{i-1},S_{i-1})$ for $i>0$.

Suppose first $i=2s$ for some integer $s$. Then apply \cref{lem:ramsey1} to $A_i$ in $G-S_i$, yielding its subset $A_{i+1}$ of size at least $Q(m,r-i-1)$ that is $(i+1)$-independent in $G-S_i$. We may set $S_{i+1}=S_i$
and proceed with the construction.

Suppose now $i=2s+1$ for some integer $s$; then $i<r$ implies that $s\leq r/2-1$, so $K_{t}\not\minor_{3s+2} G$. 
Hence we may apply \cref{lem:iterate-apex} to $A_i$ in $G-S_i$, yielding subsets $R_{s}\subseteq V(G)-S_i$ and $A_{i+1}\subseteq A_i$ 
such that $|R_{s}|\leq t$, $|A_{i+1}|\geq Q(m,r-i-1)$, each vertex of $A_{i+1}$ is at distance exactly $i+1$ from each vertex of $R_s$, and $A_{i+1}$ is $(i+1)$-independent in $(G-S_{i})-R_s$.
We may set $S_{i+1}=R_s\cup S_i$ and proceed with the construction.

Thus, after $r$ rounds we obtain subsets $S=S_r$ and $B=A_r$ such that $|B|\geq Q(m,0)=m$ and~$A'$ is $r$-independent in $G-S$. Observe that we have $S=R_1\cup R_2\cup \ldots\cup R_{\lfloor r/2\rfloor}$,
so the trivial upper bound on the size of $S$ is $rt/2$. We claim that in fact we have $|S|<t$; the argument is similar as in the proof of \cref{lem:iterate-apex}.

Suppose $|S|\geq t$ and let $v_1,\ldots,v_t$ be any $t$ distinct vertices of $S$.
Since $m\geq t$, we may arbitrarily choose $a_1,\ldots,a_t$ to be $t$ distinct vertices of $A'$.
Let $Y_i=N^{G-S}_{\lfloor r/2\rfloor}[a_i]$ for $i=1,2,\ldots,t$.
From \cref{lem:iterate-apex} it can be seen by a trivial induction on $s$ that each vertex of $R_s$ has a neighbor in each set $Y_j$, for $i,j\in \{1,2,\ldots,t\}$; 
this is because at step $i=2s+1$ of the construction we remove from the graph only vertices at distance exactly $s+1$ from each $a_j$.
Hence, the sets $X_i=Y_i\cup \{v_i\}$ for $i=1,\ldots,t$ form a depth-$(\lfloor r/2\rfloor+1)$ minor model of $K_t$ in $G$, a contradiction. 
\end{proof}

\paragraph{Algorithm.} We now argue that the proof presented above can be turned into an algorithm that, 
given $G$ and $A$, outputs sets $S$ and $B$ in time $\Oof_{r,t}(|A|^c\cdot |E(G)|)$ for some universal constant~$c$.
This boils down to implementing each step of the iterative construction from the proof.

The even steps---from $(S_{2s},A_{2s})$ to $(S_{2s+1},A_{2s+1})$---are easy, as we only have to apply Ramsey's theorem in a graph with $A$ as the vertex set, where two vertices are adjacent if and only
if they are at distance $2s$ in $G$. The distances between vertices of $A$ can be computed in time $\Oof(|A|\cdot |E(G)|)$ by running a breadth-first search from each vertex of $A$, and afterwards it remains
to emulate the proof of Ramsey's theorem algorithmically on a graph with $|A|$ vertices, which can be done in time polynomial in $|A|$.

For the odd steps---from $(S_{2s+1},A_{2s+1})$ to $(S_{2s+2},A_{2s+2})$---we need to implement algorithmically the proof of \cref{lem:iterate-apex}. 
This boils down to implementing algorithmically the proof of \cref{lem:apex}, since
the proof of \cref{lem:iterate-apex} essentially consists of applying \cref{lem:apex} at most $t$ times. For \cref{lem:apex}, we can construct the distance-$(2s+2)$ tree $T$ of $A$ in time
$\Oof(|A|\cdot |E(G)|)$, as this again only requires precomputing distances between vertices of $A$ using breadth-first search from each vertex of $A$.
Afterwards we examine the longest root-to-leaf path in $T$, which has to have length at least $2m^3$, and we classify its vertices into red and blue, as in the proof. If more than half of the vertices are red,
then they form a $(2s+2)$-independent set that can be output by the algorithm. Otherwise, the proof argues that there is vertex $v$ in the graph that is at distance exactly $s+1$ from at least $m$ blue vertices.
Such a vertex can be found by computing in time $\Oof(|A|\cdot |E(G)|)$ the distances between each blue vertex and each vertex $v\in V(G)$, again using breadth-first search from each blue vertex, and selecting
a vertex $v$ for which the number of blue vertices at distance exactly $s+1$ from $v$ is at least $m$. 


\section{VC dimension}\label{sec:vc}

We now come to the proof of \cref{thm:new-vc}. A set $X$ of vertices 
in a graph is \emph{shattered} if for every
subset $Y\subseteq X$ there exists 
a vertex $v$ such that $N[v]\cap X=Y$. The \emph{Vapnik-Chervonenkis dimension}, short \emph{VC-dimension}~\cite{chervonenkis1971theory} of a graph is the maximum size of a shattered set. 
The VC-dimension as a measure of complexity of set systems found has many applications, e.g.\ in learnability theory~\cite{haussler1987}, computational geometry~\cite{chazelle1989quasi},
and graph theory~\cite{alon2006dominating,BousquetT15,chepoi2007covering,eickmeyer2016neighborhood}.
We define notions of a {\em{$2$-shattered}} set and the {\em{2VC-dimension}} of a graph by restricting subsets $Y\subseteq X$ considered in the definition only to subsets of size exactly $2$.

The \emph{$r$th power of a graph $G$} is the graph $G^r$
with vertex set $V(G)$, where there is an edge between two 
vertices $u$ and $v$ if and only if their distance in $G$ is at most $r$. 

We observe that an argument of Bousquet and 
Thomass\'e~\cite{BousquetT15} can be slightly modified to prove that 
the $2$VC-dimension of the $r$-power graph $G^r$ of a graph $G$
with $K_t\not\minor_r G$ is small. Obviously, the $2$VC-dimension of $G$
bounds its VC-dimension. Hence, we in fact prove the following strengthening of \cref{thm:new-vc} stated in the introduction. 

\begin{theorem}
Let $r\in \N$ and let $G$ be a graph. 
If $K_t\not\minor_r G$, then the $2$VC-dimension of $G^r$
is at most $t-1$. 
\end{theorem}
\begin{proof}
Assume there is a set $A=\{a_1,\ldots, a_t\}$ of size $t$ such that
for all subsets $\{i,j\}\subseteq \{1,\ldots,t\}$ of size $2$ 
there is an vertex $v_{ij}$ with 
$N_r[v_{ij}]\cap A=\{a_i,a_j\}$.
For each subset $\{i,j\}\subseteq \{1,\ldots,t\}$ of size $2$, choose a vertex $u_{ij}$ so that:
\begin{enumerate}[(1)]
\item\label{p:i} $\dist(v_{ij},u_{ij})+\dist(u_{ij},a_i)\leq r$;
\item\label{p:j} $\dist(v_{ij},u_{ij})+\dist(u_{ij},a_j)\leq r$; and
\item\label{p:min} subject to conditions \eqref{p:i} and \eqref{p:j}, $\max(\dist(u_{ij},a_i),\dist(u_{ij},a_j))$ is minimized.
\end{enumerate}
Observe that $u_{ij}$ is well-defined since setting $u_{ij}=v_{ij}$ satisfies the first two conditions.

Let $P^i_{ij}$ and $P^j_{ij}$ be arbitrarily chosen shortest paths between $u_{ij}$ and~$a_i$, and between $u_{ij}$ and~$a_j$, respectively.
We now establish some basic properties of paths $P^i_{ij}$ and $P^j_{ij}$ following from the choice of $u_{ij}$.

\begin{claim}\label{cl:ineq}
For each vertex $x$ on $P^i_{ij}$ we have $\dist(v_{ij},x)+\dist(x,a_i)\leq r$, and
for each vertex $y$ on $P^j_{ij}$ we have $\dist(v_{ij},y)+\dist(y,a_j)\leq r$.
\end{claim}
\begin{clproof}
We prove only the first statement for the second is symmetric.
We have
$$\dist(v_{ij},x)+\dist(x,a_{i})\leq \dist(v_{ij},u_{ij})+\dist(u_{ij},x)+\dist(x,a_{i})=\dist(v_{ij},u_{ij})+\dist(u_{ij},a_{i})\leq r,$$
where the last equality is due to $x$ lying on a shortest path between $u_{ij}$ and $a_i$, and the last inequality is by condition~\eqref{p:i}.
\end{clproof}

\begin{claim}\label{cl:closer}
Suppose $x$ is a vertex on $P^i_{ij}$ that is different from $u_{ij}$. Then $\dist(x,a_i)<\dist(x,a_j)$.
Symmetrically, if $y$ lies on $P^j_{ij}$ and is different from $u_{ij}$, then $\dist(y,a_i)>\dist(y,a_j)$.
Consequently, paths $P^i_{ij}$ and $P^j_{ij}$ share only one vertex, being the endpoint $u_{ij}$.
\end{claim}
\begin{clproof}
We prove only the first claim, for the second is symmetric and the third directly follows from the first two.
Suppose for contradiction that $\dist(x,a_i)\geq \dist(x,a_j)$.
By \cref{cl:ineq} we have 
$$\dist(v_{ij},x)+\dist(x,a_i)\leq r.$$
On the other hand, since $\dist(x,a_i)\geq \dist(x,a_j)$, we have
$$\dist(v_{ij},x)+\dist(x,a_j)\leq\dist(v_{ij},x)+\dist(x,a_i)\leq r.$$
We conclude that $x$ satisfies conditions \eqref{p:i} and \eqref{p:j} from the definition of $u_{ij}$.
However, since $x\neq u_{ij}$ and $x$ lies on a shortest path between $u_{ij}$ and $a_i$, we have $\dist(x,a_i)<\dist(u_{ij},a_i)$.
Therefore,
$$\dist(x,a_j)\leq \dist(x,a_i)<\dist(u_{ij},a_i)\leq \max(\dist(u_{ij},a_i),\dist(u_{ij},a_j)).$$
Thus, the existence of $x$ contradicts condition \eqref{p:min} from the definition of $u_{ij}$.
\end{clproof}

Now, define paths $Q^i_{ij}$ and $Q^j_{ij}$ as follows:
\begin{itemize}
\item if $\dist(u_{ij},a_i)<\dist(u_{ij},a_j)$, then $Q^{i}_{ij}=P^{i}_{ij}$ and $Q^{j}_{ij}=P^{j}_{ij} - \{u_{ij}\}$;
\item if $\dist(u_{ij},a_i)>\dist(u_{ij},a_j)$, then $Q^{i}_{ij}=P^{i}_{ij} - \{u_{ij}\}$ and $Q^{j}_{ij}=P^{j}_{ij}$;
\item if $\dist(u_{ij},a_i)=\dist(u_{ij},a_j)$, then define $Q^i_{ij}$ and $Q^j_{ij}$ using any of the above.
\end{itemize}
Thus, by \cref{cl:closer} we have that paths $Q^{i}_{ij}$ and $Q^{j}_{ij}$ are disjoint. Moreover, for each vertex $x$ on $Q^{i}_{ij}$ we have $\dist(x,a_i)\leq \dist(x,a_j)$, and for each
vertex $y$ on $Q^{j}_{ij}$ we have $\dist(y,a_i)\geq \dist(y,a_j)$.

\begin{claim}\label{cl:intersect}
Let $\{i,j\}$ and $\{i',j'\}$ be two different subsets of size $2$ of $\{1,\ldots,t\}$.
Suppose that paths $Q^i_{ij}$ and $Q^{i'}_{i'j'}$ intersect.
Then $i=i'$.
\end{claim}
\begin{clproof}
Let $x$ be a vertex lying both on $Q^i_{ij}$ and $Q^{i'}_{i'j'}$. We first consider the corner case when $x=u_{ij}$.
Suppose first that $\dist(v_{ij},x)\geq \dist(v_{i'j'},x)$. Then by \cref{cl:ineq} we have
$$\dist(v_{i'j'},a_i)\leq \dist(v_{i'j'},x)+\dist(x,a_i)\leq \dist(v_{ij},x)+\dist(x,a_i)\leq r,$$
and analogously $\dist(v_{i'j'},a_{j})\leq r$. However, we assumed that $a_{i'}$ and $a_{j'}$ are the only vertices of~$A$ that are at distance at most $r$ from $v_{i'j'}$, hence $\{i,j\}=\{i',j'\}$,
a contradiction. Suppose then that $\dist(v_{ij},x)<\dist(v_{i'j'},x)$. 
Then we have
$$\dist(v_{ij},a_{i'})\leq \dist(v_{ij},x)+\dist(x,a_{i'})<\dist(v_{i'j'},x)+\dist(x,a_{i'})\leq r,$$
where the last equality follows from \cref{cl:ineq}.
Since $a_i$ and $a_j$ are the only vertices of $A$ that are at distance at most $r$ from $v_{ij}$, we infer that $i'\in \{i,j\}$. 
If $i'=i$ then we would be done, so suppose $i'=j$.
Since $x=u_{ij}$ and $x$ lies on $Q^i_{ij}$, by the definition of $Q^i_{ij}$ we have that $\dist(x,a_i)\leq \dist(x,a_j)=\dist(x,a_{i'})$. Therefore,
$$\dist(v_{i'j'},a_i)\leq \dist(v_{i'j'},x)+\dist(x,a_{i})\leq \dist(v_{i'j'},x)+\dist(x,a_{i'})\leq r.$$
where the last inequality follows from \cref{cl:ineq}.
Again, we assumed that $a_{i'}$ and $a_{j'}$ are the only vertices of $A$ that are at distance at most $r$ from $v_{i'j'}$, so $i\in \{i',j'\}$. If $i=i'$ then we are done, and otherwise we have $i=j'$.
Together with $i'=j$ this implies $\{i,j\}=\{i',j'\}$, a contradiction.

The second corner case when $x=u_{i'j'}$ leads to a contradiction in a symmetric manner.

We now move to the main case when $x\neq u_{ij}$ and $x\neq u_{i'j'}$.
Then by \cref{cl:closer} we have $\dist(x,a_i)<\dist(x,a_j)$ and $\dist(x,a_{i'})<\dist(x,a_{j'})$.
By symmetry, without loss of generality assume that $\dist(x,a_i)\leq \dist(x,a_{i'})$.
Observe now that
$$\dist(v_{i'j'},a_i)\leq \dist(v_{i'j'},x)+\dist(x,a_{i})\leq \dist(v_{i'j'},x)+\dist(x,a_{i'})\leq r,$$
where the last inequality follows from \cref{cl:ineq}.
Since we assumed that $a_{i'}$ and $a_{j'}$ are the only vertices of $A$ that are at distance at most $r$ from $v_{i'j'}$, we have $i\in \{i',j'\}$.
However, it cannot happen that $i=j'$, because $\dist(x,a_{i'})<\dist(x,a_{j'})$ and $\dist(x,a_{i'})\geq \dist(x,a_{i})$. We conclude that $i=i'$.
\end{clproof}

For each $i\in \{1,2,\ldots,t\}$ we define $X_i$ to be the union of vertex sets of paths $Q^i_{ij}$ for $j\neq i$.
Each of these paths has length at most $r$ and has $a_i$ as an endpoint, hence the subgraph induced by $X_i$ is connected and has radius at most $r$.
By \cref{cl:intersect}, sets $X_i$ are pairwise disjoint. Finally, observe that for each $\{i,j\}\subseteq \{1,\ldots,t\}$ with $i\neq j$, there is an edge between a vertex of $Q^{i}_{ij}$ and a vertex of $Q^{j}_{ij}$.
We conclude that $(X_i)_{i=1,\ldots,t}$ is a depth-$r$ minor model of $K_t$ in $G$, a contradiction.
\end{proof}

\begin{comment}
\begin{proof}
Assume there is a set $A=\{a_1,\ldots, a_t\}$ of size $t$ such that
for all subsets $\{a_i,a_j\}\subseteq A$ of size $2$ 
there is an element $v_{ij}\in V(G)\setminus A$ with 
$N_r[v_{ij}]\cap A=\{a_i,a_j\}$. Fix such $v_{ij}$ with the property
that $\max\left(\dist_G(v_{ij},a_i), \dist_G(v_{ij},a_j)\right)$ is 
minimized. 

A \emph{central walk} $W_{ij}$ is the concatenation of a minimum length
path $P_{ij}^i$ from $a_i$ to $v_{ij}$ and a minimum length path $P_{ij}^j$ from $v_{ij}$ to $a_j$. 
Note that a central walk is possibly not a path. For each pair $a_i,a_j$ fix
a central walk $W_{ij}$ and the corresponding paths $P_{ij}^i$ and $P_{ij}^j$. 

Now assume that a vertex $x$ is traversed by two different central 
walks $W_{ij}$, $W_{i'j'}$. By swapping indices if necessary, assume that $x$ lies on $P_{ij}^i$ and $P_{i'j'}^{i'}$. 

First, observe that if $\dist(x,a_i)=\dist(x,a_{i'})$, 
then $a_i=a_{i'}$. Indeed, if $\dist(x,a_i)=\dist(x,a_{i'})$ then $\dist(v_{ij},a_{i})=\dist(v_{ij},a_{i'})$, so $i'\in \{i,j\}$ because $a_i,a_j$ are the only vertices of $A$ at distance at most $r$ from $v_{ij}$.
Analogously $i\in \{i',j'\}$, so either $i=i'$, or $i'=j$ and $i=j'$. However, the latter case would imply $\{i,j\}=\{i',j'\}$, which contradicts the assumption that $W_{ij}$ and $W_{i'j'}$ are distinct. 

A similar argument yields that
$\dist(x,a_i)<\dist(x,a_j)$ 
and $\dist(x,a_{i'})<\dist(x,a_{j'})$. 
Now assume that $\dist(x,a_i)<\dist(x,a_{i'})$. By the same argument as 
above we have $a_{j'}=a_i$, hence $W_{i'j'}=W_{ij'}$. Here, we have
$\dist(x,a_i)<\dist(x,a_j)$ and $\dist(x,a_{i})<\dist(x,a_{i'})$, 
otherwise the walks are not distinct. 

Let us now construct connected subsets $X_i$ for all $1\leq i\leq t$. 
For every walk $W_{ij}$ the vertices of $W_{ij}$ closer to $a_i$ than to $a_j$ 
are added to $X_i$, the vertices of $W_{ij}$ closer to $a_j$ than to $a_i$ 
are added to $X_j$, ties are broken arbitrary.
Then the sets $X_i$ are pairwise disjoint by what we proved above. If a vertex $x$
appears in two distinct central walks, these are $W_{ij}$ and $W_{i\ell}$ for some
$i,j,\ell$ with $\dist(x,a_i)<\dist(x,a_j)$ and $\dist(x,a_i)<\dist(x,a_\ell)$. 
In both cases $x$ belongs to $X_i$. By construction, the sets $X_i$ are connected, 
have radius at most~$r$, and 
there is always an edge between a vertex of $X_i$ and a vertex of $X_j$ since $X_i\cup X_j$ 
contains the walk $W_{ij}$. Therefore, if the $2$VC-dimension is at least $t$, the 
graph contains $K_t$ as a depth-$r$ minor. 
\end{proof}
\end{comment}