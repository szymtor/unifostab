\section{Preliminaries}\label{sec:prelim}
We recall some basic notions from graph theory.

All graphs in this paper are finite, undirected and simple,
that is, they do not have loops or parallel edges. Our notation is
standard, we refer to~\cite{diestel2012graph} for more background on
graph theory.  We write $V(G)$ for the vertex set of a graph $G$ and
$E(G)$ for its edge set.  The {\em{distance}} between vertices~$u$ and
$v$ in $G$, denoted $\dist_G(u,v)$, is the length of a shortest path
between $u$ and $v$ in~$G$.  If there is no path between $u$ and $v$
in $G$, we put $\dist_G(u,v)=\infty$.  The {\em{(open) neighborhood}}
of a vertex $u$, denoted~$N(u)$, is the set of neighbors of $u$,
excluding $u$ itself.  For a non-negative integer $r$, by $N_r[u]$ we
denote the {\em{(closed) $r$-neighborhood}} of $u$ which comprises
vertices at distance at most $r$ from $u$; note that $u$ is always
contained in its closed $r$-neighborhood. The \emph{radius} of a
connected graph $G$ is the least integer $r$ such that there is some
vertex $v$ of $G$ with $N_r[v]=V(G)$.


A {\em{minor model}} of a graph $H$ in $G$ is a family
$(I_u)_{u\in V(H)}$ of pairwise vertex-disjoint connected subgraphs of
$G$, called {\em{branch sets}}, such that whenever $uv$ is an edge
in~$H$, there are $u'\in V(I_u)$ and $v'\in V(I_v)$ for which $u'v'$
is an edge in $G$.  The graph $H$ is a {\em{depth-$r$ minor}} of $G$,
denoted $H\minor_rG$, if there is a minor model $(I_u)_{u\in V(H)}$
of~$H$ in $G$ such that each $I_u$ has radius at most $r$.

A class $\CCC$ of graphs is \emph{nowhere dense} if there is a
function $t\colon \N\rightarrow \N$ such that for all $r\in \N$ it
holds that $K_{t(r)}\not\minor_r G$ for all $G\in \CCC$, where
$K_{t(r)}$ denotes the clique on $t(r)$ vertices.  The class~$\CCC$
moreover has \emph{bounded expansion} if there is a function
$d\colon\N\rightarrow\N$ such that for all $r\in \N$ and all
$H\minor_rG$ with $G\in\CCC$, the {\em{edge density}} of $H$,
i.e. $|E(H)|/|V(H)|$, is bounded by $d(r)$. Note that every class of
bounded expansion is nowhere dense. The converse is not necessarily
true in general~\cite{sparsity}.

% A set $B\subseteq V(G)$ is called {\em{$r$-independent}} in a graph
% $G$ if $\dist_G(u,v)>r$ for all distinct $u,v\in B$.

 

A set $B\subseteq V(G)$ is called {\em{$r$-independent}} in a graph
$G$ if $\dist_G(u,v)>r$ for all distinct $u,v\in B$.  A class $\CCC$
of graphs is \emph{uniformly quasi-wide} if for every $r\in \N$ there
is a function $N_r\from\N\to\N$ a number $s_r\in \N$ such that for
every $m\in \N$, graph $G\in \CCC$, and vertex subset
$A\subseteq V(G)$ of size $\abs{A}\geq N_r(m)$, there is a set
$S\subseteq V(G)$ of size $\abs{S}\leq s_r$ and a set $B\subseteq A-S$
of size $\abs{B}\geq m$ that is $r$-independent in $G-S$.  Recall that
Ne\v set\v ril and Ossona de Mendez
proved~\cite{nevsetvril2011nowhere} that nowhere dense graph classes
are exactly the same as uniformly quasi-wide classes.  The following
result of Kreutzer, Rabinovich and the second
author~\cite{siebertz2016polynomial} improves their result, by showing
that the function $N_r$ is polynomially bounded:


\begin{theorem}[\cite{siebertz2016polynomial}]\label{thm:krs}
  For every nowhere dense class $\CCC$ and for all $r\in \N$ there is
  a polynomial $N_r\from \N\to\N$ and a number $s_r\in \N$ such that
  the following holds.  Let be $G\in \CCC$ be a graph and let
  $A\subset V(G)$ be a vertex subset of size at least $N_r(m)$, for a
  given $m$.  Then there exists a set $S\subset V(G)$ of size
  $|S|<s_r$ and a set $B\subset A-S$ of size $|B|\ge m$ which is
  $r$-independent in $G-S$.
\end{theorem}

As we mentioned, the proof of Kreutzer et
al.~\cite{siebertz2016polynomial} relies on non-constructive arguments
and does not yield explicit bounds on (the degree of) of $N_r$ and
$s_r$.  In the next section, we discuss a further strengthening of
this result, by providing explicit, computable bounds on $N_r$ and
$s_r$.
