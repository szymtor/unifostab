\section{On the number of types in nowhere dense graph classes}

In this section we prove \Cref{thm:vc-density} and \Cref{thm:vc-density-lower-bound}.
\begin{change}{sz}
Throughout this section, fix a formula  $\phi(\tup x,y)$, where
$\tup x$ is an $m$-tuple of variables and $y$ is a single variable.

For a graph $G$, a set of  vertices $A\subset V(G)$, and a vertex $v\in V(G)$,
denote
\[\phi(A,v)=\setof{\tup a \in A^m }{ G\models\phi(\tup a,v)},\]
and for $W\subset V(G)$, denote
\[S_\phi(A,W)=\setof{\phi(A,v)}{ v\in W}.\]
Note that $S_\phi(A,W)$ is \emph{not} symmetric in $A$ and $W$.
We write $S_\phi(A,G)$ as a shorthand for $S_\phi(A,V(G))$.

Our goal is to prove that if $G\in \CCC$, where $\CCC$ is a nowhere dense class,
then for every $\varepsilon>0$,   $S_\phi(A,G)\le \Oof(|A|^{1+\varepsilon})$, and 
$S_\phi(A,G)\le \Oof(|A|)$ if $\CCC$ is a class of bounded expansion.
\end{change}

First, observe that if $A\subseteq B$, then the number of types
over $A$ cannot be larger than the number of types over $B$. 

\begin{lemma}\label{lem:types-over-B}
Let $G$ be a graph and let $A\subseteq B\subseteq V(G)$. Then 
$|S_\phi(A,G)|\leq |S_\phi(B,G)|$. 
\end{lemma}

We will first enlarge the set $A$ to a set $B$, called
an \emph{$r$-closure of $A$}, such 
that the connections of elements from $V(G)\setminus B$ 
towards $B$ are well controlled. This approach
was first used in Drange et al.~\cite{drange2016kernelization} for
classes of bounded expansion and extended to nowhere dense
classes in Eickmeyer et al.~\cite{eickmeyer2016neighborhood}. 
Let us recall the necessary definitions.

Let $G$ be a graph and let $B\subseteq V(G)$ be a subset of vertices. For vertices $v\in B$ and $u\in V(G)\setminus B$, a path $P$ connecting $u$ and $v$ is called {\em{$B$-avoiding}}
if all its vertices apart from~$v$ do not belong to $B$. For a positive integer $r$, the {\em{$r$-projection}} of any $u\in V(G)\setminus B$ on $B$, denoted $M^G_r(u,B)$ is the set of all vertices $v\in B$ that
can be connected to $u$ by a $B$-avoiding path of length at most $r$. 
%
%The {\em{$r$-projection profile}} of a vertex $u\in V(G)\setminus A$ on $A$ is a function $\rho^G_r[u,A]$ mapping vertices of
%$A$ to $\{0,1,\ldots,r,\infty\}$, defined as follows: for every $v\in A$, the value $\rho^G_r[u,A](v)$ is the length of a shortest $A$-avoiding path connecting $u$ and~$v$, and~$\infty$ in case this length
%is larger than $r$. 
%
We define 
\[\projnum_r(G,B)=|\{M_r^G(u,B)\colon u\in V(G)\setminus B\}|\]
%\quad\textrm{and}\quad \projprof_r(G,A)=|\{\rho_r^G[u,A]\colon %u\in V(G)\setminus A\}|\]
to be the number of different $r$-projections.
% and $r$-projection profiles realized on $B$, respectively. Clearly, it %holds that $\projnum_r(G,A)\leq \projprof_r(G,A)$.

\begin{lemma}[\cite{drange2016kernelization,eickmeyer2016neighborhood}]\label{lem:closure}
Let $\CCC$ be a class of graphs. 
\begin{enumerate}
\item If $\CCC$ has bounded expansion, then for every $r\in \N$ there is a constant $c\in \N$ such that for
every $G\in \CCC$ and $X\subseteq V(G)$ there exists a set $\cl_r(X)$, called an {\em{$r$-closure}} of $X$, with the following properties. 
\begin{itemize}
  \item $X\subseteq \cl_r(X)\subseteq V(G)$;
  \item $|\cl_r(X)|\leq c\cdot |X|$; and
  \item $|M_r^G(u,\cl_r(X))|\leq c$ for each $u\in V(G)\setminus \cl_r(X)$.
\end{itemize}
\item If $\CCC$ is nowhere dense, then for every $r\in\N$ and $\epsilon>0$ there is a 
constant $c\in\N$ such that for every $G\in \CCC$ and $X\subseteq V(G)$ there exists a set 
$\cl_r(X)$,  called an {\em{$r$-closure}} of $X$, 
with the following properties. 
\begin{itemize}
  \item $X\subseteq \cl_r(X)\subseteq V(G)$;
  \item $|\cl_r(X)|\leq c\cdot |X|^{1+\epsilon}$; and
  \item $|M_r^G(u,\cl_r(X))|\leq c\cdot |X|^{\epsilon}$ for each $u\in V(G)\setminus \cl_r(X)$.
\end{itemize}
\end{enumerate}
\end{lemma}

\begin{lemma}[\cite{drange2016kernelization,eickmeyer2016neighborhood}]\label{lem:projection-complexity}
Let $\CCC$ be a class of graphs. 
\begin{enumerate}
\item If $\CCC$ has bounded expansion, then for every $r\in \N$ there is 
  a constant $c\in\N$ such that for every graph $G\in \CCC$, and vertex subset $A\subseteq V(G)$, 
  it holds that $\projnum_r(G,A)\leq c\cdot |A|$.
  \item If $\CCC$ is nowhere dense, then for every $r\in \N$ and $\epsilon>0$ there is 
  a constant $c\in \N$ such that for every graph $G\in \CCC$, and vertex subset $A\subseteq V(G)$, 
  it holds that $\projnum_r(G,A)\leq c\cdot |A|^{1+\epsilon}$.
\end{enumerate}
\end{lemma}


\begin{change}{sz}
	Theorem~\cref{thm:vc-density} will follow easily from the above results and the following lemma.
\begin{lemma}\label{lem:num-types-same-class}
 Let $\CCC$ be a nowhere dense class,
 % There is a polynomial $f$ depending only
 % on $\CCC$ and $\phi$ such that the following holds.
let $G\in\CCC$ and let $B,C$ be sets of vertices with
$C\subset B\subseteq V(G)$.
Then, for any set of vertices $W\subset V(G)$ such that 
$M_r^G(v,B)=C$ for all $v\in W$, 
 $|S_\phi(B,W)|\le f(c)$, where $c=|C|$ and $f$ is a polynomial depending only on $\CCC$ and $\phi$.
\end{lemma}

\begin{proof}
As we are only interested in the cardinality of the set $\setof{\phi(B,u)}{u\in W}$,
we may assume  that $\phi(B,u)\neq \phi(B,v)$ 
for distinct $u, v\in W$.


Let $q$ be the quantifier rank of the formula $\phi$,
and let $r\in \N$ and~$T\from \N\times \N\to \N$  
be as described in~\cref{pro:crossing}.

As $\CCC$ is nowhere dense, there is an integer $t$ such that 
$K_t\not\minor_{\lfloor 10r/2\rfloor} G$. 
% According to
% \Cref{thm:new-uqw}, there is a polynomial  $N\colon \N\to \N$ with $N(m)=
% \Oof_{r,t}{(m^{{(2t+1)}^{2rt}})}$ such that for every set $D\subseteq V(G)$
% of size at least $N(m)$, for a given $m$, there exists a set $S\subseteq V(G)$ of size $|S|\leq t$
% and a set $X\subseteq D\setminus S$ of size $|X|\geq m$ which is $r$-independent in $G-S$.
Let $k\coloneqq T(q,t,1)$, where $q$ is the quantifier rank of $\phi$. In the parlance of~\cref{pro:crossing},
$k$ is a bound on the number of $(q,S)$-local types of vertices in a graph $G$,
for any fixed $S\subset V(G)$ with $|S|\le t$.



Let  $N\colon \N\to \N$ be the polynomial obtained by applying~\cref{thm:new-uqw} to $t$ and $2r$.
Suppose   $|W|\geq N(k+c+1)$. Then there is a  
set $S\subseteq V(G)$ of size $|S|\leq t$ 
and a set $X\subseteq W\setminus S$ of size $|X|\geq k+c+1$ which is 
$2r$-independent in $G-S$. 

Note that the function
 $v\mapsto M_r^{G-S}(v,B)$ 
 maps distinct $v,w\in X$ to pairwise disjoint subsets of $C$,
as otherwise we would have $\dist_{G-S}(v,w)\leq 2r$,
contradicting $2r$-independence.
As  $|X|\ge k+|C|+1$, it follows that 
there are at least $k+1$ vertices $v\in X$
for which ${M_r^{G-S}(v,B)=\emptyset}$. Let $Z$ denote the set of all those vertices. Then $Z$ and $B$ are $r$-separated by~$S$.


% As in the previous section, define $G^S$ as the graph $G-S$
% colored with colors $\{C_s : s\in S\}$ as follows. For each $s\in S$, a vertex $v\in V(G)\setminus S$
% is colored with color $C_s$ in $G^S$ if and only if $v$ is a neighbor of $s$ in $G$.
% According to \Cref{lem:remove-s}, we can rewrite $\phi$ to a formula $\phi'$ such that
% for every valuation $\tup a,v$ of $\tup x,y$ in $G-S$ we have
% \[(G,(\tup a,v))\models \phi\Leftrightarrow (G^S,(\tup a,v))\models \phi'.\]

Since $|Z|\geq k+1$, by the pigeon-hole principle, there are  distinct $u,v\in Z$ such that 
$u$ and $v$ have the same $(q,S)$-local types,
and from~\cref{pro:crossing}~\cref{c:confusing} it follows that $u$ and $v$
have the same quantifier rank $q$ over $B$.
In particular, $\phi(B,u)=\phi(B,v)$, contrary to the assumption made at the beginning of the proof.
\end{proof}
\end{change}

Now, it is easy to conclude the main theorem. We repeat the statement of the theorem 
for convenience.

\setcounter{theorem}{2}
\begin{theorem}
Let $\CCC$ be a class of graphs and let $\phi(\tup x,y)$ be a first-order formula, where 
$\tup x$ is an $m$-tuple of variables and $y$ is a single variable. 
\begin{enumerate}
\item If $\CCC$ is nowhere dense, then for every $\epsilon>0$ 
there exists a constant~$c$ such that for every $G\in \CCC$ and every
$A\subseteq V(G)$, we have $|S_\phi(A,G)|\leq c\cdot |A|^{1+\epsilon}.$
\item If $\CCC$ has bounded expansion, then there exists a constant~$c$ such that for every $G\in \CCC$ and every $A\subseteq V(G)$, we have $|S_\phi(A,G)|\leq c\cdot |A|$.
\end{enumerate}
\end{theorem}
\begin{proof}
Let $f$ be the polynomial from \Cref{lem:num-types-same-class},
depending on $\phi$ and $\CCC$,
and let $p$ be its degree.
 In case $\CCC$ is nowhere dense, choose
$\epsilon'$ such that $(p+2)\epsilon'+(\epsilon')^2\leq \epsilon$. In this case,
let $B$ be the closure of $A$ according to \Cref{lem:closure} with 
parameters $r$ and $\epsilon'$. In case $\CCC$ has bounded expansion, let $B$ be the closure of $A$ with parameter $r$.
According to \Cref{lem:types-over-B}, it suffices to bound the
number $|S_\phi(B,G)|$. 

Now plug the numbers of \Cref{lem:closure} and \Cref{lem:projection-complexity} 
into \Cref{lem:num-types-same-class} to conclude. 
\end{proof}

\Cref{thm:vc-density-lower-bound}, which we also repeat for
convenience, is a simple consequence of the following two
lemmas. 

\begin{theorem}
Let $\CCC$ be a class of graphs which 
is closed under taking subgraphs. 
\begin{enumerate}
\item If $\CCC$ is somewhere dense, then there is a formula 
$\phi(x,y)$ such that for every $n\in \N$ there are $G\in\CCC$ and $A\subseteq V(G)$ 
with $|A|\geq n$ and $|S_\phi(A,G)|=2^{|A|}$. 
\item If $\CCC$ has unbounded expansion, then there is a formula 
$\phi(x,y)$ such that for every function $f:\N\rightarrow \N$ 
there are $G\in\CCC$ and $A\subseteq V(G)$ 
with $|S_\phi(A,G)|>f(|\phi|)\cdot |A|$. 
\end{enumerate}
\end{theorem}

Let $\mathcal{G}_r$ be the class of $r$-subdivisions of all 
simple graphs, that is, the class comprising
all the graphs that can be obtained from any simple graph by replacing every edge by a path of
length $r$.

\begin{lemma}[\cite{nevsetvril2011nowhere}]\label{lem:lower-nd}
For every somewhere dense graph class $\CCC$ that is closed 
under taking subgraphs, there
exists an integer $r_0$ such that $\mathcal{G}_{r_0}\subseteq \CCC$.
\end{lemma}

For $r\in \N$ and a graph $G$ denote by $\nu_r(G)$ the
\emph{$r$-neighborhood complexity} of $G$ as defined
by Reidl et al.~\cite{reidl2016characterising}, that is, the number 
\[\max_{H\subseteq G,\emptyset\neq X\subseteq V(G)}\frac{|\{N_r^H[v]\cap X : v\in V(H)\}|}{|X|}.\] 

A graph $H$ is a \emph{topological depth-$r$ minor} of $G$ if
there is a mapping $\phi$ that maps vertices of~$H$ to 
vertices of $G$ such that $\phi(u)\neq \phi(v)$ for 
$u\neq v$, and edges of $H$ to paths in 
$G$ such that if $uv\in E(H)$, then $\phi(uv)$
is a path of length at most $2r$ between $u$ and $v$ in 
$G$ and furthermore, if $uv, xy\in E(H)$, then 
$\phi(uv)$ and $\phi(xy)$ are internally vertex
disjoint. We write $H\minor_r^t G$. 
Note that the above definition makes sense for 
half-integers. 

\begin{lemma}[Theorem 4 of \cite{reidl2016characterising}]\label{lem:lower-be}
Let $G$ be a graph, let $r$ be a half-integer 
and let $H\minor_r^tG$. 
Then $|E(H)|/|V(H)|\leq (2r + 1)\cdot \max \left\{\nu_1(G)^4\cdot \log^2\nu_1(G),\nu_2(G),\ldots, \nu_{\left\lceil r+1/2\right\rceil}\right\}$.
\end{lemma}

\begin{proof}[of \Cref{thm:vc-density-lower-bound}]
To prove the first statement of \Cref{thm:vc-density-lower-bound}, 
for $n\in \N$, let $\PPP(n)$ denote the graph with $n+2^n$ 
vertices $V(\PPP(n))\coloneqq \{v_1,\ldots, v_n\}\cup \{w_M : M\subseteq \{1,\ldots, n\}\}$ and edges $E(\PPP(n))\coloneqq \{v_iw_M :1\leq i\leq n, M\subseteq \{1,\ldots, n\}, i\in M\}$. 
If $\CCC$ is somewhere dense and closed under taking subgraphs, 
according to \Cref{lem:lower-nd}, there exists an integer $r_0$ 
such that $\mathcal{G}_{r_0}\subseteq \CCC$, and in particular, 
$\{\PPP(n)_{r_0} :n\in \N\}\subseteq \CCC$. Now consider 
the formula $\phi(x,y)$ stating that $x$ and~$y$ have 
distance $r_0$. Then for every $n\in \N$ we have 
$S_\phi(\PPP(n),A)=2^{|A|}$, where $A\subseteq V(\PPP(n))$ denotes the set $\{v_1,\ldots, v_n\}$, which implies the statement
of the theorem.

For the second claim, we apply a result of Dvo\v{r}\'ak~\cite{s}, 
which relates the edge density of topological depth-$r$ minors
and depth-$r$ minors. In particular, the result of Dvo\v{r}\'ak
implies that a 
class $\CCC$ of graphs has unbounded expansion if and only 
if there is $r\in \N$ such that the value $|E(H)|/|V(H)|$ is unbounded
when $H$ ranges over all graphs such that $H\minor_r^tG $ for some $G\in \CCC$.
 By applying \Cref{lem:lower-be} and 
a standard Ramsey argument we find $r_0\leq r$ such that the value
$\nu_{r_0(H)}$ is unbounded when $H$ ranges over all graphs such that $H\minor_r^t G$
for some $G\in\CCC$. As above, consider 
the formula $\phi(x,y)$ stating that $x$ and~$y$ have 
distance $r_0$ to conclude. 
\end{proof}

