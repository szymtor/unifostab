\section{On the number of types in nowhere dense graph classes}

In this section we prove \Cref{thm:vc-density} and \Cref{thm:vc-density-lower-bound}. Recall that for a
formula $\psi(\tup x,y)$, where 
$\tup x$ is an $m$-tuple of variables and $y$ is a single variable, 
\[S_\psi(A,G)=\{\{\tup a \in A^m : G\models\psi(\tup a,v)\} : v\in V(G)\}.\]

First, observe that if $A\subseteq B$, then the number of types
over $A$ cannot be larger than the number of types over $B$. 

\begin{lemma}\label{lem:types-over-B}
Let $G$ be a graph and let $A\subseteq B\subseteq V(G)$. Let 
$\psi(\tup x,y)$ be a first-order formula. Then 
$|S_\psi(A,G)|\leq |S_\psi(B,G)|$. 
\end{lemma}

We will first enlarge the set of interest such 
that the connections from outside are well controlled. This approach
was first used in Drange et al.~\cite{drange2016kernelization} for
classes of bounded expansion and extended to nowhere dense
classes in Eickmeyer et al.~\cite{eickmeyer2016neighborhood}. 

Let $G\in \CCC$ be a graph and let $A\subseteq V(G)$ be a subset of vertices. For vertices $v\in A$ and $u\in V(G)\setminus A$, a path $P$ connecting $u$ and $v$ is called {\em{$A$-avoiding}}
if all its vertices apart from~$v$ do not belong to $A$. For a positive integer $r$, the {\em{$r$-projection}} of any $u\in V(G)\setminus A$ on $A$, denoted $M^G_r(u,A)$ is the set of all vertices $v\in A$ that
can be connected to $u$ by an $A$-avoiding path of length at most $r$. The {\em{$r$-projection profile}} of a vertex $u\in V(G)\setminus A$ on $A$ is a function $\rho^G_r[u,A]$ mapping vertices of
$A$ to $\{0,1,\ldots,r,\infty\}$, defined as follows: for every $v\in A$, the value $\rho^G_r[u,A](v)$ is the length of a shortest $A$-avoiding path connecting $u$ and~$v$, and~$\infty$ in case this length
is larger than $r$. We define 
\[\projnum_r(G,A)=|\{M_r^G(u,A)\colon u\in V(G)\setminus A\}|\quad\textrm{and}\quad \projprof_r(G,A)=|\{\rho_r^G[u,A]\colon u\in V(G)\setminus A\}|\]
to be the number of different $r$-projections and $r$-projection profiles realized on $A$, respectively. Clearly, it holds that $\projnum_r(G,A)\leq \projprof_r(G,A)$.

\pagebreak
\begin{lemma}[[\cite{drange2016kernelization,eickmeyer2016neighborhood}]\label{lem:closure}
Let $\CCC$ be a class of graphs. 
\begin{enumerate}
\item If $\CCC$ has bounded expansion, then for every $r\in \N$ there is a constant $c\in \N$ such that for
every $G\in \CCC$ and $X\subseteq V(G)$ there exists a set $\cl_r(X)$, called an {\em{$r$-closure}} of $X$, with the following properties. 
\begin{itemize}
  \item $X\subseteq \cl_r(X)\subseteq V(G)$;
  \item $|\cl_r(X)|\leq c\cdot |X|$; and
  \item $|M_r^G(u,\cl_r(X))|\leq c$ for each $u\in V(G)\setminus \cl_r(X)$.
\end{itemize}
\item If $\CCC$ is nowhere dense, then for every $r\in\N$ and $\epsilon>0$ there is a 
constant $c\in\N$ such that for every $G\in \CCC$ and $X\subseteq V(G)$ there exists a set 
$\cl_r(X)$,  called an {\em{$r$-closure}} of $X$, 
with the following properties. 
\begin{itemize}
  \item $X\subseteq \cl_r(X)\subseteq V(G)$;
  \item $|\cl_r(X)|\leq c\cdot |X|^{1+\epsilon}$; and
  \item $|M_r^G(u,\cl_r(X))|\leq c\cdot |X|^{\epsilon}$ for each $u\in V(G)\setminus \cl_r(X)$.
\end{itemize}
\end{enumerate}
\end{lemma}

\begin{lemma}[\cite{drange2016kernelization,eickmeyer2016neighborhood}]\label{lem:projection-complexity}
Let $\CCC$ be a class of graphs. 
\begin{enumerate}
\item If $\CCC$ has bounded expansion, then for every $r\in \N$ there is 
  a constant $c\in\N$ such that for every graph $G\in \CCC$, and vertex subset $A\subseteq V(G)$, 
  it holds that $\projprof_r(G,A)\leq c\cdot |A|$.
  \item If $\CCC$ is nowhere dense, then for every $r\in \N$ and $\epsilon>0$ there is 
  a constant $c\in \N$ such that for every graph $G\in \CCC$, and vertex subset $A\subseteq V(G)$, 
  it holds that $\projprof_r(G,A)\leq c\cdot |A|^{1+\epsilon}$.
\end{enumerate}
\end{lemma}

We now are ready to count the number of types realized by elements
of the same projection class. 

\begin{lemma}\label{lem:num-types-same-class}
Let $\phi(\tup x,y)$ be a first-order formula of quantifier rank $q$, 
where $\tup x$ is an $m$-tuple of 
variables and $y$ is a single variable. Let $\CCC$ be a nowhere dense class of 
graphs and let $G\in\CCC$. Let 
$B\subseteq V(G)$ and assume that 
$|M_r^G(u,B)|\leq c$ for some positive integer
$c$. There are integers $k,p$ depending only on $q,m$ and $\CCC$, such that 
the following holds. Let $\{v_1,\ldots,v_\ell\}\subseteq V(G)$ be maximal such that
\begin{itemize}
\item $M_r^G(v_i,B)=M_r^G(v_j,B)$ and such that 
\item $\{\tup a \in B^m : G\models\psi(\tup a,v_i)\}\neq \{\tup a \in B^m : G\models\psi(\tup a,v_j)\}$
for all $1\leq i<j\leq \ell$. 
\end{itemize}
Then $\ell\in\Oof((k+c+1)^p)$. 
\end{lemma}
\begin{proof}
Let $r$
be the integer computable from $q$ as described in \Cref{pro:gaifman} (2). 
As $\CCC$ is nowhere dense, there is an integer $t$ such that 
$K_t\not\minor_{\lfloor 10r/2\rfloor} G$. According to 
\Cref{thm:new-uqw}, there is a polynomial  $N\colon \N\to \N$ with $N(m)=
\Oof_{r,t}{(m^{{(2t+1)}^{2rt}})}$ such that for every set $D\subseteq V(G)$
of size at least $N(m)$, for a given $m$, there exists a set $S\subseteq V(G)$ of size $|S|\leq t$ 
and a set $X\subseteq D\setminus S$ of size $|X|\geq m$ which is $r$-independent in $G-S$.
According to \Cref{pro:gaifman} (1), the set of types $\mathrm{Tp}^{q,S}_{\tup y,x}$ is 
finite and computable from $t,m$ and $q$. Let $k\coloneqq |\mathrm{Tp}^{q,S}_{\tup y,x}|$. 

Assume $\ell\geq N(k+c+1)$. Apply \Cref{thm:new-uqw} to find $S\subseteq V(G)$ of size $|S|\leq t$ 
and a set $X\subseteq \{v_1,\ldots, v_\ell\}\setminus S$ of size $|X|\geq k+c+1$ which is 
$2r$-independent in $G-S$. Let $Y\subseteq X$ be the set of vertices $v\in X$ 
with $M_r^{G-S}(v,B)\neq \emptyset$. Observe that $|Y|\leq c$, as otherwise, by the pigeon-hole 
principle, we have distinct $u,v\in Y$ with $M_r^{G-S}(u,B)\cap M_r^{G-S}(v,B)\neq \emptyset$. 
Then we have $\dist_{G-S}(u,v)\leq 2r$, contradicting the assumption that 
$X$ is $2r$-independent in $G-S$. 

As in the previous section, define $G^S$ as the graph $G-S$
colored with colors $\{C_s : s\in S\}$ as follows. For each $s\in S$, a vertex $v\in V(G)\setminus S$
is colored with color $C_s$ in $G^S$ if and only if $v$ is a neighbor of $s$ in $G$. 
According to \Cref{lem:remove-s}, we can rewrite $\phi$ to a formula $\phi'$ such that
for every valuation $\tup a,v$ of $\tup x,y$ in $G-S$ we have 
\[(G,(\tup a,v))\models \phi\Leftrightarrow (G^S,(\tup a,v))\models \phi'.\]

Let $Z\coloneqq X\setminus Y$. By assumption we have $|Z|\geq k+1$, and hence, by 
the pigeon-hole principle, we find distinct $u,v\in Z$ such that 
$(G^S,u)$ and $(G^S,v)$ have the same $(r,q)$-local types. Fix any $\tup a\in A^m$. 
Then, by construction, the sets $\{\tup a, u\}$ and $\{\tup a,v\}$ are mutually $2r$-independent
in $G-S$. 
According to \Cref{pro:gaifman} (3), the $(r,q)$-local types of $(G^S,(\tup a,u))$ and 
$(G^S,(\tup a,v))$ are equal. We apply \Cref{pro:gaifman} (2) to conclude that 
\[(G^S, (\tup a,u))\models \phi'\Leftrightarrow(G^S, (\tup a,v))\models \phi'.\]
From the construction of $\phi'$ it follows that 
\[(G, (\tup a,u))\models \phi\Leftrightarrow(G, (\tup a,v))\models \phi.\]
As the choice of $\tup a$ was arbitrary, we conclude that 
\[\{\tup a \in B^m : G\models\psi(\tup a,u)\}= \{\tup a \in B^m : G\models\psi(\tup a,v)\},\]
contradicting the assumption of the lemma. 
\end{proof}

Now, it is easy to conclude the main theorem. We repeat the statement of the theorem 
for convenience.

\setcounter{theorem}{3}
\begin{theorem}
Let $\CCC$ be a class of graphs and let $\psi(\tup x,y)$ be a first-order formula, where 
$\tup x$ is an $m$-tuple of variables and $y$ is a single variable. 
\begin{enumerate}
\item If $\CCC$ is nowhere dense, then for every $\epsilon>0$ 
there exists a constant~$c$ such that for every $G\in \CCC$ and every
$A\subseteq V(G)$, we have 
\[|S_\psi(A,G)|=|\{\{\tup a\ \in A^m : G\models\psi(\tup a,v)\} : v\in V(G)\}|\leq c\cdot |A|^{1+\epsilon}.\]
\item If $\CCC$ has bounded expansion, then there exists a constant~$c$ such that for every $G\in \CCC$ and every $A\subseteq V(G)$, we have $|S_\psi(A,G)|\leq c\cdot |A|$.
\end{enumerate}
\end{theorem}
\begin{proof}
Let $p$ be the constant from \Cref{lem:num-types-same-class} computed
from $\psi$ and $\CCC$. In case $\CCC$ is nowhere dense, choose
$\epsilon'$ such that $(p+2)\epsilon'+(\epsilon')^2\leq \epsilon$. In this case 
compute the closure $B$ of $A$ according to \Cref{lem:closure} with 
parameters $r$ and $\epsilon'$, in case $\CCC$ has bounded expansion 
compute it just with parameter $r$. 
According to \Cref{lem:types-over-B}, it suffices to bound the
number $|S_\psi(B,G)|$. 

Now plug the numbers of \Cref{lem:closure} and \Cref{lem:projection-complexity} 
into \Cref{lem:num-types-same-class} to conclude. 
\end{proof}