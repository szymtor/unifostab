\section{On the number of types in nowhere dense graph classes}

In this section we prove \Cref{thm:vc-density} and \Cref{thm:vc-density-lower-bound}. Recall that for a
formula $\psi(\tup x,y)$, where 
$\tup x$ is an $m$-tuple of variables and $y$ is a single variable, 
\[S_\psi(A,G)=\{\{a\ \in A : G\models\psi(a,\tup v)\} : \tup v\in V(G)^m\}.\]

First, observe that if $A\subseteq B$, then the number of types
over $A$ cannot be larger than the number of types over $B$. 

\begin{lemma}\label{lem:types-over-B}
Let $G$ be a graph and let $A\subseteq B\subseteq V(G)$. Let 
$\psi(\tup x,y)$ be a first-order formula. Then 
$|S_\psi(A,G)|\leq |S_\psi(B,G)|$. 
\end{lemma}

We will first enlarge the set of interest such 
that the connections from outside are well controlled. This approach
was first used in Drange et al.~\cite{drange2016kernelization} for
classes of bounded expansion and extended to nowhere dense
classes in Eickmeyer et al.~\cite{eickmeyer2016neighborhood}. 

Let $G\in \CCC$ be a graph and let $A\subseteq V(G)$ be a subset of vertices. For vertices $v\in A$ and $u\in V(G)\setminus A$, a path $P$ connecting $u$ and $v$ is called {\em{$A$-avoiding}}
if all its vertices apart from~$v$ do not belong to $A$. For a positive integer $r$, the {\em{$r$-projection}} of any $u\in V(G)\setminus A$ on $A$, denoted $M^G_r(u,A)$ is the set of all vertices $v\in A$ that
can be connected to $u$ by an $A$-avoiding path of length at most $r$. The {\em{$r$-projection profile}} of a vertex $u\in V(G)\setminus A$ on $A$ is a function $\rho^G_r[u,A]$ mapping vertices of
$A$ to $\{0,1,\ldots,r,\infty\}$, defined as follows: for every $v\in A$, the value $\rho^G_r[u,A](v)$ is the length of a shortest $A$-avoiding path connecting $u$ and~$v$, and~$\infty$ in case this length
is larger than $r$. We define 
\[\projnum_r(G,A)=|\{M_r^G(u,A)\colon u\in V(G)\setminus A\}|\quad\textrm{and}\quad \projprof_r(G,A)=|\{\rho_r^G[u,A]\colon u\in V(G)\setminus A\}|\]
to be the number of different $r$-projections and $r$-projection profiles realized on $A$, respectively. Clearly, it holds that $\projnum_r(G,A)\leq \projprof_r(G,A)$.

\pagebreak
\begin{lemma}[[\cite{drange2016kernelization,eickmeyer2016neighborhood}]\label{lem:closure}
Let $\CCC$ be a class of graphs. 
\begin{enumerate}
\item If $\CCC$ has bounded expansion, then for every $r\in \N$ there is a constant $c\in \N$ such that for
every $G\in \CCC$ and $X\subseteq V(G)$ there exists a set $\cl_r(X)$, called an {\em{$r$-closure}} of $X$, with the following properties. 
\begin{itemize}
  \item $X\subseteq \cl_r(X)\subseteq V(G)$;
  \item $|\cl_r(X)|\leq c\cdot |X|$; and
  \item $|M_r^G(u,\cl_r(X))|\leq c$ for each $u\in V(G)\setminus \cl_r(X)$.
\end{itemize}
\item If $\CCC$ is nowhere dense, then for every $r\in\N$ and $\epsilon>0$ there is a 
constant $c\in\N$ such that for every $G\in \CCC$ and $X\subseteq V(G)$ there exists a set 
$\cl_r(X)$,  called an {\em{$r$-closure}} of $X$, 
with the following properties. 
\begin{itemize}
  \item $X\subseteq \cl_r(X)\subseteq V(G)$;
  \item $|\cl_r(X)|\leq c\cdot |X|^{1+\epsilon}$; and
  \item $|M_r^G(u,\cl_r(X))|\leq c\cdot |X|^{\epsilon}$ for each $u\in V(G)\setminus \cl_r(X)$.
\end{itemize}
\end{enumerate}
\end{lemma}

\begin{lemma}[\cite{drange2016kernelization,eickmeyer2016neighborhood}]\label{lem:projection-complexity}
Let $\CCC$ be a class of graphs. 
\begin{enumerate}
\item If $\CCC$ has bounded expansion, then for every $r\in \N$ there is 
  a constant $c\in\N$ such that for every graph $G\in \CCC$, and vertex subset $A\subseteq V(G)$, 
  it holds that $\projprof_r(G,A)\leq c\cdot |A|$.
  \item If $\CCC$ is nowhere dense, then for every $r\in \N$ and $\epsilon>0$ there is 
  a constant $c\in \N$ such that for every graph $G\in \CCC$, and vertex subset $A\subseteq V(G)$, 
  it holds that $\projprof_r(G,A)\leq c\cdot |A|^{1+\epsilon}$.
\end{enumerate}
\end{lemma}

We now are ready to count the number of types realized by elements
of the same projection class. 

\begin{lemma}\label{lem:num-types-same-class}
Let $\phi(\tup x,y)$ be a first-order formula of locality radius $r$ (as defined in \Cref{pro:gaifman}). 
Let $G$ be a graph such that $K_t\not\minor_{\left\lfloor 5r/2\right\rfloor} G$. 
Let $X=\tup x\oplus y$ and denote by $\mathrm{Tp}^{q,S}_X$ the set of types over a 
set $S$ of colors of size $t$. Let 
$B\subseteq V(G)$ and assume that 
$|M_r^G(u,B)|\leq c$ for some positive integer
$c$. Let $v_1,\ldots, v_k$ be a maximal set
of elements which have different $B$-types and which
have the same $B$-projection. Then $k\in\Oof((|\mathrm{Tp}^{q,S}_X|+c+1)^{(2t+1)^{2rt}})$. 
\end{lemma}
\begin{proof}
Assume $t\geq (tp_{q,s}+c+1)^{p(r)}$. Then by uniform
quasi-wideness, there exists a set $S$ and elements 
$w_1,\ldots, w_m$ which are $2r$-independent in $G-S$. 
Here, $m\geq tp_{q,s}+c+1$. Because $|M_r(u,B)|\leq \fcl(r,\epsilon)\cdot |B|^{\epsilon}$, and no two elements $w,w'$ among
$w_1,\ldots, w_m$ can reach the same element of $B$
in $G-S$, there are at most $\fcl(r,\epsilon)|B|^\epsilon$
elements which reach an element of $B$ in $G-S$. 
Hence, there remain $tp_{q,s}$ elements which have
an empty intersection with $B$ in their $N_r^{G-S}(w)$. 
By \Cref{lem:coloring}, 
the type of the elements depends only on the local $q-S$-type. 
As we have $tp_{q,r}+1$ elements, two of them have the
same type. This is a contradiction with our assumption. 
\end{proof}

Now, it is easy to conclude the main theorem. 

\marginpar{set counter}
\begin{theorem}
Let $\CCC$ be a nowhere dense class of graphs, 
let $\epsilon>0$ and let $\psi(\tup x,y)$ be a first-order formula, where 
$\tup x$ is an $m$-tuple of variables and $y$ is a single variable. 
There exists a constant~$c$ such that for every $G\in \CCC$ 
and every
$A\subseteq V(G)$, we have 
\[|S_\psi(A,G)|=|\{\{\tup a\ \in A^m : G\models\psi(\tup a,v)\} : v\in V(G)\}|\leq c\cdot |A|^{1+\epsilon}.\]
\end{theorem}
\begin{proof}
Compute the closure $B$ of $A$ according to \Cref{lem:closure}. 
According to \Cref{lem:types-over-B}, it suffices to bound the
number $|S_\psi(B,G)|$. Now plug the numbers of 
\Cref{lem:closure} and \Cref{lem:projection-complexity} 
into \Cref{lem:num-types-same-class} and conclude by rescaling $\epsilon$. . 
\end{proof}