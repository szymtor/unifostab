% We initiate the study of {\em{VC-density}} in nowhere dense classes of graphs; nowhere denseness
% is an abstract notion of sparsity introduced by Ne\v set\v ril and Ossona de Mendez~\cite{nevsetvril2010first,nevsetvril2011nowhere}.
% Our main result states
We prove that for every class of graphs $\CCC$ which is nowhere dense, as defined by  Ne\v set\v ril and Ossona de Mendez~\cite{nevsetvril2010first,nevsetvril2011nowhere},
and for every  first order formula $\phi(\tup x,\tup y)$, 
whenever one draws a graph $G\in \CCC$ and a subset of its nodes $A$, the number of subsets of $A^{|\tup y|}$ which are of the form $\set{\tup v\in A^{|\tup y|}\, \colon\, G\models\phi(\bar u,\tup v)}$ for some valuation~$\tup u$ of $\tup x$ in $G$
is bounded by $\Oof(|A|^{|\tup x|+\epsilon})$, for every $\epsilon>0$. 
This provides optimal bounds on the VC-density of first-order definable set systems in nowhere dense graph classes.
%
We also give two new proofs of upper bounds on quantities in nowhere dense classes which are relevant for their logical treatment.
Firstly, we provide a new proof of the fact that nowhere dense classes are uniformly
quasi-wide, implying explicit, polynomial upper bounds on the functions
relating the two notions.  Secondly, we give a new combinatorial proof
of the result of Adler and Adler~\cite{adler2014interpreting} stating
that every nowhere dense class of graphs is stable. In contrast to the previous proofs of the above results,
our proofs are completely finitistic and constructive, and yield explicit and computable upper
bounds on quantities related to uniform quasi-wideness (margins) and stability (ladder indices).