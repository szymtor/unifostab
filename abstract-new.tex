\begin{abstract}
\noindent 
We study the concept of VC-density in nowhere dense classes of graphs, 
which is an abstract notion of sparsity introduced by Ne\v set\v ril and Ossona de Mendez~\cite{nevsetvril2010first,nevsetvril2011nowhere}.
Our main result states that for every nowhere dense class of graphs $\CCC$, whenever one draws a graph $G\in \CCC$ and a parameter set $A\subseteq V(G)$,
the number of subsets of $A^\ell$ which are definable by a fixed first order formula $\phi(\tup x,\tup y)$
with $\ell$ parameter variables $\tup y$ ranging over $A$ is bounded by $\Oof(|A|^{\ell+\epsilon})$, for every $\epsilon>0$. 
This provides optimal bounds on the VC-density of first-order definable sets in nowhere dense graph classes.

We also give two new proofs of upper bounds on quantities in nowhere dense classes which are relevant for their algorithmic and logical treatment.
Firstly, we provide a new proof of the fact that nowhere dense classes are uniformly
quasi-wide, implying explicit, polynomial upper bounds on the functions
relating the two notions.  Secondly, we give a new combinatorial proof
of the result of Adler and Adler~\cite{adler2014interpreting} stating
that every nowhere dense class of graphs is stable. In contrast to the previous proofs of the above results,
our proofs are completely finitistic and constructive, and yield explicit and computable upper
bounds on quantities related to uniform quasi-wideness (margins) and stability (ladder indices).
\end{abstract}