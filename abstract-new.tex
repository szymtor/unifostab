\begin{abstract}
\noindent 
We study nowhere dense classes of graphs, recently introduced by Ne\v
set\v ril and Ossona de
Mendez~\cite{nevsetvril2010first,nevsetvril2011nowhere}.  Firstly, we
provide a new proof for the fact that these classes are uniformly
quasi-wide and improve previously known bounds on the functions
relating the two notions.  Secondly, we give a new combinatorial proof
of the result of Adler and Adler~\cite{adler2014interpreting} stating
that nowhere dense classes of graphs are stable.  In contrast to the
original proof, our proof is completely finitistic and yields explicit
bounds for ladder indices of first-order formulas on nowhere dense
classes of graphs.  Finally, we establish an optimal bound on the
VC-density function of nowhere dense classes. More precisely, let
$\CCC$ be a nowhere dense class of graphs and let
$\psi(\tup x,\tup y)$ be a first-order formula, where $\tup x$ is an
$m$-tuple and $\tup y$ is an $n$-tuple of variables.  For a graph $G$
and $A\subseteq V(G)$ let
$S_\psi(A,G)=\{\{\tup a\ \in A^m : G\models\psi(\tup a,\tup b)\} :
\tup b\in V(G)^n\}$.  We prove that
\[\limsup_{a\rightarrow \infty}\max_{\substack{G\in\CCC\\A\subseteq
    V(G), |A|=a}}\frac{\log |S_\psi(A,G)|}{\log a}\leq n.\]
We want to highlight that the present paper
can be read without any knowledge of model 
theory beyond the basics of first-order logic.

%
% \sebi{Finally, let $\CCC$ be a nowhere dense class of graphs, let
% $\epsilon>0$ and let $\psi(\tup x,y)$ be a first-order formula,
% where~ $\tup x$ is an $m$-tuple of variables and $y$ is a single
% variable. We prove that there is a constant $c$ such that for every
% $G\in \CCC$ and every $A\subseteq V(G)$, the number of $\psi$-types
% over $A$, that is,
% $|S_\psi(A,G)|=|\{\{\tup a\ \in A^m : G\models\psi(\tup a,v)\} :
% v\in V(G)\}|$, is bounded by $c\cdot |A|^{1+\epsilon}$.
% \begin{change}{sz}Moreover, if $\CCC$ has bounded expansion, then
%   the $\epsilon$ in the exponent can be dropped.\end{change} For
% classes of graphs that are closed under taking subgraphs we derive
% new characterizations of bounded expansion and nowhere denseness.}
\end{abstract}