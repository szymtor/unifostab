\section{Packing and traversal numbers for nowhere dense graphs}
In this section, we give an application 
of~\cref{thm:vc-density}, proving a 
duality result for nowhere dense graph classes.

A \emph{set system} is a family  $\cal F$ of subsets of a set $X$.
Its  \emph{packing} is a subfamily of $\cal F$ of pairwise disjoint subsets, and its \emph{traversal} (or \emph{hitting set}) is a subset of $X$ which intersects every member of $\cal F$.
The \emph{packing number} of~$\cal F$, denoted $\nu(\cal F)$ is the largest cardinality of a packing,
and the \emph{transversality} of $\cal F$, denoted
$\tau(\cal F)$ is the smallest cardinality of a traversal.
Note that if $\cal G$ is a finite set system, then
$\nu({\cal G})\le \tau(\cal G)$. 
The set system $\cal F$ has the \emph{Erd{\H o}s-P\'{o}sa property} if there is a function $f\from\N\to\N$ such that every finite subfamily $\cal G$ of $\cal F$
satisfies $\tau({\cal G})\le f(\nu(\cal G))$. 

We prove that set systems defined by first order formulas in nowhere dense graph classes have the Erd{\H o}s-P\'{o}sa property, in the following sense.

\begin{theorem}\label{thm:erdos-posa}
	Fix a nowhere dense class of graphs $\CCC$ and a 
	formula $\phi(x,y)$ with free variables $x,y$.
	 % where $x$ is a single variable and $\bar y$ is a tuple of variables.
	There is a function $f\from \N\to\N$ with the following property.
	Let $G\in \CCC$ be a graph and let $\cal G$
	be a family of subsets of $V(G)$ consisting of sets of the form $\setof{b\in V(G)}{\phi(a, b)}$, where~$a\in V(G)$.
Then~$\tau({\cal G})\le f(\nu(\cal G))$.
\end{theorem}



We will apply the following result of Matou{\v s}ek,
 whose proof relies on the  proof of Alon and Kleitman~\cite{ALON1992103} of the conjecture of Hardwiger and Debrunner. 
In the result of Matou{\v s}ek, the family $\cal F$ is infinite. By $\pi_{\cal F}^*(m)$ we denote the \emph{dual shatter function}, which can be defined as the maximal number 
of occupied cells in the Venn diagram of $m$ elements of $\cal F$.


\begin{theorem}\label{thm:pq}
	Let $\cal F$ be a set system with $\pi^*_{\cal F}(m)=o(m^k)$,
	for some integer $k$, and let $p\ge k$.
	There is a constant $T$ such that the following holds for every finite family $\cal G\subset \cal F$: 
	if $\cal G$ has the $(p,k)$-property, meaning that 
	among every $p$ sets in $\cal G$ some $k$ intersect, then $\tau ({\cal G})\le T$.
\end{theorem}

%
% The dual of a set system $\cal F$ with base set $X$ is the set system $\cal F^*$, with base set $\cal F$
% and elements of the form $x^*$, where for $x\in X$, the set $x^*\subset \cal F$ is defined so that $F\in x^*$ if and only if $x\in F$,
% for $F\in \cal F$.  The \emph{shatter function} of $\cal F$ is the function $\pi_{\cal F}$
% defined so that $\pi_{\cal F}(m)$ is the maximal size of a family $\setof{A\cap F}{F\in \cal F}$,
% for $A$ ranging over subsets of $X$ of size $m$.
% Note that $\pi_{\cal F^*}=\pi_{\cal F}^*$, i.e.,
% the dual shatter function of $\cal F$ is the shatter function of the dual of $\cal F$. By dualizing~\cref{thm:pq}, we get:
%
% \begin{corollary}\label{cor:dual-pq}
% 	Let $\cal F$ be a set system with base set $X$, with $\pi_{\cal F}(m)=o(m^k)$,
% 	for some integer $k$, and let $p\ge k$.
% 	There is a constant $T$ such that the following holds for every finite set $Y\subset X$:
% 	if
% 	among every $p$ elements of $Y$, some $k$ are contained in a single element of $\cal F$, then $Y$
% 	is contained in a union of $T$ elements of $\cal F$.
% \end{corollary}


\begin{proof}
	A graph $G$ %with a distinguished set of vertices $A\subset V(G)$ 
	defines a relational structure which we denote 
	$\str A_{G}$, with universe $V(G)$,
	a binary relation $E$ interpreted as $\setof{(v,w)}{v,w\in V(G),vw\in E(G)}$,  
	% a unary predicate $U$ whose interpretation is the set $A$,
	and with a binary relation $\sim$ which is the complete relation.
	
	To avoid set-theoretic problems, we assume that all graphs $G\in \CCC$ have integers as vertices, and that two graphs $G,H$ have disjoint vertex sets.
		Define the structure $\str A_{\CCC}$ as the disjoint union of all structures $\str A_{G}$,
	for all graphs $G\in \CCC$. Therefore, $\str A_{\CCC}$
	is a structure with a binary relation symbol $E$,
	% a unary relation symbol $U$ and 
	a binary relation symbol $\sim$. Let $\cal F$ denote the
	family of subsets of $V(\str A_{\CCC})$
which are of the form
$\setof{b\in V(\str A_{\CCC})} {\phi(a,b)\land a\sim b}$, where $a$ ranges over  $V(\str A_{\CCC})$.

The following is obtained from~\cref{thm:vc-density} by symbol pushing.
\begin{claim}
	The dual shatter function of the family $\cal F$ satisfies	$\pi^*_{\cal F}(m)=O(m^{1+\epsilon})$,
	 for every fixed $\epsilon>0$.
In particular, $\pi^*_{\cal F}(m)=o(m^{2})$.
\end{claim}

To prove~\cref{thm:erdos-posa},
consider the function $f\from \N \to \N$
defined so that $f(\nu)$ is the value $T$ obtained from~\cref{thm:pq} applied to $\cal F$, to $k=2$ and $p=\nu+1$.

Suppose now that $G\in \CCC$ is a graph and $\GGG$
is a family of subsets of $V(G)$ consisting of sets of the form $\setof{b\in V(G)}{\phi(a,b)}$, for $a\in V(G)$.
We identify $\GGG$ with a subfamily of $\FFF$ in the natural way, by treating a set $\setof{b\in V(G)}{\phi(a,b)}$
as the set $\setof{b\in V(\str A_{\cal C})}{\phi(a,b)\land a\sim b}$,
where $b$ is the element of $V(G)=V(\str A_G)\subset V(\str A_{\CCC})$.

Let $\nu$ be the packing number of $\GGG$.
In particular, for every $\nu+1$ subsets of $\GGG$
there exists a vertex $v\in V(G)$
which is contained in two elements of $\GGG$.
Hence, $\GGG$ is a $(p,2)$-family for $p=\nu+1$.
By~\cref{thm:pq}, $\tau(\GGG)\le T=f(\nu)=f(\nu(G))$,  as required.
\end{proof}