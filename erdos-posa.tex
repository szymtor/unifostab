\section{Packing and covering by definable sets}
In this section, we give an application 
of~\cref{thm:vc-density}, proving a 
duality result for nowhere dense graph classes.

A \emph{set system} is a family  $\cal F$ of subsets of a set $X$.
Its  \emph{packing} is a subfamily of $\cal F$ of pairwise disjoint subsets, and its \emph{traversal} (or \emph{hitting set}) is a subset of $X$ which intersects every member of $\cal F$.
The \emph{packing number} of~$\cal F$, denoted $\nu(\cal F)$, is the largest cardinality of a packing in $\cal F$,
and the \emph{transversality} of $\cal F$, denoted
$\tau(\cal F)$, is the smallest cardinality of a traversal of $\cal F$.
Note that if $\cal G$ is a finite set system, then
$\nu({\cal G})\le \tau(\cal G)$. 
The set system $\cal F$ has the \emph{Erd{\"o}s-Pósa property} if there is a function $f\from\N\to\N$ such that every finite subfamily $\cal G$ of $\cal F$
satisfies $\tau({\cal G})\le f(\nu(\cal G))$. 

We prove that set systems defined by first order formulas in nowhere dense graph classes have the Erd{\H o}s-P{\'o}sa property, in the following sense.

\todo{wrong thm number}
\begin{theorem}\label{thm:erdos-posa}
	Fix a nowhere dense class of graphs $\CCC$ and a 
	formula $\phi(x,y)$ with two free variables $x,y$.
	Then there is a function $f\from \N\to\N$ with the following property.
	Let $G\in \CCC$ be a graph and let $\cal G$
	be a family of subsets of $V(G)$ consisting of sets of the form $\setof{v\in V(G)}{\phi(u, v)}$, where~$u$ is some vertex of $V(G)$.
Then~$\tau({\cal G})\le f(\nu(\cal G))$.
\end{theorem}

We will apply the following result of Matou{\v s}ek~\cite{Matousek:2004:BVI:1005787.1005789},
which relies on the proof of Alon and Kleitman~\cite{alon-kleitman} of the conjecture of Hardwiger and Debrunner. 
In the result of Matou{\v s}ek, the set system $\cal F$ is infinite. For $m\in \N$, by $\pi_{\cal F}^*(m)$ we denote the \emph{dual shatter function} of $\cal F$, which is defined as the maximal number 
of occupied cells in the Venn diagram of $m$ sets in $\cal F$.

\begin{theorem}[Matou{\v s}ek, \cite{Matousek:2004:BVI:1005787.1005789}]\label{thm:pq}
	Let $\cal F$ be a set system with $\pi^*_{\cal F}(m)=o(m^k)$,
	for some integer $k$, and let $p\ge k$.
	Then there is a constant $T$, depending on $p$ and $k$, such that the following holds for every finite family $\cal G\subset \cal F$: 
	if $\cal G$ has the $(p,k)$-property, meaning that 
	among every $p$ sets in $\cal G$ some $k$ have a non-empty intersection, then $\tau ({\cal G})\le T$.
\end{theorem}

\begin{proof}[of \cref{thm:erdos-posa}]
For a graph $G$, define the set system ${\cal F}_G$ on the ground set $V(G)$ as
$${\cal F}_G = \setof{\setof{v\in V(G)}{\phi(u, v)}}{u\in V(G)}.$$
Let then $\cal F$ be the disjoint union of set systems ${\cal F}_G$ for $G\in \CCC$. That is, 
the ground set of $\cal F$ is the disjoint union of the vertex sets $V(G)$ for $G\in \CCC$, and for each $G\in \CCC$ we add to ${\cal F}$
a copy of ${\cal F}_G$ over the copy of relevant $V(G)$.
Then the following claim follows directly from~\cref{thm:vc-density}.

\begin{claim}
The dual shatter function of $\cal F$ satisfies $\pi^*_{\cal F}(m)=\Oof(m^{1+\epsilon})$,
for every fixed $\epsilon>0$. In particular, $\pi^*_{\cal F}(m)=o(m^{2})$.
\end{claim}

Consider the function $f\from \N \to \N$ defined so that $f(\nu)$ is the value $T$ obtained from~\cref{thm:pq} applied to $\cal F$, $k=2$, and $p=\nu+1$.
Suppose now that $G\in \CCC$ is a graph and $\GGG\subseteq \FFF_G$
is a family of subsets of $V(G)$ consisting of sets of the form $\{v\in V(G)\,\colon\,\phi(u,v)\}$, where $u$ is some vertex of $G$.
We identify $\GGG$ with a subfamily of $\FFF$ in the natural way, following the embedding of $\FFF_G$ into $\FFF$ used in the construction of the latter.
Let $\nu$ be the packing number of $\GGG$.
In particular, for every $\nu+1$ subsets of $\GGG$
there is a vertex $v\in V(G)$
which is contained in two elements of~$\GGG$.
Hence, $\GGG$ is a $(p,2)$-family for $p=\nu+1$.
By~\cref{thm:pq}, $\tau(\GGG)\le T=f(\nu)=f(\nu(\GGG))$, as required.
\end{proof}

\begin{comment}
	A graph $G$ %with a distinguished set of vertices $A\subset V(G)$ 
	defines a relational structure which we denote 
	$\str A_{G}$, with universe $V(G)$,
	a binary relation $E$ interpreted as $\setof{(v,w)}{v,w\in V(G),vw\in E(G)}$,  
	% a unary predicate $U$ whose interpretation is the set $A$,
	and with a binary relation $\sim$ which is the complete relation.
	
	To avoid set-theoretic problems, we assume that all graphs $G\in \CCC$ have integers as vertices, and that two graphs $G,H$ have disjoint vertex sets.
	
	Define the structure $\str A_{\CCC}$ as the disjoint union of all structures $\str A_{G}$,
	for all graphs $G\in \CCC$. Therefore, $\str A_{\CCC}$
	is a structure with a binary relation symbol $E$,
	% a unary relation symbol $U$ and 
	a binary relation symbol $\sim$. Let $\cal F$ denote the
	family of subsets of $V(\str A_{\CCC})$
which are of the form
$\setof{b\in V(\str A_{\CCC})} {\phi(a,b)\land a\sim b}$, where $a$ ranges over  $V(\str A_{\CCC})$.

The following is obtained from~\cref{thm:vc-density} by symbol pushing.
\begin{claim}
	The dual shatter function of the family $\cal F$ satisfies	$\pi^*_{\cal F}(m)=O(m^{1+\epsilon})$,
	 for every fixed $\epsilon>0$.
In particular, $\pi^*_{\cal F}(m)=o(m^{2})$.
\end{claim}
To prove~\cref{thm:erdos-posa},
consider the function $f\from \N \to \N$
defined so that $f(\nu)$ is the value $T$ obtained from~\cref{thm:pq} applied to $\cal F$, to $k=2$ and $p=\nu+1$.

Suppose now that $G\in \CCC$ is a graph and $\GGG$
is a family of subsets of $V(G)$ consisting of sets of the form $\setof{b\in V(G)}{\phi(a,b)}$, for $a\in V(G)$.
We identify $\GGG$ with a subfamily of $\FFF$ in the natural way, by treating a set $\setof{b\in V(G)}{\phi(a,b)}$
as the set $\setof{b\in V(\str A_{\cal C})}{\phi(a,b)\land a\sim b}$,
where $b$ is the element of $V(G)=V(\str A_G)\subset V(\str A_{\CCC})$.

Let $\nu$ be the packing number of $\GGG$.
In particular, for every $\nu+1$ subsets of $\GGG$
there exists a vertex $v\in V(G)$
which is contained in two elements of $\GGG$.
Hence, $\GGG$ is a $(p,2)$-family for $p=\nu+1$.
By~\cref{thm:pq}, $\tau(\GGG)\le T=f(\nu)=f(\nu(G))$,  as required.
\end{proof}
\end{comment}