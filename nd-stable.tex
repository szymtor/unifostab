\documentclass[11pt, fleqn]{article}
\usepackage[english]{babel}

\usepackage[lmargin=1.1in,rmargin=1.1in,bottom=1.3in,top=1.3in,
twoside=False]{geometry}

\usepackage{relsize,xspace}
 \usepackage{xcolor}
 \usepackage{mathtools}
 \usepackage{todonotes}
 \usepackage{comment}
\usepackage{microtype}
\usepackage{amsmath}
\usepackage{amssymb}
\usepackage{amsfonts}
\usepackage{stmaryrd}
\usepackage{bm}
\usepackage{tikz}
\usepackage{refcount}
\usepackage{wrapfig}

\usepackage{marginnote}


\definecolor{blue}{rgb}{0.1,0.2,0.5}
\definecolor{brown}{rgb}{0.6,0.6,0.2}
\usepackage[ocgcolorlinks, linkcolor={blue}, citecolor={brown}]{hyperref}

\usepackage[amsmath,thmmarks,hyperref]{ntheorem}
\usepackage{cleveref}


\crefformat{page}{#2page~#1#3}%
\Crefformat{page}{#2Page~#1#3}%
\crefformat{equation}{#2(#1)#3}%
\Crefformat{equation}{#2(#1)#3}%
\crefformat{figure}{#2Figure~#1#3}%
\Crefformat{figure}{#2Figure~#1#3}%
\crefformat{section}{#2Section~#1#3}
\Crefformat{section}{#2Section~#1#3}
\crefformat{chapter}{#2Chapter~#1#3}
\Crefformat{chapter}{#2Chapter~#1#3}
\crefformat{chapter*}{#2Chapter~#1#3}
\Crefformat{chapter*}{#2Chapter~#1#3}
\crefformat{part}{#2Part~#1#3}
\Crefformat{part}{#2Part~#1#3}
\crefformat{enumi}{#2(#1)#3}
\Crefformat{enumi}{#2(#1)#3}

\usepackage{enumerate}

\usepackage{latexsym}

% BEGIN ntheorem configuration

\theoremnumbering{arabic}
\theoremstyle{plain}
\theoremsymbol{}
\theorembodyfont{\itshape}
\theoremheaderfont{\normalfont\bfseries}
\theoremseparator{}

\newtheorem{theorem}{Theorem}
\crefformat{theorem}{#2Theorem~#1#3}
\Crefformat{theorem}{#2Theorem~#1#3}

\newcommand{\newtheoremwithcrefformat}[2]{%
  \newtheorem{#1}[lemma]{#2}%
  \crefformat{#1}{##2\MakeUppercase#1~##1##3}%
  \Crefformat{#1}{##2\MakeUppercase#1~##1##3}%
}
\newcommand{\newseptheoremwithcrefformat}[2]{%
  \newtheorem{#1}{#2}%
  \crefformat{#1}{##2\MakeUppercase#1~##1##3}%
  \Crefformat{#1}{##2\MakeUppercase#1~##1##3}%
}

\newseptheoremwithcrefformat{lemma}{Lemma}
\newtheoremwithcrefformat{proposition}{Proposition}
\newtheoremwithcrefformat{observation}{Observation}
\newtheoremwithcrefformat{conjecture}{Conjecture}
\newtheoremwithcrefformat{corollary}{Corollary}
\newseptheoremwithcrefformat{claim}{Claim}
\theorembodyfont{\upshape}
\newtheoremwithcrefformat{example}{Example}
\newtheoremwithcrefformat{remark}{Remark}
\newseptheoremwithcrefformat{definition}{Definition}

\theoremstyle{nonumberplain}
\theoremheaderfont{\scshape}
\theorembodyfont{\normalfont}
\theoremsymbol{\ensuremath{\square}}
\newtheorem{proof}{Proof.}

\theoremsymbol{\ensuremath{\lrcorner}}
\newtheorem{clproof}{Proof.}

% END ntheorem configuration


%\setlength{\parskip}{0.1cm}
%\setlength{\parindent}{0cm}
%\setlength{\mathindent}{1cm}

\newcommand{\wcol}{\mathrm{wcol}}
\newcommand{\col}{\mathrm{col}}
\newcommand{\adm}{\mathrm{adm}}
\newcommand{\tw}{\mathrm{tw}}
\newcommand{\WReach}{\mathrm{WReach}}
\newcommand{\SReach}{\mathrm{SReach}}
\newcommand{\wcolorder}{\sqsubseteq}
\newcommand{\Oof}{\mathcal{O}}
\newcommand{\CCC}{\mathcal{C}}
\newcommand{\NNN}{\mathcal{N}}
\newcommand{\WWW}{\mathcal{W}}
\newcommand{\DDD}{\mathcal{D}}
\newcommand{\PPP}{\mathcal{P}}
\newcommand{\FFF}{\mathcal{F}}
\newcommand{\GGG}{\mathcal{G}}
\newcommand{\YYY}{\mathcal{Y}}
\newcommand{\nei}{\mathrm{nei}}
\renewcommand{\ker}{\mathrm{ker}}
\newcommand{\core}{\mathrm{core}}

\newcommand{\cutrk}{\mathrm{cutrk}}
\newcommand{\rank}{\mathrm{rank}}
\newcommand{\rw}{\mathrm{rw}}


\newcommand{\grad}{\nabla}
\newcommand{\ds}{\mathbf{ds}}
\newcommand{\cl}{\mathrm{cl}}
\newcommand{\cst}{\alpha}

\newcommand{\fnei}{f_{\nei}}
\newcommand{\fwcol}{f_{\wcol}}
\newcommand{\fker}{f_{\ker}}
\newcommand{\fproj}{f_{\mathrm{proj}}}
\newcommand{\fcl}{f_{\cl}}
\newcommand{\fgrad}{f_{\grad}}
\newcommand{\fpaths}{f_{\mathrm{pth}}}
\newcommand{\fapx}{f_{\mathrm{apx}}}
\newcommand{\fcore}{f_{\mathrm{core}}}
\newcommand{\ffin}{f_{\mathrm{fin}}}

\newcommand\blfootnote[1]{%
  \begingroup
  \renewcommand\thefootnote{}\footnote{#1}%
  \addtocounter{footnote}{-1}%
  \endgroup
}

\newcommand{\suchthat}{ \colon }
\newcommand{\sth}{ \colon }
\newcommand{\ie}{i.e.\@ }
\newcommand{\N}{\mathbb{N}}
\newcommand{\R}{\mathbb{R}}
\newcommand{\tup}[1]{\overline{#1}}
\renewcommand{\phi}{\varphi}
\renewcommand{\epsilon}{\varepsilon}
\newcommand{\strA}{\mathfrak{A}}
\newcommand{\strB}{\mathfrak{B}}
\newcommand{\FO}{\mathrm{FO}}
\newcommand{\minor}{\preccurlyeq}
\newcommand{\dist}{\mathrm{dist}}
\newcommand{\indx}{\mathrm{index}}
\renewcommand{\mid}{~:~}

\newcommand{\profnum}{\widehat{\nu}}
\newcommand{\projnum}{\mu}
\newcommand{\projprof}{\widehat{\mu}}

\newcommand{\abs}[1]{\ensuremath{\left\lvert#1\right\rvert}}

\newcounter{aux}

\title{On Wideness and Stability
\thanks{
The work of R.\ Rabinovich is supported by the
European Research Council (ERC) under the European Union's Horizon
2020 research and innovation programme (ERC consolidator grant DISTRUCT,
agreement No.\ 648527).
The work of M.\ Pilipczuk and S.\ Siebertz is supported by the National Science Centre of 
Poland via POLONEZ grant agreement UMO-2015/19/P/ST6/03998, 
which has received funding from the European Union's Horizon 2020 research and 
innovation programme (Marie Sk\l odowska-Curie grant agreement No.\ 665778).
M. Pilipczuk is supported by the Foundation for Polish Science (FNP) via the START stipend programme.
}}

\author{
Micha\l~Pilipczuk\thanks{Institute of Informatics, University of Warsaw, Poland, \texttt{\{michal.pilipczuk,siebertz,szymtor\}@mimuw.edu.pl}}
\and Roman Rabinovich?\thanks{Lehrstuhl f\"ur Logik und Semantik, Technische Universit\"at Berlin, \texttt{roman.rabinovich@tu-berlin.de}}
\and Sebastian Siebertz$^\dagger$
\and Szymon Torunczyk$^\dagger$}

\begin{document}

\maketitle

\begin{abstract}
\noindent 
Based on work of Podewski and Zieger~\cite{podewski1978stable}, 
Adler and Adler~\cite{adler2014interpreting}
proved that every nowhere dense class $\CCC$ of graphs is stable, that is, 
every fixed first-order interpretation of a graph $G\in \CCC$ has 
ladder index bounded by a constant depending only on $\CCC$ and 
the interpretation. The proof of this fact is based on an infinite
Ramsey argument and explicit ladder indices were never given. 
In this work, we give a combinatorial proof of Adler and Adler's result in 
the finite and work out the explicit bounds for ladder indices arising
by first-order interpretations of nowhere dense classes of graphs. 


A class $\CCC$ of graphs if uniformly quasi-wide if there are functions 
$N:\N\times\N\rightarrow\N$ and $s:\N\rightarrow\N$ such 
that for all $G\in \CCC$ and all $r,m\in \N$, if $A\subseteq V(G)$
of size at least $N(r,m)$ then there is a set $S$ of size at most $s(r)$
such that there is a subset $B\subseteq A\setminus S$ which is
$r$-independent in $G-S$. We proved in~\cite{siebertz2016polynomial} 
that we can always choose $N(r,m)\leq m^{f(r)}$. The existence of the 
function $f(r)$ follows 
from the earlier non-constructive argument of Podewski and Ziegler. 
We give a simpler proof of our earlier result with much improved bounds 
on the function $f(r)$.

Finally, we observe that a construction of Bousquet and 
Thomasse\'e~\cite{BousquetT15} can be slightly modified to prove that 
the VC-dimension of the $r$-power graph $G^r$ of a graph $G$
with $K_t\not\minor_r G$ is bounded by $t-1$.
\end{abstract}

\section{Preliminaries}

All graphs in this paper are finite, undirected and simple, that is, 
they do not have loops or parallel edges. Our notation is standard,
we refer to~\cite{diestel2012graph} for more background on 
graph theory. 
We write $V(G)$ for the vertex set of a graph $G$ and
$E(G)$ for its edge set. 
The {\em{distance}} between vertices $u$ and $v$ in $G$, denoted $\dist_G(u,v)$, is the length of a shortest path between $u$ and $v$ in~$G$.

The \emph{$r$th power of a graph $G$} is the graph $G^r$
with vertex set $V(G)$, where there is an edge between two 
vertices $u$ and $v$ if and only if their distance in $G$ is at most $r$. 

A {\em{minor model}} of a graph $H$ in $G$ is a family $(I_u)_{u\in V(H)}$ of pairwise vertex-disjoint connected subgraphs of $G$
such that whenever $uv$ is an edge in~$H$, there are $u'\in I_u$ and $v'\in I_v$ for which $u'v'$ 
is an edge in $G$.
The graph $H$ is a {\em{depth-$r$ minor}} of $G$, denoted $H\minor_rG$, if there is a minor model
$(I_u)_{u\in V(H)}$ of~$H$ in $G$ such that each subgraph $I_u$ has radius at most $r$.

A graph $H$ is a \emph{topological minor} of a graph $G$ if there is a
function~$\delta$ mapping vertices $v\in V(H)$ to vertices of $V(G)$ and 
edges $e\in E(H)$ to directed paths in $G$ such that 
$\delta(v)\neq \delta(u)$ for all distinct $u,v\in V(H)$, and 
if $e=(u,v)\in E(H)$, then $\delta(e)$ is a path from 
$\delta(u)$ to $\delta(v)$ in~$G$ which is internally vertex disjoint from all 
$\delta(e')$ with $e'\in E(H)$, $e'\neq e$. 
For $r\geq 0$, $H$ is a \emph{topological depth-$r$ minor} of $G$, 
written $H\minor_r^tG$, if it is a topological minor and all paths $\delta(e)$
have length at most $2r$. 

A class $\CCC$ of graphs is \emph{nowhere dense} if there is a function 
$f:\N\rightarrow \N$ such that for all $r\in \N$ it holds that $K_{f(r)}\not\minor_r G$
for all $G\in \CCC$. Nowhere dense classes of graphs were introduced by
Ne\v{s}et\v{r}il and Ossona de Mendez in~\cite{nevsetvril2010first,nevsetvril2011nowhere}.

Nowhere dense classes of graphs admit many equivalent characterisations, 
one of them being uniform quasi-wideness, a notion studied in 
finite model theory~\cite{dawar2010homomorphism}.  
A set $B\subseteq V(G)$ is called {\em{$r$-independent}} in $G$ if for all
distinct $u,v\in B$ we have $\dist_G(u,v)>r$.
A class $\CCC$ of graphs is \emph{uniformly quasi-wide} if there are
functions $N\colon \N\times\N\rightarrow \N$ and $s:\N\rightarrow \N$ such
that for all $r,m\in \N$ and all subsets $A\subseteq V(G)$ for
$G\in \CCC$ of size $\abs{A}\geq N(r,m)$ there is a set
$S\subseteq V(G)$ of size $\abs{S}\leq s(r)$ and a set
$B\subseteq A\setminus S$ of size $\abs{B}\geq m$ which is $r$-independent in
$G-S$. 

\begin{theorem}[Ne\v{s}et\v{r}il and Ossona de Mendez~\cite{nevsetvril2010first}]
A class $\CCC$ of graphs is nowhere dense if and only if it
is uniformly quasi-wide. 
\end{theorem}

The proof of Ne\v{s}et\v{r}il and Ossona de Mendez goes back to a construction
of Kreidler and Seese~\cite{kreidler1998monadic} (see also Atserias et al.~\cite{atserias2006preservation}), 
and uses iterated
Ramsey arguments. Hence the bounds on the function $N$ are huge. Recently, 
we proved that we may always choose $N$ to be a polynomial 
function~\cite{siebertz2016polynomial}. 

\begin{theorem}[Kreutzer, Rabinovich, and Siebertz \cite{siebertz2016polynomial}]\label{thm:uqw}
  Let $\CCC$ be a nowhere dense class of graphs and let 
  $t\,\colon\,\N\rightarrow \N$ be the function such that
  $K_{t(r)}\not\minor_r G$ for all $r\in \N$ and all $G\in \CCC$.  
  For every $r\in \N$
  there exist constants~$p(r)$ and $s(r)\leq t(r)$ such that
  for all $m\in \N$, all $G\in\CCC$ and all sets $A\subseteq V(G)$ of size at 
  least~$m^{p(r)}$, there is a set $S\subseteq V(G)$ of size at
  most $s(r)$ such that there is a set $B\subseteq A\setminus S$ of size at
  least~$m$ which is $r$-independent in $G-S$.
  
  Furthermore, there is an algorithm, that given an $n$-vertex graph
  $G\in \CCC$, $\epsilon>0$, $r\in \N$ and $A\subseteq V(G)$ of size at least
  $m^{p(r)}$, computes sets $S$ and $B\subseteq A$ as described above
  in
  time $\Oof(r\cdot t\cdot |A|^{t+1}\cdot n^{1+\epsilon})$.
\end{theorem}

We remark that the running time of the algorithm of~\Cref{thm:uqw}
is stated in the SODA version~\cite{siebertz2016polynomial} only as 
$\Oof(r\cdot t\cdot n^{t+6})$. A finer analysis with the running
times as stated above can be found in the arXiv version of that paper.

In order to prove the above bounds, we used methods from model
theory, more precisely, from stability theory. Stability is a strong tameness
property of first-order formulas, on which Shelah built his famous 
classification theory~\cite{shelah1990classification}.


\paragraph{First-order logic and stability.}
For extensive background on first-order logic, we refer the reader
to~\cite{hodges1993model}. For our purpose, it suffices to define
first-order logic over the vocabulary of graphs (with constant symbols
from a given parameter set).
 
Let $A$ be a set. We call $L(A)\coloneqq\{E\hspace{0.3mm}\}\cup A$ the \emph{vocabulary}
of graphs with parameters from $A$. \emph{First-order formulas} over $L(A)$ are
formed from atomic formulas~$x=y$ and $E(x,y)$, where $x,y$ are variables (we
assume that we have an infinite supply of variables) or elements of $A$ treated
as constant symbols, by the usual Boolean
connectives~$\neg$~(negation),~$\wedge$ (conjunction), and~$\vee$ (disjunction)
and existential and universal quantification~$\exists x,\forall x$,
respectively.  The free variables of a formula are those not in the scope of a
quantifier, and we write~$\phi(x_1,\ldots,x_k)$ to indicate that the free
variables of the formula~$\phi$ are among $x_1,\ldots,x_k$.

To define the semantics, we inductively define a satisfaction
relation~$\models$. Let $G$ be a graph and $A\subseteq V(G)$. For an
$L(A)$-formula~$\phi(x_1,\ldots,x_k)$, and
$v_1,\ldots,v_k\in V(G)$, $G\models\phi(v_1,\ldots,v_k)$
means that~$G$ satisfies~$\phi$ if the free variables~$x_1,\ldots,x_k$
are interpreted by~$v_1,\ldots,v_k$ and the parameters $a\in A$
(formally treated as constant symbols) used in the formula are
interpreted by the corresponding element of $A$ in $G$, respectively. If
$\phi(x_1,x_2)=E(x_1,x_2)$ is atomic, then $G\models\phi(v_1,v_2)$
if~$(v_1,v_2)\in E(G)$. The meaning of the equality symbol, the
Boolean connectives, and the quantifiers is as expected. For a
formula $\phi(x_1,\ldots, x_k, y_1,\ldots, y_\ell)$ and
$v_1,\ldots, v_\ell\in V(G)$ (treated as a sequence of parameters), we
write $\phi(x_1,\ldots, x_k, v_1,\ldots, v_\ell)$ for the formula with
free variables $x_1,\ldots, x_k$ where each occurrence of the variable
$y_i$ in $\phi$ is replaced by the constant symbol $v_i$.

We usually write $\tup{x}$ for a tuple $(x_1,\ldots, x_k)$ of variables. 
Usually, the length of the tuple is understood from the context. 

A first-order formula $\psi(\tup{x},\tup{y})$ has the \emph{$k$-order property}
over a graph $G$ if there are tuples $\tup{a}_1,\ldots, \tup{a}_k, \tup{b}_1,\ldots, \tup{b}_k$
of elements of $G$ such that \[\psi(\tup{a}_i,\tup{b}_j)\Longleftrightarrow i\leq j.\]
A class of graph $\CCC$ is \emph{stable} if for every formula $\psi(\tup{x},\tup{y})$ there
is a number $k$ such that for every graph $G\in \CCC$, $\psi$ does not have the $k$-order 
property.

Adler and Adler~\cite{adler2014interpreting} observed that nowhere density 
is essentially the stability theoretic notion of super flatness, introduced by
Podewski and Ziegler~\cite{podewski1978stable}. Based on the construction of 
Podewski and Ziegler they proved that every nowhere dense class of graphs is
stable. 

\begin{theorem}[Adler and Adler~\cite{adler2014interpreting}]\label{thm:adleradler}
If $\CCC$ is nowhere dense, then $\CCC$ is stable. 
\end{theorem}

Let $G$ be a graph and let $\Delta$ be a set of formulas. A sequence
$(v_1,\ldots, v_\ell)$ of vertices of~$G$ is
\emph{$\Delta$-indiscernible} if for every formula
$\phi(x_1,\ldots, x_k)\in \Delta$ with $k$ free variables and any two
increasing sequences
$1\leq i_1<\ldots <i_k\leq \ell, 1\leq j_1< \ldots< j_k\leq \ell$ of
integers, it holds that
\[G\models\phi(v_{i_1},\ldots, v_{i_k})\Leftrightarrow G\models\phi(v_{j_1},
\ldots, v_{j_k}).\]

The next theorem shows that we can find long indiscernible
sequences in stable classes of graphs. 

\begin{theorem}[Malliaris and Shelah~\cite{malliaris2014regularity}, Theorem 3.5, Item (2)]\label{thm:malshelah}
  Let $\CCC$ be a stable class of graphs and let $\Delta$ be a finite
  set of first-order formulas.  There is a polynomial $p(x)$ such that
  for all $G\in \CCC$, every positive integer $m$ and every sequence
  $(v_1,\ldots, v_\ell)$ of vertices of $G$ of length $\ell=p(m)$, there
  exists a sub-sequence $(v_{i_1},\ldots, v_{i_m})$ of
  $(v_1,\ldots, v_\ell)$ of length $m$ which is
  $\Delta$-indiscernible, $1\leq i_1<\ldots <i_m\leq \ell$.
\end{theorem}


\begin{theorem}\label{thm:extract_indiscernibles}
  Let $\CCC$ be a stable class of graphs and let $\Delta$ be a finite
  set of first-order formulas.  There is a polynomial $p(x)$ such that
  for all $G\in \CCC$, every positive integer $m$ and every sequence
  $(v_1,\ldots, v_\ell)$ of vertices of $G$ of length $\ell=p(m)$, there
  exists a sub-sequence $(v_{i_1},\ldots, v_{i_m})$ of
  $(v_1,\ldots, v_\ell)$ of length $m$ which is
  $\Delta$-indiscernible, $1\leq i_1<\ldots <i_m\leq \ell$.
\end{theorem}

As shown in~\cite{siebertz2016polynomial}, we can also compute
such sequences in polynomial time. More precisely, there is an algorithm running in time
  $\Oof(\abs{\Delta}\cdot k \cdot \ell^{k+1}
    \cdot n^{q})$, where $k$ is the maximal number of 
    free variables and $q$ is 
    the maximal quantifier-rank of a formula of $\Delta$, 
    that given an $n$-vertex graph $G\in \CCC$ and a sequence
  $(v_1,\ldots, v_\ell)\subseteq V(G)$, computes a
  $\Delta$-indiscernible sub-sequence of $(v_1,\ldots, v_\ell)$ 
  of length at least $m$.

We used the algorithmic version of~\cref{thm:extract_indiscernibles} 
to prove~\cref{thm:uqw}.

\paragraph{VC-dimension.}

Let $\FFF\subseteq 2^A$ be a family of
subsets of a set $A$. For a set $X\subseteq A$, we denote $X\cap \FFF=\{X\cap F : F\in \FFF\}$.
The set $X$ is \emph{shattered by $\FFF$} if $X\cap \FFF=2^X$.
The \emph{Vapnik-Chervonenkis dimension}~\cite{chervonenkis1971theory}, 
short \emph{VC-dimension},
of $\FFF$ is the maximum size of a set $X$ that is shattered by
$\FFF$. Note that if $X$ is shattered by $\FFF$, then also every
subset of $X$ is shattered by~$\FFF$.

For a graph $G$, the VC-dimension of $G$ is defined as the VC-dimension
of the set family $\{N[v]\colon v\in V(G)\}$ over the set $V(G)$.

Adler and Adler's result implies that any class of structures
obtained from $\CCC$ by means of a first-order interpretation has VC-dimension
bounded by a constant depending only on $\CCC$ and the interpretation.
In particular, the following is an immediate corollary of the results of Adler and Adler.

\begin{theorem}[\cite{adler2014interpreting}]\label{thm:adler}
  Let $\CCC$ be a nowhere dense class of graphs and let $\phi(x,y)$ be
  a first-order formula over the signature of graphs,
  such that for all $G \in \CCC$ and $u,v\in
  V(G)$ it holds that $G\models\phi(u,v)$ if and only if $G\models\phi(v,u)$. 
  For $G\in \CCC$, let $G_\phi$ be the graph with
  vertex set $V(G_\phi)=V(G)$ and edge set $E(G_\phi)=\{uv \colon
  G\models\phi(u,v)\}$. Then there is an integer $c$ depending only on
  $\CCC$ and $\phi$ such that $G_\phi$ has VC-dimension at most $c$.
\end{theorem}

By applying \Cref{thm:adler} to the first-order formula 
expressing that $\dist(u,v)\leq r$,
we immediately obtain the following.

\begin{corollary}\label[corollary]{crl:Gr}
  Let $\CCC$ be a nowhere dense class of graphs and let $r\in\N$. 
  Then there is an integer $c(r)$ such that $G^r$ 
  has VC-dimension at most~$c(r)$ for every $G\in \CCC$.
\end{corollary}

\paragraph{Our contributions.}

We give a new proof of~\cref{thm:uqw} which is simpler than 
the original proof and gives explicit bounds on the function $N$. 
Furthermore, the new proof leads to a much simpler algorithm 
with drastically improved running times.

\begin{theorem}\label{thm:new-uqw}
  Let $\CCC$ be a nowhere dense class of graphs and let 
  $t\,\colon\,\N\rightarrow \N$ be the function such that
  $K_{t(r)}\not\minor_r G$ for all $r\in \N$ and all $G\in \CCC$.  
  For every $r\in \N$
  there exist constants~$p(r)\leq t(r)^{t(r)}$ and $s(r)\leq t(r)$ such that
  for all $m\in \N$, all $G\in\CCC$ and all sets $A\subseteq V(G)$ of size at 
  least~$m^{p(r)}$, there is a set $S\subseteq V(G)$ of size at
  most $s(r)$ such that there is a set $B\subseteq A\setminus S$ of size at
  least~$m$ which is $r$-independent in $G-S$.
  
  Furthermore, there is an algorithm, that given an $n$-vertex graph
  $G\in \CCC$, $\epsilon>0$, $r\in \N$ and $A\subseteq V(G)$ of size at least
  $m^{p(r)}$, computes sets $S$ and $B\subseteq A$ as described above
  in
  time $\Oof(r\cdot t\cdot |A|^2\cdot n^{1+\epsilon})$.
\end{theorem}

We give an explicit proof of~\cref{thm:adler} in the finite, 
which in combination with~\cref{thm:new-uqw} for the first time
gives explicit bounds for the $k$-order property of a formula
on a nowhere dense class of graphs. 

\begin{theorem}\label{thm:new-stable}
If $\CCC$ is nowhere dense then $\CCC$ is stable. Furthermore, 
if $K_t\not\minor_{r(\phi)} G$, then $\phi$ does not have
the $k(\phi)$-order property.
\end{theorem}

Finally, we give an explicit proof of~\cref{crl:Gr} by slightly
changing a construction of Bousquet and 
Thomasse\'e~\cite{BousquetT15}. 

\begin{theorem}\label{thm:new-vc}
Let $G$ be a graph with $K_t\not\minor_r G$. Then 
the VC-dimension of $G^r$ is
bounded by $t-1$. 
\end{theorem}

\paragraph{Organisation.}
We prove~\cref{thm:new-uqw} in \cref{sec:uqw}, 
\cref{thm:new-stable} in \cref{sec:stable} and
\cref{thm:new-vc} in \cref{sec:vc}. 

\section{Uniform quasi-wideness}\label{sec:uqw}

The $k$-order property of a formula is strongly related
to its branching index. The largest $k$ such that 
$\psi$ has the $k$-order property over $G$ is
also called the \emph{ladder-index} of $\psi$ over $G$. 

If $\tau$ is a word over an alphabet $\Sigma$ and
$a\in \Sigma$, then $\tau\cdot a$ denotes the concatenation of~$\tau$
and $a$.  The \emph{branching index} of a formula $\psi(\tup{x},\tup{y}$
over a graph $G$ is the largest number
$\ell$ such that there are tuples of elements
$\tup{u}_{\sigma_1},\ldots, \tup{u}_{\sigma_{2^\ell}}\in V(G)$, indexed by the
words over the alphabet $\{0,1\}$ of length exactly $\ell$, and
tuples of elements $\tup{v}_{\tau_1},\ldots, \tup{v}_{\tau_{2^\ell-1}}$, indexed by the
words over $\{0,1\}$ of length strictly smaller than $\ell$, such that
if $\tau_j\cdot a$ is a (not necessarily proper) prefix of~$\sigma_i$, then
$G\models \psi(\tup{u}_{\sigma_i},\tup{v}_{\tau_j})$ if, and only if, $a=1$. The tuples
$\tup{u}_{\sigma_1},\ldots, \tup{u}_{\sigma_{2^\ell}}\in V(G)$ are called the 
\emph{leaves} of the tree, the tuples $\tup{v}_{\tau_1},\ldots, \tup{v}_{\tau_{2^\ell-1}}$
are its \emph{inner nodes}. Intuitively, a leaf $\tup{u}$ is connected to its 
predecessors~$\tup{v}$ such
that $\tup{u}$ is a \emph{right successor} of $\tup{v}$ and not to its predecessors such that 
it is a \emph{left
successor}. 
     

\begin{lemma}[\cite{hodges1993model}, Lemma 6.7.9, p.\
  313]\label{lem:branching}
  Let $\psi(\tup{x},\tup{y})$ be a formula and let $G$ be a graph. 
  If $\psi$ has branching index~$k$ over $G$, 
  then~$\psi$ has ladder index smaller than $2^{k+1}$ over $G$. 
  If $\psi$ has  has
  ladder index $k$ over $G$, then $\psi$ has branching index smaller than
  $2^{k+2}-2$ over $G$.
 \end{lemma}

In the proof of~\cref{thm:malshelah} we construct a type tree, in which
elements are iteratively classified according to their types. The depth of 
the type tree is directly related to the branching index of the formula. 
Our first theorem shows that we can avoid the exponential dependency 
between branching index and ladder index if we consider formulas $\psi(x,y)$
with exactly two free variables. 


Let $\psi(x,y)$ be a formula with $2$ free variables and let $(v_1,\ldots, v_n)$
be a sequence of vertices of $G$. The \emph{type tree}
of $\psi$ over $(v_1,\ldots,v_n)$ is constructed as 
follows. We make $v_1$ the root of the tree. Assume that $v_1,\ldots, v_i$
have been inserted to the tree. We follow a root-leaf path to find the
position for the next vertex $v_{i+1}$. If $G\models\psi(v_j,v_{i+1})$, we
go the right branch of the tree, otherwise we go to the left branch of 
the tree. 


\begin{theorem}
Let $\phi(x,y)$ be a formula with $2$ free variables and let
$(v_1,\ldots, v_n)$ be a sequence of elements of $G$. Then the 
largest complete binary subtree that is found as a topological minor 
of the type tree of $\psi$ over
$(v_1,\ldots, v_n)$ has depth at most twice the ladder
index of $\psi$. 
\end{theorem}
\begin{proof}
Consider the alternating path in the tree. 
\end{proof}

The following is implicit in~\cite{malliaris2014regularity}. 

\begin{lemma}[reference?]\label{lem:depth}
If a tree with $n$ vertices does not contain a complete binary 
tree of depth $k$ as a topological minor, then it has depth at most 
$x$. 
\end{lemma}

\begin{lemma}\label{lem:minor-to-tree}
Let $\phi(x,y)$ be a formula with $2$ free variables and let
$(v_1,\ldots, v_n)$ be a sequence of elements of $G$. Let $H$
be the topological minor model of a complete binary subtree in the
type tree of $\psi$ over $(v_1,\ldots, v_n)$. Denote the 
principal vertices of $H$ by $(w_1,\ldots, w_m)$. Then the type
tree of $\psi$ over $(w_1,\ldots, w_m)$ is a complete binary
tree. 
\end{lemma}

For the proof of our theorem we will use the formula 
$\dist_G(x,y)\leq 2$, for which we can give an explicit 
bound on the depth of the largest binary subtree.

\begin{theorem}
Let $\CCC$ be a nowhere dense class of graphs. Let 
$(v_1,\ldots, v_n)$ be the enumeration of an independent set 
in $G$. Assume that $K_t\not\minor_2G$. 
Then the largest complete binary subtree that is found as a topological minor 
of the type tree of $\psi$ over
$(v_1,\ldots, v_n)$ has depth at most $2(t-1)$. 
\end{theorem}
\begin{proof}
According to \cref{lem:minor-to-tree} we may assume that we find
the largest complete binary tree as a subgraph. We consider the
vertices $a_1,b_1,\ldots, a_k,b_k$ of the alternating path in the
type tree. Because $(v_1,\ldots, v_n)$ is independent in $G$, 
none of these vertices is adjacent and all of the vertices on the
paths of length $2$ which cause the creation of edges in the type
tree are distinct from $a_1,\ldots, b_k$. 

\begin{claim}
Every vertex $a_i$ is connected to every $b_j$, $j\geq i$,
via a vertex $z_{ij}$ which is not connected to any~$b_\ell$, $\ell\neq j$. 
\end{claim}

\noindent\textit{Proof.} Because $b_j$, $j\geq i$, is right of $a_i$, there is 
an element $z_{ij}$ connected to $a_i$ and $b_j$. If $z_{ij}$ was 
connected to $b_\ell$, $\ell\neq j$, then $b_\ell$ would be adjacent 
to $b_j$ in the type tree, which it is not. \hfill$\lrcorner$

\bigskip
We now contract $b_j$ and all $z_{ij}$ as well as $a_j$ to a single 
vertex for all $1\leq j\leq k$, $1\leq i\leq j$. We obtain a complete
graph $K_k$ as a depth-$2$ minor. 
\end{proof}

Note that the above proof does not give a proof that the ladder
index of the distance-$2$ formula is at most $2(t-1)$. In the type
tree we have the stronger statement that the vertices $b_i$
are not connected by a path of length $2$. 
We make no statement about the connections
of these elements in the ladder. 

\begin{theorem}
The function $N$ in the definition of uniform quasi-wideness
is small.
\end{theorem}
\begin{proof}
Assume $K_t\not\minor_2G$, in particular, $G$ excludes $K_t$
as a subgraph. We take a large independent subset $A'$ of $A$ and
enumerate it as $(a_1,\ldots, a_m)$. We build the type
tree, which has depth at least $x$ according to \cref{lem:depth}. 
We consider the longest branch of the type tree. On this branch, 
we take a set $X$ of maximum length such that every vertex has 
all its successors on the same side. This set has size at least
half the length of the branch. 

Either, $X$ is a set with all its successors on the left, then $X$ 
is a $2$-independent set and we are done with this step and
continue with $X$. Otherwise, all vertices of $X$ are at distance
$2$. Let $m=X$ and assume that $m$ is larger than $n_0$ to
be defined for neighbourhood complexity.
We claim that we find an element which is connected to at least $m^{1/3}$
of the vertices of $X$. Otherwise, every vertex
can create only $m^{2/3}$ connections, however, we need
to create $m^2$ connections. Hence, we need $m^{4/3}$ vertices
to create all connections. We have neighbourhood complexity
$m^{1+\epsilon}$ though, a contradiction. 

We delete the element of degree $m^{1/3}$ 
and continue with the subsequence
induced by its neighbours. This can happen at most 
$t(2)$ times, as we are constructing a complete minor at 
depth $2$ here. 
\end{proof}



\section{Bounds on the $k$-order property}\label{sec:stable}

\begin{theorem}
Let $\psi(\tup{x})$ be a formula and let $v_1,\ldots, v_m\in V(G)$. 
There is a formula $\vartheta(\tup{x})$ 
of the same quantifier rank as $\psi$ over a signature extended 
with $m$ colors such that for all $\tup{a}$ which do not
contain elements from $v_1,\ldots, v_m$ it holds that
$G\models\psi(\tup{a})\Leftrightarrow G\models\vartheta(\tup{a})$. 
\end{theorem}
\begin{proof}
As a preliminary step, we introduce $m$ constant symbols 
$c_1,\ldots, c_m$. We replace in $\psi$ all quantifiers 
$\exists x\xi(x,\tup{y})$ by $\exists x((x=c \wedge \xi^*(x,\tup{y})
\vee (x\neq c\wedge \xi^*(x,\tup{y}))$, 
where $\xi^*$ is the formula obtained by inductively continuing the
construction and similarly for the universal quantifier. Furthermore, 
we replace every atomic formula $E(c_i,c_j)$ by the truth value of
$E(v_i,v_j)$. Then we have
\[(G,v_1,\ldots, v_m)\models\psi^*(\tup{a})\Leftrightarrow G\models\psi(\tup{a}).\]
Obviously, the quantifier rank of $\psi^*$ is the same as that of $\psi$. 

We now build the formula $\psi^{**}$ as follows. We replace
every subformula $\exists x((x=c_i \wedge \xi^*(x,\tup{y}) \vee 
(x\neq c_i \wedge \xi^*(x,\tup{y}))$ by the formula
$\exists x(\xi^{***}(\tup{y}) \vee \xi^{**}(x,\tup{y}))$, 
where $\xi^{***}(\tup{y})$ is obtained from $\xi^{*}$ by
replacing every atom $E(c_i,y)$ by the atom $R_i(y)$. Here, 
$R_i$ is a new unary predicate which holds true exactly for 
the neighbors of $c_i$ in $G$. Note that $c_i$ occurs only together
with variables in atoms because we eliminated all other occurrences
before. Then we have 
\[G-\{v_1,\ldots, v_m\}\models \psi^{**}(\tup{a})
\Leftrightarrow (G,v_1,\ldots, v_m)\models\psi^*(\tup{a}).\]
We let $\vartheta(\tup{x})\coloneqq \psi^{**}(\tup{x})$ and conclude. 
\end{proof}

We can now translate $\vartheta(\tup{a})$ to Gaifman normal form. 
Note that the locality radius of the resulting formula does not depend
on the number $m$ of elements we delete. Only the number of local types
depends on this number, as the signature of the new formula is changed
depending on $m$. 

\begin{theorem}
Let $\psi(\tup{x})$ be a formula and let $v_1,\ldots, v_m\in V(G)$. 
There is an $r$-local formula $\vartheta(\tup{x})$ 
over a signature extended 
with $m$ colors such that for all $\tup{a}$ which do not
contain elements from $v_1,\ldots, v_m$ it holds that
$G\models\psi(\tup{a})\Leftrightarrow G\models\vartheta(\tup{a})$. 
Here, $r=2^{q+|\tup{x}|}$ by Gaifman's Theorem.
\end{theorem}

We now prove a tuple-wise uniform quasi-wideness, as Podewski and Ziegler 
prove in the infinite. For a set $S$ and two tuples $\tup{a},\tup{b}\in V(G)^k$
we write $\dist_{G-S}(\tup{a},\tup{b})>r$ if $\dist_{G-S}(x,y)>r$ for all $x,y\in 
(\tup{a}\cup\tup{b})\setminus S$. 

\begin{lemma}
For all $k,m,r\in \N$, there exist $M(k,m,r)\in \N$ and 
$s(k,m,r)\in \N$ such that if $A$ is a set of $k$-tuples
with $|A|>M$, then there exists a set $S\subseteq V(G)$
with $|S|\leq s$ such that $\dist_{G-S}(\tup{a},\tup{b})>r$
for all $\tup{a},\tup{b}\in A$. 
\end{lemma}
\begin{proof}

\end{proof}

 

\section{VC-dimension of power graphs}\label{sec:vc}

We now prove \cref{thm:new-vc}. 
In fact, we prove a slightly stronger result. The $2$VC-dimension
of a graph is the largest set which has a neighbour for each 
subset of size $2$. Obviously, the $2$VC-dimension of $G$
bounds its VC-dimension. 

\begin{theorem}
If $K_t\not\minor_rG$, then 
$G^r$ has $2$VC-dimension at most $t-1$. 
\end{theorem}
\begin{proof}
Assume there is a set $A=\{a_1,\ldots, a_t\}$ of size $t$ such that
for all subsets $\{a_i,a_j\}\subseteq A$ of size $2$ 
there is an element $v_{ij}\in V(G)\setminus A$ with 
$N_r[v_{ij}]\cap A=\{a_i,a_j\}$. Fix $v_{ij}$ with the property
that $\max\{\dist_G(v_{ij},a_i), \dist_G(v_{ij},a_j)\}$ is 
minimised. 

A \emph{central walk} $W_{ij}$ is the concatenation of a minimum length
path $P_{ij}^i$ from $a_i$ to $v_{ij}$ and a minimum length path $P_{ij}^j$ from $v_{ij}$ to $a_j$. 
Note that a central walk is possibly not a path. For each pair $a_i,a_j$ fix
a central walk $W_{ij}$ and the corresponding paths $P_{ij}^i$ and $P_{ij}^j$. 

Now assume that a vertex $x$ belongs to two distinct central 
walks $W_{ij}$, $W_{i'j'}$. Assume that $x$ lies on $P_{ij}^i$ and $P_{ij}^{i'}$,
otherwise, rename the elements. First, observe that if $\dist(x,a_i)=\dist(x,a_{i'})$, 
then $a_i=a_{i'}$. Otherwise, $\dist(v_{ij},a_{i})=\dist(v_{ij},a_{i'})$ and hence 
$a_j=a_i$, and analogously, $a_{j'}=a_j$, contradicting the assumption 
that $W_{ij}$ and $W_{i'j'}$ are distinct. By the same argument we have 
$\dist(x,a_i)<\dist(x,a_j)$ and $\dist(x,a_{i'})<\dist(x,a_{j'})$. 
Now assume that $\dist(x,a_i)<\dist(x,a_{i'})$. By the same argument as 
above we have $a_{j'}=a_i$, hence $W_{i'j'}=W_{ij'}$. Here, we have
$\dist(x,a_i)<\dist(x,a_j)$ and $\dist(x,a_{i})<\dist(x,a_{i'})$, 
otherwise the walks are not distinct. 

Let us now construct connected subsets $X_i$ for all $1\leq i\leq t$. 
For every walk $W_{ij}$ the vertices of $W_{ij}$ closer to $a_i$ than to $a_j$ 
are added to $X_i$, the vertices of $W_{ij}$ closer to $a_j$ than to $a_i$ 
are added to $X_j$, ties are broken arbitrary.
Then the sets $X_i$ are pairwise disjoint by what we proved above. If a vertex $x$
appears in two distinct central walks, these are $W_{ij}$ and $W_{i\ell}$ for some
$i,j,\ell$ with $\dist(x,a_i)<\dist(x,a_j)$ and $\dist(x,a_i)<\dist(x,a_\ell)$. 
In both cases $x$ belongs to $X_i$. By construction, the sets $X_i$ are connected, 
have radius at most~$r$, and 
there is always an edge between a vertex of $X_i$ and a vertex of $X_j$ since $X_i\cup X_j$ 
contains the walk $W_{ij}$. Therefore, if the $2$VC-dimension is at least $t$, the 
graph contains $K_t$ as a depth-$r$ minor. 
\end{proof}

\bibliographystyle{abbrv}
\bibliography{ref} 


\end{document}
%%% Local Variables:
%%% mode: latex
%%% TeX-master: t
%%% End:
