\section{Introduction}

Nowhere dense classes of graphs were introduced 
by Ne\v set\v ril and Ossona de 
Mendez~\cite{nevsetvril2010first,nevsetvril2011nowhere} as a very 
general model
for uniform sparseness of graphs. These classes generalize many 
familiar classes of sparse graphs, such as planar graphs, graphs 
of bounded treewidth,  graphs of bounded degree, and, in fact, 
all classes that exclude a fixed 
topological minor.
Formally, a class $\CCC$ of graphs is {\em{nowhere dense}} if there is a function $t\colon \N\to \N$ such that for every $r\in \N$, every graph $G$ from $\CCC$ does not admit $K_{t(r)}$ as an {\em{$r$-shallow minor}}
(denoted $K_{t(r)}\not\minor_r G$),
that is, there is no minor model of $K_{t(r)}$ in $G$ where every branch set has radius at most $r$.

The concept of nowhere denseness
turns out to be very robust, as witnessed by the fact that it is equivalent 
to multiple other concepts studied in different areas of mathematics. 
One can equivalently characterize nowhere dense graph classes 
by bounds on the density of shallow (topological)
minors~\cite{nevsetvril2010first,nevsetvril2011nowhere},
quasi-wideness~\cite{nevsetvril2011nowhere} (a notion introduced by
Dawar~\cite{dawar2010homomorphism} in his study of homomorphism
preservation properties), low tree-depth
colorings~\cite{nevsetvril2008grad}, generalized coloring
numbers~\cite{zhu2009coloring}, sparse neighborhood
covers~\cite{GroheKRSS15,grohe2014deciding}, by a game called the
splitter game~\cite{grohe2014deciding}, and by the model-theoretic
concepts of stability and independence~\cite{adler2014interpreting}.
For a broader discussion on the graph theoretic sparsity we refer to the book
of Ne\v{s}et\v{r}il and Ossona de Mendez~\cite{sparsity}.

\begin{comment}
These alternative characterizations have been very useful in 
the design of efficient algorithms. For instance, 
the {\sc{Subgraph Isomorphism}} and {\sc{Homomorphism}} problems 
are fixed-parameter tractable on any nowhere dense
class, parameterized by the size of the pattern graph~\cite{nevsetvril2010first}
and so is the {\sc Distance-$r$ Dominating Set} problem, parameterized
by the size of the solution~\cite{DawarK09}. In fact, 
the {\sc Distance-$r$ Dominating Set} problem admits
polynomial kernels~\cite{siebertz2016polynomial} and even 
almost linear kernels on nowhere dense classes of 
graphs~\cite{eickmeyer2016neighborhood}
(see also~\cite{drange2016kernelization} for the case $r=1$). 
It was shown in~\cite{grohe2014deciding}
that every first-order definable problem can be decided in
almost linear time on any nowhere dense graph class.

It is a natural question to ask for the most general classes of graphs
which admit efficient solutions for certain problems, or to 
classify them into tractable and intractable classes. It was shown 
that for the first-order model-checking problem~\cite{dvovrak2013testing} and for
the {\sc Distance-$r$ Dominating Set} problem~\cite{drange2016kernelization} 
the dividing line for algorithmic tractability 
on subgraph closed classes of graphs is exactly between the
nowhere dense and somewhere dense graph classes. 
\end{comment}

%\bigskip

In this work we revisit the connections between the notion of nowhere 
denseness and notions from (finite) model theory.
We first consider uniform quasi-wideness, a
notion was introduced by Dawar~\cite{dawar2010homomorphism}, who 
proved that every quasi-wide class that is closed under taking substructures
and disjoint unions has the homomorphism preservation property. It was 
later proved by Ne\v{s}et\v{r}il and Ossona de Mendez that 
the notions of uniform quasi-wideness and nowhere denseness coincide for 
graphs~\cite{nevsetvril2011nowhere}. 
Formally, a class $\CCC$ of graphs is \emph{uniformly quasi-wide} if there are
functions $N\colon \N\times\N\rightarrow \N$ and $s\colon \N\rightarrow \N$ such
that for all $r,m\in \N$ and all subsets $A\subseteq V(G)$ for
$G\in \CCC$ of size $\abs{A}\geq N(r,m)$ there is a set
$S\subseteq V(G)$ of size $\abs{S}\leq s(r)$ and a set
$B\subseteq A\setminus S$ of size $\abs{B}\geq m$ which is $r$-independent in
$G-S$. Recall that a set $B\subseteq V(G)$ is called {\em{$r$-independent}} in $G$ if for all
distinct $u,v\in B$ we have $\dist_G(u,v)>r$.

The second topic of study is the
connection between nowhere denseness and stability theory. 
Let $\cal C$ be a class of structures over a relational vocabulary $\Sigma$. Let 
$\phi(\tup{x},\tup{y})$ be a $\Sigma$-formula with the free variables
divided into two groups $\tup{x}, \tup{y}$. A \emph{$\phi$-ladder}
of length $n$ is a sequence $(\tup{a}_1,\ldots, \tup{a}_{n},
\tup{b}_1,\ldots, \tup{b}_{n})$ of tuples in some model $\strA\in \cal C$, 
such that for all $1\leq i,j\le n$,
\[\strA\models\phi(\tup{a}_i,\tup{b}_j)\Longleftrightarrow i\leq j. \]
The least  $n$ for which 
there is no $\phi$-ladder of length $n$ is 
the \emph{ladder index} 
of $\phi(\tup{x},\tup{y})$ in $\cal C$ (which may depend on the way we split the
variables).
Based on work of Podewski and Ziegler~\cite{podewski1978stable}, 
Adler and Adler~\cite{adler2014interpreting}
proved that every nowhere dense class $\CCC$ of graphs is stable, that is, 
the ladder index of every first-order formula $\phi(\tup{x},\tup{y})$ over
graphs from $\CCC$ is bounded by a constant depending only on $\phi$ 
and~$\CCC$. In fact, for a subgraph-closed class $\CCC$, the notions of nowhere denseness and stability coincide.

We remark that the above connections between nowhere denseness and notions from model theory have recently found algorithmic applications.
Both uniform quasi-wideness and stability techniques are key tools used in the study of the complexity of the {\sc Distance-$r$ Dominating Set} problem on nowhere dense graph classes,
and in particular in the design of polynomial kernelization procedures for this problem~\cite{DawarK09,drange2016kernelization,eickmeyer2016neighborhood,siebertz2016polynomial}.

\paragraph{Our contribution.} 
Our first result is  a new proof of a result of
Ne\v{s}et\v{r}il and Ossona de Mendez~\cite{nevsetvril2010first},
which states that a class $\CCC$ of graphs is nowhere dense if and only if it
is uniformly quasi-wide. The proof of Ne\v{s}et\v{r}il 
and Ossona de Mendez goes back to a construction
of Kreidler and Seese~\cite{kreidler1998monadic} (see also Atserias et al.~\cite{atserias2006preservation}), 
and uses iterated Ramsey arguments. Hence the original bounds on 
the function $N$ are huge. Recently, 
it was proved that we may always choose~$N$ to be a polynomial 
function~\cite{siebertz2016polynomial}. However, the degree of the polynomial 
in~\cite{siebertz2016polynomial} was  not specified, as its existence 
depends on the non-constructive arguments used by Adler and Adler's in showing that nowhere dense classes of graphs
are stable~\cite{adler2014interpreting}. We give a new construction 
which is considerably simpler than that of~\cite{siebertz2016polynomial}
and which gives explicit bounds on the degree of the polynomial. 
Precisely, we prove the following theorem; here, the notation $\Omega_{r,t}(\cdot)$ hides factors depending on $r$ and $t$.

\begin{theorem}\label{thm:new-uqw}
For all $r,t\in \N$ there exists a function $N\colon \N\to \N$ with $N(m)\in \Omega_{r,t}(m^{(6t+3)^{r}})$ such that the following holds.
Let $G$ be a graph such that $K_t\not\minor_{\lfloor 3r/2\rfloor} G$, and
let $A\subseteq V(G)$ be a vertex subset of size at least $N(m)$, for a given $m$.
Then we can find a set $S\subseteq V(G)$ of size $|S|\leq t$ and a set $B\subseteq A\setminus S$ 
of size $|B|\geq m$ which is $r$-independent in $G-S$.
Moreover, given $G$ and $A$, such sets $S$ and $B$ can be computed in time $\Oof_{r,t}(|A|^c\cdot |E(G)|)$, where $c$ is a universal constant.
\end{theorem}

Let us remark
that even though the techniques employed to prove \cref{thm:new-uqw} are inspired by the methods from stability theory, 
we use only very simple graph theoretic notions. In particular, the
proof can easily be turned into an efficient algorithm which does not
call the model-checking algorithm as a subroutine. 

The proof of Podewski and Ziegler~\cite{podewski1978stable} that flat graphs are stable uses an 
infinite Ramsey argument and compactness, and the observation of Adler and Adler~\cite{adler2014interpreting} was that
this can be translated to classes $\CCC$ of finite graphs by considering an infinite graph that is the disjoint union of graphs from $\CCC$.
Based on the approach of Podewski and Ziegler~\cite{podewski1978stable}, we give a combinatorial 
proof that every first-order formula has finite ladder index on every
nowhere dense class of graphs, which does not involve infinite combinatorics and model theory.
In particular, instead of compactness we use Gaifman's Locality Theorem for
first-order logic~\cite{gaifman1982local}. Precisely, we prove the following theorem.

\begin{theorem}\label{thm:new-stable}
  There are computable functions $f\colon \N^3\to\N$ and $g\colon\N\to\N$ with the following property.
If $\phi(\bar x)$ is a formula of quantifier rank $q$ and with $d$ free variables,
and  $G$ is a graph such that $K_t\not\minor_{g(q)} G$, then the ladder index of $\phi$ on $G$ is at most $f(q,d,t)$. 
\end{theorem}

Finally, we observe that an argument of Bousquet and 
Thomasse\'e~\cite{BousquetT15} can be slightly modified to prove that 
the VC-dimension of the $r$th power $G^r$ of a graph $G$
with $K_t\not\minor_r G$ is bounded by $t-1$.
Boundedness of VC-dimension is a weaker condition than the stability of a class, however it is sufficient in many context.
Therefore, we consider investigating explicit bounds on the VC-dimension potentially useful for future research.

\begin{theorem}\label{thm:new-vc}
Let $G$ be a graph such that $K_t\not\minor_r G$. Then the
VC-dimension of $G^r$, the $r$th power of the graph~$G$, is bounded by $t-1$. 
\end{theorem}

\paragraph{Organization.} We give background from graph theory in \cref{sec:uqw}, where we also
prove \cref{thm:new-uqw}. The proof of \cref{thm:new-vc} is in \cref{sec:vc}. We provide background 
on logic and prove \cref{thm:new-stable} in \cref{sec:stable}. 



