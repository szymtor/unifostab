%\section{Types and locality}\label{sec:gaifman}
\section{Bounds on the number of types}\label{sec:types}

In this section we prove Theorem~\ref{thm:vc-density} and Theorem~\ref{thm:vc-density-lower-bound}. Let us quickly recall the required notions from
logic. 
%
%
%In this section, we develop auxiliary tools concerning first order logic on graphs, 
%in particular we develop a convenient abstraction for Gaifman's locality property that can be easily combined with the notion of $r$-separation.
%We begin by recalling some standard notions from logic.
%
\subsection{Logical notions}

\paragraph{Formulas.}
All formulas in this paper are first order formulas on graph,
i.e., they are built using variables (denoted $x,y,z$, etc.),
atomic predicates $x=y$ or $E(x,y)$,
where the latter denotes the existence of an edge between two nodes, quantifiers $\forall x,\exists x$, and boolean connectives $\lor,\land,\neg$. 
Let $\phi(\bar x)$ be a formula with free variables 
$\bar x$. (Formally, the free variables form a set.
To ease notation, we identify this set with a tuple by fixing any its enumeration.)
If $\bar w\in V^{|\bar x|}$ is a tuple of vertices of some graph $G=(V,E)$ (treated as a valuation of the free variables $\bar x$), then we write $G,\bar w\models \phi(\bar x)$
to denote that the valuation $\bar w$ satisfies the formula $\phi$ in the graph $G$.
%The following example should clarify our notation.
%
%\begin{example}\label{ex:dist-formula}
%The formula
%$$\phi(x,y)\equiv \exists z_1\, \exists z_2\, (E(x,z_1)\lor (x=z_1))\land (E(z_1,z_2)\lor (z_1=z_2))\land (E(z_2,y)\lor (z_2=y))$$
%with free variables $x,t$ expresses that $x$ and $y$ are at distance at most $3$.
%That is, for two vertices $u,v$ of a graph $G$,
%the relation $G,u,v\models \phi(x,y)$ holds 
%if and only if the distance between $u$ and $v$ is at most $3$ in $G$.
%\end{example}

We will consider also \emph{colored graphs},
where we have a fixed set of colors $\Lambda$ and every vertex is assigned a subset of colors from 
$\Lambda$. If $C\in \Lambda$ is a color then the atomic formula $C(x)$ holds in a vertex $x$ if and only if $x$ has color $C$.

Finally, we will consider \emph{formulas with parameters}
from a set $A$, which is a subset of vertices of some graph.
Formally, such formula with parameters is a pair consisting of a (standard) formula $\phi(\bar x,\bar y)$
with a partitioning of its free variables into $\bar x$ and $\bar y$,
and a valuation $\bar v\in A^{|\bar y|}$ of the free variables $\bar y$ in $A$.
We denote the resulting formula with parameters by $\phi(\bar x,\bar v)$, and say that its free variables 
are $\bar x$. For a valuation $\bar u\in A^{|\bar x|}$,
we write $G,\bar u\models \phi(\bar x,\bar v)$
iff $G,\bar u\bar v\models \phi(\bar x,\bar y)$. Here and later on, we write $\bar u\bar v$ for the concatenation of tuples $\bar u$ and $\bar v$.

\newcommand{\tp}{\mathrm{tp}}

\paragraph{Types.}
Fix a formula $\phi(\bar x,\bar y)$ together with a distinguished partitioning of its free variables into 
\emph{object variables} $\bar x$ and \emph{parameter variables} $\bar y$. 
Let $G=(V,E)$ be a graph, and let $A\subset V$.
If $\bar u\in V^{|\bar y|}$ is a tuple of 
nodes of length $|\bar y|$, then the 
\emph{$\phi$-type of $\bar u$ over $A$},
denoted $\tp^\phi_G(\bar u/A)$,
is the set of all
formulas $\phi(\bar x,\bar v)$,
with parameters $\bar v\in A^{|\bar y|}$
replacing the parameter variables $\bar z$,
such that $G,\bar u\models \phi(\bar x,\bar v)$.
%Note that since $\phi$ is fixed in this definition, formulas $\phi(\bar x,\bar v)$ belonging to the $\phi$-type of $\bar u$ are in one-to-one correspondence
%with tuples $\bar v\in A^{|\bar y|}$ satisfying $G,\bar u,\bar v\models \phi(\bar u,\bar v)$.
%Therefore, up to this bijection, we have the following identification:
%\begin{equation}\label{eq:bijection}
%\tp^\phi_G(\bar u/A)\quad\leftrightarrow\quad\setof{\bar v\in  A^{|\bar y|}}{G, \bar u\bar v\models \phi(\bar x,\bar y)}.
%\end{equation}

%If $q\in \N$ is a number and $\bar u\in  V^{d}$
%is a tuple of some length~$d$, then by $\tp^q_G(\bar u/A)$  we denote the set of all formulas $\phi(\bar x,\bar v)$
%of quantifier rank at most $q$, with parameters $\bar v$ from $A$, and with $|\bar x|=d$,
%such that $G,\bar u\models \phi(\bar y,\bar v)$.
%Therefore, up to the correspondence \eqref{eq:bijection}, we have the following identification:
%\begin{equation*}
%\tp^q_G(\bar u/A)\quad\leftrightarrow\quad\set{\tp^\phi(\bar u/A)}_{\phi(\bar x,\bar y)},
%\end{equation*}
%where $\phi(\bar x,\bar y)$ ranges over all formulas of quantifier rank $q$, and all partitions of its free variables into two sets $\bar x,\bar y$,
%where $|\bar x|=d$. 
%In particular, the set $\tp^q_G(\bar u/A)$ is infinite.
%It is not difficult to see, however, that in the case when $A$ is finite,
%the set $\tp^q_G(\bar u/A)$ is uniquely determined by its finite subset, since up to syntactic equivalence, 
%there are only finitely many formulas of quantifier rank $q$ with $|\bar u|$ free variables and parameters from $A$
%(we can assume that each such formula has $|A|+|\bar u|$ free variables).
%In particular, the set of all possible types 
%$\tp^q_G(\bar u/A)$ has cardinality upper bounded by some number 
% computable from $q,|\bar u|$ and $|A|$.

%When $\Delta$ is either a formula $\phi(\bar x,\bar y)$ with a distinguished partitioning of its free variables, or a number $q$,
%we simply write $\tp^\Delta(\bar u/A)$ if the graph $G$
%is clear from the context.
%In the case $A=\emptyset$, we omit it from the notation, 
%and simply write $\tp^\Delta(\bar u)$ or $\tp^\Delta_G(\bar u)$.
%Observe that in particular, if $\Delta=q$ and $A=\emptyset$, then $\tp^q_G(\bar u)$ consists of all first order formulas $\phi(\bar x)$ of quantifier rank at most~$q$ and with $|\bar x|=|\bar u|$
%such that $G,\bar u\models \phi(\bar x)$. This coincides with the standard notion of the first order type of quantifier rank $q$ of the tuple $\bar u$.
%
%\begin{example}
%Let $\phi(x,y)$	be the formula from~\autoref{ex:dist-formula}, denoting that the distance between~$x$ and $y$ is at most $3$.
%We  partition  the free variables of $\phi$
%into $x$ and $y$.
%Let $A$ be a subset of vertices of a graph $G=(V,E)$
%and $u\in V$ be a single vertex.
%The $\phi$-type of $u$ over $A$
%corresponds, via the said bijection, to the set of those vertices in $A$
%whose distance from $u$ is at most $3$ in $G$.
%\end{example}

For a fixed formula $\phi(\bar y,\bar z)$,  graph $G=(V,E)$ and sets $A,W\subset V$, define
 $S^\phi(W/A)$ as the set of all $\phi$-types of tuples from $W$ over $A$ in $G$; that is, 
\begin{equation*}
S^\phi(W/A)=\setof{\tp^\phi_G(\bar u/A)}{\bar u\in W^{|\bar y|}}.
\end{equation*}
Although not visible in the notation, the set $S^\phi(W/A)$ depends on the chosen partitioning $\bar x,\bar y$ of the free variables of~$\phi$.
In case $W=V(G)$ we write $S^{\phi}_d(G/A)$ instead of $S^{\phi}_d(W/A)$.
Note that this definitions differs syntactically from the one given in \autoref{sec:intro}, as here $S^{\phi}(G/A)$ consists of $\phi$-types, and not of subsets of tuples.
However, %as we argued, 
it is easy to see that there is a one-to-one correspondence between 
the two notions. %them, as expressed in~\eqref{eq:bijection}.
%
The following lemma is immediate.
\begin{lemma}\label{lem:types-over-B}
Let $G$ be a graph and let $A\subseteq B\subseteq V(G)$. Then for each formula $\phi(\bar x,\bar y)$, it holds that
$|S^\phi(G/A)|\leq |S^\phi(G/B)|$. 
\end{lemma}

\subsection{Locality}
We will use the following intuitive notion of functional determination.
Suppose $X,A,B$ are sets and we have two functions: $f\colon X\to A$ and $g\colon X\to B$.
We say that $f(x)$ {\em{determines}} $g(x)$ for $x\in X$ if for every pair of elements $x,x'\in X$ the following implication holds: $f(x)=f(x')$ implies $g(x)=g(x')$.
Equivalently, there is a function $h\colon A\to B$ such that $g=h\circ f$.
%Note that this relation is transitive: if $f(x)$ determines $g(x)$ and $g(x)$ determines $h(x)$, then $f(x)$ determines $h(x)$.

Recall that if $A,B,S$ are subsets of vertices of a graph $G$ and $r\in\N$,
then $A$ and $B$ are $r$-separated by $S$ in $G$
if every path from $A$ to $B$ of length at most $r$ contains a vertex from~$S$.

\medskip
The following lemma is the main result of this subsection. 

\begin{lemma}[$\star$]\label{lem:types}
For any given numbers $q$ and $d$
one can compute numbers $p$ and $r$ with the following properties.
Let $G=(V,E)$ be a fixed graph and let $A,B,S\subset V$ be fixed subsets of its vertices
such that $A$ and $B$ are $r$-separated by~$S$ in $G$.
Then, for tuples $\bar u\in A^{d}$, the type $\tp^q(\bar u/B)$ is determined by the type $\tp^{p}(\bar u/S)$.
\end{lemma}

%We will only use the following consequence of~Theorem~\ref{lem:types}.
%
\begin{corollary}[$\star$]\label{cor:bound}
For every formula $\phi(\bar x,\bar y)$ 
and number $s\in \N$
there exist numbers $T,r\in \N$,
where~$r$ is computable from $\phi$ and $T$ is computable from $\phi$ and $s$,
  such that the following holds. For every graph $G$ and vertex subsets $A,B,S\subset V(G)$ 
  where~$S$ has at most $s$ vertices and $r$-separates $A$ from $B$, we have $|S^\phi(A/B)|\le T$.
\end{corollary}
%\begin{proof}
%Apply~Theorem~\ref{lem:types} to $q$ being the quantifier rank of $\phi$ and $d=|\bar x|$, yielding numbers $p$ and $r$.
%By~Theorem~\ref{lem:types} we have $|S^\phi(A/B)|\leq |S^\phi(A/S)|$.
%However, $|S^\phi(A/S)|$ is the number of quantifier rank $p$ types of $d$-tuples of elements over a set of parameters of size $s$, and, as we argued, this number is bounded by a value computable from $p$, $d$, and $s$.
%\end{proof}
%
%The remainder of this section is devoted to the proof of~Theorem~\ref{lem:types}.
%This result is a consequence of two fundamental properties of first order logic:
%Gaifman's locality and Feferman-Vaught compositionality. We recall these results now.
%The following statement is an immediate corollary of the main result in a paper of Gaifman~\cite{gaifman1982local}.
%
%\begin{lemma}[Gaifman locality,~\cite{gaifman1982local}]\label{lem:gaif}
%  For all numbers $d,q\in \N$ there exists a number $t\in \N$, computable from $d$ and $q$, such that the following holds.
%  Let $G=(V,E)$ be a graph colored by a fixed set colors, and $A\subset V$ be a set of vertices of $G$.
%  Then, for tuples $\bar u\in V^d$, the type  $\tp^q(\bar u)$ is determined by the type $\tp^{t}(B^r(\bar u))$, where $r=7^q$.
%\end{lemma}
%
%The next result expresses compositionality of first order logic. Its proof is a standard application of Ehrenfeucht-Fra\"iss\'e games, so we only sketch it for completeness.
%
%\begin{lemma}[Feferman-Vaught]\label{lem:fv}
%  Let $G,H$ be two fixed vertex-disjoint graphs colored by a fixed set of colors $\Lambda$, and let 
%  $c,d\in\N$  be numbers.
%  Then, for valuations $\bar u\in V(G)^{c}$ and $\bar v\in V(H)^{d}$, 
% the type 
% $\tp^q_{G\cup H}(\bar u\bar v)$
% is determined by the pair of types $\tp^q_G(\bar u)$ and $\tp^q_H(\bar v)$.
%\end{lemma}
%\begin{proof}[sketch]The proof proceeds by applying the following, well-known characterization of $\tp^q_G(\bar w)$ in terms of Ehrenfeucht-Fra\"iss\'e games:
%$\tp^q_{G}(\bar w)=\tp^q_{G}(\bar w')$
%if and only if duplicator has a strategy to survive for $q$-rounds in a certain pebble game.
%To prove the lemma, we combine two strategies of duplicator: one on $G$ and one on $H$.
%\end{proof}
%
%Before proving~Theorem~\ref{lem:types}, we introduce the following notions.
%Fix a graph $G=(V,E)$.
%For a set of vertices $S\subset V$, define the color set $\Lambda_S=\{C_s\colon s\in S\}$, where we put one color $C_s$ for each vertex $s\in S$.
%Define a graph $G^S$ colored with $\Lambda_S$, which is 
%the subgraph of $G$ induced by $V-S$
%in which, additionally, for every vertex $s\in S$, every vertex $v\in V-S$ which is a neighbor of $s$ in $G$ is colored by color $C_s$. 
%In other words, every vertex $v$ of $G^S$ is colored with a subset of colors from $\Lambda_S$ corresponding to the neighborhood of $v$ in $S$.
%
%A sequence of elements of $S\cup\set\star$,
%where $\star$ is a fixed placeholder symbol,
%will be called an \emph{$S$-signature}.
%If $H$ is any (colored) graph with vertex set contained in $V-S$,
%and $\bar u\in V^d$ is a $d$-tuple of vertices,
%define the {$S$-signature} of $\bar u$
%as the tuple $\bar s\in (S\cup\set\star)^d$ obtained from $\bar u$ by replacing the vertices in $V-S$ by the symbol~$\star$.
%Define $\tp^q[H,\bar u]$ as the
%pair consisting of the following components:
%\begin{itemize}
%	\item the type $\tp^q_H(\bar v)$,
%	where $\bar v$ is the tuple obtained from $\bar u$
%	by removing the vertices which belong to $S$; and
%	\item the $S$-signature of $\bar u$.
%\end{itemize}
%
%Given a graph $G$, a subset of its vertices $S$, and a tuple of vertices $\bar u$,
%by $N^r_S(\bar u)$ we denote the subgraph of $G^S$ induced by the set of vertices reachable from a vertex in $\bar u$ by a path of length at most $r$
%in $G-S$ (the graph $G$ is implicit in the notation $N^r_S(\bar u)$). Note that $N^r_S(\bar u)$ is a colored graph, with colors inherited from $G^S$.
%
%\begin{comment}
%We will prove the following strengthening of~Theorem~\ref{lem:types}:
%
%\begin{lemma}%[Gaifman locality $\lor$ Feferman-Vaught]
%  \label{lem:types1}
%For any given number $q\in\N$ one can compute 
% a number $r\in\N$ with the following property.
%	For any graph $G=(V,E)$, sets of vertices $A,B,S\subset V$	
%	such that $A$  and $B$ are $r$-separated by $S$,
%	for every tuple $\bar u\in A^{d}$, 
%	the type $\tp^q_G(\bar u/B)$
%	is computable from $\tp^{q}[B^r_S(\bar u), \bar u]$, and $G$ and $S$.
%	
%	
%	
%		%
%	% and any $\bar u,\bar v\in B^{\bar y}$, the following implication holds:
%	% $$\text{if\quad}\tp^q(\bar u/S)=\tp^q(\bar v/S)\text{\quad then\quad}
%	% 	\tp^q(\bar u/A)=\tp^q(\bar v/A).$$
%\end{lemma}
%
%To show that~Theorem~\ref{lem:types1} implies~Theorem~\ref{lem:types}, define $p$ as $q\cdot r$. It suffices to show that
%$\tp^{q}[B^r_S(\bar u), \bar u]$ is computable from $\tp_G^p(\bar u/S)$, and $G$ and $S$. This is the case, since
%a formula $\phi(\bar y)$
%can be relativized to $B^r_S(\bar u)$
%by replacing each quantifier $\exists x$ by a formula
%$\exists x\exists x_1\ldots\exists x_r\psi (x,x_1,\ldots,x_r)$,
%where $\psi$ specifies that $x_1,\ldots,x_r,x$ form a path
%starting in one of the vertices in $\bar u$, ending in $x$,
%and omitting all vertices in $S$, which are enumerated as parameters.
%
%It remains to prove~Theorem~\ref{lem:types1}.
%% We will use slightly stronger variants
%% of~Theorem~\ref{lem:gaif}
%% and~Theorem~\ref{lem:fv}, where the  graphs are colored with a fixed number of colors; the  types computed in the statements then depend also on the number of colors.
%\end{comment}
%
%
%With all these definitions and results in place, we may proceed to the proof of Theorem~\ref{lem:types}.
%
%\begin{proof}[of~Theorem~\ref{lem:types}]
%We prove the lemma for $r=7^q$.
%Let $t$ be the constant given by Gaifman's lemma, Theorem~\ref{lem:gaif}, for $q$ and $d$, and let $p=t+r$.
%
%Fix $G,A,B,S$ as in the statement of the lemma, and fix a tuple $\bar w\in B^\ell$, for some length $\ell$.
%To prove the lemma, it is enough to show that
%for tuples $\bar u\in A^d$,
%the type $\tp^q_G(\bar u\bar w)$ is determined by the type $\tp^p(\bar u/S)$.
%Indeed, applying this to every tuple $\bar w$ of parameters from $B$ implies that $\tp^q_G(\bar u/B)$ is determined by $\tp^q_G(\bar u/S)$, as requested.
%
%We will prove the following sequence of determinations,
%where an arrow $a\rightarrow b$ signifies that $b$ is determined by $a$:
%\begin{align*}
%	\tp^p_G(\bar u/S)
%  \ \longrightarrow\ 
%	\tp^{t}[N^r_S(\bar u), \bar u]
%  \ \stackrel{\textrm{(\ref{lem:fv})}}{\longrightarrow}\ 
%	\tp^{t}[N^r_S(\bar u\bar w), \bar u\bar w] \ \stackrel{\textrm{(\ref{lem:gaif})}}\longrightarrow\ 
%	\tp^q[G^S, \bar u\bar w] \ \longrightarrow\ 
%	\tp^q_G(\bar u\bar w).
%\end{align*}
%The second arrow follows from Feferman-Vaught's lemma, Theorem~\ref{lem:fv},
%as the colored graph $N^r_S(\bar u\bar w)$
%is the disjoint union of the colored graphs 
%$N^r_S(\bar u)$ and $N^r_S(\bar w)$,
%because $\bar u$ and $\bar w$ are $r$-separated by $S$.
%The third arrow is directly implied by the Gaifman's lemma,~Theorem~\ref{lem:gaif}.
%We are left with arguing the first and the last arrow, which both follow from simple rewriting arguments, presented below.
%
%\medskip
%For the first arrow, obviously already $\tp^0_G(\bar u/S)$ determines the $S$-signature of $\bar u$. Let $\bar s$ be any enumeration of $S$.
%To see that $\tp^p_G(\bar u/S)$ determines $\tp^{t}_{N^r_S(\bar v)}(\bar v)$, where $\bar v$ is $\bar u$ with vertices of~$S$ removed, take any formula $\phi(\bar x)$ with $|\bar x|=|\bar v|$.
%Let $\phi'(\bar x,\bar s)$ be the formula with parameters~$\bar s$ from $S$ that is syntactically derived from $\phi(\bar x)$ as follows: 
%to every quantification $\exists y$ in $\phi(\bar x)$ we add a guard $\delta(y,\bar s)$ stating that there is a path from some element of $\bar x$ to $y$ that has length at most $r$ and does not pass through
%any vertex of $\bar s$; it is easy to see that there is such a guard $\delta(y,\bar s)$ with quantifier rank $r$.
%Then $\phi'(\bar x,\bar s)$ has quantifier rank at most $t+r=p$, and it is straightforward to see that for every $\bar v\in (A-S)^{|\bar v|}$, we have $G,\bar v\models \phi'(\bar x,\bar s)$ if and only if
%$N^r_S(\bar v),\bar v\models \phi(\bar x)$. Therefore, to check whether $\phi(\bar x)$ belongs to $\tp^{t}_{N^r_S(\bar v)}(\bar v)$ it suffices to check 
%whether $\phi'(\bar x,\bar s)$ belongs to $\tp^p_G(\bar u/S)$, so the latter type determines the former.
%
%\medskip
%
%The argument for the last arrow is provided by the following claim.
%
%\begin{claim}\label{cl:rewrite}
%  Let $\phi$ be a formula
%  with $k$ free variables and quantifier rank at most $q$, 
%  and let $\sigma$ be an $S$-signature of length $k$.
%  One can compute a formula $\phi^S$ of quantifier rank at most $q$
%  whose free variables correspond to the $\star$'s in $\sigma$,
%  such that for every tuple $\bar v$ of elements of $G$
%  whose $S$-signature is~$\sigma$,
%   $\phi(\bar v)$ holds in $G$
%  if and only if $\phi^S(\bar v^S)$ holds in $G^S$, where $\bar v^S$ is obtained from $\bar v$ by removing 
%  those elements that belong to $S$.
%\end{claim}
%\begin{clproof}[Sketch]
%The proof proceeds by a straightforward induction on the structure of the formula $\phi$.
%In essence, every quantification $\exists y$ of a vertex~$y$ in $G$ is replaced by quantification of $y$ in $G-S$ plus a disjunction over $s\in S$ of formulas where we assume $y=s$.
%Atomic formulas of the form $E(x,y)$ and $x=y$ have to be replaced accordingly. Say for $E(x,y)$: if both~$x$ and~$y$ are assumed to be in $G-S$, then we leave $E(x,y)$ intact;
%if $x$ is assumed to be in $S$ (say we assume $x=s$) and $y$ is assumed to be in $G-S$, then we substitute $E(x,y)$ by $C_s(y)$; and if both~$x$ and~$y$ are assumed to be in $S$, then 
%we replace $E(x,y)$ by $\bot$ or $\top$ depending on whether the vertices assumed to be equal to $x$ and $y$ are adjacent or not.
%We leave the details to the reader.
%\end{clproof}
%\begin{comment}
%\begin{clproof}
%The proof proceeds by induction on the structure of the formula $\phi$. 
%
%If $\phi$ is an atomic formula $E(x,x')$ or $x=x'$, then the formula $\phi^S$ is constructed by case analysis. If $\alpha(x),\alpha(x')\in Y$ then $\phi^S$
%is obtained from $\phi$ by substituting the variables $x,x'$ with variables from $Y$ according to~$\alpha$. If  $\alpha(x),\alpha(x')\in S$ then $\phi'$ is the truth value $\bot$ or $\top$ of 
%the formula $\phi$ in the graph $G$ under the valuation which maps $x$ to $\alpha(x)$ and $x'$ to $\alpha(x')$. 
%Finally, suppose that $\alpha(x)=y\in Y$ and $\alpha(x')=s\in S$. If $\phi$ is $E(x,x')$ then $\phi'$ is the formula $C_{s}(y)$, and if $\phi$ is $x=x'$ then $\phi'$ is the formula $\bot$.
% 
%For the inductive step, we consider two cases.
%If $\phi$ is a boolean combination of formulas $\phi_1,\ldots,\phi_k$, then 
%apply the inductive assumption to each formula $\phi_i$,
%yielding formulas $\phi_1',\ldots,\phi_k'$. Then let $\phi'$ be the analogous boolean combination of the formulas $\phi_1',\ldots,\phi_k'$.
%
%Finally, suppose that $\phi$ is of the form $\exists x\, \psi$, where   $Y$ are the free variables of $\phi$ and $x\not \in Y$.
% For $w$ being either the variable $x$ 
%or an element $s\in S$, 
%let $\psi^w$ be the formula obtained from the inductive assumption applied to the formula $\psi$ 
%and pre-valuation $\alpha$ extended to a valuation which maps  $x$ to $w$. 
%Then let $\phi'$
%be the formula $\exists x\, \psi^x \lor \bigvee_{v\in S}\psi^v$.
%The case of $\forall$ is dual.
%
%In each case, it follows from the inductive assumption that $\phi'$ 
%satisfies the required condition.
%\end{clproof}
%\end{comment}
%
%\autoref{cl:rewrite}, applied to $k=|\bar u\bar w|$, implies that 
%$\tp^q_G(\bar u\bar w)$ is determined by $\tp^q[G^S, \bar u\bar w]$, finishing the proof of~Theorem~\ref{lem:types}.	
%\end{proof}
%
%We remark that in all the results of this section, whenever some type determines some other type, it is actually true that the latter type can be {\em{computed}} given the former type together with the graph $G$ 
%and, if applicable, also the set of vertices $S$. For Gaifman's locality lemma, the effectiveness follows from the original proof of Gaifman~\cite{gaifman1982local}, and it is not hard to see that the proof of the Feferman-Vaught lemma (Theorem~\ref{lem:fv}) can be also made effective. By examining our proofs, one can readily verify that all the stated determination relations can be made effective in this sense.

