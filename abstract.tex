
\begin{abstract}
\noindent 
Nowhere dense classes of graphs were introduced 
by Nešetřil and Ossona de 
Mendez~\cite{nevsetvril2010first,nevsetvril2011nowhere} as a model
for uniform sparseness in graphs. The concept of nowhere denseness
turns out to be very robust as witnessed by the fact that it is equivalent 
to multiple other concepts studied in different areas of mathematics. 
In this work we revisit the connections between the notion of nowhere 
denseness and notions from (finite) model theory.

Based on work of Podewski and Zieger~\cite{podewski1978stable}, 
Adler and Adler~\cite{adler2014interpreting}
proved that every nowhere dense class $\CCC$ of graphs is stable, that is, 
the ladder index of every first-order formula $\phi(\tup{x},\tup{y})$ over
graphs from $\CCC$ is bounded by a constant depending only on $\phi$ and
$\CCC$. The original proof of this fact is based on an infinite
Ramsey argument and explicit bounds on ladder indices were never given. 
We give a combinatorial proof of Adler and Adler's result in 
the finite and work out the explicit bounds for ladder indices of
first-order formulas on nowhere dense classes of graphs. 

A class $\CCC$ of graphs if uniformly quasi-wide if there are functions 
$N:\N\times\N\rightarrow\N$ and $s:\N\rightarrow\N$ such 
that for all $G\in \CCC$ and all $r,m\in \N$, if $A\subseteq V(G)$ is
of size at least $N(r,m)$ then there is a set $S$ of size at most $s(r)$
such that there is a subset $B\subseteq A\setminus S$ which is
$r$-independent in $G-S$. It was proved in~\cite{siebertz2016polynomial} 
that we can always choose $N(r,m)\leq m^{f(r)}$, for a function $f$ whose
existence follows from the earlier non-constructive argument of Podewski 
and Ziegler. We give a simpler proof of this result with much improved bounds 
on $f(r)$.

Finally, we observe that an argument of Bousquet and 
Thomasse\'e~\cite{BousquetT15} can be slightly modified to prove that 
the VC-dimension of the $r$-power graph $G^r$ of a graph $G$
with $K_t\not\minor_r G$ is bounded by $t-1$.
\end{abstract}