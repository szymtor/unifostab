\begin{abstract}
\noindent 
Nowhere dense classes of graphs were introduced 
by Ne\v set\v ril and Ossona de 
Mendez~\cite{nevsetvril2010first,nevsetvril2011nowhere} as a model
for uniform sparseness of graphs. The concept of nowhere denseness
turns out to be very robust, as witnessed by the fact that it is equivalent 
to multiple other concepts studied in different areas of mathematics. 
In this work we revisit connections between nowhere 
denseness and notions from (finite) model theory.

Based on the work of Podewski and Ziegler~\cite{podewski1978stable}, 
Adler and Adler~\cite{adler2014interpreting}
proved that every nowhere dense class $\CCC$ of graphs is stable; that is, 
the ladder index of every first-order formula $\phi(\tup{x},\tup{y})$ over
graphs from $\CCC$ is bounded by a constant depending only on $\phi$ and
$\CCC$. The original proof of this fact is based on an infinite
Ramsey argument that does not yield any explicit bounds on ladder indices. 
We give a combinatorial proof of the result of Adler and Adler in 
the finite and work out explicit bounds for ladder indices of
first-order formulas on nowhere dense classes of graphs. 

A class $\CCC$ of graphs if uniformly quasi-wide if there are functions 
$N\colon \N\times\N\rightarrow\N$ and $s\colon \N\rightarrow\N$ such 
that for all $G\in \CCC$ and all $r,m\in \N$, if $A\subseteq V(G)$ is
of size at least $N(r,m)$ then there are subsets $S\subseteq V(G)$ and $B\subseteq A\setminus S$, of sizes at most $s(r)$ and at least $m$ respectively, such that $B$ is
$r$-independent in $G-S$. It was proved by Ne\v set\v ril and Ossona de Mendez~\cite{nevsetvril2010first} that
nowhere denseness is equivalent to uniform quasi-wideness, while recently Kreutzer et al.~\cite{siebertz2016polynomial} 
showed that we can always choose $N(r,m)\leq m^{f(r)}$, for a function $f$ whose
existence follows from the earlier non-constructive argument of Podewski 
and Ziegler. We give a combinatorial proof of this result with explicit and much improved bounds 
on $f(r)$.

Finally, we observe that an argument of Bousquet and 
Thomass\'e~\cite{BousquetT15} can be slightly modified to prove that 
the VC-dimension of the $r$-power graph $G^r$ of a graph $G$
not admitting $K_t$ as an $r$-shallow minor is bounded by $t-1$.
\end{abstract}