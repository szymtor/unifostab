All graphs in this paper are finite, undirected and simple, that is, 
they do not have loops or parallel edges. Our notation is standard,
we refer to~\cite{diestel2012graph} for more background on 
graph theory. 
We write $V(G)$ for the vertex set of a graph $G$ and
$E(G)$ for its edge set. 
The {\em{distance}} between vertices $u$ and $v$ in $G$, denoted $\dist_G(u,v)$, is the length of a shortest path between $u$ and $v$ in~$G$.

The \emph{$r$th power of a graph $G$} is the graph $G^r$
with vertex set $V(G)$, where there is an edge between two 
vertices $u$ and $v$ if and only if their distance in $G$ is at most $r$. 

A {\em{minor model}} of a graph $H$ in $G$ is a family $(I_u)_{u\in V(H)}$ of pairwise vertex-disjoint connected subgraphs of $G$
such that whenever $uv$ is an edge in~$H$, there are $u'\in I_u$ and $v'\in I_v$ for which $u'v'$ 
is an edge in $G$.
The graph $H$ is a {\em{depth-$r$ minor}} of $G$, denoted $H\minor_rG$, if there is a minor model
$(I_u)_{u\in V(H)}$ of~$H$ in $G$ such that each subgraph $I_u$ has radius at most $r$.

A graph $H$ is a \emph{topological minor} of a graph $G$ if there is a
function~$\delta$ mapping vertices $v\in V(H)$ to vertices of $V(G)$ and 
edges $e\in E(H)$ to directed paths in $G$ such that 
$\delta(v)\neq \delta(u)$ for all distinct $u,v\in V(H)$, and 
if $e=(u,v)\in E(H)$, then $\delta(e)$ is a path from 
$\delta(u)$ to $\delta(v)$ in~$G$ which is internally vertex disjoint from all 
$\delta(e')$ with $e'\in E(H)$, $e'\neq e$. 
For $r\geq 0$, $H$ is a \emph{topological depth-$r$ minor} of $G$, 
written $H\minor_r^tG$, if it is a topological minor and all paths $\delta(e)$
have length at most $2r$. 

A class $\CCC$ of graphs is \emph{nowhere dense} if there is a function 
$f:\N\rightarrow \N$ such that for all $r\in \N$ it holds that $K_{f(r)}\not\minor_r G$
for all $G\in \CCC$. Nowhere dense classes of graphs were introduced by
Ne\v{s}et\v{r}il and Ossona de Mendez in~\cite{nevsetvril2010first,nevsetvril2011nowhere}.

Nowhere dense classes of graphs admit many equivalent characterisations, 
one of them being uniform quasi-wideness, a notion studied in 
finite model theory~\cite{dawar2010homomorphism}.  
A set $B\subseteq V(G)$ is called {\em{$r$-independent}} in $G$ if for all
distinct $u,v\in B$ we have $\dist_G(u,v)>r$.
A class $\CCC$ of graphs is \emph{uniformly quasi-wide} if there are
functions $N\colon \N\times\N\rightarrow \N$ and $s:\N\rightarrow \N$ such
that for all $r,m\in \N$ and all subsets $A\subseteq V(G)$ for
$G\in \CCC$ of size $\abs{A}\geq N(r,m)$ there is a set
$S\subseteq V(G)$ of size $\abs{S}\leq s(r)$ and a set
$B\subseteq A\setminus S$ of size $\abs{B}\geq m$ which is $r$-independent in
$G-S$. 

\begin{theorem}[Ne\v{s}et\v{r}il and Ossona de Mendez~\cite{nevsetvril2010first}]
A class $\CCC$ of graphs is nowhere dense if and only if it
is uniformly quasi-wide. 
\end{theorem}

The proof of Ne\v{s}et\v{r}il and Ossona de Mendez goes back to a construction
of Kreidler and Seese~\cite{kreidler1998monadic} (see also Atserias et al.~\cite{atserias2006preservation}), 
and uses iterated
Ramsey arguments. Hence the bounds on the function $N$ are huge. Recently, 
we proved that we may always choose $N$ to be a polynomial 
function~\cite{siebertz2016polynomial}. 

\begin{theorem}[Kreutzer, Rabinovich, and Siebertz \cite{siebertz2016polynomial}]\label{thm:uqw}
  Let $\CCC$ be a nowhere dense class of graphs and let 
  $t\,\colon\,\N\rightarrow \N$ be the function such that
  $K_{t(r)}\not\minor_r G$ for all $r\in \N$ and all $G\in \CCC$.  
  For every $r\in \N$
  there exist constants~$p(r)$ and $s(r)\leq t(r)$ such that
  for all $m\in \N$, all $G\in\CCC$ and all sets $A\subseteq V(G)$ of size at 
  least~$m^{p(r)}$, there is a set $S\subseteq V(G)$ of size at
  most $s(r)$ such that there is a set $B\subseteq A\setminus S$ of size at
  least~$m$ which is $r$-independent in $G-S$.
  
  Furthermore, there is an algorithm, that given an $n$-vertex graph
  $G\in \CCC$, $\epsilon>0$, $r\in \N$ and $A\subseteq V(G)$ of size at least
  $m^{p(r)}$, computes sets $S$ and $B\subseteq A$ as described above
  in
  time $\Oof(r\cdot t\cdot |A|^{t+1}\cdot n^{1+\epsilon})$.
\end{theorem}

We remark that the running time of the algorithm of~\Cref{thm:uqw}
is stated in the SODA version~\cite{siebertz2016polynomial} only as 
$\Oof(r\cdot t\cdot n^{t+6})$. A finer analysis with the running
times as stated above can be found in the arXiv version of that paper.

In order to prove the above bounds, we used methods from model
theory, more precisely, from stability theory. Stability is a strong tameness
property of first-order formulas, on which Shelah built his famous 
classification theory~\cite{shelah1990classification}.


\paragraph{First-order logic and stability.}
For extensive background on first-order logic, we refer the reader
to~\cite{hodges1993model}. For our purpose, it suffices to define
first-order logic over the vocabulary of graphs (with constant symbols
from a given parameter set).
 
Let $A$ be a set. We call $L(A)\coloneqq\{E\hspace{0.3mm}\}\cup A$ the \emph{vocabulary}
of graphs with parameters from $A$. \emph{First-order formulas} over $L(A)$ are
formed from atomic formulas~$x=y$ and $E(x,y)$, where $x,y$ are variables (we
assume that we have an infinite supply of variables) or elements of $A$ treated
as constant symbols, by the usual Boolean
connectives~$\neg$~(negation),~$\wedge$ (conjunction), and~$\vee$ (disjunction)
and existential and universal quantification~$\exists x,\forall x$,
respectively.  The free variables of a formula are those not in the scope of a
quantifier, and we write~$\phi(x_1,\ldots,x_k)$ to indicate that the free
variables of the formula~$\phi$ are among $x_1,\ldots,x_k$.

To define the semantics, we inductively define a satisfaction
relation~$\models$. Let $G$ be a graph and $A\subseteq V(G)$. For an
$L(A)$-formula~$\phi(x_1,\ldots,x_k)$, and
$v_1,\ldots,v_k\in V(G)$, $G\models\phi(v_1,\ldots,v_k)$
means that~$G$ satisfies~$\phi$ if the free variables~$x_1,\ldots,x_k$
are interpreted by~$v_1,\ldots,v_k$ and the parameters $a\in A$
(formally treated as constant symbols) used in the formula are
interpreted by the corresponding element of $A$ in $G$, respectively. If
$\phi(x_1,x_2)=E(x_1,x_2)$ is atomic, then $G\models\phi(v_1,v_2)$
if~$(v_1,v_2)\in E(G)$. The meaning of the equality symbol, the
Boolean connectives, and the quantifiers is as expected. For a
formula $\phi(x_1,\ldots, x_k, y_1,\ldots, y_\ell)$ and
$v_1,\ldots, v_\ell\in V(G)$ (treated as a sequence of parameters), we
write $\phi(x_1,\ldots, x_k, v_1,\ldots, v_\ell)$ for the formula with
free variables $x_1,\ldots, x_k$ where each occurrence of the variable
$y_i$ in $\phi$ is replaced by the constant symbol $v_i$.

We usually write $\tup{x}$ for a tuple $(x_1,\ldots, x_k)$ of variables. 
Usually, the length of the tuple is understood from the context. 

A first-order formula $\psi(\tup{x},\tup{y})$ has the \emph{$k$-order property}
over a graph $G$ if there are tuples $\tup{a}_1,\ldots, \tup{a}_k, \tup{b}_1,\ldots, \tup{b}_k$
of elements of $G$ such that \[\psi(\tup{a}_i,\tup{b}_j)\Longleftrightarrow i\leq j.\]
A class of graph $\CCC$ is \emph{stable} if for every formula $\psi(\tup{x},\tup{y})$ there
is a number $k$ such that for every graph $G\in \CCC$, $\psi$ does not have the $k$-order 
property.

Adler and Adler~\cite{adler2014interpreting} observed that nowhere density 
is essentially the stability theoretic notion of super flatness, introduced by
Podewski and Ziegler~\cite{podewski1978stable}. Based on the construction of 
Podewski and Ziegler they proved that every nowhere dense class of graphs is
stable. 

\begin{theorem}[Adler and Adler~\cite{adler2014interpreting}]\label{thm:adleradler}
If $\CCC$ is nowhere dense, then $\CCC$ is stable. 
\end{theorem}

Let $G$ be a graph and let $\Delta$ be a set of formulas. A sequence
$(v_1,\ldots, v_\ell)$ of vertices of~$G$ is
\emph{$\Delta$-indiscernible} if for every formula
$\phi(x_1,\ldots, x_k)\in \Delta$ with $k$ free variables and any two
increasing sequences
$1\leq i_1<\ldots <i_k\leq \ell, 1\leq j_1< \ldots< j_k\leq \ell$ of
integers, it holds that
\[G\models\phi(v_{i_1},\ldots, v_{i_k})\Leftrightarrow G\models\phi(v_{j_1},
\ldots, v_{j_k}).\]

The next theorem shows that we can find long indiscernible
sequences in stable classes of graphs. 

\begin{theorem}[Malliaris and Shelah~\cite{malliaris2014regularity}, Theorem 3.5, Item (2)]\label{thm:malshelah}
  Let $\CCC$ be a stable class of graphs and let $\Delta$ be a finite
  set of first-order formulas.  There is a polynomial $p(x)$ such that
  for all $G\in \CCC$, every positive integer $m$ and every sequence
  $(v_1,\ldots, v_\ell)$ of vertices of $G$ of length $\ell=p(m)$, there
  exists a sub-sequence $(v_{i_1},\ldots, v_{i_m})$ of
  $(v_1,\ldots, v_\ell)$ of length $m$ which is
  $\Delta$-indiscernible, $1\leq i_1<\ldots <i_m\leq \ell$.
\end{theorem}


\begin{theorem}\label{thm:extract_indiscernibles}
  Let $\CCC$ be a stable class of graphs and let $\Delta$ be a finite
  set of first-order formulas.  There is a polynomial $p(x)$ such that
  for all $G\in \CCC$, every positive integer $m$ and every sequence
  $(v_1,\ldots, v_\ell)$ of vertices of $G$ of length $\ell=p(m)$, there
  exists a sub-sequence $(v_{i_1},\ldots, v_{i_m})$ of
  $(v_1,\ldots, v_\ell)$ of length $m$ which is
  $\Delta$-indiscernible, $1\leq i_1<\ldots <i_m\leq \ell$.
\end{theorem}

As shown in~\cite{siebertz2016polynomial}, we can also compute
such sequences in polynomial time. More precisely, there is an algorithm running in time
  $\Oof(\abs{\Delta}\cdot k \cdot \ell^{k+1}
    \cdot n^{q})$, where $k$ is the maximal number of 
    free variables and $q$ is 
    the maximal quantifier-rank of a formula of $\Delta$, 
    that given an $n$-vertex graph $G\in \CCC$ and a sequence
  $(v_1,\ldots, v_\ell)\subseteq V(G)$, computes a
  $\Delta$-indiscernible sub-sequence of $(v_1,\ldots, v_\ell)$ 
  of length at least $m$.

We used the algorithmic version of~\cref{thm:extract_indiscernibles} 
to prove~\cref{thm:uqw}.

\paragraph{VC-dimension.}

Let $\FFF\subseteq 2^A$ be a family of
subsets of a set $A$. For a set $X\subseteq A$, we denote $X\cap \FFF=\{X\cap F : F\in \FFF\}$.
The set $X$ is \emph{shattered by $\FFF$} if $X\cap \FFF=2^X$.
The \emph{Vapnik-Chervonenkis dimension}~\cite{chervonenkis1971theory}, 
short \emph{VC-dimension},
of $\FFF$ is the maximum size of a set $X$ that is shattered by
$\FFF$. Note that if $X$ is shattered by $\FFF$, then also every
subset of $X$ is shattered by~$\FFF$.

For a graph $G$, the VC-dimension of $G$ is defined as the VC-dimension
of the set family $\{N[v]\colon v\in V(G)\}$ over the set $V(G)$.

Adler and Adler's result implies that any class of structures
obtained from $\CCC$ by means of a first-order interpretation has VC-dimension
bounded by a constant depending only on $\CCC$ and the interpretation.
In particular, the following is an immediate corollary of the results of Adler and Adler.

\begin{theorem}[\cite{adler2014interpreting}]\label{thm:adler}
  Let $\CCC$ be a nowhere dense class of graphs and let $\phi(x,y)$ be
  a first-order formula over the signature of graphs,
  such that for all $G \in \CCC$ and $u,v\in
  V(G)$ it holds that $G\models\phi(u,v)$ if and only if $G\models\phi(v,u)$. 
  For $G\in \CCC$, let $G_\phi$ be the graph with
  vertex set $V(G_\phi)=V(G)$ and edge set $E(G_\phi)=\{uv \colon
  G\models\phi(u,v)\}$. Then there is an integer $c$ depending only on
  $\CCC$ and $\phi$ such that $G_\phi$ has VC-dimension at most $c$.
\end{theorem}

By applying \Cref{thm:adler} to the first-order formula 
expressing that $\dist(u,v)\leq r$,
we immediately obtain the following.

\begin{corollary}\label[corollary]{crl:Gr}
  Let $\CCC$ be a nowhere dense class of graphs and let $r\in\N$. 
  Then there is an integer $c(r)$ such that $G^r$ 
  has VC-dimension at most~$c(r)$ for every $G\in \CCC$.
\end{corollary}

\paragraph{Our contributions.}

We give a new proof of~\cref{thm:uqw} which is simpler than 
the original proof and gives explicit bounds on the function $N$. 
Furthermore, the new proof leads to a much simpler algorithm 
with drastically improved running times.

\begin{theorem}\label{thm:new-uqw}
  Let $\CCC$ be a nowhere dense class of graphs and let 
  $t\,\colon\,\N\rightarrow \N$ be the function such that
  $K_{t(r)}\not\minor_r G$ for all $r\in \N$ and all $G\in \CCC$.  
  For every $r\in \N$
  there exist constants~$p(r)\leq t(r)^{t(r)}$ and $s(r)\leq t(r)$ such that
  for all $m\in \N$, all $G\in\CCC$ and all sets $A\subseteq V(G)$ of size at 
  least~$m^{p(r)}$, there is a set $S\subseteq V(G)$ of size at
  most $s(r)$ such that there is a set $B\subseteq A\setminus S$ of size at
  least~$m$ which is $r$-independent in $G-S$.
  
  Furthermore, there is an algorithm, that given an $n$-vertex graph
  $G\in \CCC$, $\epsilon>0$, $r\in \N$ and $A\subseteq V(G)$ of size at least
  $m^{p(r)}$, computes sets $S$ and $B\subseteq A$ as described above
  in
  time $\Oof(r\cdot t\cdot |A|^2\cdot n^{1+\epsilon})$.
\end{theorem}

We give an explicit proof of~\cref{thm:adler} in the finite, 
which in combination with~\cref{thm:new-uqw} for the first time
gives explicit bounds for the $k$-order property of a formula
on a nowhere dense class of graphs. 

\begin{theorem}\label{thm:new-stable}
If $\CCC$ is nowhere dense then $\CCC$ is stable. Furthermore, 
if $K_t\not\minor_{r(\phi)} G$, then $\phi$ does not have
the $k(\phi)$-order property.
\end{theorem}

Finally, we give an explicit proof of~\cref{crl:Gr} by slightly
changing a construction of Bousquet and 
Thomasse\'e~\cite{BousquetT15}. 

\begin{theorem}\label{thm:new-vc}
Let $G$ be a graph with $K_t\not\minor_r G$. Then 
the VC-dimension of $G^r$ is
bounded by $t-1$. 
\end{theorem}