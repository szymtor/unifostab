
\section{From uniform quasi-wideness to stability}\label{sec:stable}

In this section we focus on proving the following result, observed earlier by Adler and Adler~\cite{adler2014interpreting}.

\begin{theorem}\label{thm:uqw-stable}
	Let $\cal C$ be a uniformly quasi-wide class of graphs.
	Then $\cal C$ is stable. % More precisely, if $\phi(\bar u,\bar v)$ is a formula with $d$ free variables and quantifier rank $q$, and $N^d(r,m)$ and $s^d(r)$ are as in Proposition~\ref{prop:uqw-tuples}, then for all graphs $G\in\cal C$,  the ladder index of $\phi$ is at most $N^d(2r,(2T)^d)$,
% where $r= 7^q$,
%  $T$ is the number of all quantifier rank $q$ types with one free  variable over the signature of  graphs colored with $s^d(2r)$ colors.
%\marginpar{shall we recall the definition of stability and ladders?}
\end{theorem}

Recall that a class $\cal C$ is stable if and only if for every first-order formula $\varphi(\bar x,\bar y)$, its ladder index over graphs from $\cal C$ is bounded by a constant depending only on $\cal C$ and $\varphi$;
see \cref{sec:intro} to recall the background on stability.
Thus \cref{thm:uqw-stable} is implied by \cref{thm:new-stable} (stated in \cref{sec:intro}), and is weaker in the following sense: \cref{thm:new-stable} asserts in addition that there is a computable bounds on the ladder index
of any formula that depends only on the size of an excluded clique minor on a levels bounded in terms of formula's quantifier rank and number of free variables. At the end of the proof we argue that
the obtained bounds in fact also imply the stronger statement of \cref{thm:new-stable}, but for the clarity of presentation we find it more instructive to work with the cleaner formulation of \cref{thm:uqw-stable}.

The plan is as follows. In \cref{sec:uqw-tuples} we formulate a variant of uniform quasi-wideness tailored to tuples of vertices. Using this and Gaifman's Locality Theorem, we prove \cref{thm:uqw-stable} in \cref{sec:uqw-stable}.


\subsection{Uniform quasi-widness for tuples}\label{sec:uqw-tuples}
Fix a graph $G$, the dimension $d\in\N$, and the radius $r\in \N$.
If $S\subset V(G)$ is a set of vertices and $A\subset V(G)^d$ is a set of $d$-tuples of vertices,
then we say that $A$ is \emph{mutually $r$-independent} in $G-S$ 
if for every two distinct $(u_1,\ldots,u_d),(v_1,\ldots,v_d)\in A$
and for all $1\le i,j\le d$, the distance between the vertices $u_i$ and $v_j$ in the graph $G-S$
is larger than $r$. Throughout this section we use the convention that if $x\in S$, then the distance in $G-S$ between $x$ and any vertex, including~$x$ itself, is infinity. 
For instance, in the definition above, the tuples from $A$ may contain vertices of~$S$, and such a vertex is considered infinitely far from every vertex.

We now prove the following proposition, which can be viewed as an extension of uniform quasi-wideness to tuples.
The proof is based on translating the approach of Podewski and Ziegler~\cite{podewski1978stable} to the finite.

\begin{proposition}\label{prop:uqw-tuples}
	Let $\cal C$ be a uniformly quasi-wide class of graphs and $d\in\N$ be an integer.
	Then there are functions $N^d\colon \N\times\N\to\N$ and $s^d\colon \N\to\N$
	such that for all $r,m\in\N$ and all subsets $A\subset V(G)^d$
	with $|A|\ge N^d(r,m)$ there  is a set $S\subset V(G)$
	of size $|S|\le s^d(r)$ and a subset $B\subset A$ of size $|B|\ge m$ which is mutually $r$-independent in $G-S$. Moreover, $N^d$ is polynomial in the second argument and $s^d$ is polynomial. 
\end{proposition}

The rest of this section is devoted to the proof of \cref{prop:uqw-tuples}.
Fix a uniformly quasi-wide class $\cal C$ and functions $N\colon \N\times\N\to \N$
	and $s\colon \N\to\N$ as in the definition of uniform quasi-wideness.
	Let $d\in \N$ be a fixed dimension.
		For a fixed graph $G\in \cal C$  and
	  coordinate $i\in\set{1,\ldots,d}$, let $\pi_i$ denote the projection from $V(G)^d$ onto the $i$th coordinate.

	


\begin{lemma}\label{lem:step1} For any $r,m\in \N$ there is an integer $K(r,m)$ such that
	for any given $A\subset V(G)^d$ with $|A|\ge K(r,m)$,
	there is a set $B\subset A$ with $|B|\ge m$ and a set $S\subset V(G)$ with $|S|\le d\cdot s(r)$, 
	such that for each coordinate $i\in\set{1,\ldots,d}$ and all distinct $\bar x,\bar y\in B$,
 $\pi_i(\bar x)$ and $\pi_i(\bar y)$ are at distance greater than $r$ in $G-S$. 
\end{lemma}
\begin{proof}
We will iteratively apply the following claim.

%Let $f\colon \N\to \N$ be defined as $f(m)=N(r,m)\cdot m$ for $m\in\N$.

\begin{claim}\label{claim:ith-coord}
Fix a coordinate $i\in\set{1,\ldots,d}$, an integer $m'\in\N$, and a  set $A'\subset V(G)^d$ with  $|A'|\ge N(r,m')\cdot m'$.
Then there is a set $B'\subset D$ with $|B'|\ge m'$
and a set $S'\subset V(G)$ with $|S'|\le  s(r)$, such that for all distinct $\bar x,\bar y\in B$,
 $\pi_i(\bar x)$ and $\pi_i(\bar y)$ are at distance greater than $r$ in $G-S$. 
\end{claim}
\begin{clproof}
We consider two cases.

If $\pi_i(A')\subset V(G)$ has at least $N(r,m')$ elements, then we apply the definition of uniform quasi-wideness to $\pi_i(A')\subset V(G)$. This yields sets $S'\subset V(G)$ and $B''\subset \pi_i(A')$
such that $|B''|\ge m'$, $|S'|\le s(r)$, and $B''$ is $r$-independent in $G-S'$. 
Let $B'\subseteq A'$ be a subset of tuples constructed as follows: for each $u\in B''$, include in $B'$ one arbitrarily chosen tuple $\bar x\in A'$ such that $\pi_i(\bar x)=u$.
Clearly $|B'|=|B''|\ge m'$ and for all distinct $\bar x,\bar y\in B'$, we have that $\pi_i(\bar x)$ and $\pi_i(\bar y)$ are distinct and at distance greater than $r$ in $G-S$; this is because $B''$ is $r$-independent
in $G-S$. Hence $B'$ and $S'$ satisfy all the required properties.

If $\pi_i(A')$ has less than $N(r,m')$ elements, then choose the element $a\in\pi_i(A')$ whose inverse image $\pi_i^{-1}(\set a)\cap A'$ has the largest cardinality. Let $S'=\set{a}$ 
and let $B'=\pi_i^{-1}(\set a)$. Then $$|B'|\ge \frac{|A'|}{|\pi_i(A')|}\ge \frac{|A'|}{N(r,m')}\ge \frac {N(r,m')\cdot m'}{N(r,m')}=m',$$
and $|S'|=1$. Observe that $\pi_i(\bar x)=a$ for all $\bar x\in A'$. As $a\in S$, by the adopted convention, $\pi_i(\bar x)$ and $\pi_i(\bar y)$ are at infinite distance for all distinct $\bar x,\bar y\in B$.
\end{clproof}

We proceed with the proof of \cref{lem:step1}.
Let $f(m')=N(r,m')\cdot m'$ for $m'\in\N$; by $f^k$ we denote the $k$-fold composition of $f$ with itself.
Let $A\subset V(G)^d$ be such that $|A|\ge f^d(m)$. 
Define $B_0=A$, $S_0=\emptyset$, and for $i=1,\ldots,d$,
let $B_{i}$ and $S_i$ be the $B'$ and $S'$ obtained from \cref{claim:ith-coord} applied to the set of tuples $B_{i-1}\subset V(G)^d$, the coordinate $i$, and $m'=f^{d-i}(m)$. 
The invariant is that $|B_i|\ge f^{d-i}(m)$.
In particular, 
taking $B=B_d$ and $S=S_1\cup\ldots \cup S_d$, we obtain that $|B|\ge m$ and $|S|\le d\cdot s(r)$, and, by construction, $\pi_i(B)$
is $r$-independent in $G-S$ for every coordinate $i\in\set{1,\ldots,d}$. Letting $K(r,m)=f^d(m)$ yields the lemma.
\end{proof}


\begin{lemma}\label{lem:step2}
	Let $B\subset V(G)^d$ and $S\subset V(G)$ be such that 
  %\begin{itemize}
	%\item
   for all $i\in \set{1,\ldots,d}$,
	$\pi_i(B)$ is $2r$-independent in $G-S$.
  % \item for each tuple $\bar a\in B$, the set of entries of $\bar a$ is $r$-independent in $G-S$.
%	\end{itemize}
	Then there is a set $C$ with $C\subset B$ 
	such that $C$ is mutually $r$-independent in $G-S$
	and $|C|\geq\frac{|B|}{d^2+1}$.
\end{lemma}
\begin{proof}
We construct a sequence $C_0\subset C_1\subset \ldots$ of subsets of $B$ which are mutually $r$-independent in $G-S$, as follows.

We start with $C_0=\emptyset$. Suppose that $C_s\subset B$ is 
 already constructed for some $s\ge 0$
 and is mutually $r$-independent in $G-S$; we construct $C_{s+1}$. With each element $a\in B-C_s$,
we associate an arbitrarily chosen function $f_a\colon \set{1,\ldots,d}^2\to C_s\cup \set{\bot}$
with the following properties:
\begin{itemize}
	\item If $f_a(i,j)=b$ then the $i$th coordinate of $a$
	and the $j$th coordinate of $b$ are at distance at most $r$
	in $G-S$.
	\item If $f_a(i,j)=\bot$ then there is no element $b\in C_s$ 
	such that the $i$th coordinate of $a$ and the $j$th coordinate of $b$ are at distance at most $r$ in $G-S$.	
\end{itemize}
Observe that whenever $a_1, a_2$ are two distinct elements of $B-C_s$,
then for all $i,j\in \set{1,\ldots,d}^2$, the values $f_{a_1}(i,j)$ and $f_{a_2}(i,j)$
cannot be equal to the same element $b\in C_s$:
otherwise, we would have that the $i$th coordinate of $a_1$
and the $i$th coordinate of $a_2$ are at distance at most~$2r$
in $G-S$, which is impossible by the assumption on $B$.
In particular, if $|B-C_s|> |C_s|\cdot d^2$
then there must be some element  $a\in B-C_s$  
such that $f_a(i,j)=\bot$  for all $i,j\in\set{1,\ldots,d}$.
Let $C_{s+1}=C_s\cup \set a$.
By construction, $C_{s+1}$ is mutually $r$-independent in $G-S$.

We may repeat the construction as long as $|B|>|C_s|\cdot (d^2+1)=s\cdot (d^2+1)$, and we stop when this inequality no longer holds. Define the set $C$ as the last constructed set $C_s$.
By construction, $|C_s|=s\ge 
\frac{|B|}{d^2+1}$.	
\end{proof}

To finish the proof of \cref{prop:uqw-tuples},
given a set $A\subset V(G)^d$ and integers $r,m\in\N$,
first apply 
\cref{lem:step1} 
  with $r'=2r$ and
 $m'= m\cdot (d^2+1)$.
 Assuming that $|A|\ge K(r',m')$, 
we obtain a set $B\subset A$ with $|B|\ge m\cdot (d^2+1)$ and a set $S\subset V(G)$ with $|S|\le s(2r)$.
Apply \cref{lem:step2} to $B$ and $S$, yielding a set $C\subset B$ which is mutually $r$-independent in $G-S$ and has size at least $m$. This concludes the proof of \cref{prop:uqw-tuples},
where the obtained bounds are $N^d(r,m)=K(r',m')=K(2r,m\cdot (d^2+1))$ and $s^d(r)=d\cdot s(2r)$.


\subsection{Excluding long ladders}
\label{sec:uqw-stable}



% In this language,
% assuming that  $X$ is partitioned into disjoint sets $Y,Z$, a $\phi$-ladder of length $n$ in a graph $G$ consists of two sequences of valuations $u_1,\ldots,u_n\in V(G)^Y$
%     and $v_1,\ldots,v_n\in V(G)^Z$
%      such that $G,u_i\oplus v_j\models \phi$ if and only if $i\le j$.
    
    
    
	% 	Let $u,v\in V(G)^X$, $u_Y$ and $u_Z$ be the restrictions of $u$ to $Y$ and $Z$, respectively,
% and $v_Y$ and $v_Z$ be the restrictions of $v$ to $Y$ and $Z$, respectively.
% 		We say that valuations $u$ and $v$ are \emph{confusing} for $\phi$
% 		if \[G,u_Y\oplus v_Z\models \phi\iff G,v_Y\oplus u_Z\models \phi.\]
		%\marginpar{$G,u_Y\oplus v_Z\models \phi\iff G,u_Z\oplus v_Y\models \phi$}

% Before proving~\cref{pro:crossing}, we show how it yields  \cref{thm:uqw-stable}.
We now prove~\cref{thm:uqw-stable}, i.e.,
that any uniformly quasi-wide  class of graphs $\CCC$ is stable.

\begin{proof}%[of \cref{thm:uqw-stable}]
Fix a formula $\phi$ of quantifier rank $q$ and with free variables $X$, partitioned into disjoint sets $Y,Z$.
For $d=|X|$,
let $N^d(\cdot,\cdot)$ and $s^d(\cdot)$ be functions yielded by \cref{prop:uqw-tuples}.
Let $T(\cdot,\cdot,\cdot)$ and $r(\cdot)$ be the functions given by \cref{pro:crossing}.

We show that 
every $\phi$-ladder in a graph $G\in\cal C$ has length smaller than $\ell$. 
For the sake of contradiction, assume that there is a graph $G\in\cal C$, a number $k\ge \ell$,
and valuations $u_1,\ldots,u_k\in V(G)^Y$ and $ v_1,\ldots, v_k\in V(G)^Z$
which form a $\phi$-ladder in $G$.
Denote $w_i=u_i\oplus v_i$, for $i=1,\ldots,k$.
	Let $A=\set{ w_i\colon i=1,\ldots,k}\subset V(G)^X$; note that $|A|=k$.
Applying \cref{prop:uqw-tuples} to the set $A$, radius $2r$, and target size $m$
		 yields a set $S\subset V(G)$ with $|S|\le s^d(2r)$
	and a set $B\subset A$ with $|B|\geq m$ 
such that for all $i,j$ such that $w_i,w_j\in B$
	and $w_i\neq w_j$,
	the valuations $u_i$ and $v_j$ are $2r$-independent in $G-S$.
		Let $M=\bigcup_{w\in B} w(Y)$
		and let $N=\bigcup_{w\in B} w(Z)$.
	 Then $M$ and $N$ are $r$-separated by $S$.
	
% Replacing $A$ by $B$, we may assume that $\bar u_1,\ldots,\bar u_m\in V(G)^{\bar u}$ and $\bar v_1,\ldots,\bar v_k\in V(G)^{\bar v}$
% form a $\phi$-ladder of length $m$ in $G$, and
% that $A$ is totally $2r$-independent in $G-S$.
%
%
% An \emph{$q$-local formula} $\phi(\bar x)$ is a
% formula such that for every colored graph $G$ and tuple of vertices $\bar a\in V(G)^{\bar x}$, the following equivalence holds:
% $$G,\bar a\models \phi\qquad\textit {if and only if }\qquad G[N^{r}(\bar a)],\bar a\models \phi.$$
%
% \begin{theorem}\label{thm:gaifman}
% 	Let $\phi(\bar x)$ be a formula over the signature of colored graphs of quantifier rank $q$.
% 	There is a  number $r\le 7^{q}$
% 	% , and $s\le q+|\bar x|$,
% 	such that $\phi$ is equivalent to a Boolean combination of sentences and $q$-local formulas $\psi^{(r)}(\bar x)$.% the following:
% % 	\begin{itemize}
% % 		\item $q$-local formulas ;
% % 		\item sentences.%  of the form $$\exists y_1,\ldots,y_s
% % % \bigwedge_{1\le i\le s} \alpha^{(r)}(y_i)\land
% % % \bigwedge_{1\le i<j\le s} d^{>2r}(y_i,y_j),$$
% % % where $\alpha^{(r)}(y)$ is $q$-local and in one variable,
% % %  and $d^{>2r}(v,w)$ expresses the property that $\set{v,w}$ is $2r$-independent.
% % 	\end{itemize}
% \end{theorem}
With each tuple $v_i\in B$ we associate its $(q,S)$-type, as described in~\cref{pro:crossing}.
Since $B$ has at least $m>2T(q,|S|,d)$ elements, by
a variant of the pigeonhole principle, there are $i,j,e$ 
with $i<e<j$ such that $ w_i,w_e,w_j\in B$ and $v_i$ and $v_j$ have the same $(q,S)$-local types. 
Let $M=w_e(X)$ and let $N=w_i(X)\cup w_j(X)$.
Then $M$ and $N$ are $r$-separated by $S$,
since $w_i$ and $w_j$ are
mutually $r$-independent in $G-S$.

From~\cref{pro:crossing} it follows that $w_i$ and $w_j$ have the same quantifier rank $q$ type over $M$.
In particular, $u_e\oplus w_i\models\phi$ if and only if 
$u_e\oplus w_j\models\phi$,
contradicting the assumption that $u_1,\ldots, u_k$ and $ v_1,\ldots, v_k$ form a $\phi$-ladder. This finishes the proof of \cref{thm:uqw-stable}.
\end{proof}





We conclude by proving \cref{thm:new-stable}, promised in \cref{sec:intro}, which follows by tracking the precise dependencies on the parameters in the proof of \cref{thm:uqw-stable}.

\begin{proof}[of \cref{thm:new-stable}]
  Let $r(\cdot)$ be the function described in~\cref{pro:crossing}. We 
  define the function~$g(\cdot)$ from the statement of the theorem by $g(q)=10\cdot r(q)$.
	Let $\phi$ be the given formula of quantifier rank $q$ and with  free variables $X$, partitioned into $Y$ and $Z$.
  Denote $r=r(q)$. From now on we consider only graphs $G$ such that $K_t\not\minor_{10r} G$.

We first examine \cref{prop:uqw-tuples}, and in particular the dependence of the yielded functions 
$N^d(\cdot,\cdot)$ and $s^d(\cdot)$ on the assumed quasi-wideness properties of the class $\CCC$.
Precisely, having assumed that the underlying class $\CCC$ is quasi-wide with functions $N(\cdot,\cdot)$ and $s(\cdot)$,
we have~obtained:
$$
N^d(2r,m)=K(4r,m(d^2+1))\qquad\textrm{and}\qquad s^d(2r)=d\cdot s(4r),
$$
where $K(4r,m')$ is the $d$-fold composition of the function $f(m')=N(4r,m')\cdot m'$.
Thus, when establishing the values of $N^d(2r,m)$ and $s^d(2r)$, we refer to the quasi-wideness of $\CCC$ only by using numbers $s(4r)$ and $N(4r,m')$ for $m'\in \N$.
By \cref{thm:new-uqw}, it suffices to assume that $K_t\not\minor_{10r} G$ to have $s(4r)\leq t$ and $N(4r,m')\leq c(r,t)\cdot (m')^{(8t+1)^{2rt}}$ for some computable function $c(r,t)$.
Hence, this supposition alone, instead of full quasi-wideness of $\CCC$, is sufficient to claim that the conclusion of \cref{thm:new-uqw} holds with
$s^d(2r)$ bounded by a computable function of $t$, $d$, and $q$, and $N^d(2r,m)$ bounded by a computable function of $m$, $t$, $d$, and $q$.

	% In the proof of \cref{thm:uqw-stable} we have introduced the following parameters:
%   \begin{eqnarray*}
%   s=s^d(2r), \qquad T_i\le (s+d_i)^{d_i}\cdot |\mathrm{Tp}_{d_i}^{q,s}|\ \textrm{ for $i=1,2$},\qquad\textrm{and}\qquad m=T_1T_2+1.
%   \end{eqnarray*}
%   As we argued, $s$ is bounded by a computable function of $t$, $d$, and $q$.
%   It is well known that given $d_1,d_2,q,s$, one can compute the number of quantifier rank $q$ types $|\mathrm{Tp}_{d_1}^{q,s}|$ and $|\mathrm{Tp}_{d_2}^{q,s}|$; see e.g.~\cite{libkin}.
%   Thus, $m$ is also bounded by a computable function of $t$, $d$, and $q$.
	
Finally, in the proof of \cref{thm:uqw-stable} we have argued that the ladder index of $\phi$ is bounded by $N^d(2r,m)$, where $d=|X|$, $m=T(s^d(2r),q,d)+1$
and $T(\cdot,\cdot,\cdot)$ is a computable function,
and the only reference to the quasi-wideness of $\CCC$ in the proof
is by invoking \cref{prop:uqw-tuples}.
As we argued above, in \cref{prop:uqw-tuples} it suffices to assume $K_t\not\minor_{10r} G$ to claim that $N^d(2r,m)$ is bounded by a computable function of $m$, $t$, $d$, and~$q$.
Since $m$ is bounded by a computable function of $t$, $d$, and $q$, the obtained upper bound on the ladder index of $\varphi$ depends in a computable way on $t$, $d$, and~$q$,
and to derive it we only need to assume that $K_t\not\minor_{10r} G$. This concludes the proof.
\end{proof}

\begin{comment}
$N(\cdot,\cdot)$ be the function given by \cref{thm:new-uqw}.
	Following the proof of \cref{prop:uqw-tuples}, we see that for the functions $N^d(r,m)$ and $s^d(r)$, we can take the following functions:%
		\newcommand{\pow}{\ \uparrow\ }%
	\begin{align*}
	 N^d(r,m)&= c\cdot (m d)\pow(2(t+4))\pow(d(t+2r)),\\
	  s^d(r)&= d\cdot t,
	\end{align*}
where $x\pow y$ denotes $x^y$ and associates to the right, i.e., $x\pow y\pow z=x^(y^z)$.
In particular, the proof of \cref{prop:uqw-tuples} shows the following property
($\ast$)
for every graph $G$ such that $K_t\not\minor_{3r+1} G$ 
and for every set of $d$-tuples $A\subset V(G)^d$ of size at least $N^d(r,m)$
there is a subset $B\subset A$ of size $m$ and $S\subset V(G)$ of size at most $s^d(r)$
such that $B$ is mutually $r$-independent in $G-S$.

	
As in the proof of \cref{thm:uqw-stable}, let $r=7^q$,  
$s=s^d(2r)=d\cdot t$,  $T=(s+d)^d\cdot |\mathrm{Tp}_d^{q,s}|$ and $m=T+1$.



Let $\cal C$ be the class of all graphs $G$
such that $K_t\not\minor_{3r+1} G$, i.e., $K_t\not\minor_{3\cdot 7^q+1} G$. We repeat the argument presented in the second paragraph of the proof of \cref{thm:uqw-stable}, and assume that there is a $\phi$-ladder in $G\in \cal C$ of length larger than $N^d(2r,m)$.
 Instead of applying \cref{prop:uqw-tuples} we apply  ($\ast$) above
and obtain sets  $S$ and $B$ with the required properties. As before, this yields a contradiction. 


In particular, every $\phi$-ladder in a graph $G\in \cal C$ has length at most $N^d(2r,m)$, 
which is at most $$c'\cdot (d^{d+1}\cdot (t+1)^d\cdot |\mathrm{Tp}_d^{q,d\cdot t}|)\pow(2t+8)\pow(d\cdot t+2d\cdot 7^q),$$
where $c'$ is some constant.
As the number $|\mathrm{Tp}_d^{q,s}|$ is computable given $d,q,s$, this 
 yields \cref{thm:new-stable}.
\end{proof}
\end{comment}
