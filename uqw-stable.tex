
\section{From uniform quasi-wideness to stability}\label{sec:stable}

In this section we focus on proving the following result, observed earlier by Adler and Adler~\cite{adler2014interpreting}.

\begin{theorem}\label{thm:uqw-stable}
	Let $\cal C$ be a uniformly quasi-wide class of graphs.
	Then $\cal C$ is stable. % More precisely, if $\phi(\bar u,\bar v)$ is a formula with $d$ free variables and quantifier rank $q$, and $N^d(r,m)$ and $s^d(r)$ are as in Proposition~\ref{prop:uqw-tuples}, then for all graphs $G\in\cal C$,  the ladder index of $\phi$ is at most $N^d(2r,(2T)^d)$,
% where $r= 7^q$,
%  $T$ is the number of all quantifier rank $q$ types with one free  variable over the signature of  graphs colored with $s^d(2r)$ colors.
%\marginpar{shall we recall the definition of stability and ladders?}
\end{theorem}

Recall that a class $\cal C$ is stable if and only if for every first-order formula $\varphi(\bar x,\bar y)$, its ladder index over graphs from $\cal C$ is bounded by a constant depending only on $\cal C$ and $\varphi$;
see \cref{sec:intro} to recall the background on stability.
Thus \cref{thm:uqw-stable} is implied by \cref{thm:new-stable} (stated in \cref{sec:intro}), and is weaker in the following sense: \cref{thm:new-stable} asserts in addition that there is a computable bounds on the ladder index
of any formula that depends only on the size of an excluded clique minor on a levels bounded in terms of formula's quantifier rank and number of free variables. At the end of the proof we argue that
the obtained bounds in fact also imply the stronger statement of \cref{thm:new-stable}, but for the clarity of presentation we find it more instructive to work with the cleaner formulation of \cref{thm:uqw-stable}.

% The plan is as follows. In \cref{sec:uqw-tuples} we formulate a variant of uniform quasi-wideness tailored to tuples of vertices. Using this and Gaifman's Locality Theorem, we prove \cref{thm:uqw-stable} in \cref{sec:uqw-stable}.





% \subsection{Excluding long ladders}
% \label{sec:uqw-stable}



% In this language,
% assuming that  $X$ is partitioned into disjoint sets $Y,Z$, a $\phi$-ladder of length $n$ in a graph $G$ consists of two sequences of valuations $u_1,\ldots,u_n\in V(G)^Y$
%     and $v_1,\ldots,v_n\in V(G)^Z$
%      such that $G,u_i\oplus v_j\models \phi$ if and only if $i\le j$.
    
    
    
	% 	Let $u,v\in V(G)^X$, $u_Y$ and $u_Z$ be the restrictions of $u$ to $Y$ and $Z$, respectively,
% and $v_Y$ and $v_Z$ be the restrictions of $v$ to $Y$ and $Z$, respectively.
% 		We say that valuations $u$ and $v$ are \emph{confusing} for $\phi$
% 		if \[G,u_Y\oplus v_Z\models \phi\iff G,v_Y\oplus u_Z\models \phi.\]
		%\marginpar{$G,u_Y\oplus v_Z\models \phi\iff G,u_Z\oplus v_Y\models \phi$}

% Before proving~\cref{pro:crossing}, we show how it yields  \cref{thm:uqw-stable}.
We now prove~\cref{thm:uqw-stable}, i.e.,
that any uniformly quasi-wide  class of graphs $\CCC$ is stable.

\begin{proof}%[of \cref{thm:uqw-stable}]
Fix a formula $\phi(\bar y,\bar z)$ of quantifier rank $q$ and
a partitioning of its 
free variables $\bar x$ into tuples $\bar y$ and $\bar z$.
For $d=|\bar x|$,
let $N^d(\cdot,\cdot)$ and $s^d(\cdot)$ be functions yielded by \cref{prop:uqw-tuples}.
Let $p,r\in \N$ be the numbers given by \cref{lem:types}, and let
$T(\cdot,\cdot,\cdot)$ be the function given by \cref{lem:types}.

We show that 
every $\phi$-ladder in a graph $G\in\cal C$ has length smaller than $\ell$. 
For the sake of contradiction, assume that there is a graph $G\in\cal C$, a number $k\ge \ell$,
and tuples $u_1,\ldots,u_k\in V(G)^{|\bar y|}$ and $ v_1,\ldots, v_k\in V(G)^{|\bar z|}$
which form a $\phi$-ladder in $G$.
Denote $w_i=(u_i, v_i)$, for $i=1,\ldots,k$.
	Let $A=\set{ w_i\colon i=1,\ldots,k}\subset V(G)^X$; note that $|A|=k$.
Applying \cref{prop:uqw-tuples} to the set $A$, radius $2r$, and target size $m$
		 yields a set $S\subset V(G)$ with $|S|\le s^d(2r)$
	and a set $B\subset A$ with $|B|\geq m$ 
such that for all $i,j$ such that $w_i,w_j\in B$
	and $w_i\neq w_j$,
	the tuples $u_i$ and $v_j$ are $2r$-independent in $G-S$.
		Let $M=\bigcup_{w\in B} w(Y)$
		and let $N=\bigcup_{w\in B} w(Z)$.
	 Then $M$ and $N$ are $r$-separated by $S$.
	
% Replacing $A$ by $B$, we may assume that $\bar u_1,\ldots,\bar u_m\in V(G)^{\bar u}$ and $\bar v_1,\ldots,\bar v_k\in V(G)^{\bar v}$
% form a $\phi$-ladder of length $m$ in $G$, and
% that $A$ is totally $2r$-independent in $G-S$.
%
%
% An \emph{$q$-local formula} $\phi(\bar x)$ is a
% formula such that for every colored graph $G$ and tuple of vertices $\bar a\in V(G)^{\bar x}$, the following equivalence holds:
% $$G,\bar a\models \phi\qquad\textit {if and only if }\qquad G[N^{r}(\bar a)],\bar a\models \phi.$$
%
% \begin{theorem}\label{thm:gaifman}
% 	Let $\phi(\bar x)$ be a formula over the signature of colored graphs of quantifier rank $q$.
% 	There is a  number $r\le 7^{q}$
% 	% , and $s\le q+|\bar x|$,
% 	such that $\phi$ is equivalent to a Boolean combination of sentences and $q$-local formulas $\psi^{(r)}(\bar x)$.% the following:
% % 	\begin{itemize}
% % 		\item $q$-local formulas ;
% % 		\item sentences.%  of the form $$\exists y_1,\ldots,y_s
% % % \bigwedge_{1\le i\le s} \alpha^{(r)}(y_i)\land
% % % \bigwedge_{1\le i<j\le s} d^{>2r}(y_i,y_j),$$
% % % where $\alpha^{(r)}(y)$ is $q$-local and in one variable,
% % %  and $d^{>2r}(v,w)$ expresses the property that $\set{v,w}$ is $2r$-independent.
% % 	\end{itemize}
% \end{theorem}
With each tuple $v_i\in B$ we associate
$\tp^p(v_i/S)$, its
quantifier rank $p$ type over $S$, as described in~\cref{lem:types}.
Since $B$ has at least $m>2T(q,d,|S|)$ elements, by
a variant of the pigeonhole principle, there are $i,j,e$ 
with $i<e<j$ such that $ w_i,w_e,w_j\in B$ and $v_i$ and $v_j$ have the same quantifier rank $p$ types over $S$.
Let $M=w_e(X)$ and let $N=w_i(X)\cup w_j(X)$.
Then $M$ and $N$ are $r$-separated by $S$,
since $w_i$ and $w_j$ are
mutually $r$-independent in $G-S$.

From~\cref{lem:types} it follows that $w_i$ and $w_j$ have the same quantifier rank $q$ type over $M$.
In particular, $\phi(u_e, w_i)$ holds in $G$ if and only if 
$\phi(u_e, w_j)$ holds in $G$,
contradicting the assumption that $u_1,\ldots, u_k$ and $ v_1,\ldots, v_k$ form a $\phi$-ladder. This finishes the proof of \cref{thm:uqw-stable}.
\end{proof}





We conclude by proving \cref{thm:new-stable}, promised in \cref{sec:intro}, which follows by tracking the precise dependencies on the parameters in the proof of \cref{thm:uqw-stable}.

\begin{proof}[of \cref{thm:new-stable}]We 
  define the function~$g(\cdot)$ from the statement of the theorem as follows. Fix $q\in \N$.  
  Let $p,r$ be the numbers described in~\cref{lem:types}. Define $g(q)\coloneqq 10\cdot r$.
	Let $\phi$ be the given formula of quantifier rank $q$ and with  free variables $X$, partitioned into $Y$ and $Z$.
 From now on we consider only graphs $G$ such that $K_t\not\minor_{10r} G$.

We first examine \cref{prop:uqw-tuples}, and in particular the dependence of the yielded functions 
$N^d(\cdot,\cdot)$ and $s^d(\cdot)$ on the assumed quasi-wideness properties of the class $\CCC$.
Precisely, having assumed that the underlying class $\CCC$ is quasi-wide with functions $N(\cdot,\cdot)$ and $s(\cdot)$,
we have~obtained:
$$
N^d(2r,m)=K(4r,m(d^2+1))\qquad\textrm{and}\qquad s^d(2r)=d\cdot s(4r),
$$
where $K(4r,m')$ is the $d$-fold composition of the function $f(m')=N(4r,m')\cdot m'$.
Thus, when establishing the values of $N^d(2r,m)$ and $s^d(2r)$, we refer to the quasi-wideness of $\CCC$ only by using numbers $s(4r)$ and $N(4r,m')$ for $m'\in \N$.
By \cref{thm:new-uqw}, it suffices to assume that $K_t\not\minor_{10r} G$ to have $s(4r)\leq t$ and $N(4r,m')\leq c(r,t)\cdot (m')^{(8t+1)^{2rt}}$ for some computable function $c(r,t)$.
Hence, this supposition alone, instead of full quasi-wideness of $\CCC$, is sufficient to claim that the conclusion of \cref{thm:new-uqw} holds with
$s^d(2r)$ bounded by a computable function of $t$, $d$, and $q$, and $N^d(2r,m)$ bounded by a computable function of $m$, $t$, $d$, and $q$.

	% In the proof of \cref{thm:uqw-stable} we have introduced the following parameters:
%   \begin{eqnarray*}
%   s=s^d(2r), \qquad T_i\le (s+d_i)^{d_i}\cdot |\mathrm{Tp}_{d_i}^{q,s}|\ \textrm{ for $i=1,2$},\qquad\textrm{and}\qquad m=T_1T_2+1.
%   \end{eqnarray*}
%   As we argued, $s$ is bounded by a computable function of $t$, $d$, and $q$.
%   It is well known that given $d_1,d_2,q,s$, one can compute the number of quantifier rank $q$ types $|\mathrm{Tp}_{d_1}^{q,s}|$ and $|\mathrm{Tp}_{d_2}^{q,s}|$; see e.g.~\cite{libkin}.
%   Thus, $m$ is also bounded by a computable function of $t$, $d$, and $q$.
	
Finally, in the proof of \cref{thm:uqw-stable} we have argued that the ladder index of $\phi$ is bounded by $N^d(2r,m)$, where $d=|X|$, $m=T(s^d(2r),q,d)+1$
and $T(\cdot,\cdot,\cdot)$ is a computable function,
and the only reference to the quasi-wideness of $\CCC$ in the proof
is by invoking \cref{prop:uqw-tuples}.
As we argued above, in \cref{prop:uqw-tuples} it suffices to assume $K_t\not\minor_{10r} G$ to claim that $N^d(2r,m)$ is bounded by a computable function of $m$, $t$, $d$, and~$q$.
Since $m$ is bounded by a computable function of $t$, $d$, and $q$, the obtained upper bound on the ladder index of $\varphi$ depends in a computable way on $t$, $d$, and~$q$,
and to derive it we only need to assume that $K_t\not\minor_{10r} G$. This concludes the proof.
\end{proof}

\begin{comment}
$N(\cdot,\cdot)$ be the function given by \cref{thm:new-uqw}.
	Following the proof of \cref{prop:uqw-tuples}, we see that for the functions $N^d(r,m)$ and $s^d(r)$, we can take the following functions:%
		\newcommand{\pow}{\ \uparrow\ }%
	\begin{align*}
	 N^d(r,m)&= c\cdot (m d)\pow(2(t+4))\pow(d(t+2r)),\\
	  s^d(r)&= d\cdot t,
	\end{align*}
where $x\pow y$ denotes $x^y$ and associates to the right, i.e., $x\pow y\pow z=x^(y^z)$.
In particular, the proof of \cref{prop:uqw-tuples} shows the following property
($\ast$)
for every graph $G$ such that $K_t\not\minor_{3r+1} G$ 
and for every set of $d$-tuples $A\subset V(G)^d$ of size at least $N^d(r,m)$
there is a subset $B\subset A$ of size $m$ and $S\subset V(G)$ of size at most $s^d(r)$
such that $B$ is mutually $r$-independent in $G-S$.

	
As in the proof of \cref{thm:uqw-stable}, let $r=7^q$,  
$s=s^d(2r)=d\cdot t$,  $T=(s+d)^d\cdot |\mathrm{Tp}_d^{q,s}|$ and $m=T+1$.



Let $\cal C$ be the class of all graphs $G$
such that $K_t\not\minor_{3r+1} G$, i.e., $K_t\not\minor_{3\cdot 7^q+1} G$. We repeat the argument presented in the second paragraph of the proof of \cref{thm:uqw-stable}, and assume that there is a $\phi$-ladder in $G\in \cal C$ of length larger than $N^d(2r,m)$.
 Instead of applying \cref{prop:uqw-tuples} we apply  ($\ast$) above
and obtain sets  $S$ and $B$ with the required properties. As before, this yields a contradiction. 


In particular, every $\phi$-ladder in a graph $G\in \cal C$ has length at most $N^d(2r,m)$, 
which is at most $$c'\cdot (d^{d+1}\cdot (t+1)^d\cdot |\mathrm{Tp}_d^{q,d\cdot t}|)\pow(2t+8)\pow(d\cdot t+2d\cdot 7^q),$$
where $c'$ is some constant.
As the number $|\mathrm{Tp}_d^{q,s}|$ is computable given $d,q,s$, this 
 yields \cref{thm:new-stable}.
\end{proof}
\end{comment}
