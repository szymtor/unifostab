
\section{Bounds on the $k$-order property}\label{sec:stable}

\begin{theorem}
Let $\psi(\tup{x})$ be a formula and let $v_1,\ldots, v_m\in V(G)$. 
There is a formula $\vartheta(\tup{x})$ 
of the same quantifier rank as $\psi$ over a signature extended 
with $m$ colors such that for all $\tup{a}$ which do not
contain elements from $v_1,\ldots, v_m$ it holds that
$G\models\psi(\tup{a})\Leftrightarrow G\models\vartheta(\tup{a})$. 
\end{theorem}
\begin{proof}
As a preliminary step, we introduce $m$ constant symbols 
$c_1,\ldots, c_m$. We replace in $\psi$ all quantifiers 
$\exists x\xi(x,\tup{y})$ by $\exists x((x=c \wedge \xi^*(x,\tup{y})
\vee (x\neq c\wedge \xi^*(x,\tup{y}))$, 
where $\xi^*$ is the formula obtained by inductively continuing the
construction and similarly for the universal quantifier. Furthermore, 
we replace every atomic formula $E(c_i,c_j)$ by the truth value of
$E(v_i,v_j)$. Then we have
\[(G,v_1,\ldots, v_m)\models\psi^*(\tup{a})\Leftrightarrow G\models\psi(\tup{a}).\]
Obviously, the quantifier rank of $\psi^*$ is the same as that of $\psi$. 

We now build the formula $\psi^{**}$ as follows. We replace
every subformula $\exists x((x=c_i \wedge \xi^*(x,\tup{y}) \vee 
(x\neq c_i \wedge \xi^*(x,\tup{y}))$ by the formula
$\exists x(\xi^{***}(\tup{y}) \vee \xi^{**}(x,\tup{y}))$, 
where $\xi^{***}(\tup{y})$ is obtained from $\xi^{*}$ by
replacing every atom $E(c_i,y)$ by the atom $R_i(y)$. Here, 
$R_i$ is a new unary predicate which holds true exactly for 
the neighbors of $c_i$ in $G$. Note that $c_i$ occurs only together
with variables in atoms because we eliminated all other occurrences
before. Then we have 
\[G-\{v_1,\ldots, v_m\}\models \psi^{**}(\tup{a})
\Leftrightarrow (G,v_1,\ldots, v_m)\models\psi^*(\tup{a}).\]
We let $\vartheta(\tup{x})\coloneqq \psi^{**}(\tup{x})$ and conclude. 
\end{proof}

We can now translate $\vartheta(\tup{a})$ to Gaifman normal form. 
Note that the locality radius of the resulting formula does not depend
on the number $m$ of elements we delete. Only the number of local types
depends on this number, as the signature of the new formula is changed
depending on $m$. 

\begin{theorem}
Let $\psi(\tup{x})$ be a formula and let $v_1,\ldots, v_m\in V(G)$. 
There is an $r$-local formula $\vartheta(\tup{x})$ 
over a signature extended 
with $m$ colors such that for all $\tup{a}$ which do not
contain elements from $v_1,\ldots, v_m$ it holds that
$G\models\psi(\tup{a})\Leftrightarrow G\models\vartheta(\tup{a})$. 
Here, $r=2^{q+|\tup{x}|}$ by Gaifman's Theorem.
\end{theorem}

We now prove a tuple-wise uniform quasi-wideness, as Podewski and Ziegler 
prove in the infinite. For a set $S$ and two tuples $\tup{a},\tup{b}\in V(G)^k$
we write $\dist_{G-S}(\tup{a},\tup{b})>r$ if $\dist_{G-S}(x,y)>r$ for all $x,y\in 
(\tup{a}\cup\tup{b})\setminus S$. 

\begin{lemma}
For all $k,m,r\in \N$, there exist $M(k,m,r)\in \N$ and 
$s(k,m,r)\in \N$ such that if $A$ is a set of $k$-tuples
with $|A|>M$, then there exists a set $S\subseteq V(G)$
with $|S|\leq s$ such that $\dist_{G-S}(\tup{a},\tup{b})>r$
for all $\tup{a},\tup{b}\in A$. 
\end{lemma}
\begin{proof}

\end{proof}
